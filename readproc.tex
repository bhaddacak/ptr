\chapter{Process of P\=ali text reading}\label{chap:readproc}

Now we will take P\=ali text into consideration, after we went through a long, and quite boring to some, theoretical journey. You are supposed to know those foundations, so you should not wonder why at some parts I am so fussy and critical about certain points. What I call `reading' here means more or less (crude) translating. The end result of reading is understanding, not a beautiful translation.

As laid down in the principle (Chapter \ref{chap:principle}), reading belongs to Step 2 of the process. However, as the process runs in circle, `reading' here means more than mere reading. It involves all steps mentioned in the principle. So, you are supposed to do other steps in the process as well to make the reading complete. To be precise on the part of reading, I set up its own process step-by-step as follows:

\paragraph*{1.\ Section analysis} This is the first thing we do when we encounter any text. This corresponds to text composition analysis mentioned in Chapter \ref{chap:analysis} as an intratextual factor. We have to know when the text starts and ends, how many parts are there, which part is heading, which is explanation, which is definition, which is anecdotal story, which is dialogue, which is poetry, which is summary, and so on. Some types of section are easy to recognize, for example, definition and dialogues can be detected by `\pali{-ti}' or `\pali{iti},' and verses can be seen directly by their arrangement. Some sections may be more difficult to identify. Just note them as unclear, and come to those again when helpful information is available.

\paragraph*{2.\ Breaking down sentences} This step is easy when we read modern P\=ali text collection. P\=ali sentences can be separated by a period, like normal English sentences. We have to know where a sentence starts and ends. Challenging aspects of P\=ali sentences is some of them are very long and complex, and it is not necessary that one sentence will have only one finite verb; some even have none. One complex sentence may comprise several relative clauses. We have to decompose those clauses as well (after part of speech analysis is done). In verses, sentence decomposition can be a little more difficult.

\paragraph*{3.\ Sentence decomposition} When all sentences are broken down, then we read them one by one. The next step we do is to break down each sentence further to its components. However, we cannot do this without knowing part of speech of the components. So, this step and the next one are processed in tandem. A guideline to sentence decomposition is shown in Chapter \ref{chap:sentence}.

\paragraph*{4.\ Part of speech analysis} This is the main work of reading process. To understand a sentence in any language, we have to know what its components do. We have to tell which part is noun, which is verb, which is adjective, adverb and so on. The knowledge essential to do this analysis is P\=ali grammar. That is really a lot to know, and I will not deliberately talk about it anymore in this book. For more information, see PNL. However, a quick summary is drawn as a guideline in Chapter \ref{chap:pos}.

\paragraph*{5.\ Draft translation} When we know all parts of a sentence, then we can give certain meaning to it. Word-for-word translation is the most suitable method in this step. If we can do well in sentence decomposition and part of speech analysis, this step should be easy to manage. By word-for-word translation, it does not mean you just translate one word after the other and string them together in the exact original order. That will make no sense. And you can do it as such in sentence decomposition, if you like.

Word-for-word translation, rather, is an attempt to keep the grammatical structure and the meaning of the original as much as possible, yet the outcome is intelligible. There can be a matter of degree how much we trade off between intactness and comprehensibleness. In my application, I value comprehensibleness more. So, as we shall see fully in Part III and subsequent chapters with P\=ali translations, if necessary, I will change the original order, or even structure, to gain a more intelligible translation. The main reason of this, to me, is translations are always tentative, and we should go back to P\=ali sources when we have a serious reading.

Sometimes, we may encounter multiple options, so alternative translations may also be presented. And do not be discouraged in ugly, but understandable, translation.
