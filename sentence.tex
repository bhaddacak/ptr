\chapter{Sentence decomposition}\label{chap:sentence}

Sentence is the fundamental unit of reading and translation. Any meaning makes sense only at sentence level, not word level. Sometimes the context is necessary too. So, nearby sentences can help in sense making. In this chapter we will focus on sentence-level translation.

Before we read a sentence, we have to break down sentences from paragraphs. In modern P\=ali collection, it is quite easy to do because of the use of periods, or sentence-ending markers in other scripts. So, assume we have demarcated sentences and we will read them one by one. What should we do to accomplish the reading? You may think ``just read it.'' Even if it sounds true, a guideline can be very helpful to new learners. I take the guidance below from Thai tradition.

From Thai translators' perspective, there is the best practice of what should be translated first, and what should be done in the succession. I try to keep the order, but applying this to English translation can be irrelevant. So, for English translators, we will loosely follow the sequence with a wide leeway. You can make your own decision if you see it fits to your specific purposes.

Steps enumerated by Thai P\=ali teachers concerning sentence translation have no standard version. Some may give eight items, some nine, and some others. But the main idea is all the same. This list comes from my own adaptation for English translation.

Remember that not every sentence has all these components. Some simple sentences may have only one or two parts. And to ease my diagramming, I will also use the shortened names of these systematically (see the example section below).

\paragraph*{1.\ Addressing words (\textbf{A})} Basically, nouns in vocative case are used in this purpose. Also some particles can do the job, such as \pali{bhante, \=avuso, bha\d ne, ambho, he, re,} etc. They are easy to detect mostly. Normally, P\=ali sentences have an addressing word at the beginning, maybe the first or a few next position. Sometimes it is placed at the end. You may follow this strict position or pick it first to translate (as Thais do).

\paragraph*{2.\ Opening particles (\textbf{P})} Some particles are often used at the beginning of sentences (it may be not at the first position, but close). These function like adverb or conjunction in English. Some particles in this group are listed below (some may have multiple meaning). If these particles are found in a sentence, they should be translated soon after addressing words.\par
\begin{compactenum}[(1)]
\item Hearsay (as it is heard): \pali{kira, khalu, suda\d m}\par
\item Supposition (if, whether): \pali{atha, yadi, ce, sace, appevan\=ama, yann\=una, ud\=ahu, \=ad\=u}\par
\item Response (yes): \pali{\=ama, \=amant\=a, eva\d m}\par
\item Persuasion (shall we): \pali{handa, i\.ngha, taggha} \par
\item Connector (but, also, furthermore, because): \pali{ca, pana, hi, tu, apica, athav\=a, tath\=a} \par
\item Cause (because): \pali{tasm\=a, tena, tenahi} \par
\item Negation (not): \pali{na, m\=a} \par
\item Other: \pali{atho} (then, also), \pali{aho} (oh), \pali{ki\~nc\=api} (whatever, however much, but), \pali{tath\=api} (even so), \pali{iti} (thus) \par
\end{compactenum}

All these can be called `sentence particles,' which affect the meaning of the whole sentence. Other particles that are used at word or phrase level, such as \pali{ca, v\=a, pi, viya, eva, saddhi\d m}, etc., will be treated as a part of the related words (see examples below). By this treatment, we do not need to classify these particles by their function, like traditional students do. That saves us a lot of energy.

\paragraph*{3.\ Locative markers (\textbf{L})} In Thai tradition, only time markers (\pali{k\=alasattam\=i}) are mentioned. I think we should include place markers here too. Locative markers have two kinds. The first comes from terms in locative, accusative, or instrument case. The second comes from some particles.\footnote{For more information, see locative particles in Appendix \externalref{F} of PNL.} By grammatical function, these markers can be grouped as a part related to the main verb (see below). They are singled out because they should be translated as soon as the opening is done. In Thai, it sounds natural when we mention time in this step (following the P\=ali sentences). But in English, if it sounds sensible, following the position of the markers in P\=ali sentences is better. If not, you can shift the translation to a more suitable position.\par
\begin{compactenum}[(1)]
\item Time markers with declension, e.g.\ \pali{ekasmi\d m samaye} (in one occasion), \pali{eka\d m samaya\d m} (in one occasion), \pali{ekena samayena} (by one occasion), \pali{ekadivasa\d m} (in one day), \pali{ta\d m divasa\d m} (in that day), \pali{athekadivasa\d m} (in that time one day)\par
\item Place markers with declension (mostly loc.), e.g.\ \pali{S\=avatthiya\d m} (in S\=avatth\=i), \pali{Ves\=aliya\d m} (in Ves\=al\=i), \pali{vih\=are} (in a dwelling place), \pali{antar\=amagge} (on the road)\par
\item Particles of time, e.g.\ \pali{atha} (in that time), \pali{p\=ato} (in the morning), \pali{s\=aya\d m} (in the evening), \pali{div\=a} (by day), \pali{ajja} (today), \pali{suve} (tomorrow), \pali{hiyyo} (yesterday), \pali{tato} (from that time)\par
\item Particles of place, e.g.\ \pali{samant\=a} (everywhere), \pali{ekamanta\d m} (in one proper area), \pali{parito} (in the surrounding area), \pali{upari/uddha\d m} (in the upper part), \pali{adho} (in the lower part)\par
\end{compactenum}

\paragraph*{4.\ Subject (\textbf{S})} Basically, nouns (including compounds) and pronouns in nominative case function as subject of sentences. However, some nouns can be formed derivatively. So, some of them may look similar to verbs, for example, \pali{kara\d na\d m} (doing), \pali{gamana\d m} (going), \pali{gantabba\d m} (going). 

Sometimes there are several potential subjects in a sentence. To make sure which one is really the subject, we have to check it with the main verb (if any). The subject of a sentence has to agree with the verb in number and person, e.g.\ \pali{puriso gacchati} (sg., 3rd), \pali{tumhe gacchatha} (pl., 2nd).

In many sentences, there is only verb, no subject mentioned. By the verb form, we can know the unsaid subject. And it is suggested that we should insert the omitted subject back to the sentences to clarify our reading, e.g.\ \pali{eva\d m [tva\d m] vadehi} ([You] say in this way).

In case the verb takes a derivative form (\pali{an\=iya, tabba,} or \pali{ta}), the subject has to agree with this in gender, number, and case, e.g.\ \pali{bhojan\=ani bhu\~njitabb\=ani} (Foodstuffs should be eaten).

Apart from terms in nominative case, some particles can be seen as subject in particular contexts. These are \pali{ala\d m} (enough), \pali{sakk\=a} (able to), \pali{tath\=a} (thus), \pali{eva\d m} (thus), \pali{ajja} (today), for example. Moreover, verbs in \pali{tu\d m} form can also be treated as subject in some cases, like we use an infinitive as subject of a sentence.

There is a different concern between the tradition and English when we talk about subject. In impersonal passive sentence (\pali{bh\=avav\=acaka}), like \pali{may\=a sayate} (Sleeping is done by me), the instrumental agent is counted as subject by the tradition.\footnote{This is called \pali{anabhihitakatt\=a}.} In English it is unusual to see as such because in that sentence there is no subject at all. In practice, we can force the active meaning to the sentence, hence ``I sleep'' which costs a grammatical loss. My treatment is we change the passive verb to action (\pali{bh\=ava}) noun. For more information, see Chapter \externalref{32} and \externalref{38} of PNL.

\paragraph*{5.\ Subject modifiers (\textbf{M})} Once we identify the subject, the parts modifying it can be seen accordingly. Basically they are adjectives which take the same gender, number, and case as the subject, e.g.\ \pali{th\=ulo bi\d l\=aro} (a fat cat). Sometimes the modifiers can be other kinds of word, such as pronominal adjective, \pali{so bi\d l\=aro} (that cat); derivative noun, \pali{nagarav\=asino jano} (an urban person); participial phrase, \pali{g\=ama\d m gato coro} (a thief having gone to the village), \pali{dh\=avanto bi\d l\=alo}\footnote{Normally if present participle, (\pali{anta} and \pali{m\=ana} verbs), is placed before a noun, it is said to be modifier. If it is placed after a noun, it is a verb of subordinate clause, e.g.\ \pali{bi\d l\=alo dh\=avanto bh\=ayati} (A cat, running, fears).} (a running cat); numeral, \pali{sat\=a bi\d l\=al\=a} (100 cats). Subject complements\footnote{Subject complements are noun or adjective in nom.\ that completes verb `to be' in a sentence. In the same manner, object complements modify the object of verb `to do/make.' But they can be in both nom.\ (in passive structure) and acc.\ (in active structure). For some more information, see Chapter \ref{chap:pos}.} are also counted as modifier, as well as object complements in passive sentences.

If the modifiers are also nouns, they can be used by putting them in an appropriate case\footnote{In P\=ali, two nouns in the same case do not modify each other. Instead, one is seen as apposition to the other.}, for example, \pali{d\=arakassa bi\d l\=alo} (a boy's cat), \pali{g\=ame geha\d m} (a house in the village), \pali{kapa\d n\=ana\d m bhojana\d m} (food for beggars).

\paragraph*{6.\ Subordinate clauses and absolute constructions (\textbf{C})} \ \par Subordinate clauses embedded in a P\=ali sentence are very common. These clauses can take several forms described below. Once we see a subordinate clause, we also see its components like those in a normal sentence. So, we can analyze the clause recursively like we do to the outer sentence, except the two parts use different verb forms. 
\begin{compactenum}[(1)]
\item With verbs in \pali{anta} and \pali{m\=ana} form, e.g.\ \pali{bi\d l\=alo dh\=avam\=ano sunakh\=a bh\=ayati} (A cat, running, fears [from] a dog.)\par
\item With verbs in \pali{ta} form, e.g.\ \pali{bi\d l\=alo sunakh\=a bh\=ayito dh\=avi} (A cat, having feared a dog, ran.)\par
\item With verbs in \pali{tv\=a} form, e.g.\ \pali{bi\d l\=alo sunakh\=a bh\=ayitv\=a dh\=avati} (A cat, having feared a dog, runs.)\par
\item Absolute construction (in gen.), e.g.\ \pali{bi\d l\=alassa dh\=avantassa, sunakho anudh\=avati} (When a cat runs, a dog runs after.)\par
\item Absolute construction (in loc.), e.g.\ \pali{sunakhasmi\d m anudh\=avantasmi\d m, bi\d l\=alo bh\=ayati} (When a dog runs after, a cat fears.)\par
\end{compactenum}

\paragraph*{7.\ Main verb (\textbf{V})} This is the most important part of a sentence. Even though a P\=ali sentence can be verbless, normally we can see the main verb in a sentence, in one form or another. In the early stage of practice, it is suggested that the omitted verb should be inserted to clarify our understanding. In most cases, we can use \pali{hoti, bhavati,} or \pali{atthi} with their corresponding forms as the left out. There are some nuances of these, though. If the verb is about a state of being, \pali{hoti} or \pali{bhavati} is used.\footnote{We use \pali{hoti} most of the time for its handiness. Omitted verbs can also belong to subordinate clauses. In this case, \pali{hutv\=a} is inserted.} If the verb implies a possession, \pali{atthi} is preferred. 

When we encounter a sentence, the verb should be located immediately, even though it will be translated in a later step. What counts as main verb can be listed as follows:
\begin{compactenum}[(1)]
\item \pali{\=Akhay\=ata}, e.g.\ \pali{dh\=avati} (to run)\par
\item Verbs in \pali{ta} form, e.g.\ \pali{bi\d l\=alo sunakh\=a bh\=ayito} (A cat feared a dog.)\par
\item Verbs in \pali{an\=iya} and \pali{tabba} form (only passive), e.g.\ \pali{bi\d l\=alena sunakho bh\=ayitabbo} (Dog should be feared by cat.)\par
\item Verbs in \pali{tv\=a} form (rare), see \pali{\=arabbha} in the example section below\par
\end{compactenum}

\paragraph*{8.\ Parts related to the main verb (\textbf{R})} These can mean several things, for example, object of the verb (noun in acc.), object complements (noun/adj.\ in acc.), adverbials (adjective in acc.\ and particles), etc. Locative markers that suitably belong to the whole sentence are excluded from this group, as we have mentioned earlier. In addition, clauses of direct speech which relate to the verb somehow are so widely used that they should be considered as an item of their own.

\paragraph*{9.\ Direct speech (\pali{iti}) clauses (\textbf{I})} These clauses are normally marked by \pali{iti} or just \pali{-ti} welded to the preceding word. When we find a direct speech part in a sentence, we just take it out and run the process over it as we do to a normal sentence. There can be multiple layers of \pali{iti} clause. So, we process them recursively as needed.

\section*{Examples}

To illustrate the process, several examples will be given here. Some are simple, some are complex. All examples are taken from the story of Cakkhup\=ala, the first one in the commentary to Dhammapada (Dhp-a\,1.1). I use text from the commentary for a number of reasons. First, it is extensively studied by Thai tradition. So, it is quite well-documented. Second, it contains all kinds of structures we have learned. Generally, sentences in commentaries are more complex than in the canon. So, if we can tackle the commentaries, reading the canon itself will be relatively easy. Third, the stories in this commentary are enjoyable to read. It is better to read them as parables rather than historical records, though.

The readers may feel baffled at first by my diagramming scheme of sentence decomposition. To very new learners, this activity is very helpful to systematize their reading. I do not want to put much explanation on this because the diagram is self-explaining. And it is better to learn by practice. So, I recommend that the readers should follow my analysis meticulously, especially in complicated sentences. And you may read through these examples roughly in the first reading. Until you finish Chapter \ref{chap:pos}, your understanding should be better in the subsequent readings.

\bigskip
\setcounter{sennum}{1}
\arabic{sennum}. \pali{Aya\d m dhammadesan\=a kattha bh\=asit\=a'ti? S\=avatthiya\d m. Ka\d m \=arabbh\=a'ti? Cakkhup\=alatthera\d m.}\\[1mm]
\small
\fbox{\pali{Aya\d m} [M]} \fbox{\pali{dhammadesan\=a} [S]} = this; teaching\\[0.5mm]
\fbox{\pali{kattha} [L]} \fbox{\pali{bh\=asit\=a} [V]} = in what place; was said\\[0.5mm]
\fbox{\pali{iti} [P]} = thus\\[1mm]
\normalsize
$\triangleright$ Thus, where was this teaching delivered?\\[1.5mm]
\small
\fbox{\pali{S\=avatthiya\d m} [L]} = in S\=avatth\=i\\[1mm]
\normalsize
$\triangleright$ [This teaching was delivered] in S\=avatth\=i.\\[1.5mm]
\small
\fbox{\pali{Ka\d m} [R]} \fbox{\pali{\=arabbha} [V]}\footnote{This is a \pali{tv\=a} form of \pali{\=arabhati} (to begin). See \pali{\=arabbha} in PTSD. In this sentence, it functions as the main verb.} \fbox{\pali{iti} [P]} = to whom; referred; thus\\[1mm]
\normalsize
$\triangleright$ Thus, to whom did [the teaching] refer to?\\[1.5mm]
\small
\fbox{\pali{Cakkhup\=alatthera\d m} [R]} = Ven.\ Cakkhup\=ala\\[1mm]
\normalsize
$\triangleright$ [The teaching referred to] the Venerable Cakkhup\=ala.\\[1.5mm]

\stepcounter{sennum}\medskip
\arabic{sennum}. \pali{S\=avatthiya\d m kira mah\=asuva\d n\d no n\=ama ku\d tumbiko ahosi a\d d\d dho mahaddhano mah\=abhogo aputtako.}\\[1mm]
\small
\fbox{\pali{S\=avatthiya\d m} [L]} \fbox{\pali{kira} [P]} = in S\=avatth\=i; it is heard thus\\[0.5mm]
\fbox{\pali{mah\=asuva\d n\d no n\=ama} [M]} = named Mah\=asuva\d n\d na\\[0.5mm]
\fbox{\pali{ku\d tumbiko} [S]} \fbox{\pali{ahosi} [V]} = a house-holder; was\\[0.5mm]
\fbox{\pali{a\d d\d dho mahaddhano mah\=abhogo aputtako} [M]}\\[0.5mm]
= wealthy; having great riches; having great wealth; having no son\\[1mm]
\normalsize
$\triangleright$ In S\=avatth\=i, as it is heard, there was a house-holder called Mah\=asuva\d n\d na, very wealthy, childless.\\[1.5mm]

\stepcounter{sennum}\medskip
\arabic{sennum}. \pali{So ekadivasa\d m nh\=anatittha\d m nhatv\=a natv\=a \=agacchanto antar\=amagge sampannapattas\=akha\d m eka\d m vanappati\d m disv\=a ``aya\d m mahesakkh\=aya devat\=aya pariggahito bhavissat\=i''ti tassa he\d t\d th\=abh\=aga\d m sodh\=apetv\=a p\=ak\=araparikkhepa\d m k\=ar\=apetv\=a v\=aluka\d m okir\=apetv\=a dhajapa\d t\=aka\d m uss\=apetv\=a vanappati\d m ala\.nkaritv\=a a\~njali\d m karitv\=a ``sace putta\d m v\=a dh\=itara\d m v\=a labheyya\d m, tumh\=aka\d m mah\=asakk\=ara\d m kariss\=am\=i''ti patthana\d m katv\=a pakk\=ami.}\\[1mm]
\small
\fbox{\pali{So} [S]} \fbox{\pali{ekadivasa\d m} [L]} = He; in one day\\[0.5mm]
\fbox{\fbox{\pali{nh\=anatittha\d m} [C1.R]} \fbox{\pali{nhatv\=a} [C1.V]} [C1]}\footnote{In Thai, it is \pali{nh\=atv\=a}.}\\[0.5mm]
= at bathing waterside; having bathed\\[0.5mm]
\fbox{\pali{natv\=a} [C2]}\footnote{This word is dubious. Literally, it means `having bent/stooped' which has no sense in this context. In Thai, it appears as \pali{gantv\=a} instead. I think it is possibly \pali{y\=atv\=a} (having gone). In Thai, the order is different, thus ``\pali{nh\=anatittha\d m gantv\=a nh\=atv\=a}'' (having gone to the bathing waterside, having bathed). That makes more sense, because \pali{nh\=anatittha\d m} is in acc.\ not loc.} = ? having gone\\[0.5mm]
\fbox{\pali{\=agacchanto} [C3]} = [while] coming back [home]\\[0.5mm]
\fbox{\fbox{\pali{antar\=amagge} [C4.L]}} = on the road\\[0.5mm]
\fbox{\fbox{\pali{sampannapattas\=akha\d m} [C4.R]}} = full of spreading branches\\[0.5mm]
\fbox{\fbox{\pali{eka\d m vanappati\d m} [C4.R]}} = one big tree\\[0.5mm]
\fbox{\fbox{\pali{disv\=a} [C4.V]} [C4]} = having seen\\[0.5mm]
\fbox{\fbox{[\pali{cintetv\=a}] [C5.V]} [C5]}\footnote{The following direct speech clause is in thinking, so this verb should be added.} = having thought\\[0.5mm]
\fbox{\fbox{\pali{aya\d m} [C5.I.S]}} = this [big tree]\\[0.5mm]
\fbox{\fbox{\pali{mahesakkh\=aya devat\=aya pariggahito} [C5.I.R]}}\\[0.5mm]
= by a powerful god; was possessed\\[0.5mm]
\fbox{\fbox{\pali{bhavissati} [C5.I.V]} \fbox{\pali{iti} [C5.P]}} = will be\footnote{This future verb does not express a prediction, but it is a certain speculation. We can render it as `must be.''}; thus\\[0.5mm]
\fbox{\fbox{\pali{tassa he\d t\d th\=abh\=aga\d m} [C6.R]} \fbox{\pali{sodh\=apetv\=a} [C6.V]} [C6]}\\[0.5mm]
= of that [tree]; the ground under; having made clean\\[0.5mm]
\fbox{\fbox{\pali{p\=ak\=araparikkhepa\d m} [C7.R]} \fbox{\pali{k\=ar\=apetv\=a} [C7.V]} [C7]}\\[0.5mm]
= surrounding wall; having made create\\[0.5mm]
\fbox{\fbox{\pali{v\=aluka\d m} [C8.R]} \fbox{\pali{okir\=apetv\=a} [C8.V]} [C8]}\\[0.5mm]
= sand; having made spread\\[0.5mm]
\fbox{\fbox{\pali{dhajapa\d t\=aka\d m} [C9.R]} \fbox{\pali{uss\=apetv\=a} [C9.V]} [C9]}\\[0.5mm]
= a flag and banner; having made put up\\[0.5mm]
\fbox{\fbox{\pali{vanappati\d m} [C10.R]} \fbox{\pali{ala\.nkaritv\=a} [C10.V]} [C10]}\\[0.5mm]
= the big tree; having decorated\\[0.5mm]
\fbox{\fbox{\pali{a\~njali\d m} [C11.R]} \fbox{\pali{karitv\=a} [C11.V]} C11]}\\[0.5mm]
= homage with lifting joined palms; having done\\[0.5mm]
\fbox{\fbox{[\pali{cintetv\=a}] [C12.V]} [C12]} = having thought\\[0.5mm]
\fbox{\fbox{\pali{sace} [C12.I.P]} \fbox{\pali{putta\d m v\=a dh\=itara\d m v\=a} [C12.I.R]}}\\[0.5mm]
= if; son or daughter\\[0.5mm]
\fbox{\fbox{\pali{labheyya\d m} [C12.I.V]}} = [I] should get\\[0.5mm]
\fbox{\fbox{\pali{tumh\=aka\d m mah\=asakk\=ara\d m} [C12.I.R]}} = for you; a great honor\\[0.5mm]
\fbox{\fbox{\pali{kariss\=ami} [C12.I.V]} \fbox{\pali{iti} [C12.P]}} = [I] will do; thus\\[0.5mm]
\fbox{\fbox{\pali{patthana\d m} [C13.R]} \fbox{\pali{katv\=a} [C13.V]} [C13]} = vow, having made\\[0.5mm]
\fbox{\pali{pakk\=ami} [V]} = left\\[1mm]
\normalsize
$\triangleright$ Having bathed at the waterside, [while] coming back [home], having seen a big tree full of spreading branches, [having thought] thus ``this [big tree] must be possessed by a powerful god,'' [then] having made [someone] clean the ground under it, having made [someone] create surrounding wall, having made [someone] spread sand [within the wall], having made [someone] put up a flag and banner, having decorated the big tree, having paid homage with lifting joined palms, [then] having thought thus ``If I get a son or daughter, I will pay you a great honor,'' having made the vow, he left.\\[1.5mm]

\stepcounter{sennum}\medskip
\arabic{sennum}. \pali{``Ki\d m kathesi bh\=atika, tva\d m me m\=atari mat\=aya m\=at\=a viya, pitari mate pit\=a viya laddho, gehe te mah\=avibhavo, sakk\=a geha\d m ajjh\=avasanteheva pu\~n\~n\=ani k\=atu\d m, m\=a eva\d m karitth\=a''ti.}\footnote{This P\=ali sentence is a run-on. It can be broken into several English sentences with their own verb, five in total. Note on the numbers used. If you see the leading `I' below is redundant, you can omit it in your own work. I want to make everything clear here. Or you may translate sentence by sentence and leave out the running numbers.}\\[1mm]
\small
[ \fbox{\pali{kani\d t\d thabh\=at\=a} [S]} \fbox{\pali{vadi} [V]} ]\footnote{This part is not in the text, but it is added for clarification.} = the younger brother; said\\[0.5mm]
\fbox{\pali{Ki\d m} [I.R1]} [ \fbox{\pali{tva\d m} [I.S1]} ] \fbox{\pali{kathesi} [I.V1]} \fbox{\pali{bh\=atika} [I.A1]}\\[0.5mm]
= what; [you]; said; brother\\[0.5mm]
\fbox{\pali{tva\d m} [I.S2]} \fbox{\pali{me} [I.R2]} = you; by me (ins.)\\[0.5mm]
\fbox{\fbox{\pali{m\=atari} [I.C1.S]} \fbox{\pali{mat\=aya} [I.C1.V]} [I.C1]} = [when] mother; died\\[0.5mm]
\fbox{\pali{m\=at\=a viya} [I.C1.M]}\footnote{See more on simile in Chapter \ref{chap:simile}.} = like mother\\[0.5mm]
[ \fbox{\pali{laddho} [I.V1]} ] = [was] got (treated)\\[0.5mm]
\fbox{\fbox{\pali{pitari} [I.C2.S]} \fbox{\pali{mate} [I.C2.V]} [I.C2]} = [when] father; died\\[0.5mm]
\fbox{\pali{pit\=a viya} [I.C2.M]} = like father\\[0.5mm]
\fbox{\pali{laddho} [I.V2]}\footnote{This is in passive structure.} = [was] got (treated)\\[0.5mm]
\fbox{\pali{gehe} [I.L3]} \fbox{\pali{te} [I.M3]} \fbox{\pali{mah\=avibhavo} [I.S3]} [ \fbox{\pali{atthi} [I.V3]} ]\\[0.5mm]
= in the house; your; great wealth; [exists]\\[0.5mm]
\fbox{\pali{sakk\=a} [I.V4]} = be able\\[0.5mm]
\fbox{\pali{geha\d m ajjh\=avasante eva} [I.S4]} = even those living in the house\\[0.5mm]
\fbox{\pali{pu\~n\~n\=ani k\=atu\d m} [I.R4]} = to do meritorious deeds\\[0.5mm]
\fbox{\pali{m\=a} [I.P5]} [ \fbox{\pali{tva\d m} [I.S5]} ] \fbox{\pali{eva\d m} [I.R5]} \fbox{\pali{karittha} [I.V5]}\\[0.5mm]
= do not; [you]; as such; do\\[0.5mm]
\fbox{\pali{iti} [P]} = thus\\[1mm]
\normalsize
$\triangleright$ [The younger brother said] thus, ``What did you say, brother? When mother dies, you was treated by me as mother. When father died, you was treated as father. There is great wealth in your house. Even those living in the house can do meritorious deeds. Do not do that.''\\[1.5mm]

\stepcounter{sennum}\medskip
\arabic{sennum}. \pali{``T\=ata, mahallakassa hi attano hatthap\=ad\=api anassav\=a honti, na attano vase vattanti, kima\.nga\d m pana \~n\=atak\=a, sv\=aha\d m tava katha\d m na karomi, sama\d napa\d tipatti\d myeva p\=uress\=ami''.}\\[1.5mm]
\pali{Jar\=ajajjarit\=a honti, hatthap\=ad\=a anassav\=a;}\\
\pali{Yassa so vihatatth\=amo, katha\d m dhamma\d m carissati.}\\[1.5mm]
\pali{``Pabbajiss\=amev\=aha\d m, t\=at\=a''ti.}\footnote{In the text we have, the \pali{iti} clause does not end here, but it should. The stanza above just interrupts the conversation to stress the point. This looks like a dramatic play to me.}\\[1mm]
\small
[ \fbox{\pali{Mah\=ap\=ala} [S]} \fbox{\pali{vadi} [V]} ] = Mah\=ap\=ala; said\\[0.5mm]
\fbox{\pali{T\=ata} [I.A]} \fbox{\pali{mahallakassa} [I.M1]} \fbox{\pali{hi} [I.P1]}\\[0.5mm]
= Dear brother; of an old person; but\\[0.5mm]
\fbox{\pali{attano} [I.M1]} \fbox{\pali{hatthap\=ad\=a pi} [I.S1]}\\[0.5mm]
= of one's own; even hands and feet\\[0.5mm]
\fbox{\pali{anassav\=a} [I.M1]} \fbox{\pali{honti} [I.V1]} = disobedient; are\\[0.5mm]
[ \fbox{\pali{hatthap\=ad\=a} [I.S2]} ] = hands and feet\\[0.5mm]
\fbox{\pali{na} [I.P2]} \fbox{\pali{attano} [I.R2]} \fbox{\pali{vase} [I.R2]}\\[0.5mm]
= not; of one's own; in control\\[0.5mm]
\fbox{\pali{vattanti} [I.V2]} = exist\\[0.5mm]
\fbox{\pali{kima\.nga\d m pana} [I.P3]} \fbox{\pali{\~n\=atak\=a} [I.R3]} [ \fbox{\pali{vattanti} [I.V3]} ]\\[0.5mm]
= how [do that happen to]; relatives\\[0.5mm]
\fbox{\pali{so} [I.M4]} \fbox{\pali{aha\d m} [I.S4]} \fbox{\pali{tava katha\d m} [I.R4]}\\[0.5mm]
= [that]; I; your word\\[0.5mm]
\fbox{\pali{na karomi} [I.V4]} = not do\\[0.5mm]
\fbox{\pali{sama\d napa\d tipatti\d m eva} [I.R5]} = only the path of renouncers\\[0.5mm]
\fbox{\pali{p\=uress\=ami} [I.V5]} = [I] will fulfill\\[0.5mm]
\normalsize
$\triangleright$ [Mah\=ap\=ala said], ``Dear brother, but, even an old man's own hands and feet are disobedient. They are not in one's own control, let alone relatives. I will not do [by] your word. I will fulfill only the path of renouncers.''\\[1.5mm]
\small
\fbox{\pali{Jar\=ajajjarit\=a} [M1]} \fbox{\pali{honti} [V1]} = weakened by old age; are\\[0.5mm]
\fbox{\pali{hatthap\=ad\=a} [S1]} \fbox{\pali{anassav\=a} [M1]} = hands and feet; disobedient\\[0.5mm]
\fbox{\pali{Yassa} [P1]}\footnote{This is a marker of \pali{ya-ta} structure, or correlative sentences (see Chapter \externalref{16} of PNL). It pairs with \pali{so} below.} = of which [person]\\[0.5mm]
\fbox{\pali{so} [S2]} \fbox{\pali{vihatatth\=amo} [M2]} [ \fbox{\pali{hoti} [V2]} ]\\[0.5mm]
= that [person]; having impaired strength; is\\[0.5mm]
\fbox{\pali{katha\d m} [P3]} \fbox{\pali{dhamma\d m} [R3]} \fbox{\pali{carissati} [V3]}\\[0.5mm]
= how; the Dhamma; [that person] practices\\[0.5mm]
\normalsize
$\triangleright$ Hands and feet of which person are disobedient, weakened by old age. That person has impaired strength. How does he practice the Dhamma?\\[1.5mm]
\small
\fbox{\pali{Pabbajiss\=ami eva} [I.V]} \fbox{\pali{aha\d m} [I.S]} = will only go forth; I\\[0.5mm]
\fbox{\pali{t\=ata} [I.A]} \fbox{\pali{iti} [P]} = Dear Brother, thus\\[1mm]
\normalsize
$\triangleright$ [Mah\=ap\=ala said] thus, ``I will go forth anyway.''\\[1.5mm]
