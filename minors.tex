\chapter{Other minor concerns}\label{chap:minors}

There are some remarks made by Thai P\=ali teachers which give us useful information, but they are not big enough to have topics of their own. So, I collect them here. These include some common idioms which are hard to find an explanation in basic textbooks.

\phantomsection
\addcontentsline{toc}{section}{Noun as modifier}
\section*{Noun as modifier}

As I have said elsewhere, nouns in P\=ali do not modify each other like in English, say, `world leader.' When we have to do as such, we use a suitable case for the modifier, hence `\pali{lokassa n\=ayako}' (world's leader); or we make them a compound, i.e.\ `\pali{lokan\=ayako}.' However, nouns in the same case can be put side by side as apposition, e.g.\ `\pali{lokassa n\=ayako p\=alo}' (the world's leader, the protector). Sometimes this kind of modifier does not sit close to the main noun, and it can come as a noun phrase. This is what we are going to talk about. By technical terminology, it is called `\pali{visesanal\=abh\=i}.' One thing to remember is this modifier noun has to be in the same case as the main term, even if it has different gender and number.\footnote{Strictly speaking, for some Thai teachers, \pali{visesanal\=abh\=i} means the modifier that agrees in number. If not, it is called \pali{sar\=upa} instead. To make our life simple, we group them together here.} Let us see some examples:

- \pali{ya\d m yadeva avasi\d t\d tha\d m hoti \textbf{a\.nkuro v\=a patta\d m v\=a taco v\=a}, ta\d m ta\d m kh\=aditv\=a}\footnote{Dhp-a\,2.32. In Thai version, there is no repetition. So, only \pali{yadeva} (ya\d m + eva) and one \pali{ta\d m} are presented. Note the genders of the modifier nouns. All these take nominative case. The exact main noun is omitted. We know only that it has neuter gender. Some teachers suggest \pali{rukkhaj\=ata\d m} as the hidden subject.} (Which [part of a tree] is left over, shoot or leaf or bark, that [part] having been eaten, \ldots)\par
- \pali{ete, bhikkhave, dve sah\=ayak\=a \=agacchanti \textbf{kolito ca upatisso ca}}\footnote{Dhp-a\,1.11} (Monks, these two friends, Kolita and Upatissa, come.)\par
- \pali{\textbf{Aha\d m} so, bhante}\footnote{Dhp-a\,2.21} (Sir, that [person] [is] I.)\par
- \pali{Sv\textbf{\=aha\d m} kusala\d m karitv\=a kamma\d m, tidas\=ana\d m sahabyata\d m gato}\footnote{Pv\,197. Here, \pali{sv\=aha\d m} = \pali{so + aha\d m}.} (That [person] [is] I who, having done good action, [then] became a friend of gods in T\=avasi\d msa.)\par
- \pali{Yassa ete dhan\=a atthi, \textbf{itthiy\=a purisassa v\=a}; Adaliddo'ti ta\d m \=ahu, amogha\d m tassa j\=ivita\d m.}\footnote{A7\,5} (Which [person], woman or man, has these [seven] fortunes. [They] said that [person] [is] `not poor,' his life [is] not worthless.)\par

\phantomsection
\addcontentsline{toc}{section}{Past participle in compounds}
\section*{Past participle in compounds}

This matter is quite trivial to talk about, but it may help new learners to realize the possibility of its use. When we find a compound with a \pali{-ta} word, we can read it like we use past participle in English, for example:

\begin{quote}
\pali{Atha ratti\d m cor\=a ekasmi\d m gehe sandhi\d m chinditv\=a bha\d n\d daka\d m gahetv\=a lohabh\=ajanasaddena pabuddhehi s\=amikehi anubaddh\=a \textbf{gahita}bha\d n\d da\d m cha\d d\d detv\=a pal\=ayi\d msu.}\footnote{Dhp-a\,5.161}\\
In that time, thieves, having broken into a house in the night, having taken articles, [then] having been followed by the owners who have been awakened by the clank of metal articles, having abandoned the \textbf{taken articles}, ran away.
\end{quote}

As we have seen, \pali{gahitabha\d n\d da\d m} has passive meaning. So, traditional teachers suggest that we should add its instrumental agent to make the sentence clear. In this case, \pali{attan\=a attan\=a}\footnote{For repetition, see Chapter \externalref{18} of PNL. By its meaning, \pali{atta} is always singular. When we use it in plural sense, we double the term.} (by each of them) is added, then we get \pali{[attan\=a attan\=a] gahita\-bha\d n\d da\d m} (the articles taken [by each of them]). Sometimes the instrumental agent is specified in the text, for example:

\begin{quote}
\pali{may\=a \textbf{pesita}k\=ale gamissasi}\footnote{Dhp-a\,2.21}\\
{[You]} go in the time that was \textbf{sent} by me.
\end{quote}

This reminds us that in compounds with a \pali{-ta} word, sometimes another term outside the compounds has to be related. To put in another way, when you see a term with instrumental case, it may relate to just half a term of a compound. So, be careful of this. Here is another example:

\begin{quote}
\pali{bhante, may\=a \textbf{laddha}sampatti\d m m\=a n\=asetha}\footnote{Dhp-a\,9.118}\\
Sir, do not ruin the fortune \textbf{received} by me.
\end{quote}

Intransitive verbs can be formed as a \pali{-ta} word composed in compounds, like this one.

\begin{quote}
\pali{\textbf{nibbatta}satt\=a panettha kammabalena m\=atukucchigat\=a viya na vil\=iyanti}\footnote{Dhp-a\,1.15}\\
\textbf{Beings born} in that [hell] do not perish by the power of karma, like [those beings] having go into the mother's womb.
\end{quote}

Sometimes a \pali{-ta} word composed in compounds modifies the term that does not relate to it directly, for example:

\begin{quote}
\pali{tiracch\=anayoniya\d m \textbf{nibbatta}k\=alepi mayha\d m putto mama santika\d m \=agacchanto sobhiyeva}\footnote{Dhp-a\,1.17}\\
My child, coming to my presence, even in the \textbf{time [it was] born} in the animal realm, looked beautiful indeed.
\end{quote}

\phantomsection
\addcontentsline{toc}{section}{\pali{Iti}, the content marker}
\section*{\pali{Iti}, the content marker}

I have explained the uses of \pali{iti}, mainly as direct speech marker, in Chapter \externalref{35} of PNL. I will not reproduce anything here, except some uses that were not mentioned before or that are worth emphasizing. 

\paragraph*{Simple explanation} This use of \pali{iti} can be found here and there because it is handy to use. Technically, this is called \pali{sar\=upa}. Let us see some examples to make the idea clearer. And to make the readers familiar with the real texts, I will not mark out \pali{iti} in this group of examples. Be careful, do not mistake it for a verb.\par
- \pali{Ganthadhura\d m vipassan\=adhuran\textbf{ti} dveyeva dhur\=ani bhikkhu}\footnote{Dhp-a\,1.1} (Monks, the only two burdens [are] scriptural study and insight meditation.)\par
- \pali{Tassa eko dve tayo\textbf{ti} eva\d m anupabbajja\d m pabbajit\=a catusattatisahassamatt\=a ja\d til\=a ahesu\d m.}\footnote{Dhp-a\,1.11} (There were matted-haired ascetics around 74,000, having gone forth following his ordination in this manner ``[from] one [person], two, three.'')\par
- \pali{So atthato tayo ar\=upino khandh\=a vedan\=akkhandho sa\~n\~n\=akkhan\-dho sa\.nkh\=arakkhandho\textbf{ti}.}\footnote{Dhp-a\,1.1} (By the meaning, that [condition] is three non-material aggregates, thus feeling, perception, and mental formation.)\par
- \pali{So pana mah\=anubh\=avo at\=ite catt\=al\=isa, an\=agate catt\=al\=is\=a\textbf{ti} as\=itik\-appe anussarati.}\footnote{Dhp-a\,1.3} (He, [having] great power, remembers 80 eons, 40 in the past [and] 40 in the future.)\par
- \pali{Setabyanagarav\=asino hi c\=u\d lak\=a\d lo, majjhimak\=a\d lo, mah\=ak\=a\d lo\textbf{ti} tayo bh\=ataro ku\d tumbik\=a.}\footnote{Dhp-a\,1.7} (As the story goes, three householders, residents of Setabya City, [are] brothers, namely C\=u\d lak\=a\d la, Majjhimak\=a\d la, [and] Mah\=ak\=a\d la)\par

\medskip
Sometimes, even \pali{iti} is not necessary if the context makes the sense clear\footnote{See also \emph{Noun as modifier} above.}, for example:\par
- \pali{Kosambiya\~nhi ghosit\=ar\=ame pa\~ncasatapa\~ncasatapariv\=ar\=a dve bhikkh\=u vihari\d msu vinayadharo ca dhammakathiko ca.}\footnote{Dhp-a\,1.6} (As the story goes, two monks, an expert in the Vinaya and a Dhamma preacher, having around 500 followers each, lived in Ghosita mona\-stery, Kosamb\=i.)\par

\paragraph*{\pali{Iti} as `to make known that'} In this use, \pali{iti} does not signal an actual conversation but a message, which may be not really a verbal expression. It may be a hint to something unsaid directly. It is better to see some examples:\par
- \pali{`Gandhase\d t\d thino bhu\~njanal\=i\d la\d m olokent\=u'\textbf{ti} nagare bheri\d m car\=a\-pesi.}\footnote{Dhp-a\,10.145} ([They] have the drum [parade] travel in the city [to make known that], ``See, [people], the beauty of Gandhase\d t\d th\=i's consumption.'')\par
- \pali{mayha\~nca `ida\~ncida\~nca may\=a katan'\textbf{ti} s\=asana\d m pesetu}\footnote{Dhp-a\,2.21} (Send the message for me [to make known that], ``[the work], this and this, was done by me.'')\par
- \pali{Thero `n\=aya\d m mama vacana\d m su\d n\=at\=i'ti cintetv\=a `tuva\d m pam\=a\-\d na\d m na j\=an\=as\=i'\textbf{ti} acchara\d m pahari.}\footnote{Dhp-a\,9.118} (The elderly monk, having thought that ``This [girl] does not listen to my words,'' made a snap [to make known that], ``You do not know [your] measure.'')\par

\paragraph*{\pali{Iti} as `called'} This use might be already mentioned somewhere else. Because this is seen frequently in the commentaries, it should be stressed here. When a term in the canon is taken to make a remark or explanation, it is marked with \pali{iti}, also with boldface in our text collection. In a simple structure, say, ``\pali{xxxti} such and such,'' it can be read as ``Such and such thing is \emph{called} 'xxx'\,'' or ``\,`xxx' \emph{means} such and such.'' But it can mean other things in more complex structures (see further in Chapter \ref{chap:commentary}). New students should be careful of this, because there is a good chance that the word is mistook as a verb. Let us see an example:\par
- \pali{Tattha \textbf{mano}ti k\=am\=avacarakusal\=adibheda\d m sabbampi catubh\=umikacitta\d m}\footnote{Dhp-a\,1.1} (In that [verse], even all kinds of consciousness in the four realms of existence [are] called `\pali{mano}.')\par

\phantomsection
\addcontentsline{toc}{section}{\pali{Pageva/kuto pana} (let alone)}
\section*{\pali{Pageva/kuto pana} (let alone)}

This couple of words means roughly `let alone' in English. You cannot know their idiomatic use by dictionary definitions. The first one, \pali{pageva} (\pali{p\=a + eva}) literally means `only prior to' which makes no sense in the context. Seeing an example is better for understanding.

\begin{quote}
\pali{S\=ama\d nera\~nhi tasmi\d m k\=ale sinerun\=a avattharantopi m\=aretu\d m samattho n\=ama natthi, \textbf{pageva} asin\=a.}\footnote{Dhp-a\,8.110}\\
In that time, there was no such ability to kill the novice, even [by] crushing [him] with Mt.\ Sineru, let alone by a sword.
\end{quote}

The second term, \pali{kuto pana}, literally means `but from where?' When we use it as `let alone,' it means like ``How possibly will this happen in that way?'' Sometimes, only \pali{kuto} works fine in this meaning. Let us see an example.

\begin{quote}
\pali{Na puttamiccheyya \textbf{kuto} sah\=aya\d m,}\\
\pali{eko care khaggavis\=a\d nakappo.}\footnote{Snp\,35}\\
One does not [even] desire a child, let alone a friend.\\
{[One]} should travel alone like the [single] horn of a rhinoceros.\\
\end{quote}

In the commentary to Suttanip\=ada, the explanation of the stanza above, in conversational form, goes like this:

\begin{quote}
\pali{aha\d m id\=ani atraj\=ad\=isu ya\d m ki\~nci puttampi na iccheyya\d m, \textbf{kuto pana} tumh\=adisa\d m sah\=aya\d m?}\footnote{Snp-a\,35}\\
``I now do not desire even any child, of my own, etc., how possibly [= let alone] a friend like yourself?''\\
\end{quote}

For some more information and examples from the canon, see Chapter \externalref{35} of PNL.

\phantomsection
\addcontentsline{toc}{section}{\pali{Kima\.nga\d m pana} (why [not])}
\section*{\pali{Kima\.nga\d m pana} (why [not])}

This is another idiom that causes a headache to new learners. By its letters, it means ``What factor?'' An elaborate version is something like ``What factor could be the cause of that?'' That sounds confusing in English. I find that it is close when we say ``Why \ldots?'' or ``Why not \ldots?,'' which is in a way similar to \pali{pageva} and \pali{kuto pana} above. Let us see examples for more understanding.

- \pali{evar\=upassa n\=ama ka\d t\d thakali\.ngarass\=api jar\=a \=agacchati, \textbf{kima\.nga\d m pana} attabh\=avassa}\footnote{Dhp-a\,11.151} (Decay comes to even such a piece of wood, why not to [our] individuality?)\par
- \pali{mahallakassa hi attano hatthap\=ad\=api anassav\=a honti, na attano vase vattanti, \textbf{kima\.nga\d m pana} \~n\=atak\=a}\footnote{Dhp-a\,1.1} (But even an old man's own hands and feet are disobedient. They are not in one's own control, why relatives [should be]?)\footnote{In Chapter \ref{chap:sentence}, I translate this instance using `let alone' for \pali{kima\.nga\d m pana}.}\par
- \pali{Ime hi n\=ama sakyakum\=ar\=a evar\=upa\d m sampatti\d m pah\=aya im\=ani anaggh\=ani \=abhara\d n\=ani khe\d lapi\d n\d da\d m viya cha\d d\d detv\=a pabbajissanti, \textbf{kima\.nga\d m pan\=a}ha\d m}\footnote{Dhp-a\,1.17} \ \ ([Even] these [young] men of the S\=akya, having discard such a fortune [and] these priceless ornaments like [spitting] saliva, will go forth. Why shouldn't I?)\par

\phantomsection
\addcontentsline{toc}{section}{\pali{Ki\d m} (what)}
\section*{\pali{Ki\d m} (what)}

This is a common word for questioning. You can review the use of \pali{ki\d m} in Chapter \externalref{15} and \externalref{27} of PNL. It can appear in some idiomatic units, like \pali{kima\.nga\d m pana} above and \pali{ki\d m k\=ara\d n\=a} (why?). Here, I will remind you only one use when \pali{ki\d m} starts a sentence. If it is not mark a simple sentence, like ``\pali{ki\d m gacchasi?}'' (Will you go?), it can mean ``What is the use?'' Thai P\=ali teachers suggest \pali{payojana\d m} insertion, hence \pali{ki\d m [pajojana\d m]}. Let us see some examples:

- \pali{\textbf{ki\d m} [payojana\d m] me ghar\=av\=asena pabbajiss\=ami}\footnote{Dhp-a\,1.1} (What is the use for me by living in the house? [I] will go forth.)\par
- \pali{samma, \textbf{ki\d m} [payojana\d m] te sama\d nena gotamena, ta\d m upasa\.nkamitv\=a ki\d m labhissasi}\footnote{Dhp-a\,4.58} (Dear friend, what is the use for you with ascetic Gotama? Approaching him, what will you get?)\footnote{The second \pali{ki\d m} is a normal `what' question word.}\par
- \pali{ettak\=ana\d m bhikkh\=una\d m \textbf{ki\d m} [payojana\d m] eva\d mbahukehi y\=agu\=a\-d\=ihi}\footnote{Dhp-a\,7.93} (What is the use for this amount of monks by such plenty of rice-gruel etc.)\par
- \pali{\textbf{ki\d m} [payojana\d m] me janassa saddh\=adeyya\d m niv\=asetv\=a vicara\d n\-ena, attano pilotikameva niv\=asess\=ami}\footnote{Dhp-a\,10.143} (What is the use for me by going about, having put on [the cloth] given by people's faith? [I] will just wear my own old rag.)\par
- \pali{`ki\d m nu kho core v\=aremi, ud\=ahu v\=a\d nijan'ti cintetv\=a, `\textbf{ki\d m} [payojana\d m] me corehi, \ldots'}\footnote{Dhp-a\,9.123} ([He], having thought that ``Will I prevent the thieves or the merchant?'', [then having thought] ``What is the use for me by [preventing] the thieves? \ldots'')\footnote{In this instance, the first \pali{ki\d m} is a simple question marker, often accompanied with \pali{nu kho} or only \pali{nu}. The second \pali{ki\d m} is a reflection that ``It is not my ability to prevent the thieves.''}\par

\phantomsection
\addcontentsline{toc}{section}{\pali{Attho} (need)}
\section*{\pali{Attho} (need)}

We normally find \pali{attha} (m., nt.) in three senses: (a) benefit, gain, or use; (b) need or want; (c) sense or meaning. The first is general meaning. In the canon, it is more or less a synonym of \pali{hita} (nt.). And we often find this stock phrase, ``\pali{atth\=aya hit\=aya sukh\=aya devamanuss\=ana\d m}''\footnote{Mv\,1.32} (for benefit, welfare, happiness of gods and human beings).

The last sense is often used in commentaries (see Chapter \ref{chap:commentary}). The second sense is what I want to stress here. It may sound close to the first one, but the context suggests that it is better to mean `need/want,' because the object of need is specified by a word of instrumental case. Here are some examples:

- \pali{Tena kho pana samayena gil\=an\=ana\d m bhikkh\=una\d m m\=ulehi bhesajjehi \textbf{attho} hoti.}\footnote{Mv\,6.263.} (In that time, there is a need of plant roots [as] medicine for sick monks.)\par
- \pali{candimas\=uriyehi me \textbf{attho}, te me dehi}\footnote{Dhp-a\,1.2} (My need of moon and sun [exists], give those to me.)\par
- \pali{Kittakehi te bhikkh\=uhi \textbf{attho} up\=asaka}\footnote{Dhp-a\,1.9} (How many monks is the need of you, lay devotee?)\par
- \pali{deva, amh\=aka\d m a\~n\~nena kenaci \textbf{attho} natthi}\footnote{Dhp-a\,1.11} \ \ (Your Majesty, my need of any other thing does not exist.)\par
- \pali{dhanena me \textbf{attho}, dhana\d m me dehi}\footnote{Dhp-a\,1.13} (My need of wealth [exists], give me wealth.)\par

\medskip
Even with an instrumental object, in some context it is better to means `benefit' like this one:\par
- \pali{atthi nu kho mayha\d m ettha gatapaccayena \textbf{attho}}\footnote{Dhp-a\,1.2} (Is there a benefit for me by the cause of going here?)\footnote{However, you can insist that it means `a need for me' or `my need.'}\par

\phantomsection
\addcontentsline{toc}{section}{\pali{Bh\=ava} (state)}
\section*{\pali{Bh\=ava} (state)}

Often found in compounds, \pali{bh\=ava} (m.) denotes a state of being, for example, \pali{sama\d nabh\=avo} (a state of being an ascetic), \pali{\=agatabh\=avo} (a state of being one who has come), \pali{gamanabh\=avo} (a state of going), \pali{atthibh\=avo} (a state of existing). Some derivative nouns also have the same meaning, particularly those ending with \pali{-tta} (nt.) and \pali{-t\=a} (f.). Here are some examples of the use of nouns to express certain states.

- \pali{mayha\d m kani\d t\d tho pana tumhe disv\=a pucchissati, athassa mama cakkh\=una\d m \textbf{parih\=inabh\=ava\d m} \=aroceyy\=atha}\footnote{Dhp-a\,1.1} \ \ (My younger brother, having seen you, will ask. Then, may you tell him [about] my eyes' impaired condition.)\par
- \pali{Tath\=agatassa tattha hatthin\=agena upa\d t\d thiyam\=anassa \textbf{vasanabh\=avo} sakalajambud\=ipe p\=aka\d to ahosi.}\footnote{Dhp-a\,1.6} (There was the state of living, well-known in whole India, of the Buddha being attended by the noble elephant there.)\par
- \pali{Brahmadattena d\=igh\=itissa kosalara\~n\~no rajja\d m acchinditv\=a a\~n\~n\=atakavesena vasantassa [pituno] \textbf{m\=aritabh\=ava\~n}ceva}\footnote{Dhp-a\,1.6. There is no \pali{pituno} in Thai edition. The term is out of place here.} \\(The condition of being killed of D\=igh\=iti, a king of Kosala who was overthrown by king Brahmadatta, living by unrecognizable dressing, \ldots)\par
- \pali{dvinna\d m kul\=ana\d m \textbf{gu\d namahattata\d m} pa\d ticca s\=avatthi\d m niss\=aya pa\~ncav\=isativass\=ani vass\=av\=asa\d m vasi}\footnote{Dhp-a\,1.1. Here, \pali{gu\d namahattata\d m} is acc.\ of \pali{gu\d namahattat\=a}. In Thai edition, it is \pali{gu\d namahantata\d m} instead.} ([The Buddha] has lived in S\=avatth\=i for 25 years because of the [state of] great virtue of two families.)\par
- \pali{Ayamassa bhamarena saddhi\d m \textbf{madhukara\d nasarikkhat\=a} veditabb\=a.}\footnote{Dhp-a\,4.49} (The state of making honey by a bee should be known [like] this [living] of that [arhant].)\par
- \pali{Tasm\=a pamatt\=a j\=ati\=ad\=ihi \textbf{aparimuttatt\=a} j\=ivant\=api mat\=ayeva n\=ama.}\footnote{Dhp-a\,2.21. Here, \pali{aparimuttatt\=a} is abl.\ of \pali{aparimuttatta}.} (Therefore the careless [beings], even living, are regarded as dead, because of the state of not being free from birth, etc.)\par

\phantomsection
\addcontentsline{toc}{section}{\pali{Eva\d m sante} (if it is so)}
\section*{\pali{Eva\d m sante} (if it is so)}

In grammatical terms, this is an absolute construction with locative case, but only an \pali{anta} verb is present, i.e.\ \pali{sante} or \pali{sati}.\footnote{For the declensional paradigm of this word, see irregular noun \pali{santa} in Appendix \externalref{B} of PNL.} This clause can be translated literally as ``When it is being/existing as such, \ldots'' In an idiomatic use, often found in conversations, we can put it simply as ``When/If it is so, \ldots''\footnote{See also \pali{tena hi} in \pali{Nip\=ata} section of Chapter \ref{chap:pos}.}

- \pali{\textbf{eva\d m sante} mayha\d m aph\=asuka\d m bhavissati}\footnote{Dhp-a\,1.1} (If it is so, it will be inconvenient for me.)\par
- \pali{\textbf{eva\d m sati} imassa santike brahmacariyav\=aso niratthako}\footnote{Dhp-a\,1.11} (If it is so, living a religious life in this place [is] useless.)\par
- \pali{\textbf{Eva\d m sante} mah\=adukkha\d m n\=ama anubhosi}\footnote{Dhp-a\,1.14} (If it is so, [you] underwent such great suffering[?])\par
- \pali{\textbf{eva\d m sante} ahampi te kattabba\d m j\=aniss\=ami}\footnote{Dhp-a\,1.14} (When it it so, I will know what should be done for you.)\par
- \pali{\textbf{Eva\d m sante} saha gabbhena j\=ivitakkhaya\d m p\=apu\d niss\=ami}\footnote{Dhp-a\,2.21} (If it is so, I will reach the dissolution of life with the fetus.)\par

\medskip
Sometimes, there is no \pali{sante}, but it is \pali{yadi eva\d m} instead. This straightly means `if it is so.' It can be used as a substitute of \pali{tena hi}. And sometimes even \pali{eva\d m} alone can be used in this sense.\par
- \pali{\textbf{yadi eva\d m}, s\=adhu, s\=igha\d m \=anetha}\footnote{Dhp-a\,4.47} (If it is so, that's good, bring [her here] quickly.)\par
- \pali{[Yadi] \textbf{eva\d m} kasm\=a purimadivase satthu santike nisinno ma\d m disv\=a na u\d t\d thahi}\footnote{Dhp-a\,4.51. There is no \pali{yadi} in Thai edition.} (If it is so, why in the previous day, having sat in the Buddha's place, having seen me, didn't you rise up [to respect me]?)\par
- \pali{s\=a ta\d m aggi\d m nibb\=apetu\d m kasm\=a n\=asakkhi, \textbf{yadi eva\d m} mah\=anu\-bh\=av\=a}\footnote{Dhp-a\,5.60} (Why were those [gods] unable to make that fire quenched, if they are so powerful?)\par

\medskip
In some case, \pali{atha} can be used in the same meaning\footnote{See also \pali{atha ca pana} in Chapter \ref{chap:pos}.}, like this instance.\par
- \pali{Nanu c\=ayasm\=a mah\=apa\~n\~no, \textbf{atha} kasm\=a mah\=amoggall\=anato ciratarena s\=avakap\=arami\~n\=a\d na\d m p\=apu\d ni?}\footnote{Dhp-a\,1.11} (Isn't the Venerable [S\=ar\=iputta] [has] great wisdom? If it is so [= Then], why did [he] attain the perfection of disciple knowledge slower than Ven.\,Mah\=a\-moggall\=ana.)\par
