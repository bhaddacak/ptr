\chapter{(Critical) discourse analysis}\label{chap:discourse}

In Chapter \ref{chap:howunderman}, we discussed discourse and ideology in the line that our use of language contributes to the construction of our social reality. Now we will put this notion into practice. By principle, discourse analysis is not directly related to translation studies. And the application of the concept in the field is quite old.\footnote{For a recent treatment, see \citealp{mundayzhang:discourse}.}

In Chapter \ref{chap:analysis} we have learned how to analyze text. By that process, we gain understanding of the text and its environment. As a result, we can make a better judgement what it should mean in a particular context. That seems enough as a typical reading/translation goes. But somethings are still left unaddressed, resulting in a tendency of covert, unhealthy textual manipulation.

Roughly speaking, discourse analysis addresses the relation between the use of language and social phenomena. In other words, it treats language as a tool for construction of social reality. A good summary of what I mean by discourse analysis can be shown as follows:\footnote{\citealp[p.~2]{paltridge:discourse}, my itemization}

\begin{compactenum}[(1)]
\item Discourse analysis examines patterns of language across texts and considers the relationship between language and the social and cultural contexts in which it is used. 
\item Discourse analysis also considers the ways that the use of language presents different views of the world and different understandings.
\item It examines how the use of language is influenced by relationships between participants as well as the effects the use of language has upon social identities and relations.
\item It also considers how views of the world, and identities, are constructed through the use of discourse.
\end{compactenum}

The first item above presents the general idea as I have mentioned. The second one asserts that when language is used differently, it constitutes various world-views and understandings. A simple example is we know God exists only because it is stated discursively in religious canons. Even if we have a supramundane experience, we have to check it against the canons whether it is a `proper' God or not. This reminds us of linguistic determinism discussed in Chapter \ref{chap:equivalence}.

The third item focuses on effects of the discourse made upon the individual. By discursive power, certain identities and relations between individuals are established. For example, as said in the P\=ali canon, by different characteristics, some people are ordinary, and some people are saintly. Hence, ordinary people are supposed to support saintly people, and the latter are supposed to teach the former or to be an exemplar. These identities and relationships are constructed discursively.

The last item is somewhat redundant. It suggests that what we are and how we see the world are conditioned by the discourse. Thus, the task of discourse analysis is to uncover those constructions.

A shorter guideline is presented by Norman Fairclough, as he said, when language is used, it always constitutes three things:

\begin{quote}
(a) Social identities\\
(b) Social relations\\
(c) System of knowledge and belief\footnote{\citealp[p.~134]{fairclough:marketization}} 
\end{quote}

To make it easily applicable, we will stick to this three notions: \emph{knowledge}, \emph{identities}, and \emph{relations}. So, when a text is analyzed, we should ponder that what kind of knowledge or belief, what kind of identities, and what kind of relations are established by the text.

Now we will consider what discourse analysis has things to do with `critical.' By the fact that `critical' means different things to different scholars, to make it simple I put it in this way. Critical discourse analysis puts more focus on normative judgement. It explores the issues of politics, economics, racism, genders, religious discrimination, inequalities, and other ideologies.

Critical discourse analysis may include ``tracing underlying ideologies from the linguistic features of a text, unpacking particular biases and ideological presuppositions underlying the text, and relating the text to other texts and to people's experiences and beliefs.''\footnote{\citealp[p.~186]{paltridge:discourse}}

It is true by the term that critical discourse analysis is more `critical' than usual. It should be stressed again that `critical' does not mean fault finding but rather to put more effort on investigating the reasons behind the meaning of the text.

By the three keywords given above, when discourse analysis is done critically, the system of knowledge constituted by the discourse looks more ideological, so are social identities and relations. That is to say, we have to keep suspicious eyes, rather than naive eyes, when we read a text. Or if you want to be less suggestive, you may draw two pictures for comparison. The first comes from naive reading, and the second from critical reading.

How exactly do we do discourse analysis? This question is difficult to answer, because there is no concrete method of discourse analysis. Basically, discourse analysis provides a conceptual framework, and it often combines with other methodology, notably conversation analysis. As long as P\=ali texts are concerned, conversation analysis has little use. However, we can use other methods to get a similar result, for example, non-verbal elements analysis, peculiar terms analysis, sentence structure analysis, and so on. Only thing to keep in mind when we do discourse analysis is the intention to uncover what kind of social condition is made effective by the text.

In the light of speech act theory, it can be said roughly that discourse analysis is ``the study how to do things with words.''\footnote{\citealp[p.~134]{hjelm:discourse}} That means you can use speech act analysis\footnote{see Chapter \ref{chap:howmisunder}} (micro level) and text function analysis\footnote{see Chapter \ref{chap:analysis}} (macro level) as a guide for discourse analysis.

\clearpage
\phantomsection
\addcontentsline{toc}{section}{Techniques for critical discourse analysis}
\section*{Techniques for critical discourse analysis}

There are some strategies useful for analyzing written text suggested by Thomas N.\ Huckin.\footnote{\citealp{huckin:cda}} I find these applicable and not too much theoretical. 

Huckin suggests two phases of reading. The first is from a typical reader's perspective. It goes normally uncritical. I call this `naive reading.' The second phase is done critically by adopting various perspectives: asking questions about it, imagining how it can be constructed in a different way, etc. I call this `critical reading.' The latter stage can be done in multiple levels: text as a whole, sentences, and words. These following tactics may repeat some analytical strategies mentioned elsewhere. To do critical discourse analysis effectively, these factors have to be taken into consideration.

\paragraph*{1.\ Text type} Normally, this means the recognition of genre in literature. To our concern, this can be done by text function analysis mentioned in Chapter \ref{chap:analysis}. This analysis is done at text level. Roughly speaking, we have to know whether the text is giving information (informative function), or expressing certain feelings (expressive function), or arousing certain actions (operative function).

\paragraph*{2.\ Framing} This is also done at text level. Framing refers to how the content of the text is presented, what sort of perspective (angle or slant) the writer is taking. It is like when you take a photo and you frame the picture in a particular way. It presents what you want the viewer to see via your perspective. Framing can be done in layers, as exemplified by Huckin:

\begin{quote}
For example, a news report might be framed as a narrative, or story; and within that frame it might set up a Good Guys vs.\ Bad Guys frame with one group of participants being given favorable treatment over the other.\footnote{\citealp[p.~82]{huckin:cda}}
\end{quote}

Visual aids like diagrams, pictures, and other embellishments can enhance framing effect considerably. But for ancient texts, these non-textual elements are rarely used.

\paragraph*{3.\ Foregrounding/backgrounding} This tactic is closely related to framing analysis. In analogy of taking a photo, when you make a shot, you have to emphasize a certain part, marking it into foreground. At the same time, you de-emphasize other unwanted parts, fading them into background.

Likewise, when a text is composed, certain ideas or information may be presented more vividly than others. At sentence level, some sentences can be stressed by putting them at the beginning of a paragraph, or repeating them several times. Unstressed sentences are often put in the middle, with bland wording, and sometimes with vague meaning. In Huckin's terms, foregrounding at sentence level is called \emph{topicalization} instead.\footnote{Perhaps, \emph{de-topicalization} can be used for backgrounding at sentence level.}

Another powerful way of backgrounding is omission. When the author does not mention something, it does not come to the readers' mind, thus not under their examination. You cannot question what is not said.

Backgrounding at sentence level can also be seen in agent-patient relations. Agent is the doer of the verb in active sentences. Patient is the receiver of the action. We can see patient as the subject of a passive sentence which the agent can be left out. By this deletion or omission of agents, we can leave something unsaid and put more focus on something else. For example, we often see a headline like ``25 Villagers Massacred,'' which the victims are highlighted, not those guilty of the crime. 

\paragraph*{4.\ Presuppositions} We have met this before as an intratextual factor mentioned in Chapter \ref{chap:analysis}. By Huckin's explanation, presupposition is ``the use of language in a way that appears to take certain ideas for granted, as if there were no alternative.''\footnote{\citealp[p.~82]{huckin:cda}} A common example is found in advertisements which posit the product as the best of its kind, without any rival. Detecting presuppositions is important in discourse analysis, because it can uncover what is held as true but unmentioned directly. Presuppositions make discursive practice powerful and difficult to resist.

\paragraph*{5.\ Discursive differences} Texts can contain more than one `voice' by utilizing multiple styles or registers. These discursive differences can be used to manipulate the reader in various ways. An example illustrated by Huckin is quoted in full below.

\begin{quote}
For example, an advertisement for a medical product might be written partly in the voice of a typical user (``Some seasonal allergy medicines used to make me feel drowsy\ldots\ Then I woke up to HISMANAL'') and partly in the voice of the medical scientist (``The reported incidence of drowsiness with HISMANAL [7.1\%] in clinical studies involving more than 1600 patients did not differ significantly from that reported in patients receiving placebo [6.4\%].''). The first `voice' emphasizes the helplessness of the ordinary citizen; the second emphasizes the authority and expertise of the scientific community. [Quotes from TIME, 4/24/95, p.\ 57.]\footnote{\citealp[p.~83]{huckin:cda}}
\end{quote}

The ad quoted by Huckin has two voices. The first is from an ordinary person who feels that HISMANAL does not cause drowsiness. The second is from the scientist who asserts scientifically that HISMANAL may cause drowsiness but it is insignificant when compared with no drug used. The first voice targets common people who can mostly be moved by emotional appeal. The second voice targets higher educated people who need some reliable information before making a judgement. The ad works magically, despite the fact that using HISMANAL can cause drowsiness in some cases.

\paragraph*{6.\ Insinuations} Like presuppositions, insinuations are difficult to be challenged. They are comments that are slyly suggestive, and typically have double meanings. Therefore, the writer can claim to have only one meaning when the other meaning, the intended one, is under attack.

\paragraph*{7.\ Connotations} This analysis is done at word or phrase level. Connotation is ``[t]he set of associations implied by a word in addition to its literal meaning.''\footnote{The American Heritage Dictionary, \url{https://www.ahdictionary.com/word/search.html?q=connotation}} It is another way to say things indirectly. Labeling is a powerful technique to make a connotation. For example, in P\=ali \pali{titthiya} means an adherent to other religions. When this is applied to someone, it associates that person with a wrong view, often degenerated or ridiculous. Sometimes, connotations can be carried out by the use of figures of speech like metaphor.

\paragraph*{8.\ Register} This is related to lexis, an intratextual factor mentioned in Chapter \ref{chap:analysis}. Register refers to the level of formality or informality, the degree of technicality, or the subject domain of a text. For example, this present book is in semi-formal, semi-technical, applied linguistic and philological register. Using different registers can have a manipulative effect, as noted by Huckin:

\begin{quote}
Writers can deceive readers by affecting a phony register, one that induces a certain misplaced trust. Typical examples of this would include advertisements written either in a friendly `conversational' register or in an authoritative `expert' register.\footnote{\citealp[p.~84]{huckin:cda}}
\end{quote}

\paragraph*{9.\ Modality} This means the use of moods in sentences. In P\=ali, it can be imperative, optative, and conditional mood. Moods can express the degree of certitude and authority. For example, the use of imperative mood has strong, authoritative voice. Optative mood is less pressing. And conditional mood has low degree of certainty comparing to simple present or past tense. Modality can also be conveyed by certain particles, for example, \pali{ekanta\d m, niyata\d m, duva\d m} (definitely).
