\markboth{}{Preface}
\cleardoublepage
\phantomsection
\addcontentsline{toc}{chapter}{Preface}
\chapter*{Preface}

This book is a sequel of \emph{P\=ali for New Learners} (PNL)\footnote{\url{bhaddacak.github.io/pnl}}, which is intended to be a friendly P\=ali primer, yet a comprehensive one. Learning how to speak the language first in a familiar way is the main aim of that book. Foundations of P\=ali grammar, as taught by the tradition, are also presented in that book thoroughly. When finishing PNL, the learners should understand P\=ali enough to form a simple conversation and to use the basic grammar correctly. Speaking the language fluently is not the objective of our learning.\footnote{Although some great scholars may can, but I still cannot say things in P\=ali instantly without pondering.} Instead, approaching the language, or any language in this matter, by trying to say it first is the fastest way to learn it.

In this volume, the objective is different. I expect that the readers are no more new learners.\footnote{This is the main reason I change the title of the book from the second book in the series to the present one.} Foundations laid in PNL are the prerequisite to this one. Now we will learn to read P\=ali texts in a rigorous way, I mean, a critical way. It is not the way the tradition does it, so there are several things we have to learn before we get into the texts.

To understand our situation, let me explain how the tradition approaches P\=ali scriptures. In the elementary level, students, mostly in monastic settings, have to learn grammatical rules and commit them to memory. At the same time, or at least when students get acquainted with some basic rules, they learn to translate texts directly.\footnote{In Thai tradition, the first text to be learned is the commentary to Dhammapada.} The concept of translation is basically based on a belief that there is a high degree of equivalence between source language (P\=ali) and target language (English, Thai, or whatever).\footnote{We will discuss more on \emph{equivalence} in Chapter \ref{chap:equivalence}.} Thus, the translation process looks transparent. It is just a mechanical transformation from one language to the other. Meanings are mostly retained and conveyed.

As we shall learn, this naive view is now challenged in our time and in several fronts. So, in our course, before we learn to read or translate a text, we should know what translation is all about first. This involves a multi-disciplinary approach led by \emph{literary studies}.\footnote{Some may see differently on this point. In my concern, first and foremost we have to accept that religious texts, no matter how sacred they are, are a kind of literature, a product of authoring. So, I see literary studies as the main entry point.} No matter how P\=ali texts come or how authoritative they are, a brute fact is that the texts are all literary works. All texts have their author, and authoring process is selective by nature. That means when texts portray certain information something has to be included, and something has to be excluded; something is stressed, and something is played down. So, in a good reading of text, these should be kept in mind by the readers. That enables us to read beyond the text itself and to get better understanding.

Does that modern way of reading texts undermine the authority (possibly read, sacredness) of the scriptures, and undermine the religion in turn? This question may arise to those of the traditional mind who prefer the old way of reading under the umbrella of traditional guidelines. From my position, scriptures have no authority by its own. Religious authority comes from religious organization. And every organization has political dimension. Religious scripture instrumentally serves its organization, hence serves politics in turn. Reading texts against the established guidelines may challenge the institution's power, but it cannot do harm to any scripture, because by its nature text can be read in many ways depending on whose benefit. I think a great number of portions in religious texts have just an expressive or artistic intention, but they are used as a political or proselytizing instrument.

How about soteriological reading of texts, then? Is there such a thing? From my Buddhist position, it is simple: salvation or awakening does not require much reading or any firm belief in certain things. We just unlearn our previous unhealthy misreading\footnote{As we shall see later on, all readings are a form of misreading. Read more in Chapter \ref{chap:translation}.} and live our life mindfully. That is to say, to be liberated you need unlearning rather than learning. But the process of unlearning something seems far more difficult than learning new things. By this, it renders that there is no soteriological reading of texts. That is my position. Reading texts is for intellectual purpose only, if not purely poetic. However, like water can wash away dirt, learning texts critically can do away our misreading, and uproot unhealthy beliefs. So, soteriological purpose of reading texts can be achieved indirectly, at best.

With the above concerns, our lessons in this volume will not only deal with translation as a technical process, but also the meaning, purpose, and possibility of the translation itself that I think they are more important than just how to make sense out of certain texts. Hence, I divide the book into three parts.

In the first part, we will deal with theories about foreign text reading. This involves a vast area of knowledge, including semiotics, hermeneutics, literary theory, communication studies, translation studies, religious studies, applied linguistics, and so on. Even cognitive psychology has something to say about the topic. However, since this book is not primarily about such theories, only an introduction of these in a digestible amount is portrayed. And because I do not position myself as a philosophy teacher, when I discuss controversial issues, I will not suspend my judgement. I will clearly express my stance with reasons concerning the issues straightly. This does not mean I am totally impartial, because everyone is biased in some way sometimes, but rather I want to give the readers information as much as possible. By their consideration, the readers can decide to agree with me or not.

In the second part, the process of P\=ali translation will be described. This includes the overall process, sentence decomposition, part of speech analysis, and some grammatical highlights. This part, to a large extent, conforms to the traditional guidelines for reading P\=ali texts. So, it is the part that traditional students expect to see. However, I have my own formulation of the process aiming mainly to a critical reading of texts. Moreover, I introduce a graphic way of doing sentence decomposition (Chapter \ref{chap:sentence}) that can help the learners understand what they are doing clearly.

The third part is all about demonstration. Excerpts from P\=ali texts in various groups will be elaborately analyzed and translated. The readers have an opportunity here to see texts composed in different styles. However, I have to admit that the amount and variety of texts brought here are not vast enough to see the distinction between different authors. At least, we can discern the differences between the canon and post-canonical literature.

Like for PNL, my main target readers, apart from those who study P\=ali in academic settings, are those who adhere to the tradition but want to learn more than the tradition can teach. And those who want to explore P\=ali texts more critically will find my approach helpful.

For some technical terms mentioned but unexplained in this book, the readers can skip them safely because they have little to do with the main content. If they are really curious, consulting Wikipedia suffices their need in most cases, particularly with terms used in philosophy, linguistics, psychology, and other academic disciplines.

For very new learners who come across this book, do not forget to use our companion program, \textsc{P\=ali\,Platform}.\footnote{\url{https://bhaddacak.github.io/paliplatform}} It contains a number of essential corpora of P\=ali texts and helpful learning tools such as dictionaries and many others.

Finally, because I have done all my works alone from gathering ideas to typesetting the whole books. They are inevitably far from perfect. If the readers see flaws in the books and the program and wish they could be better, please take notes and send back to the author.\footnote{\texttt{bhaddacak} at \texttt{proton} dot \texttt{me}} Discussions and suggestions are all gratefully welcome.
