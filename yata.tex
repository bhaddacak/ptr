\chapter{\pali{Ya-ta} structure revisited}\label{chap:yata}

In Chapter \externalref{16} of PNL, I have stressed that \pali{ya-ta}, or correlative, structure is very important in P\=ali. It helps us create complex sentences easily and gracefully. As the structure is alien to English, new learners have to practice before they can think it automatically. This chapter is not a review in a strict sense, but rather supplementary illustrations and emphases. I will show some uses of \pali{ya-ta} structure we often encounter in text reading. Some are so idiomatic that an explanation is needed. For the basic explanation, please refer to PNL. 

\setcounter{parnum}{1}
\paragraph*{\arabic{parnum}. \pali{Ya-ta} pair can be unmatched in form.} The first thing I have to remind new students is that \pali{ya} and \pali{ta} can come in different forms depending on the meaning both parts relate to each other. So we have to identify them carefully, particularly when they are joined with other words. Let us see some examples:\par
- \pali{punadivase \textbf{yattha} therassa kani\d t\d tho vasati, \textbf{ta\d m} v\=ithi\d m pi\d n\d d\-\=aya pavisi\d msu}\footnote{Dhp-a\,1.1} (In the following day, where the younger brother of the senior monk lives, [they] went to that street for alms.)\par
- \pali{\textbf{yena} kilesava\d t\d tasa\.nkh\=atena m\=aradheyyeneva paripphandati, \textbf{ta\d m} pah\=atabba\d m}\footnote{Dhp-a\,3.34} (By which realm of M\=ara, the so-called circle of impurity, [one] trembles, that should be abandoned.)\par
- \pali{agandhakapuppha\d m \textbf{yo} na\d m dh\=areti, \textbf{tassa} sar\=ire gandha\d m na pharati}\footnote{Dhp-a\,4.51} (Which person wears that scentless flower, the scent does not spread over his body.)\par
- \pali{na cirasseva[,] \textbf{yassa}tth\=aya kulaputt\=a sammadeva ag\=arasm\=a anag\=ariya\d m pabbajanti, \textbf{tad}anuttara\d m brahmacariyapariyos\=ana\d m di\d t\d theva dhamme saya\d m abhi\~n\~n\=a sacchikatv\=a upasampajja vih\=asi}\footnote{Dhp-a\,1.13} (Not long, [Nanda], having seen that excellent [condition], of which gain the sons of the family going forth properly from the house to the homelessness [should get], having realized special knowledge, lived with calm by himself in that goal of the religious life realized.)\footnote{This instance is a difficult one. There is a small sentence, marked by \pali{pabbajanti}, embedded in the big one, marked by \pali{vih\=asi}. The subject is \pali{Nando} (not shown here). Some key bunches of words are \pali{yassatth\=aya} (\pali{yassa + atth\=aya}), \pali{tadanuttara\d m} (\pali{ta\d m + anuttara\d m}). I see \pali{tadanuttara\d m brahmacariyapariyos\=ana\d m} (acc.) as object of \pali{di\d t\d theva} (\pali{di\d t\d the + eva}), hence ``that goal of the religious life realized.'' In turn, \pali{di\d t\d the} modifies \pali{dhamme} (loc.), which denotes the state that he lived in.}\par

\stepcounter{parnum}
\paragraph*{\arabic{parnum}. \pali{Ya\d m-ta\d m} can stand for clauses.} Basically, the \pali{ya-ta} pair works as pronoun, which relates somehow to other noun. In some cases, these pronouns do not relate to any particular noun but the whole clause. The pair normally appears as \pali{ya\d m-ta\d m} in this use. It is no better than seeing some examples:\par
- \pali{Idha kho \textbf{ta\d m}, bhikkhave, sobhetha, \textbf{ya\d m} tumhe eva\d m sv\=akh\=ate dhammavinaye pabbajit\=a sam\=an\=a kham\=a ca bhaveyy\=atha sorat\=a ca}\footnote{Dhp-a\,1.6} (Monks, that you, having gone forth in this well-preached religion, are being patient and gentle, is made beautiful here.)\par
Let me explain this a little more. I do not use `which-that' pair in this translation, like I normally do. Only one `that,' which stands for the \pali{ya\d m} clause in the latter part, is used here. The hidden subject of the former part is `you [all]' agreeing with causative plural verb \pali{sobhetha} (to make beautiful). But by the sentence's structure in English, the translation of the first part has to be in passive voice. In active sense, it can be translated as ``Monks, you make `that' beautiful here.'' But unpacking `that' in this sentence is more difficult to do in English.\par
The main verb in \pali{ya\d m} clause is \pali{bhaveyy\=atha}. The use of optative mood here can be seen as a suggestion or a wish, but I put it simply in present tense.\par
- \pali{anacchariya\d m kho pan\textbf{eta\d m} bhikkhu, \textbf{ya\d m} tva\d m m\=adisa\d m \=acariya\d m labhitv\=a appiccho ahosi}\footnote{Dhp-a\,2.32. In this sentence and the next one, \pali{ya\d m} pairs with \pali{eta\d m}.} (That you, monk, having got a teacher like me, was an ungreedy person, [is] not astonishing.)\par
- \pali{\d Th\=ana\d m kho pan\textbf{eta\d m} vijjati, \textbf{ya\d m} tva\d m kum\=arova sam\=ano k\=ala\d m kareyy\=asi}\footnote{Dhp-a\,1.17} (That you, being just a child, shall die, is not possible.)\footnote{For \pali{\d th\=ana\d m vijjati} as ``It is possible,'' see Chapter \externalref{22} of PNL.}\par
- \pali{\textbf{Ya\d m} bhikkh\=una\d m akkosakaparibh\=asako ahosi, \textbf{tena}ssa mukh\-ato duggandho v\=ayati.}\footnote{Dhp-a\,24.334} (That [he] was one who abused and scolded monks, bad smell emits from his mouth by that [action].)\par
- \pali{\~N\=atam\textbf{eta\d m} kuru\.ngassa, \textbf{ya\d m} tva\d m sepa\d n\d ni siyyasi}\footnote{Ja\,1:21} \\(Sepa\d n\d n\=i tree, that you drop [your fruit], the deer knew it.)\par

\stepcounter{parnum}
\paragraph*{\arabic{parnum}. \pali{Yasm\=a-tasm\=a} marks a cause or reason.} I prefer the translation of this pair as `from which/that reason' to reflect the P\=ali terms. However, the two terms can be rendered in a more familiar way as `because' and `therefore' respectively. It is also sensible when only one word is used alone. Furthermore, grammatically speaking, \pali{yena-tena} (ins.) and \pali{yasmi\d m-tasmi\d m} (loc.), including its indeclinable equivalents like \pali{yato-tato} and \pali{yattha-tattha}, also have the same meaning, if the pair controls the whole clause, not related to a word in particular.\par
- \pali{\textbf{Yasm\=a} ca kho, bhikkhave, atthi cakkhussa ass\=ado \textbf{tasm\=a} satt\=a cakkhusmi\d m s\=arajjanti.}\footnote{S4\,17, SN\,35} (From which reason [= Because], monks, the enjoyment of the eye exists, from that reason [= therefore] beings are attached to the eye.)\footnote{Or you can put it simpler as ``Monks, beings are attached to the eye because the enjoyment of the eye exists.''}\par
- \pali{Tattha buddh\=anamupp\=ado'ti \textbf{yasm\=a} buddh\=a uppajjam\=an\=a mah\-\=ajana\d m r\=agakant\=ar\=ad\=ihi t\=arenti, \textbf{tasm\=a} buddh\=ana\d m upp\=ado sukho uttamo.}\footnote{Dhp-a\,14.194. This is an explanation in a commentary. For more detail, see Chapter \ref{chap:commentary}.} (In that [verse], [what is meant by] `\pali{buddh\=anamupp\=ado}' is this: From which reason the Buddhas, arising, make the masses cross [from] the desert of lust etc., from that reason the arising of the Buddhas is the highest happiness.)\par
- \pali{Ki\d m k\=ara\d n\=a? Mara\d nanta\~nhi j\=ivita\d m \textbf{yasm\=a} sabbasatt\=ana\d m j\=ivita\d m mara\d napariyos\=anamev\=a'ti vutta\d m hoti.}\footnote{Dhp-a\,11.148} (Why? [Because] life [has] death as the end. Because there is an explanation that ``the life of all beings [has] death as the end.'')\par

\stepcounter{parnum}
\paragraph*{\arabic{parnum}. \pali{Yath\=a-tath\=a} is used in metaphors.} Sometimes, \pali{seyyath\=a} is used instead of \pali{yath\=a}, and \pali{eva\d m} instead of \pali{tath\=a}. And sometimes, only one part of the pair is present. See these examples for better understanding.\par
- \pali{\textbf{Yath}eva tumhe ta\d m na passatha, \textbf{tath}eva sopi te p\=a\d ne na passati.}\footnote{Dhp-a\,1.1} (In which way you do not see him, in that way [= likewise] he does not see those living beings.)\par
- \pali{\textbf{Yath\=a} hi cor\=ad\=ina\d m coraje\d t\d thak\=adayo adhipatino se\d t\d th\=a. \textbf{Ta\-th\=a} tesampi mano adhipati manova se\d t\d th\=a.}\footnote{Dhp-a\,1.1} (Like gang leaders etc., the rulers, are superior to thieves etc.; in that manner the mind [is] the ruler of those [entities]; only the mind [is] superior.)\par
- \pali{\textbf{Yath\=a} pana d\=aru\=ad\=ihi nipphann\=ani t\=ani t\=ani bha\d n\d d\=ani d\=arumay\=ad\=ini n\=ama honti, \textbf{tath\=a} tepi manato nipphannatt\=a manomay\=a n\=ama.}\footnote{Dhp-a\,1.1} (Also, like those things produced from wood etc.\ are called made-of-wood etc.; in that manner [entities] produced from the mind [are] called made-from-mind.)\par
- \pali{\textbf{seyyath\=a}pi s\=a, bhante, chinnaka\d n\d nan\=asana\.ngu\d t\d th\=a palu\d t\d tha\-makka\d t\=i, \textbf{evam}eva kho, bhante, s\=akiy\=an\=i janapadakaly\=a\d n\=i}\footnote{Dhp-a\,1.13} (Sir, that female monkey having crippled ear, nose, and tail, [is] like Janapadakaly\=a\d n\=i of the S\=akya.)\par
- \pali{\textbf{Yath\=a} hi duggatamanuss\=a [yattha] katthaci gantuk\=am\=a khippameva nikkhamanti, r\=aj\=una\d m pana hatthiv\=ahanakappan\=adi\d m mahanta\d m parikamma\d m laddhu\d m va\d t\d tati, \textbf{eva\d m}sampadamida\d m veditabba\d m.}\footnote{Dhp-a\,1.11. There is no \pali{yattha} in Thai edition.} (Like poor people, having a need to go somewhere, just go out immediately, but it is suitable to get a big preparation of the elephant, vehicle, harness, etc., for kings. This metaphor is should be known in this way.)\par
- \pali{\textbf{yath\=a}, mah\=ar\=aja, k\=u\d t\=ag\=aramatto p\=as\=a\d nopi nerayikaggimhi pak\-khitto kha\d nena vilaya\d m gacchati, nibbattasatt\=a panettha kammabalena m\=atukucchigat\=a viya na vil\=iyanti}\footnote{Dhp-a\,1.15. This instance has neither \pali{tath\=a} nor \pali{eva\d m}. Thai teachers insist that students should insert it when they work on this. Some might see \pali{viya} as a substitute, but this particle normally effects on word level not sentence level. We can leave out \pali{m\=atukucchigat\=a viya}, but the main metaphor is still intact.} \\(Your Majesty, like a castle-size stone thrown into the fire of hell goes dissolved in a moment, but beings born there do not dissolve [as such] like those born in the mother's womb, by the power of karma.)\par

\medskip
In some context, however, \pali{yath\=a-tath\=a} is better translated as `in which/that way,' like these examples:\par
- \pali{\textbf{Yath\=a} me dhanacchedo na hoti, \textbf{tath\=a} kariss\=ami}\footnote{Dhp-a\,1.2} (In which way my lack of wealth will not happen, I will do in that way.)\par
- \pali{\textbf{Yath\=a} d\=araka\d m na labhati, \textbf{tath}eva na\d m k\=atu\d m va\d t\d tati}\footnote{Dhp-a\,1.5} (In which way [she] will not get a baby, to make her in such a way is suitable.)\footnote{Put it simply, this means ``I should prevent her from having a baby.''}\par
- \pali{\textbf{Yath\=a} nissadd\=a hutv\=a ga\d nhanti, \textbf{tath\=a} me up\=ayo kato}\footnote{Dhp-a\,2.21} (In which way [they], having no sound, take [the food], in that way the method was done by me.)\footnote{Put it understandably, this means ``I have done in the way that people take [the food] quietly.''}\par
- \pali{Atha satth\=a tass\=agamanabh\=ava\d m \~natv\=a \textbf{yath\=a} attano santike nisinn\=a bhikkh\=u na pa\~n\~n\=ayanti, \textbf{evam}ak\=asi.}\footnote{Dhp-a\,6.79} (Then the Buddha, having known her [state of] coming, in which way monks sitting in his own place disappear, has done in that way.)\par

\stepcounter{parnum}
\paragraph*{\arabic{parnum}. \pali{Y\=ava-t\=ava} marks a boundary.} The pair may be translated strictly as `to which extent' and `to that extent' respectively, or simply `until', `as long as,' or in some contexts, `inasmuch as', `insofar as.' They often appear in the context of time, but they can also be used in other contexts that have a kind of limit to mark. See some examples below.\par
- \pali{\textbf{y\=ava} mahanta\d m \=aka\.nkhasi, \textbf{t\=ava} mahanta\d m kariss\=ami}\footnote{Dhp-a\,1.2} (How big [you] want, [I] will make [it] that big.)\par
- \pali{\textbf{y\=ava} ida\d m bandhana\d m na va\d d\d dhati, \textbf{t\=ava}deva na\d m chindiss\=ami}\footnote{Dhp-a\,1.11. In Thai edition, \pali{bandhati} is used instead of \pali{va\d d\d dhati}.} (To which extent this fetter does not grow, I will cut it to that extent.)\footnote{Simply, this means ``I will cut the fetter until it cannot bind me.''}\par
- \pali{na \textbf{t\=av}ima\d m palla\.nka\d m bhindiss\=ami, \textbf{y\=ava} me anup\=ad\=aya \=asavehi citta\d m na muccissati}\footnote{Dhp-a\,1.11} (I will not give up this sitting as long as my mind will not be free, being unattached, from intoxicants.)\par
- \pali{\textbf{y\=ava} thokampi kar\=isa\d m atthi, \textbf{t\=ava} \=avila\d m hutv\=a nikkhamati}\footnote{Dhp-a\,1.15} (To which extent even little feces exists, to that extent [the water], being dirty, goes out.)\footnote{An understandable rendition of this is ``As long as there is even little feces, dirty [water] goes out.''}\par
- \pali{\textbf{Y\=ava} tass\=a pavatti\d m na su\d n\=ama. \textbf{T\=ava} neva yuddha\d m dass\-\=ama, na rajja\d m}\footnote{Dhp-a\,2.21} (As long as we do not hear her happening, we will not give neither the battle nor the kingship.)\par
- \pali{So ekasmi\d m samaye gadrabhabh\=arakehi saddhi\d m takkasila\d m gantv\=a \textbf{y\=ava} bha\d n\d dassa vissajjana\d m, \textbf{t\=ava} gadrabha\d m caritu\d m vissajjesi.}\footnote{Dhp-a\,1.14} (In one occasion, he, having gone to Takkasil\=a together with a donkey loaded with goods, sent off the donkey to walk until the discharge of the goods.)\footnote{If you take \pali{y\=ava-t\=ava} seriously, their part can be rendered awkwardly as ``\ldots\ to which extent the goods are not sold out, to that extent [he] sent off the donkey to walk.''}\par

\medskip
This pair works nicely with \pali{\=agameti} (to wait, to expect) as these examples show:\par
- \pali{\=agamehi \textbf{t\=ava}, meghiya, ekakomhi \textbf{y\=ava} a\~n\~nopi koci bhikkhu \=agacchati}\footnote{Dhp-a\,3.33} (Wait! Meghiya, until some other monk comes. I am alone.)\par
- \pali{Tena hi, s\=ami, \=agamehi \textbf{t\=ava}, \textbf{y\=av}\=aha\d m kucchigata\d m d\=araka\d m vij\=ay\=ami}\footnote{Dhp-a\,6.84} (If it is so, master, [please] wait until I give birth to a child who has come to my womb.)\par

\stepcounter{parnum}
\paragraph*{\arabic{parnum}. \pali{Pa\d t\d th\=aya-y\=ava} marks a period of time.} In the context of time, \pali{y\=ava} can pair with \pali{pa\d t\d th\=aya}\footnote{This term is a \pali{tv\=a} form of \pali{pa\d t\d thahati} (to put down), but it is seen as a particle in this use.} to mark a period of time, in the sense of `from the time of' and `up to.' If a range of time is mentioned, both come together. If only a point of time is intended, only one of them comes. Terms related to these words normally take ablative case, if they are not ablative particles (with \pali{-to} ending). In addition, \pali{y\=ava} can be used in contexts other than time. In this case, it can pair with \pali{\=adi\d m katv\=a} (see the last example below).\par
- \pali{mah\=abhinikkhamanato \textbf{pa\d t\d th\=aya y\=ava} ajap\=alanigrodham\=ul\=a m\=arena anubaddhabh\=ava\d m}\footnote{Dhp-a\,2.21} (The state of being followed by the king of death from the time of the great renouncement to the root of the goatherd's banyan tree.)\par
- \pali{Devadattassa vatthu pabbajitak\=alato \textbf{pa\d t\d th\=aya y\=ava} pathavippavesan\=a devadatta\d m \=arabbha bh\=asit\=ani sabb\=ani j\=atak\=ani vitth\=aretv\=a kathita\d m.}\footnote{Dhp-a\,1.17} (The story of Devadatta, beginning from the time of going forth to entering the earth [= dying], was told. [The Buddha] explained all birth stories, mentioning Devadatta.)\footnote{The structure of this sentence is strange. In Thai edition, it is \pali{vatthu\d m} (acc.) not \pali{vatthu} (nom.). So, the Buddha is the hidden subject. But the verb \pali{kathita\d m} is suspicious. I make the sentence passive instead, and cut the \pali{tv\=a} clause as another sentence. This is not a good translation because the two sentences look unconnected. You may try another way.}\par
- \pali{[C\=atumah\=ar\=ajika\d m \=adi\d m katv\=a] \textbf{y\=ava} akani\d t\d thabh\=avan\=a pana ekaninn\=ada\d m kol\=ahala\d m agam\=asi.}\footnote{Dhp-a\,1.6. In Thai edition, there is no \pali{C\=atumah\=ar\=ajika\d m} part.} (The unified sound of uproar went up to heaven Akani\d t\d tha, [starting from C\=atumah\=ar\=ajik\=a].)\par
- \pali{tasmi\d m uppanne makkhik\=a \=adi\d m katv\=a \textbf{y\=ava} g\=av\=a pa\d thama\d m tiracch\=anagat\=a maranti}\footnote{Dhp-a\,2.24} (When that [disease] happened, animals die first, starting from flies up to cattle.)\par

\stepcounter{parnum}
\paragraph*{\arabic{parnum}. \pali{Ya\~nce-seyyo} is used in comparison.} This is a special case of \pali{ya-ta} structure, mostly found in verses. It is used to compare two conditions that one of them is better (\pali{seyyo}). Hence, The \pali{ta} word left out is \pali{tato} or \pali{tasm\=a} (than that).\footnote{Learn more about adjective comparison in Chapter \externalref{18} of PNL.} Here are some examples:\par
- \pali{Da\d n\d dova kira me [tato] \textbf{seyyo}, \textbf{ya\~nce} putt\=a anassav\=a}\footnote{S1\,200 (SN\,7)} ([It is said that] my [walking] stick is better than disobedient children.)\footnote{Or you can translate it in this way: ``In which condition disobedient children [exist], my stick is better than that, it is said. In Thai translation, the \pali{ya\~nce} part is often changed to a question, thus ``It is said that my stick is better, what is good about disobedient children?''}\par
- \pali{\textbf{Seyyo} amitto medh\=av\=i, \textbf{ya\~nce} b\=al\=anukampako}\footnote{Ja\,1:45} (A wise enemy is better than a foolish compassionate one.)\par
- \pali{Sa\.ng\=ame me mata\d m \textbf{seyyo}, \textbf{ya\~nce} j\=ive par\=ajito}\footnote{Thag\,194} (Dying in the battle by me is better than having been defeated, living.)\par
