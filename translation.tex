\chapter{On translation}\label{chap:translation}

I will use this chapter to bring theories discussed in \emph{translation studies} into our consideration. This field of study is vast and complicated nowdays. It is quite irrelevant to review all issues in the field. But ignoring perspectives given by the field is not a good idea either. In fact, some parts of theories are overlapping with what we have discussed previously, particularly the notion of ideology and discourse. To reduce the learners' burden, I will bring only some parts essential to our approach by not locating us in the theoretical space.\footnote{Several terms used in the field are baffling to the outsiders, such as descriptive translation, skopos (functional) theory, polysystem theory, etc. I will not mention strange names even if our discussion has something to do with them one way or another. For an up-to-date introduction to the field, see \citealp{munday:translation}.}

Since the present book is not really on translation \textit{per se}, we will not talk about how to do a good English translation. Rather, we will focus here only the ideas that affect interlingual reading. As I have said, translation in our concern is just a specific case of reading.\footnote{Strictly speaking, output of translation has to be in written form. When we just read text in other language and try to make sense in our own language without writing it down, this is commonly called \emph{interpretation}. Yet the definition of interpretation can go more specific in interpretation studies \citep[see][p.~8]{munday:translation}. I do not keep the distinction clearly in this book. So, reading, translation, and interpretation can mean the same thing in our course.} Our purpose of this book is therefore aimed at how to read P\=ali texts. By using English as the medium, we inevitably regard English as our target language. However, the ideas discussed here can be applied to other target languages as well.

\section*{What is translation, really?}

If we study translation academically, particularly from translation studies, defining `translation' exactly is somewhat troublesome, in the same way as defining `culture' in cultural studies or `religion' in religious studies. Even though we do not approach the subject that way, a basic definition has to be mentioned. The most cited explanation when translation is defined is from Roman Jakobson (1896--1982). According to Jakobson, there are three kinds of translations:\footnote{\citealp[p.~261]{jakobson:selected2}; \citealp[p.~127]{jakobson:translation} (first published in 1959)}

\begin{compactenum}[(1)]
\item Intralingual translation or \emph{rewording} is an interpretation of verbal signs by means of other signs of the same language.
\item Interlingual translation or \emph{translation proper} is an interpretation of verbal signs by means of some other language.
\item Intersemiotic translation or \emph{transmutation} is an interpretation of verbal signs by means of signs of nonverbal sign systems.
\end{compactenum}

In our concern, when we talk about translation, generally we mean interlingual translation or \emph{translation proper} in Jakobson's terms. When P\=ali language is considered, this is almost always the case in modern time, because we do not understand P\=ali texts by P\=ali language itself. So, reading a P\=ali text is more or less equal to translating it to a modern language.

However, for knowledgeable scholars or advanced learners, intralingual translation of P\=ali can also be the case. Sometimes we have to reword or rephrase an obscure word or sentence to a more comprehensible one. Or in some cases, we may need to convert a prose into a verse or vice versa.

Finally, for our main material is only in textual form, no nonverbal sign is used, intersemiotic translation will never occur in our activity.

\begin{figure}[!hbt]
\centering
\setlength{\unitlength}{1mm}
\begin{picture}(80,40)(0,0)
\thinlines
\put(0,20){\makebox(0,0)[l]{Author}}
\put(12,20){\vector(1,0){5}}
%% paper
\thicklines
\put(17,16){\line(1,0){6}}
\put(17,16){\line(0,1){8}}
\put(17,24){\line(1,0){6}}
\put(23,16){\line(0,1){8}}
\thinlines
\put(18,18){\line(1,0){4}}
\put(18,19){\line(1,0){4}}
\put(18,20){\line(1,0){4}}
\put(18,21){\line(1,0){4}}
\put(18,22){\line(1,0){4}}
%%
\put(20,20){\makebox(0,25)[c]{Source Text}}
\put(20,20){\makebox(0,15)[c]{(ST)}}
\put(20,20){\makebox(0,-15)[c]{Source Language}}
\put(20,20){\makebox(0,-25)[c]{(SL)}}
%%
\put(23,20){\vector(1,0){35}}
\put(40,20){\makebox(0,5)[c]{Translation}}
\put(40,20){\makebox(0,-5)[c]{(Translator)}}
%% paper
\thicklines
\put(58,16){\line(1,0){6}}
\put(58,16){\line(0,1){8}}
\put(58,24){\line(1,0){6}}
\put(64,16){\line(0,1){8}}
\thinlines
\put(59,18){\line(1,0){4}}
\put(59,19){\line(1,0){4}}
\put(59,20){\line(1,0){4}}
\put(59,21){\line(1,0){4}}
\put(59,22){\line(1,0){4}}
%%
\put(61,20){\makebox(0,25)[c]{Target Text}}
\put(61,20){\makebox(0,15)[c]{(TT)}}
\put(61,20){\makebox(0,-15)[c]{Target Language}}
\put(61,20){\makebox(0,-25)[c]{(TL)}}
%%
\put(64,20){\vector(1,0){5}}
\put(70,20){\makebox(0,0)[l]{Reader}}
\end{picture}
\caption{The overall process of translation}
\label{fig:overall}
\end{figure}

To see the big picture of interlingual translation, let us see Figure \ref{fig:overall}. In our concern, target text here can be of an immediate situation that deliberate edition or even jotting down is not feasible.\footnote{In interpretation studies, this is more suitably called `interpretation' \citep[see][pp.~10--1]{pochhacker:interpreting}. But we do not follow that.} Hence translation may happen only in memory. In this case, the translator is the same as the reader. When a full-fledged translation is produced, the target text is normally in written form that can be read by others.

There are some issues that we should know:

\paragraph*{1.\ Translation continues the life of the original.} This idea is proposed by Walter Benjamin (1892--1904). The main argumentation of this is translated text does not exist to give the meaning or information of the original. Rather, it stands apart from, but in conjunction with, the original as a separate work. By this, the translation continues the life of the original which is barely understood, yet regarded as translatable. 

\begin{quote}
It is clear that a translation, no matter how good, cannot have any significance for the original. Nevertheless, it has the closest connection with the original by virtue of the latter's translatability.\footnote{\citealp[p.~76]{benjamin:task} (first published in 1923)}
\end{quote}

For example, most Buddhists who know little about P\=ali read the canon by its translations. Still, the original scriptures have to be kept, studied, and respected. If the canon cannot be understood, it does not exists. When a translation can be made in some way, well or poorly, the canon's existence is visible by its translation. That is to say, the translations prolong the life of the canon by making it seemingly understood in one way or another.

Moreover, in Benjamin's view, the way to achieve `true' translation is through \emph{word-for-word} rendering.\footnote{\citealp[p.~81]{benjamin:task}} By this way, the source and target language can be harmonized.\footnote{To understand this entails a philosophical explanation. I see it too technical and unnecessary for us. So, I will make no attempt to make it clearer. For a quick treatment, see \citealp[pp.~260--4]{munday:translation}.}

\paragraph*{2.\ Translation is in fact a rewriting.} When we read a smooth translation of certain work, we experience the translation as original. This effect is called `illusion of transparency' by Lawrence Venuti.

\begin{quote}
A translated text, whether prose or poetry, fiction or nonfiction, is judged acceptable by most publishers, reviewers and readers when it reads fluently, when the absence of any linguistic or stylistic peculiarities makes it seem transparent, giving the appearance that it reflects the
foreign writer's personality or intention or the essential meaning of the foreign text -- the appearance, in other words, that the translation is not in fact a translation, but the `original.'\footnote{\citealp[p.~1]{venuti:invisibility}}
\end{quote}

When ordinary people read a translation of the P\=ali canon and they feel its readability, the translator is virtually invisible to them because they read the work as the canon itself. In fact, as P\=ali students know, translation of ancient text is not that easy. To make it easy to read, many things have to be neglected, sometimes, new elements have to be added, the structure of sentences has to be changed, and original meaning has to be domesticated. All these are unaware to untrained readers, making the illusion of transparency so apparent.

In the same line as Walter Benjamin's thought, translations stand by themselves as separate work. They are product of \emph{rewriting}, not just a delivery of messages of the original in other forms. From a single source, translations can be made in many different ways. Sometimes they are different considerably. We can see this in various translations of Buddhist scriptures or the Bible.

As we have seen in previous chapters, transmission of intact source texts through a long period of time is impossible. Furthermore, it is likely that the original itself might be a translation from other source or even other translation.\footnote{When the issue of the language of the Buddha used in teaching is taken into consideration, it is likely that the teaching was rendered into several languages or dialects to make local people understand them. So, it is logical to think that the P\=ali canon itself is by and large a product of translation, perhaps from a close related languages.} A conspicuous case is Chinese Buddhist scriptures which their indic sources had been lost. By the widespread practice of relay translation at the time, the source itself might be also a translation.\footnote{\citealp[p.~47]{neather:buddhism}}

Consequences of seeing translation as rewriting is revealing, and disturbing for some. A notable result is that translation inevitably brings about some form of loss or change. Or to put is more drastically, human translation is always to some extent `false.'\footnote{\citealp[p.~26]{fawcett:translation}. This echoes deconstruction in literary theory. To Paul de Man (1919--1983), reading is always necessarily `misreading' (\citealp[p.~172]{selden:literary}). But some `misreading' may be better than others.} This does not mean we cannot understand original text in some way. It reminds us the illusion of transparency. That is to say, in translation, manipulation of meaning is always the case.

The point I want to stress here is when translation is a rework, decisions made by translators are determined by their perspective and interest. This leads us to the next issue.

\paragraph*{3.\ Translation is always ideological.} We have already talked about ideology and discourse in Chapter \ref{chap:howunderman}. Here I will focus more on translation. When translation is a kind of authoring new text, within a certain boundary\footnote{Comparing this to a historical fiction, we can get a clearer picture. When a historical fiction is written, some obvious facts have to be retained, vague facts have to be made clear, missing facts have to be made up reasonably, the event is dramatized, a purpose (what we call `theme' in literature) is set.}, choices made by the author/translator are not neutral because we all have different knowledge structure or prejudices in Gadamer's terms.

You might think what can be translated has to be limited in the language used in the text, not anything can be added. That is true for benign points. When a conflict occurs, Andr\'e Lefevere notes that:

\begin{quote}
On every level of the translation process, it can be shown that, if linguistic considerations enter into conflict with considerations of an ideological and/or poetological nature, the latter tend to win out.\footnote{\citealp[p.~39]{lefevere:rewriting}}
\end{quote}

When we make a translation of certain sacred text, on the point that can be rendered against the established norms or beliefs, we tend to make it more conformable, or to water down the point, if not to neglect the point all together. And when you make a poet out of such text, word choices are mostly determined by rules governing the poetic form, not the exact content of the text.

There are two main factors, in Lefevere's view, that control the literary system: (1) professionals within the literary system, who partly determine the dominant poetics; and (2) patronage outside the literary system, which partly determines the ideology.\footnote{\citealp[p.~200]{munday:translation}}

Bringing this to religious scene, we can see a parallel picture. What determines the direction of translations of religious texts are: (1) religious authority who sets criteria of `good' translation, and (2) the king, government, or other form of patronage supporting the religious institute. This issue is closely linked to the ethics of translation which we will discuss in Chapter \ref{chap:ethics}.
