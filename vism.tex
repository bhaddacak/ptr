\chapter{Visuddhimagga}\label{chap:vism}

After we have seen several examples from the canon, we now know that reading canonical text is not so difficult as new P\=ali students might think. Today all crucial materials in the P\=ali canon are translated, so it is easy, or at least manageable, for new learners to study the canon by their own with the knowledge we have learned together so far. We can use existing translations as a guideline when encountering some difficult points. Or better, we can evaluate how good or bad those translations are. And if you a P\=ali or Buddhist scholar, you can research into the P\=ali canon with confidence.

We have already read some parts of commentaries. Now we will read more post-canonical literature. The most important author is undoubtedly Budhhaghosa of Mah\=avih\=ara. In this chapter we will study his remarkable work, \emph{Visuddhimagga} (the Path of Purification). For it is impossible to bring a full chapter of it here, I select only one portion as an example. We will learn a later style of P\=ali composition. That can equip us to be a competent P\=ali readers (only if you study various texts further by your own).

\phantomsection
\addcontentsline{toc}{section}{Pre-reading introduction}
\section*{1.\ Pre-reading introduction}

\paragraph*{About the text} Visuddhimagga is a treatise composed by Buddhaghosa. It is modeled after Vimuttimagga of Upatissa.\footnote{Only Chinese and partly Tibetan translations survive.} So, it is not original by its three-fold structure, namely \pali{s\=ila} (morality), \pali{sam\=adhi} (concentration), and \pali{pa\~n\~n\=a} (wisdom), but the content is more elaborated. In modern context, Visuddhimagga can be seen like a dissertation or a research report to summarize the main ideas of the P\=ali canon.

The excerpt we study here is the last part of Chapter 18, \pali{Di\d t\d thivisuddhiniddeso}. This chapter concerns mainly about name-and-form (\pali{n\=amar\=upa}), and the part I bring here is about metaphors of name-and-form (Vism\,18.673--677). I find this part interesting because it depicts vivid pictures to help us understand the relation between name and form.

\paragraph*{About the author} Buddhaghosa, an Indian monk, came to Sri Lanka in the reign of king Mah\=an\=ama (AD 409--431 or AD 349--371\footnote{\citealp[p.~102]{hinuber:literature}. In the legend, 956 years after the death of the Buddha is mentioned as the year of coronation. By Thai calculation, it is AD 413.}). He is the most prominent commentator in Therav\=ada tradition, but little is known about detail of his life. He composed the majority of commentaries to the P\=ali canon, including the Vinaya, the first four Nik\=ayas, some parts of Khuddakanik\=aya, and the Abhidhamma. Oskar von Hin\"uber gives us AD 370 to 450 approximately for Buddhaghasa's dates.\footnote{\citealp[p.~103]{hinuber:literature}} Unlike texts in the canon, we can say that the sender and text producer of Visuddhimagga is Buddhaghosa himself. And the text medium is clearly written type. It is a product of literary culture, so are all post-canonical texts.\footnote{This is a fact. No one can compose any text only with memory. There must be something to write down and revise later. This fact can support that texts produced by oral culture are transmitted, not composed. But it does not guarantee that the content of oral transmission does not suffer distortion or even fabulation.}

\paragraph*{About the audience} According to the legend\footnote{A detailed account on Buddhaghosa in P\=ali is found as a supplement to Thai edition of Visuddhimagga. It was compiled or recomposed from various old sources by Somdet Phra Vannarat, Heng Khemac\=ar\=i (1882--1943).}, when Buddhagh\-osa came to Sri Lanka in order to study Sinhalese commentaries, he was tested by the Sangha before the study was allowed. And Visuddhimagga was the thesis Buddhaghosa proposed to the Sangha of Mah\=avih\=ara. By this account, the original audience of this work is senior monks of the Sangha in that time. By virtue of his articulation, Visuddhimagga then spread throughout the Therav\=ada world. However, by its highly scholastic style, the main audience of the text is limited to learned monks and lay scholars. Particularly, the part of meditation methods appeals to forest monks who take meditation seriously, and it has been referred as the Buddhist meditation bible up to these days.
 
\paragraph*{About time and place} The exact year of composition is unknown, von Hin\"uber surmises it was around AD 400.\footnote{\citealp[p.~126]{hinuber:literature}. This seems too early if we take the coronation year of king Mah\=an\=ama into consideration.} The place is rather unquestionably at Mah\=avih\=ara monastery in Anur\=adhapura, Sri Lanka.

\paragraph*{About motives} From the legendary account, the main motive of this text is to qualify the author for accessing the Sinhalese materials. That is the good part of legends. At least, we have something to say about uncertain or unknown fact.

\paragraph*{About text function} Informative function can be a marked type of this text. If we bring the competition between two big monasteries (Mah\=avih\=ara vs.\ Abhayagir\=i) at the time into consideration, we can also attribute operative function to the text. That is to say, it can be used to promote Mah\=avih\=ara position. If this is true, the result was very successful, as we have seen that Mah\=avih\=ara has dominated Therav\=ada world until today.

\phantomsection
\addcontentsline{toc}{section}{Reading with a draft translation}
\section*{2.\ Reading with a draft translation}

The format used here follows the previous example.

\bigskip
\begin{center}
\textbf{\pali{Upam\=ahi n\=amar\=upavibh\=avan\=a}}\par
$\triangleright$ Explanation of name-and-form from metaphors
\end{center}

\pali{\fbox{\stepcounter{sennum}\arabic{sennum}} 673. Eva\d m anekasatehi suttantehi n\=amar\=upameva d\=ipita\d m, na satto na puggalo. \fbox{\stepcounter{sennum}\arabic{sennum}} Tasm\=a yath\=a akkhacakkapa\~njara\=is\=ad\=isu a\.ngasambh\=aresu eken\=ak\=arena sa\d n\d thitesu ratho'ti voh\=aramatta\d m hoti, paramatthato ekekasmi\d m a\.nge upaparikkhiyam\=ane ratho n\=ama natthi. \fbox{\stepcounter{sennum}\arabic{sennum}} Yath\=a ca ka\d t\d th\=ad\=isu gehasambh\=aresu eken\=ak\=arena \=ak\=asa\d m pariv\=aretv\=a \d thitesu gehan'ti voh\=aramatta\d m hoti, paramatth\-ato geha\d m n\=ama natthi. \fbox{\stepcounter{sennum}\arabic{sennum}} Yath\=a ca a\.ngulia\.ngu\d t\d th\=ad\=isu eken\=ak\=a\-rena \d thitesu mu\d t\d th\=i'ti voh\=aramatta\d m hoti. \fbox{\stepcounter{sennum}\arabic{sennum}} Do\d nitanti\=ad\=isu v\=i\d n\=a'ti. Hatthiass\=ad\=isu sen\=a'ti. P\=ak\=aragehagopur\=ad\=isu nagaran'ti. \fbox{\stepcounter{sennum}\arabic{sennum}} Khandhas\=akh\=apal\=as\=ad\=isu eken\=ak\=arena \d thitesu rukkho'ti voh\=aramatta\d m hoti, paramatthato ekekasmi\d m avayave upaparikkhiy\-am\=ane rukkho n\=ama natthi. \fbox{\stepcounter{sennum}\arabic{sennum}} Evameva\d m pa\~ncasu up\=ad\=anakkhandhesu sati `satto, puggalo'ti voh\=aramatta\d m hoti, paramatthato ekekasmi\d m dhamme upaparikkhiyam\=ane `asm\=iti v\=a ahanti v\=a'ti g\=ahassa vatthubh\=uto satto n\=ama natthi. Paramatthato pana n\=amar\=upamattameva atth\=i'ti \fbox{\stepcounter{sennum}\arabic{sennum}} Eva\d m passato hi dassana\d m yath\=a\-bh\=utadassana\d m n\=ama hoti.}

\addtocounter{sennum}{-8}
$\triangleright$ \fbox{\stepcounter{sennum}\arabic{sennum}} 673. In this way, name-and-form was elucidated in several hundred of suttas as not being, not person. \fbox{\stepcounter{sennum}\arabic{sennum}} Therefore, just like when axle, wheel, body, pole, and other components were composed together, a mere designation of `chariot' happens. [But] from the ultimate sense of truth, when each part is being investigated, the label `chariot' does not exist. \fbox{\stepcounter{sennum}\arabic{sennum}} Also, in the same way, when pieces of wood and other house components, having enclosed a space together, were erected, a mere designation of `house' happens. [But] from the ultimate sense of truth, the label `house' does not exist. \fbox{\stepcounter{sennum}\arabic{sennum}} Also, likewise, when fingers and thumb came together, a mere designation of `fist' happens. \fbox{\stepcounter{sennum}\arabic{sennum}} [In the same way,] when body and strings [were put together], a `lute' [is called]. When elephants and horses [were put together], an `army' [is called]. When wall, houses, and gates [were put together], a `city' [is called]. \fbox{\stepcounter{sennum}\arabic{sennum}} When trunk, branches, and leaves were composed together, a mere designation of `tree' happens. [But] from the ultimate sense of truth, when each part is being investigated, the label `tree' does not exist. \fbox{\stepcounter{sennum}\arabic{sennum}} In the same manner, when the attached five aggregates exist, a mere designation of `being, person' happens. [But] from the ultimate sense of truth, when each entity is being investigated, the being, the object of grasping thus `I am or I,' does not exist. From the ultimate sense of truth, however, only name-and-form exists. \fbox{\stepcounter{sennum}\arabic{sennum}} Thus, the view of [one who is] seeing as such is called ``seeing things as they really are.''\\

\begin{longtable}[c]{|p{0.9\linewidth}|}
\hline
\hspace{5mm}\small For those who are still baffled with compounds, \pali{akkhacakkapa\~njara\=is\=ad\=isu} (pl.\,loc.) = \pali{akkha + cakka + pa\~njara + \=is\=a + \=adi + su}. Locative case used here is to mark an absolute construction. Precisely, \pali{eken\=ak\=arena} (\pali{ekena + \=ak\=arena}) means `by one manner/condition.' I use simply `together' here.\\
\hspace{5mm}\small Towards the end of this paragraph, \pali{passato} is a genitive form of \pali{passanta} (seeing). It is used here as a noun meaning `one who is seeing.' In this sentence, \pali{eva\d m} is translated as `as such,' and `thus' is related to \pali{iti} at the end of the previous sentence.\\
\hline
\end{longtable}

\pali{\fbox{\stepcounter{sennum}\arabic{sennum}} 674. Yo paneta\d m yath\=abh\=utadassana\d m pah\=aya `satto atth\=i'ti ga\d nh\=ati. So tassa vin\=asa\d m anuj\=aneyya avin\=asa\d m v\=a. \fbox{\stepcounter{sennum}\arabic{sennum}} Avin\=asa\d m anuj\=ananto sassate patati. Vin\=asa\d m anuj\=ananto ucchede patati. \fbox{\refstepcounter{sennum}\arabic{sennum}\label{sen:anvaya}} Kasm\=a? Kh\=iranvayassa dadhino viya tadanvayassa a\~n\~nassa abh\=avato. \fbox{\stepcounter{sennum}\arabic{sennum}} So `sassato satto'ti ga\d nhanto ol\=iyati n\=ama. `Ucchijjat\=i'ti ga\d nhanto atidh\=avati n\=ama. \fbox{\stepcounter{sennum}\arabic{sennum}} Ten\=aha bhagav\=a --}

\addtocounter{sennum}{-5}
$\triangleright$ \fbox{\stepcounter{sennum}\arabic{sennum}} 674. Which [person], having renounced this seeing-things-as-they-are, holds that `being exists,' that [person] should accept its destructibility or indestructibility [too]. \fbox{\stepcounter{sennum}\arabic{sennum}} When [one] accepts indestructibility, [he or she] falls into eternalism. When [one] accepts destructibility, [he or she] falls into nihilism. \fbox{\stepcounter{sennum}\arabic{sennum}} Why? [Because] the other continuity of that is absent, like curd [is] milk's continuity. \fbox{\stepcounter{sennum}\arabic{sennum}} That who holds `being is eternal' is called [he or she] retards. [And that] who holds `[being is] annihilated' is called [he or she] overruns. \fbox{\stepcounter{sennum}\arabic{sennum}} The Blessed One said as follows:\\

\begin{longtable}[c]{|p{0.9\linewidth}|}
\hline
\hspace{5mm}\small In \fbox{\ref{sen:anvaya}}, it means like if we accept that curd is a successive state of milk, we should accept that something can become other thing as well. The belief in eternalism or nihilism violates this fact. For more understanding of compounds used here, see \pali{anvaya} in PTSD.\\
\hline
\end{longtable}

\pali{\fbox{\stepcounter{sennum}\arabic{sennum}} Dv\=ihi, bhikkhave, di\d t\d thigatehi pariyu\d t\d thit\=a devamanuss\=a ol\=iyanti eke, atidh\=avanti eke, cakkhumanto ca passanti. \fbox{\stepcounter{sennum}\arabic{sennum}} Katha\~nca, bhikkhave, ol\=iyanti eke? Bhav\=ar\=am\=a, bhikkhave, devamanuss\=a bhavarat\=a bhavasamudit\=a. Tesa\d m bhavanirodh\=aya dham\-me desiyam\=ane citta\d m na pakkhandati nappas\=idati na santi\d t\d thati n\=adhimuccati. Eva\d m kho, bhikkhave, ol\=iyanti eke. \fbox{\refstepcounter{sennum}\arabic{sennum}\label{sen:atidhaavanti}} Katha\~nca, bhikkhave, atidh\=avanti eke? Bhaveneva kho paneke a\d t\d t\=iyam\=an\=a har\=ayam\=an\=a jiguccham\=an\=a vibhava\d m abhinandanti, `yato kira bho aya\d m att\=a k\=ayassa bhed\=a ucchijjati vinassati, [tato] na hoti para\d mmara\d n\=a, eta\d m santa\d m, eta\d m pa\d n\=ita\d m, eta\d m y\=ath\=avan'ti. Eva\d m kho, bhikkhave, atidh\=avanti eke. \fbox{\stepcounter{sennum}\arabic{sennum}} Katha\~nca, bhikkhave, cakkhumanto passanti? Idha, bhikkhave, bhikkhu bh\=uta\d m bh\=utato passati, bh\=uta\d m bh\=utato disv\=a bh\=utassa nibbid\=aya vir\=ag\=aya nirodh\=aya pa\d tipanno hoti. Eva\d m kho, bhikkhave, cakkhumanto passant\=i'ti}\footnote{This quotation is from It\,49, but the texts are slightly different.}

\addtocounter{sennum}{-4}
$\triangleright$ \fbox{\stepcounter{sennum}\arabic{sennum}} Monks, gods and humans were possessed by two views, some retard [and] some overrun, but those having eyes [can] see. \fbox{\stepcounter{sennum}\arabic{sennum}} Monks, who are the retarding ones? Monks, [some] gods and humans [are] those who are delighted, pleased, [and] satisfied in existence. When a teaching for the cessation of existence is being explained, their mind does not jump into it, not be pleased, not be settled, [and] not incline towards it. Monks, this is the retarding ones. \fbox{\stepcounter{sennum}\arabic{sennum}} Monks, who are the overrunning ones? Monks, [some] gods and humans [are] those who, being distressed, depressed, disgusted only by existence, rejoice at non-existence, [thinking that] ``It is said, friend, from which this self gets annihilated [and] destroyed because of the disintegration of the body, [from that] there is no afterlife. That is peaceful. That is excellent. That is certain. Monks, this is the overrunning ones. \fbox{\stepcounter{sennum}\arabic{sennum}} Monks, who are those having eyes see? In this [religion], monks, a monk sees the five aggregates as they realy are. Having seen the five aggregates as they really are, [the monk] becomes the one who enter the course for the aversion, dispassionateness, [and] cessation of the five aggregates. Monks, this is those having eyes see.\\

\begin{longtable}[c]{|p{0.9\linewidth}|}
\hline
\hspace{5mm}\small In \fbox{\ref{sen:atidhaavanti}}, I insert \pali{tato} to make \pali{ya-ta} structure more visible. In the canon we have, it is split \pali{para\d m mara\d n\=a} instead. This can also be translated as `further death.' \\
\hspace{5mm}\small In the last part, the commentary explains \pali{bh\=uta} as ``\pali{Bh\=utan'ti khandhapa\~ncaka\d m}.''\footnote{It-a\,49} So, I replace \pali{bh\=uta} with the five aggregates. The generic meaning of the term is so vague that it can mean anything that has a certain state of being.\\
\hline
\end{longtable}

\pali{\fbox{\stepcounter{sennum}\arabic{sennum}} 675. Tasm\=a yath\=a d\=aruyanta\d m su\~n\~na\d m nijj\=iva\d m nir\=ihaka\d m, atha ca pana d\=arurajjukasam\=ayogavasena gacchatipi ti\d t\d thatipi. Sa\-\=ihaka\d m saby\=ap\=ara\d m viya kh\=ayati, \fbox{\stepcounter{sennum}\arabic{sennum}} evamida\d m n\=amar\=upampi su\~n\~na\d m nijj\=iva\d m nir\=ihaka\d m, atha ca pana a\~n\~nama\~n\~nasam\=ayogavasena gacchatipi ti\d t\d thatipi. Sa\=ihaka\d m saby\=ap\=ara\d m viya kh\=ayat\=i'ti da\d t\d thabba\d m. \fbox{\stepcounter{sennum}\arabic{sennum}} Ten\=ahu por\=a\d n\=a --}

\addtocounter{sennum}{-3}
$\triangleright$ \fbox{\stepcounter{sennum}\arabic{sennum}} 675. Therefore, by which manner a wooden puppet [is] empty, lifeless, [and] motionless, but by the united composition of wood and rope [it can] walk and stand. It appears as if [it is] movable [and] workable. \fbox{\stepcounter{sennum}\arabic{sennum}} By such a manner, name-and-form [is] empty, lifeless, motionless, but by the united composition mutually [between name and form] [it can] walk and stand. It is should be seen thus, it appears as if [it is] movable [and] workable. \fbox{\stepcounter{sennum}\arabic{sennum}} By that, said ancient [sages]:\\

\pali{\fbox{\stepcounter{sennum}\arabic{sennum}} \\
N\=ama\~nca r\=upa\~nca idhatthi saccato,\\
Na hettha satto manujo ca vijjati;\\
Su\~n\~na\d m ida\d m yantamiv\=abhisa\.nkhata\d m,\\
Dukkhassa pu\~njo ti\d naka\d t\d thas\=adiso'ti.}

\addtocounter{sennum}{-1}
$\triangleright$ \fbox{\stepcounter{sennum}\arabic{sennum}} \\
In reality, name and form exist in this [world],\\
Also, no being and human being is found here;\\
It [is] empty, made up like a machine,\\
{[It is]} a pile of suffering, like [a pile of] grass or wood, etc.\\

\begin{longtable}[c]{|p{0.9\linewidth}|}
\hline
\hspace{5mm}\small Normally, \pali{n\=amar\=upa} is used as a singular unit, because it is inseparable, doctrinally speaking. That is why \pali{atthi}\footnote{In fact, \pali{atthi} can be used both in singular and plural sense. In case of the latter case, it is counted as verbal particle.} and \pali{ida\d m} are used here.\\
\hspace{5mm}\small In the second line, I translate \pali{hi} as `also.' Mostly, I ignore this particle unless it can be put into a simple word.\\
\hline
\end{longtable}

\pali{\fbox{\stepcounter{sennum}\arabic{sennum}} Na kevala\~nceta\d m d\=aruyantupam\=aya, a\~n\~n\=ahipi na\d lakal\=ap\=i\=ad\-\=ihi upam\=ahi vibh\=avetabba\d m -- \fbox{\stepcounter{sennum}\arabic{sennum}} yath\=a hi dv\=isu na\d lakal\=ap\=isu a\~n\~nama\~n\~na\d m niss\=aya \d thapit\=asu ek\=a ekiss\=a upatthambho hoti, ekiss\=a patam\=an\=aya itar\=api patati, evameva\d m pa\~ncavok\=arabhave n\=amar\=upa\d m a\~n\~nama\~n\~na\d m niss\=aya pavattati, eka\d m ekassa upatthambho hoti. Mara\d navasena ekasmi\d m patam\=ane itarampi patati. \fbox{\stepcounter{sennum}\arabic{sennum}} Ten\=ahu por\=a\d n\=a --}

\addtocounter{sennum}{-3}
$\triangleright$ \fbox{\stepcounter{sennum}\arabic{sennum}} Not only by that wooden-puppet metaphor, [but name-and-form] also should be explained by other metaphors, [like] a bundle of reeds, for example. \fbox{\stepcounter{sennum}\arabic{sennum}} Just like when two bundles of reeds mutually upheld each other, one is another's support. When one [bundle] is falling, the other also falls. By that manner, name-and-form in the five constituents of existence exists by mutual support. One is another's support. When one is falling by death, the other also falls. \fbox{\stepcounter{sennum}\arabic{sennum}} By that, said ancient [sages]:\\

\pali{\fbox{\stepcounter{sennum}\arabic{sennum}} \\
Yamaka\d m n\=amar\=upa\~nca, ubho a\~n\~no\~n\~nanissit\=a;\\
Ekasmi\d m bhijjam\=anasmi\d m, ubho bhijjanti paccay\=a'ti.}

\addtocounter{sennum}{-1}
$\triangleright$ \fbox{\stepcounter{sennum}\arabic{sennum}} \\
Name and form [exists] in a pair, both depends on each other;\\
When one is breaking, the both supports are destroyed.\\

\pali{\fbox{\stepcounter{sennum}\arabic{sennum}} 676. Yath\=a ca da\d n\d d\=abhihata\d m bheri\d m niss\=aya sadde pavattam\=ane a\~n\~n\=a bher\=i, a\~n\~no saddo, bherisadd\=a asammiss\=a, bher\=i saddena su\~n\~n\=a, saddo bheriy\=a su\~n\~no, evameva\d m vatthudv\=ar\=aramma\d nasa\.nkh\=ata\d m r\=upa\d m niss\=aya n\=ame pavattam\=ane a\~n\~na\d m r\=upa\d m, a\~n\~na\d m n\=ama\d m, n\=amar\=up\=a asammiss\=a, n\=ama\d m r\=upena su\~n\-\~na\d m, r\=upa\d m n\=amena su\~n\~na\d m, apica kho bheri\d m pa\d ticca saddo viya r\=upa\d m pa\d ticca n\=ama\d m pavattati. \fbox{\stepcounter{sennum}\arabic{sennum}} Ten\=ahu por\=a\d n\=a --}

\addtocounter{sennum}{-2}
$\triangleright$ \fbox{\stepcounter{sennum}\arabic{sennum}} 676. In addition, just like when sound is being produced by a drum struck with a stick, the drum is one [thing], the sound is another, drum and sound [are] not mixed. The drum [is] empty of the sound. The sound [is] empty of the drum. By that manner, when name is existing by form---namely the base, the entry point, and the sense-object---form is one [thing], name is another, name and form [are] not mixed. Name [is] empty of form. Form [is] empty of name. However, like sound depends on a drum, depending on form, name exists. \fbox{\stepcounter{sennum}\arabic{sennum}} By that, said ancient [sages]:\\

\begin{longtable}[c]{|p{0.9\linewidth}|}
\hline
\hspace{5mm}\small We find a technical compound here, \pali{vatthudv\=ar\=aramma\d nasa\.nkh\=ata\d m} (\pali{vatthu + dv\=ara + \=aramma\d na + sa\.nkh\=ata}). In this context, \pali{vatthu} (object, field) does not mean object of perception, but the physical base of perception.\\
\hspace{5mm}\small I put \pali{dv\=ara} (door) as `the entry point.' And it is \pali{\=aramma\d na} that means the object of perception, hence `the sense-object.' At the end, \pali{sa\.nkh\=ata} just means like `so-called' or `namely.' This term is found only in compounds, and do not be confused with \pali{sa\.nkhata} (conditioned).\\
\hline
\end{longtable}

\pali{\fbox{\stepcounter{sennum}\arabic{sennum}} \\
Na cakkhuto j\=ayare phassapa\~ncam\=a,\\
Na r\=upato no ca ubhinnamantar\=a;\\
Hetu\d m pa\d ticcappabhavanti sa\.nkhat\=a,\\
Yath\=api saddo paha\d t\=aya bheriy\=a.}\\[1.5mm]
\pali{Na sotato j\=ayare phassapa\~ncam\=a,\\
Na saddato no ca ubhinnamantar\=a\ldots\\
Na gh\=anato j\=ayare phassapa\~ncam\=a,\\
Na gandhato no ca ubhinnamantar\=a\ldots\\
Na jivh\=ato j\=ayare phassapa\~ncam\=a,\\
Na rasato no ca ubhinnamantar\=a\ldots\\
Na k\=ayato j\=ayare phassapa\~ncam\=a,\\
Na phassato no ca ubhinnamantar\=a\ldots}\\[1.5mm]
\pali{Na vatthur\=up\=a pabhavanti sa\.nkhat\=a,\\
Na c\=api dhamm\=ayatanehi niggat\=a;\\
Hetu\d m pa\d ticcappabhavanti sa\.nkhat\=a,\\
Yath\=api saddo paha\d t\=aya bheriy\=a'ti.}\\

\addtocounter{sennum}{-1}
$\triangleright$ \fbox{\stepcounter{sennum}\arabic{sennum}} \\
The pentad of sense-impression does not arise from eyes,\\
Neither from shape, nor from between the two;\\
They originate from a cause, being conditioned,\\
Just like sound [originates] from a drum being hit.\\[1.5mm]
The pentad of sense-impression does not arise from ears,\\
Neither from sound, nor from between the two; \ldots\\
The pentad of sense-impression does not arise from nose,\\
Neither from scent, nor from between the two; \ldots\\
The pentad of sense-impression does not arise from tongue,\\
Neither from taste, nor from between the two; \ldots\\
The pentad of sense-impression does not arise from the body,\\
Neither from touch, nor from between the two; \ldots\\[1.5mm]
The conditioned [things] do not originate from material form,\\
Neither also [they] came from the mental objects;\\
They originate from a cause, being conditioned,\\
Just like sound [originates] from a drum being hit.\\[1.5mm]

\begin{longtable}[c]{|p{0.9\linewidth}|}
\hline
\hspace{5mm}\small Here is a short description of the pentad of sense-impression (\pali{phassapa\~ncamaka}) from Ven.\,Nyanatiloka: ``[\pali{Phassa} is] the first factor in the pentad of sense-impression (\pali{phassa-pa\~ncamaka}), together with feeling, perception, volition and consciousness.''\footnote{\citealp[p.~142]{nyanatiloka:dict}. The explanation appears in the entry of \pali{phassa}.} It means, so to speak, the four mental aggregates plus contact. In the verse, \pali{phassapa\~ncam\=a} is plural, so it refers to each entity separately. I use it as singular collectively in the first line, but in the third line I turn back to the plural sense.\\
\hline
\end{longtable}

\pali{\fbox{\stepcounter{sennum}\arabic{sennum}} 677. Apicettha n\=ama\d m nitteja\d m na sakena tejena pavattitu\d m sakkoti, na kh\=adati, na pivati, na by\=aharati, na iriy\=apatha\d m kappeti. R\=upampi nitteja\d m na sakena tejena pavattitu\d m sakkoti. Na hi tass\=a kh\=adituk\=amat\=a, n\=api pivituk\=amat\=a, na by\=aharituk\=amat\=a, na iriy\=apatha\d m kappetuk\=amat\=a, atha kho n\=ama\d m niss\=aya r\=upa\d m pavattati, r\=upa\d m niss\=aya n\=ama\d m pavattati, n\=amassa kh\=adituk\=amat\=aya pivituk\=amat\=aya by\=aharituk\=amat\=aya iriy\=apatha\d m kappetuk\=amat\=aya sati r\=upa\d m kh\=adati, pivati, by\=aharati, iriy\=apatha\d m kappeti.}

\addtocounter{sennum}{-1}
$\triangleright$ \fbox{\stepcounter{sennum}\arabic{sennum}} Furthermore, name [is] powerless. [It] cannot move by its own power, nor eat, nor drink, nor talk, nor make a posture. Form [is] also powerless. [It] cannot move by its own power. [It] has no desire to eat, no desire to drink, no desire to talk, and no desire to make a posture. Then depending on name, form moves; depending on form, name moves. [Only when] name has a desire to eat, to drink, to talk, to make a posture, form eats, drinks, talks, and makes a posture.\\

\begin{longtable}[c]{|p{0.9\linewidth}|}
\hline
\hspace{5mm}\small Difficult to put in English, \pali{pavattati} can generally mean `to exist,' like \pali{atthi} or \pali{hoti}. But it is not just an acquirement of certain state of being. It also includes the state of maintaining the existence.\\
\hline
\end{longtable}

\pali{\fbox{\stepcounter{sennum}\arabic{sennum}} Imassa panatthassa vibh\=avanatth\=aya ima\d m upama\d m ud\=aharanti -- \fbox{\refstepcounter{sennum}\arabic{sennum}\label{sen:jaccandho}} yath\=a jaccandho ca p\=i\d thasapp\=i ca dis\=apakkamituk\=am\=a assu, \fbox{\stepcounter{sennum}\arabic{sennum}} jaccandho p\=i\d thasappi\d m evam\=aha `aha\d m kho bha\d ne, sakkomi p\=adehi p\=adakara\d n\=iya\d m k\=atu\d m, natthi ca me cakkh\=uni yehi samavisama\d m passeyyan'ti. \ \fbox{\stepcounter{sennum}\arabic{sennum}} P\=i\d thasapp\=ipi jaccandha\d m evam\=aha `aha\d m kho bha\d ne, sakkomi cakkhun\=a cakkhukara\d n\=iya\d m k\=atu\d m, natthi ca me p\=ad\=ani yehi abhikkameyya\d m v\=a pa\d tikkameyya\d m v\=a'ti. \fbox{\stepcounter{sennum}\arabic{sennum}} So tu\d t\d thaha\d t\d tho jaccandho p\=i\d thasappi\d m a\d msak\=u\d ta\d m \=aropesi. P\=i\d thasapp\=i jaccandhassa a\d msak\=u\d te nis\=iditv\=a evam\=aha `v\=ama\d m mu\~nca dakkhi\d na\d m ga\d nha, dakkhi\d na\d m mu\~nca v\=ama\d m ga\d n\-h\=a'ti. \fbox{\stepcounter{sennum}\arabic{sennum}} Tattha jaccandhopi nittejo dubbalo na sakena tejena sakena balena gacchati, p\=i\d thasapp\=ipi nittejo dubbalo na sakena tejena sakena balena gacchati, na ca tesa\d m a\~n\~nama\~n\~na\d m niss\=aya gamana\d m nappavattati, \fbox{\refstepcounter{sennum}\arabic{sennum}\label{sen:naamampi}} evameva\d m n\=amampi nitteja\d m na sakena tejena uppajjati, na t\=asu t\=asu kiriy\=asu pavattati. R\=upampi nitteja\d m na sakena tejena uppajjati, na t\=asu t\=asu kiriy\=asu pavattati, na ca tesa\d m a\~n\~nama\~n\~na\d m niss\=aya uppatti v\=a pavatti v\=a na hoti. \fbox{\stepcounter{sennum}\arabic{sennum}} Teneta\d m vuccati --}

\addtocounter{sennum}{-8}
$\triangleright$ \fbox{\stepcounter{sennum}\arabic{sennum}} Moreover, to make this explanation clear, [teachers] articulate this metaphor. \fbox{\stepcounter{sennum}\arabic{sennum}} Just like there are a blind person and a cripple, [and] they may have a desire to go to [certain] directions. \fbox{\stepcounter{sennum}\arabic{sennum}} The blind one said to the cripple thus, ``My friend, I can do pedestrian work by feet, but I have no eyes by which [I] should see the [path] even or uneven.'' \fbox{\stepcounter{sennum}\arabic{sennum}} The cripple said to the blind one thus, ``My friend, I can do visual work by eyes, but I have no feet by which I should go forward or backward.'' \fbox{\stepcounter{sennum}\arabic{sennum}} That blind person, satisfied and joyful, put the cripple on the shoulder. Sitting on the blind person's shoulder, the cripple said thus, ``Avoid the left, take the right; Avoid the right, take the left.'' \fbox{\stepcounter{sennum}\arabic{sennum}} Here, even though the blind person, powerless [and] feeble, does not go by his own power [and] strength, [and] the cripple, powerless [and] feeble, does not go by his own power [and] strength, but their going is not hindered by mutual support. \fbox{\stepcounter{sennum}\arabic{sennum}} By this manner, even though powerless name does not arise by its own power, does not proceed such and such actions; even though powerless form does not arise by its own power, does not proceed such and such actions; but their arising and proceeding happen by mutual support. \fbox{\stepcounter{sennum}\arabic{sennum}} By that, this is said:\\

\newpage
\begin{longtable}[c]{|p{0.9\linewidth}|}
\hline
\hspace{5mm}\small In \fbox{\ref{sen:jaccandho}}, we have to split this part into two sentences: ``\pali{jaccandho ca p\=i\d thasapp\=i ca [honti], [tesa\d m] dis\=apakkamituk\=am\=a assu}.'' As plural optative verb, \pali{assu} indicates a speculation.\\
\hspace{5mm}\small In \fbox{\ref{sen:naamampi}}, the last part has a double negative, but I turn it to positive meaning.\\
\hline
\end{longtable}

\pali{\fbox{\stepcounter{sennum}\arabic{sennum}} \\
Na sakena balena j\=ayare,\\
Nopi sakena balena ti\d t\d thare;\\
Paradhammavas\=anuvattino,\\
J\=ayare sa\.nkhat\=a attadubbal\=a.\\[1.5mm]
Parapaccayato ca j\=ayare,\\
Para\=aramma\d nato samu\d t\d thit\=a;\\
\=Aramma\d napaccayehi ca,\\
Paradhammehi cime pabh\=avit\=a.}\\[1.5mm]

\addtocounter{sennum}{-1}
$\triangleright$ \fbox{\stepcounter{sennum}\arabic{sennum}} \\
{[They are]} not born by [their] own strength,\\
Also [they do] not stand by [their] own strength;\\
{[They are]} obedient to other factor,\\
Conditioned things are born strengthless by themselves.\\[1.5mm]
{[They are]} born from other cause,\\
Having arisen from other sense-object;\\
Also from sense-object as causes,\\
And from other factors, these [conditioned things] have arisen.\\

\pali{\fbox{\stepcounter{sennum}\arabic{sennum}} \\
Yath\=api n\=ava\d m niss\=aya, manuss\=a yanti a\d n\d nave;\\
Evameva r\=upa\d m niss\=aya, n\=amak\=ayo pavattati.\\
Yath\=a ca manusse niss\=aya, n\=av\=a gacchati a\d n\d nave;\\
Evameva n\=ama\d m niss\=aya, r\=upak\=ayo pavattati.\\
Ubho niss\=aya gacchanti, manuss\=a n\=av\=a ca a\d n\d nave;\\
Eva\d m n\=ama\~nca r\=upa\~nca, ubho a\~n\~no\~n\~nanissit\=a'ti.}

\addtocounter{sennum}{-1}
$\triangleright$ \fbox{\stepcounter{sennum}\arabic{sennum}} \\
Human beings go into the ocean by support of a ship;\\
Likewise, the heap of name proceeds by support of form.\\
A ship goes into the ocean by support of human beings;\\
Likewise, the heap of form proceeds by support of name.\\
Human beings and ship go into the ocean by support of each other;\\
Likewise, name and form [are] mutually dependent on each other.\\

\pali{\fbox{\stepcounter{sennum}\arabic{sennum}} Eva\d m n\=an\=anayehi n\=amar\=upa\d m vavatth\=apayato [ya\d m dassana\d m] sattasa\~n\~na\d m abhibhavitv\=a asammohabh\=umiya\d m \d thita\d m n\=amar\=up\=ana\d m y\=ath\=avadassana\d m [hutv\=a] [ta\d m dassana\d m] di\d t\d thivisuddh\=i'ti veditabba\d m. \fbox{\stepcounter{sennum}\arabic{sennum}} N\=amar\=upavavatth\=anan'tipi sa\.nkh\=araparicchedo'tipi etasseva adhivacana\d m.}

\addtocounter{sennum}{-2}
$\triangleright$ \fbox{\stepcounter{sennum}\arabic{sennum}} From the analysis of name-and-form in various ways as such, [which view], having overcome the (mis)conception of being, having stayed on a confusion-free ground, seeing name-and-form as they really are, [that view] should be known as `the purity of view.' \fbox{\stepcounter{sennum}\arabic{sennum}} Also `the analysis of name-and-form' and `the decomposition of conditioned things' are the terms for it.\\

\begin{longtable}[c]{|p{0.9\linewidth}|}
\hline
\hspace{5mm}\small The first part is a bit tricky. The subject is not a person, but the view when one understood the explanation so far. I insert \pali{ya-ta} markers to make the sentence more discernable.\\
\hspace{5mm}\small I use \pali{dassana\d m} (nt.) as the omitted noun to make gender agreeable. To be more complete, we can insert \pali{hoti} or \pali{hutv\=a} after \pali{y\=ath\=avadassana\d m}, because there is no verb for this word. If we use \pali{hoti}, we make two correlated sentences. Whereas using \pali{hutv\=a}, we treat it as one big sentence.\\
\hline
\end{longtable}

\pali{\fbox{\stepcounter{sennum}\arabic{sennum}} \\
Iti s\=adhujanap\=amojjatth\=aya kate visuddhimagge\\
Pa\~n\~n\=abh\=avan\=adhik\=are\\
Di\d t\d thivisuddhiniddeso n\=ama\\
A\d t\d th\=arasamo paricchedo.}\\

\addtocounter{sennum}{-1}
$\triangleright$ \fbox{\stepcounter{sennum}\arabic{sennum}} \\
Thus in Visuddimagga done for the joy of faithful people,\\
In the guideline of wisdom development,\\
Called the analytic explanation of the purity of view,\\
The eighteenth chapter.

\phantomsection
\addcontentsline{toc}{section}{Conclusion and discussion}
\section*{3.\ Conclusion and discussion}

As we have read so far, it might be too little that the writing style of Buddhaghasa can be discerned by new learners. You have to read more than this, including what he wrote in the commentaries. As far as I can tell, the sentence formation in Visuddhimagga, also in the commentaries, is more sophisticated than that in the canon. In some parts, Visuddhimagga can be relatively difficult to read, particularly the highly technical ones.

The excerpt we have read here shows that using metaphors in explanations is fashionable and quite effective. We can grasp the picture quickly. A famous one is the simile of chariot, originally found in Sa\d myuttanik\=aya\footnote{In Vajir\=asutta, S1\,171 (SN\,5), there is ``\pali{Yath\=a hi a\.ngasambh\=ar\=a, hoti saddo ratho iti.}''}, and elaborated later in Milindapa\~nh\=a\footnote{Mil\,2-3.1.1 (Pa\~n\~nattipa\~nho)}.

After reading this, those who know Western philosophy may ask ``Does this show that Buddhism is mind-body dualism or else?'' It is rather clear that name and form are treated as different things.\footnote{The difference is even conspicuous in the Abhidhamma.} But the marked point here is they are powerless if they do not interact with each other. They are mutually dependent and inseparable. So, as far as we can see, dualism seems to be a wrong word here, at least in Western philosophical sense. They are different by their quality and function, but inseparable by their existence. Asking whether this conception is dualism or not is therefore misleading.\footnote{Dualism is not suitable term for Buddhism because dualism implies substances and Buddhists do not like substances. Perhaps, classifying the Buddhist view on mind-body problem as \emph{functionalism} is more appropriate. What mind and body are made of is not really the issue, but rather how they work together. Some Buddhists still resist a comparative classification between Western and Buddhist ideas, like I try to do here. They might feel it is irrelevant or incommensurable. My view is different. If we cannot tell the difference or similarity between two systems, we do not understand them enough. Comparing Buddhist ideas with other systems therefore should make us understand the religion better.}

Still, many Buddhists believe that mind and body can be separated, obviously when someone dies. Some teachers reconcile the idea by proposing a kind of subtle body that we acquire immediately after death. I do not want to talk much in detail about this, because it leads us to a metaphysical speculation. One thing Therav\=ada Buddhists should know is that by orthodox position mind and body cannot be separated, as shown by what we have read.

Another point worth a note is, to Buddhaghasa, name includes what we call energy today. But energy is categorized as material form in modern science. In the account of name-and-form, form has no desire or power to move or to eat by its own. This implies that even a virus has mental components. And a person in a coma or a persistent vegetative state has certain mental activities. That sounds odd to us. The tradition seems to be aware of the problem, particularly when thinking about the state of deep sleep, and a kind of unconscious mind (\pali{bhava\.nga}) is suggested.

I think the very point Buddhaghosa tried to stress here is the interdependence of things, the underlying concept of dependent origination. The main message is ``Nothing can happen by its own nature. Everything needs other things else.'' The analysis of name and form is useful as long as it benefits someway to practical matters, as we find in Vipassan\=a meditation today. Overanalysis of this can end up with a mere metaphysical talk.
