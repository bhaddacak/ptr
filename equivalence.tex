\chapter{On equivalence}\label{chap:equivalence}

When we think about translation, basically we mean mapping the meaning from source language to target language. This concept of equivalence is central to the problem of translation.\footnote{The basic story is that all the theories respond in different ways to one central problem: translation can be defined by equivalence \citep[p.~xiii]{pym:exploring}.} Although equating equivalence to translation is less fashionable nowadays, the concept is still important to the field. So, we inevitably have to deal with it.

Generally speaking in theoretical space, there are two poles relating to equivalence: linguistic relativity/determinism and linguistic universalism.

On the first view, widely known as Sapir-Whorf hypothesis, Edward Sapir asserts that language is a guide to social reality:

\begin{quote}
The fact of the matter is that the `real world' is to a large extent unconsciously built up on the language habits of the group. No two languages are ever sufficiently similar to be considered as representing the same social reality. The worlds in which different societies live are distinct worlds, not merely the same world with different labels attached.\footnote{\citealp[p.~69]{sapir:culture}}
\end{quote}

By this view, different languages constitute different worlds. We cannot map reality presented in one language into another language perfectly. This simply means exact translation is untenable.\footnote{As put by Roman Jakobson, ``there is ordinarily no full equivalence between code-units'' \citep[p.~127]{jakobson:translation}. This is not implied that interlingual rendition is impossible. Finding perfectly matched substitution word-for-word may cannot be done, but we can somehow map one concept to a combination of concepts (see below shortly).} Because of its strong relativistic flavor, this position is rarely accepted nowadays.\footnote{A good reading on linguistic determinism is \citealp[pp.~124--51]{pinker:stuff}.}

On the other hand, linguistic universalism asserts that although languages differ in form but there is a way of thinking or experiencing the world shared by all human beings. Therefore, it is not necessary to map two languages word-for-word to obtain the meaning. We can more or less map a word to a phrase or sentence to make the intended meaning accessible. As put by Jakobson, ``All cognitive experience and its classification is conveyable in any existing language.''\footnote{\citealp[p.~128]{jakobson:translation}} Translation in this case looks more like explanation or commentation. For Jakobson, only poetry is regarded as untranslatable and it needs creative transposition instead.\footnote{\citealp[p.~131]{jakobson:translation}}

Seeing equivalence as a hallmark of translation, Anothony Pym divides equivalence into two types: natural and directional.\footnote{\citealp{pym:exploring}} Natural equivalence denotes reciprocality of signs not depending on languages and existing prior to translation. An example suggested by Pym is road signs, `Stop' for instance. We can render this sign in many ways. All mean you have to do this `Stop' thing.

On the other hand, directional equivalence does not guarantee such reciprocality. When you render a word to TL, it is difficult, if not impossible, to back-translate to yield the same term in SL. For example, you can translate \pali{nibb\=ana} as `cessation,' but translating `cessation' back to P\=ali does not guarantee that \pali{nibb\=ana} is the right word in that context.

Concerning the problem of equivalence, there are many other things to discuss in theoretical space, because the concept of equivalence itself is multifarious. Terms, or texts at large, can be equivalent in many ways, denotatively, connotatively, performatively, normatively, or aesthetically, etc.

For practical purpose, however, we can say that perfect equivalence is an untenable concept. So, the very problem is shifted to ``What is the best way to achieve suitable equivalence?'' Hence, translatability becomes a matter of degree and adequacy suitable to the context.\footnote{\citealp[See][p.~61]{munday:translation}.} Susan Bassnett puts it in this way:

\begin{quote}
Translation involves far more than replacement of lexical and grammatical items between languages and, as can be seen in the translation of idioms and metaphors, the process may involve discarding the basic linguistic elements of the SL text \dots\ But once the translator moves away from close linguistic equivalence, the problems of determining the exact nature of the level of equivalence aimed for begin to emerge.\footnote{\citealp[p.~34]{bassnett:translation}}
\end{quote}

To conclude, when equivalence is seen as the very goal of translation, we face a difficulty of non-mechanical nature of translation. Translation is not an engineering project that you can calculate in advance and get the result exactly as you expect.\footnote{Scientific approaches to translation do exist in the field. For instance, in Eugene Nida's \textit{Toward a Science of translating} \citealp{nida:toward}, Noam Chomsky's syntactic structure is incorporated into his framework of Bible translation. Other attempts can be seen in polysystem theory of Itamar Even-Zohar and descriptive translation studies of Gideon Toury, for example. All of these are not `science' with airtight objectivity. They just look more systematic and technical. So, I do not bring these into our discussion. See \citealp[p.~169--96]{munday:translation} for more information.} We have to know our allowable leeway when we make certain decision among choices. That is to say, translation cannot go totally objective or impartial. There must be a subjective or ideological motive in decision making of the translator. This brings us again the fact that any translation is vulnerable to manipulation. Hence, translation ethics is another topic that we have to take into consideration.
