\chapter{Principle of P\=ali text reading}\label{chap:principle}

From the theoretical foundations we have discussed so far, now it is the time to wrap up and apply them to P\=ali things. I will not mainly aim to translation but rather on reading, the state before any refined translation is produced. However, there is no clear demarcation between reading and translation when P\=ali is concerned. Reading a P\=ali text involves certain translation inevitably. To get a clearer picture, let us see the translation process in Figure \ref{fig:process}.\footnote{In \citealp[p.~34]{nord:analysis}, this is called the \emph{two-phase model} of translation process. There are other models mentioned in the book, such as the three-phase model, and the looping model. To lessen the technical burden, I show only the simplest one here.}

\begin{figure}[!hbt]
\centering
\setlength{\unitlength}{1mm}
\begin{picture}(80,30)(0,0)
\thinlines
%% paper
\thicklines
\put(0,16){\line(1,0){6}}
\put(0,16){\line(0,1){8}}
\put(0,24){\line(1,0){6}}
\put(6,16){\line(0,1){8}}
\thinlines
\put(1,18){\line(1,0){4}}
\put(1,19){\line(1,0){4}}
\put(1,20){\line(1,0){4}}
\put(1,21){\line(1,0){4}}
\put(1,22){\line(1,0){4}}
\put(3,13){\makebox(0,0)[c]{ST}}
%
\put(6,20){\vector(1,0){4}}
\put(10,15){\framebox(25,10)[c]{\textbf{Analysis}}}
\put(10,11){\parbox[t]{35mm}{\small\raggedright (decoding or comprehension phase)}}
\put(35,20){\vector(1,0){10}}
\put(45,15){\framebox(25,10)[c]{\textbf{Synthesis}}}
\put(45,11){\parbox[t]{35mm}{\small\raggedright (encoding, reconstruction or reverbalization phase)}}
\put(70,20){\vector(1,0){4}}
%% paper
\thicklines
\put(74,16){\line(1,0){6}}
\put(74,16){\line(0,1){8}}
\put(74,24){\line(1,0){6}}
\put(80,16){\line(0,1){8}}
\thinlines
\put(75,18){\line(1,0){4}}
\put(75,19){\line(1,0){4}}
\put(75,20){\line(1,0){4}}
\put(75,21){\line(1,0){4}}
\put(75,22){\line(1,0){4}}
\put(77,13){\makebox(0,0)[c]{TT}}
%
\end{picture}
\caption{The translation process}
\label{fig:process}
\end{figure}

As shown in the figure, what I mean by `reading' covers mainly the analysis phase. So, the intended end result is comprehension or understanding of the text. I mostly leave the synthesis phase to the learners. You have to find out by yourselves how to produce a `good' translation. Even though a rough synthesis into English has to be done inevitably, we will do it here only for understanding, not for creating an elegant, or even a readable translation.
 
When we read a text, we get certain understanding, perhaps more than one interpretation. Mapping `reading' to `interpretation,' I would like to stress that a text can be read in several ways resulting in several readings or interpretations. One reading/interpretation can produce yet several (concrete) translations. We will talk less about how a translation should be produced, but rather more about how possibly we can read a text to make certain understanding, and as a consequence, how to manipulate the meaning to gain the intended interpretation.\footnote{Manipulation of meaning sounds a little pessimistic and Machiavellian, but everyone does this by not thinking or saying as such (invisibility of the translator in effect). It should to be repeated again that manipulation can yield a good result, but manipulation nonetheless.}

As far as I know, there is no principle of P\=ali text reading laid down by any scholar. There is no need for principle, because in common practice translators just do a translation and that is all. It is the time to think differently about the matter. That is why I spent a lot of space to lay a strong foundation to make our reading practice more effective. Here is my attempt to systematize the reading by using well-established knowledge, not just the knowledge of the language itself. The principle may look idiosyncratic and very unfamiliar to traditional students. But if the learners are willing to invest the effort to learn new things, their understanding of P\=ali and Buddhism at large will be liberated and healthy.

Before we go further, from what we have learned so far, we can summarize underlying assumptions concerning text reading as follows:

\begin{compactenum}[1.]
\item Understanding is private to us. To make others understand likewise, we have to convert mental content into signs regulated by certain code. One understanding can be materialized into various textual forms.
\item To the text readers, reproduction of understanding does not always yield the original, for a number of reasons, i.e.\ degree of ambiguity of textual formation, the ever changing code used in reading, psychological states and cultural preferences of the readers, manipulation of meaning, and so on. The more contexts of the source and the readers are distant in space and time, the more understanding reproduced is different.
\item By the fact that original intention is difficult to determine, the reading can be worthwhile only when the benefit is on the readers' side. Although manipulated reading cannot be avoided, being aware of its influence makes us less to be exploited. So, a healthy reading should incorporate \emph{critical discourse analysis}.
\end{compactenum}

\phantomsection
\addcontentsline{toc}{section}{A guideline to P\=ali text reading}
\section*{A guideline to P\=ali text reading}

Now I will show you a simple guideline for P\=ali text reading step-by-step as follows:

\paragraph*{Step 1: Analyzing extratextual factors} First we should know all the things about text environment as described in Chapter \ref{chap:analysis}. It is no need to do an extensive research on a particular factor to gain the most accurate answer. But you should know the text well from various perspectives. In the first round, you do it roughly and try to provide several possible answers. For some factors you cannot know for sure, just mark them as uncertain and try to fulfill these in the next rounds when you gain some information from intratextual analysis. A quick checklist can be as the following:

\begin{compactenum}[(1)]
\item Who is the sender? Is the sender the text producer?
\item What is (are) the possible sender's intention(s)? What is our intention in this reading or translation?
\item Who is the possible ST audience? Who is our target TT audience?
\item What is the text medium, oral or writing or something else?
\item What is the possible place of ST? How far is that place from the target audience?
\item What is the possible time of ST? How far is that time from the target audience?
\item What could be the reason(s) behind ST production?
\item What is the possible ST function, informative, expressive, operative, ceremonial, etc.?
\end{compactenum}

\paragraph*{Step 2: Reading} Once we know about text environment, then we start going through the text, roughly in the first round and more deliberately afterwards. No translation output is not yet produced. We just read it, take notes, and try to understand it in various possible choices of meaning. This step can take a considerable time. When finished, we should know what the text is all about.

\paragraph*{Step 3: Analyzing intratextual factors} This step can be done while we do Step 2, taking notes when some ideas come up. Or we can do after the rough reading is finished. A quick checklist of intratextual factors can be as the following:

\begin{compactenum}[(1)]
\item What is the subject matter? What is the text about?
\item What possibly does the text mean?
\item What are presuppositions the audience expects to know/accept in advance? Should extra explanation be added in TT?
\item How is the text organized?
\item What are the non-verbal elements, if any?\footnote{This includes suprasegmental features, e.g.\ punctuation, explained on page \pageref{par:suprasegment}.}
\item What are conspicuous, idiosyncratic word choices? What possibly do these imply?
\item What kind of sentence structure is used in the text mostly? What are the marked sentence forms used in the text?
\end{compactenum}

\paragraph*{Step 4: Doing (critical) discourse analysis} After we get what possibly the text means, we then have to go deeper by putting three keywords, mentioned in Chapter \ref{chap:discourse}, into perspective: (a) what kind of system of knowledge or belief, (b) what kind of social identities, and (c) what kind of social relations is established by the text. The result may repeat some of former analyses, such as presupposition analysis. But this time we put more focus on ideological issues which affect the society at large.

\paragraph*{Step 5: Producing a translation} Now you know the text and need to record your understanding or present it to others. You have to choose among possible options you realize in analyzing process. One option makes one translation corresponding to the translator's intention. You can choose to follow the sender's intention or your own specific purpose, so to speak. Translation does not has one correct answer. It depends on what effect you want to make upon the target reader. Translation techniques mentioned in Chapter \ref{chap:ethics} and Chapter \ref{chap:techniques} can be used in this step to make your translation more effective.

I divide the end product of translation into to two types: draft and refined translation.

(1) Draft translation. When certain reading is obtained, we have to take notes for further use. This is suitably done by word-for-word translation. In our course, just this kind of translation is enough for text reading. Draft or rough translation is appropriate for textual study because it has low contamination. Hence, I also call this `clean' translation.

(2) Refined translation. To make the end product accessible to people, producing a readable outcome is inevitable, because no one will read our ugly reading notes. As we have discussed earlier, the translator can avoid responsibility of pushing certain ideas by making itself invisible. To be ethical and honest to the readers, we have to be judicious and make clear what we are doing, for what purpose, to what kind of readers. We may also use side text, like preface or footnotes, to express this concern. Ironically, refined translation is hard to make `clean.' It can spoil clear understanding. Hence, it is a perfect tool for manipulation.

\paragraph*{Step 6: Starting over again} After we finish the first round of reading, ending up in the first version of draft translation, we should know much more than when we have not read it. Upon this understanding, we run the process again from Step 1 to fulfill what we have left out or marked it as unclear. We should gain better understanding in the recurring reading, as well as a better translation. If we are not satisfied with the reading, do it again and again, until no more understanding can be obtained. Then we can finish the reading and enjoy the result.

\section*{Concluding remarks}

I will end the theoretical part of P\=ali text reading with this chapter. Even though I tried to water down the content, there are still a lot to learn. Only the theoretical part itself can be a book of its own. Despite the vast area of the theoretical ground, it does not meant to be applied in every detail. We have to choose what is applicable to our text, and in what degree it should be applied. For example, doing discourse analysis in every sentence, or analyzing every word meticulously can be too fussy and entail too much work. In practice, therefore, the process of reading can be tailored to suit the text and our purpose. In learning, however, we have to know all in the theoretical space. 

To my approach, the essential theories\footnote{All things put in Part I are counted as essential. So new learners should know all of them, even if some of them are hardly put into practice.} should be studied before a systematic P\=ali translation is undertaken. However, I have never seen this kind of prerequisite course in the traditional learning system. Even hermeneutics is hardly mentioned in a helpful way. In practice, traditional students are exposed to P\=ali texts immediately. They just try to tackle them as if a `right' translation is hidden somewhere. If certain problems occur, they just follow the commentaries and textbooks. If there is no suitable solution, students may improvise by their own judgement. But if the result is not in line with the traditional tenets, it is unlikely to be accepted. That is to say, traditional learning promotes conformity to the power structure rather than rationality.

With a strong background in theories described, I hope we will see P\=ali texts more critically. Understanding of texts should be (epistemic) understanding by itself, not just acceptance under the yoke of authority. I by no means imply that authoritative power is bad, but rather we should know that manipulation occurs all the time. We have to be aware of it and accept it only for a good, healthy outcome. Do not allow surreptitious, unhealthy manipulation goes unnoticed. And, more importantly, do not do it as such by yourselves because from now on such a practice will be caught by well-equipped P\=ali learners.

