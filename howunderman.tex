\chapter{How is understanding manipulated?}\label{chap:howunderman}

As we have seen that our understanding is far from objectively transferable, by its actively constructive nature, consequently understanding can be manipulated. I use `to manipulate' here in neutral sense. It means roughly `to control.' Manipulation can be done for a good result or a bad result, for one's own benefits or for the benefits of all.

Let me draw a simple picture first. Every culture has a highly valued text, in a form of narration or canon, for example. This text is regarded as `sacred' somehow. To read certain meaning out of that text is not a trivial matter. It has to be done deliberately. As we have seen previously, a text can produce a number of understandings according to the existing knowledge of the individuals. Having different understanding in crucial things in the society is not good. So, the understand has to be unified. Then comes the authority who determines what the text should be meant and what understanding should be rendered. That is textual manipulation in process.

A clear example of this is law. The interpretation of legal text is crucial and has to be unified, otherwise the society will fall apart. Law has highly manipulative nature, so to speak. What's about religious text? This kind of text is variegated. Some religions have highly manipulative text, e.g.\ Islam and Christianity (in some places and eras at least). Some have less manipulative text, such as Hinduism and Buddhism. In the later cases, a variety of interpretations is allowed, or tolerable. However, when certain position is asserted, a degree of manipulation can be read off nonetheless.

The readers may have this question in mind: ``Is there any sincere interpretation of text?'' If `sincere' means `without any hidden agenda' and `for the benefits of the many,' most interpretations are sincere. Very few downright evil religious interpretations have been found in our history. Yet, people can be sincerely wrong all the time. Therefore, sincerity does not guarantee truthfulness.

When you hold that your interpretation of some text is authentic and want to propagate the idea to your fellow adherents, is this a kind of manipulation? Religious people normally think what is good for them is good for all, and making others believe the `true' thing is meritorious. What do you think when you are happy in your own way and some sincerely say that you are not really happy and you have to adopt their belief to be `genuinely' happy? If you understand this situation, you get what I mean by manipulation.

Now let us discuss in detail how understanding is manipulated by signs. In Saussurean principle of semiotics, relation between representation (signifier) and concept (signified) is arbitrary. If `dog' had not been used for that shaggy thing, another word can do the same job if everyone agrees on that, and other languages have their own equivalent of `dog.' That is arbitrariness meant by Saussure. He describes it as `unmotivated'\footnote{\citealp[p.~69]{saussure:course}} because there is no intrinsic connection between signifier and signified.

The principle of arbitrariness has been rated as `illusory' by Roman Jakobson (1896--1982)\footnote{\citealp[p.~524]{jakobson:selected2}}, and more recently it is argued by Gunther Kress as follows:

\begin{quote}
The relation of signifier to signified, in all human semiotic systems, is always motivated, and is never arbitrary.\footnote{\citealp[p.~173]{kress:against}}
\end{quote}

In Kress's view, what make signs motivated are the sign producer's interest and characteristics of the object. See it another way, sign using in communication is far from neutral. In its good part, communication holds social stability and make transference of civilization possible, as we clearly see in educational systems. Also, communication ``can be used for manipulation, deceit, display of wit, seduction and maintenance of social relationships, all of which have fitness consequences.''\footnote{\citealp[p.~141]{origgisperber:evolution}} Co-operation is only one side of communication, but it is also vulnerable to free-riding, which communication becomes a tool for manipulation and deception.

\clearpage
\phantomsection
\addcontentsline{toc}{section}{Introduction to ideology and discourse}
\section*{Introduction to ideology and discourse}

The notion of `ideology' can elucidate our concern at this point. For the term is used in various ways, it is better to know the whole range of its meaning as listed by Terry Eagleton\footnote{\citealp[pp.~1--2]{eagleton:ideology}} as follows:

\begin{quote}
\begin{compactenum}[(a)]
\item the process of production of meanings, signs and values in social life
\item a body of ideas characteristic of a particular social group or class
\item ideas which help to legitimate a dominant political power
\item false ideas which help to legitimate a dominant political power
\item systematically distorted communication
\item that which offers a position for a subject
\item forms of thought motivated by social interests
\item identity thinking
\item socially necessary illusion
\item the conjuncture of discourse and power
\item \textit{the medium in which conscious social actors make sense of their world} \label{item:ideomedium}
\item action-oriented sets of beliefs
\item the confusion of linguistic and phenomenal reality
\item semiotic closure
\item the indispensable medium in which individuals live out their relations to a social structure
\item \textit{the process whereby social life is converted to a natural reality} \label{item:ideoprocess}
\end{compactenum}
\end{quote}

As we can see, ideology covers most of area of our life, individual and social. It is so common that ``all of our thinking might be said to be ideological.''\footnote{\citealp[p.~4]{eagleton:ideology}} And our common sense is substantially, though not entirely, ideological.\footnote{\citealp[p.~84]{fairclough:power}} Communication as a social process is also ideological in nature.\footnote{To understand this, \emph{interpellation} in Althusser's sense and \emph{hegemony} in Gramsci's sense have to be taken into account on the background of Marx's theory of ideology as \emph{false consciousness}. For a quick treatment, see \citealp[pp.~172--8]{fiske:communication}.}

To make our course simpler, from now on I will focus on only two meanings of ideology given above, i.e.\ (\ref{item:ideomedium}) and (\ref{item:ideoprocess}). The common theme of these two items I want to stress is ideology as the reality maker. We can only perceive the world as real through ideology, so to speak. This leads us to another term that I will use more often, i.e.\ \emph{discourse} used by Michel Foucault (1926--1984). I find this definition captures nicely what I want to say.

\begin{quote}
For Foucault, a discourse is a strongly bounded area of social knowledge, a system of statements within which the world can be known. The key feature of this is that the world is not simply `there' to be talked about; rather, it is through discourse itself that the world is brought into being.\footnote{\citealp[p.~83]{ashcroft:postcolonial}}
\end{quote}

From now on, when I use `discourse,' it can mean two things. First, in general parlance it means simply utterances that convey any kind of meaning. And second, it means, more specifically, utterances ``which have meaning, force and effect\footnote{These remind us to illocutionary \emph{force} and percutionary \emph{effect} as we have seen in speech act theory discussed in Chapter \ref{chap:howmisunder}. That is to say, discourse is performativity done at social level.} within a social context.''\footnote{\citealp[p.~11]{mills:discourse}} For example, when we say ``The Buddha delivered a discourse,'' by the first meaning it means the Buddha said something. And by the second meaning, it means a certain statement is made by a text, and the statement affects people's worldview in some way. Nowadays, particularly in social sciences, we rarely use `discourse' to means only just utterances. The second meaning is always implied. So, it is not really the textual form of speech or writing. Rather, discourse is statement that set up certain reality.

Roughly speaking, we can say that discourse is text read in a certain way. In general, text is neutral artifact. We can read it in many ways, good, bad, or ugly. A particular reading of text yields a particular understanding depending on interests and prejudices. A result of the reading is a discourse. When a text ``is realised in a knowable context,'' it is a discourse.\footnote{\citealp[p.~8]{greenlebihan:critical}} So, a text can produce many discourses. That is the very reason why from one religious text, many sects or denominations are established by different readings. Each sect holds different reality preconditioned by its inclination.

Another familiar discourse in our time is scientific discourse which is characterized by its empirical orientation. Reality established by scientific discourse is powerful and reliable, and, to my view, is the most impartial one. This by no means implies that science holds the absolute truth. For example, by exercising modern medical knowledge, a kind of scientific discourse, doctors have power to say whether who is sick or not. And the line between healthy and sick persons sometimes is blurry. By traditional medical discourse, one may not be sick as such, or may be sick in a different way. Behind the decision to classify a sick person, it is not just medical knowledge. Economic status of medical staff and hospitals, profits of pharmaceutical industry, policy of the government, culturally valued knowledge, and so on, play a role in defining the reality of sickness.

As the example shows, we can see that discourse and power are closely related. Ordinary people are not in the position capable to question or challenge the medical authority. This makes people of authority can exercise, or even abuse, their power by controlling discourses. It is true in other domains as well, such as in religion, in politics, in economics, in education, and so on. However, in Foucault's view, power is pervasive. That means power can be challenged by counter-discourse, as he puts it in this way:

\begin{quote}
Discourse transmits and produces power; it reinforces it, but also undermines it and exposes it, renders it fragile and makes it possible to thwart it.\footnote{\citealp[p.~101]{foucault:sexuality}}
\end{quote}

Do I give you a pessimistic view of the world? Far from that, I hold the view as realistic. I just point out that there is no escape from discourses. You have to encounter them one way or another. Some discourses make good results, for benefits of the many. Some discourses make bad results, for benefits for the few at the expense of the many. Unfortunately, we are usually unaware that we are stuck in certain discourses, because everything looks real and natural.

A well-known discourse in action is \emph{Orientalism}\footnote{\citealp{said:orientalism}, first published in 1978} proposed by Edward Said (1935--2003). By his analysis, the Eastern world is defined by colonialism as savage, strange, and powerless. This definition justifies its control over the area. Another well-known example given by Foucault himself is that `homosexuality' is defined as a species by social discourses, not just a kind of sexual behavior.\footnote{\citealp[p.~101]{foucault:sexuality}}

One problematic consequence when discourse is brought into play is it can slipperily entail cultural relativism---reality is made and depends on context. I think this line of thought itself is a discourse in action. The problem is in fact not binary that has either black or white answer. When we talk about physical reality, it is better to follow scientific discourse, not religious one. For example, we should hold that our world system is solar-centered, not earth-centered as described in the P\=ali canon. And when we talk about social reality, because of its complexity, several discourses should be put into play. I make myself clear here that I do not hold any relativistic position, even though I entertain considerably postmodern thought. We all have common reality based on our human physicality. On that ground, cultures make the reality express itself differently in various ways.

With the notion of discourse elaborated so far, we can apply it in our textual analysis. Norman Fairclough gives us a useful guideline. He says when language is used, it always constitutes three things: (a) social identities, (b) social relations, and (c) system of knowledge and belief.\footnote{\citealp[p.~134]{fairclough:marketization}} In reading a text, therefore we should identify that what kind of belief, what kind of identity, and what kind of relation the text tries to constitute.
