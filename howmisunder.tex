\chapter{How does misunderstanding happen?}\label{chap:howmisunder}

As we have seen, communication is not a foolproof activity. It is vulnerable to miscarriage. In this chapter we will discuss factors contributing to how we misunderstand each other.

\paragraph*{1.\ The received meaning is newly created, not delivered.} When we get certain understanding through communication, there is an interplay between external information, the message received in this case, and the existing knowledge structure in our mind, which is unique individually. By the same message with a different decoding system, it is likely that we can create different understanding out of it. We see this frequently: when two persons hear the same story, they can understand it differently. Furthermore, when we are sad, happy, angry, hungry, or drowsy, we create the meaning differently depending on our mood. There are other many factors that can affect our interpretation, for example, relationship between the listener and the speaker. We hear different thing from the one you love and the one you hate, even though the message is exactly the same.

\paragraph*{2.\ Memory is not videorecorder.} This is a widespread misunderstanding, and many people still hold this assumption. By our best knowledge nowadays, memory works in a reconstructive way.\footnote{Reconstruction [of memory] is often driven by background knowledge that suggests plausible inferences. Such inferences may even lead us to believe we are remembering something when we are not \citep[p.~216]{baddeley:memory}. In other words, instead of \emph{reproducing} the original event or story, we derive a \emph{reconstruction} based on our existing presuppositions, expectations and our `mental set' \citep[p.~12]{foster:memory}.} It is ``a dynamic activity or process rather than as a static entity or thing.''\footnote{\citealp[p.~8]{foster:memory}} When we commit something into memory, the information is encoded and linked in a way that later retrieval is possible and effective. Sometimes separate memories are `knitted' together, as we often mix episodes from different times and places into one story. When a certain idea is recalled, sometimes we mistake its source (this is called misattribution, see below). For example, we may think it comes from our direct experience, but in fact we saw it on TV, read it from a book, or even dreamed of it. Memory is vulnerable to false suggestion. Even false memory can be implanted. Moreover, people are normally overconfident in their memory. They usually think their recalling is impeccable. We all are blind to our false memory.

Daniel Schacter summarizes seven factors that make memory tend to fault, known as seven sins of memory\footnote{\citealp{schacter:seven}; \citealp[p.~82]{foster:memory}}, as follows:

\begin{compactenum}[(1)]
\item{Transience: Memory can be weakened and lost over time.}
\item{Absent-mindedness: Without attention, information is not put into memory or recalling is not properly done.}
\item{Blocking: Searching for information can be impeded, like `tip of the tongue.'}
\item{Misattribution: Sources of memory can be confused and misattributed.}
\item{Suggestibility: Memories can be implanted by leading questions, comments or suggestions.}
\item{Bias: Our current knowledge and beliefs play a role on how we remember our pasts.}
\item{Persistence: Recurring recall of disturbing information, as in post-traumatic stress disorder, can affect memory.}
\end{compactenum}

Although our memory is prone to error, it is quite useful and good enough for our living. It makes our learning possible and help us navigate through the world easily. But taking reliability of memory too seriously can end up in tragedy, like false eyewitness testimonies have put many innocent people into jail.\footnote{For a quick treatment, it is worth seeing Elizabeth Loftus in TED talk (\url{www.ted.com}) on ``How reliable is your memory?''}

To our concern, memory is essential in immediate conversation. We cannot engage in any conversation without memory. In such situation mutual understanding is easily obtained, because with turn taking we can check the correctness of the message by asking for confirmation, or observe non-verbal responses of the interlocutor. However, retelling from long past memory is a different story. Reconstruction based on our knowledge structure, as well as emotional state, inevitably occurs in this case.

\paragraph*{3.\ Language using can be ambiguous at all levels.}

This is an undeniable fact. All human languages, except languages in computer programming, are ambiguous.\footnote{Language is rife with ambiguities at every level of analysis, both in input and output \citep[p.~373]{baarsgage:cognition}.} At letter level, for example, we can be easily confused with `k' and `c' and `ch' and `q' when we listen to unfamiliar words. In P\=ali, this ambiguity is even wilder. The retroflexes (\pali{\d t \d th \d d \d dh \d n}) can be easily misplaced with the dentals (\pali{t th d dh n}). Often, \pali{\d l} becomes \pali{l} or vice versa. And \pali{r} is difficult to tell from \pali{l} when said.

At word level, P\=ali has many shared forms of declension. For example, several cases of feminine nouns look alike, and dative case uses mostly the same forms as genitive case. At sentence level, since words can be arranged in a number of ways and words themselves are ambiguous, P\=ali sentences often have multiple meaning. Comparing to English, P\=ali seems to have more ambiguity traps.\footnote{As we shall see, from power's point of view, it is good to have ambiguous religious language, because intended `discourse' is easily made. Vague language is easy to exploited than a clearer one, so to speak.}

\phantomsection
\addcontentsline{toc}{section}{Introduction to speech act theory}
\section*{Introduction to speech act theory}

\paragraph*{4.\ Language using has performative aspect.}

This is quite technical, but so important that we cannot ignore it. In a nutshell, when we say something, we do not always give some information or assert some fact. We do other things as well with words, such as promising, asking, apologizing, naming, and so on. Sentences that assert certain facts descriptively, which can be verified in some way, do \emph{constative} job, technically speaking. And sentence that do other things mentioned above do \emph{performative} jobs. This terminology is a part of \emph{speech acts theory}, introduced by J.\,L.\ Austin (1911--60).

Here is a simple constative sentence: ``The sun rises in the east.'' The sentence can be said to be `true' or `false' by observation or logical inference. Examples of performative sentences can be as follows:\footnote{adapted from \citealp[p.~5]{austin:how}}

\begin{compactenum}[(1)]
\item{In a wedding ceremony, the couple say ``I do'' to assert their commitment.}
\item{In christening a ship, one says ``I name this ship the \emph{Queen Elizabeth}'' while smashing the bottle against the stem.}
\item{In a will, one can write ``I give and bequeath my watch to my brother.''}
\item{In making a wager, one says ``I bet you sixpence it will rain tomorrow.''}
\end{compactenum}

Concerning performative aspect of language used, to make a conversation effective we have to recognize the right intention of utterances. Basically we observe the verbs used in sentences, but sometimes it is tricky. For example, in ``I bet you sixpence it will rain tomorrow,'' the speaker may not really want to make a bet, but just asserts a belief. Hence it can be read as ``I am sure that it will rain tomorrow.'' Seeing the sentence does not guarantee that the right intention will be obtained. We have to take the context into consideration.

Distinguishing constative from performative sentences is too coarse to be very useful. A more refined treatment from speech act theory is to recognize \emph{illocutionary acts}.\footnote{In theory, there are three parts. First, saying something meaningful---\emph{locutionary act}. Second, the saying has certain `force' to make things happen---\emph{illocutionary act}. And third, by saying so, certain effect happens---\emph{percutionary act}. See \citealp[pp.~101--2]{austin:how} for some examples. I will not go to all these in detail. Only illocutionary acts are useful to know here.} Put it simply, we have to know the `force' or intention of the utterances. Several schemes of the acts are proposed.\footnote{Austin himself suggests five kinds of illocutionary acts: verdictives, exercitives, commissives, behavitives, and expositives \citep[p.~150]{austin:how}. Some of these are difficult to understand. So, John Searle gives us alternative taxonomy: assertives, directives, commissives, expressives, and declarations \citep[pp.~12--20]{searle:speech}.} Here, I will follow the taxonomy presented by Kent Bach and Robert Harnish, because of its extensiveness. By this scheme, there are two groups: communicative and conventional illocutionary acts. The former succeeds by means of recognition of intention, the latter by satisfying a convention.\footnote{\citealp[p.~110]{bachharnish:speech}} There are four categories in the former, namely, constatives, directives, commissives, and acknowledgements; and two in the latter, namely, effectives and verdictives. All these are summarized in Table \ref{tab:illoacts}.\footnote{adapted from \citealp[pp.~40ff, 110ff]{bachharnish:speech}}

\bigskip
\begin{longtable}[c]{@{}%
	>{\raggedright\arraybackslash}p{0.35\linewidth}%
	>{\raggedright\arraybackslash}p{0.5\linewidth}%
	@{}}
\caption{Taxonomy of illocutionary acts}\label{tab:illoacts}\\
\toprule
\bfseries Act & \bfseries Examples/Typical Verbs\\ \midrule
\endfirsthead
\multicolumn{2}{c}{\tablename\ \thetable: Taxonomy of illocutionary acts (contd\ldots)}\\
\toprule
\bfseries Act & \bfseries Examples/Typical Verbs\\ \midrule
\endhead
\bottomrule
\ltblcontinuedbreak{2}
\endfoot
\bottomrule
\endlastfoot
%%
\mbox{1. Constatives} & \\
\hspace{5mm}- Assertives & affirm, allege, assert, aver, avow, claim, declare, deny, indicate, maintain, propound, say, state, submit \\
\hspace{5mm}- Predictives & forecast, predict, prophesy \\
\hspace{5mm}- Retrodictives & recount, report \\
\hspace{5mm}- Descriptives & appraise, assess, call, categorize, characterize, classify, date, describe, diagnose, evaluate, grade, identify, portray, rank \\
\hspace{5mm}- Ascriptives & ascribe, attribute, predicate \\
\hspace{5mm}- Informatives & advise, announce, apprise, disclose, inform, insist, notify, point out, report, reveal, tell, testify \\
\hspace{5mm}- Confirmatives & appraise, assess, bear witness, certify, conclude, confirm, corroborate, diagnose, find, judge, substantiate, testify, validate, verify, vouch for \\
\hspace{5mm}- Concessives & acknowledge, admit, agree, allow, assent, concede, concur, confess, grant, own \\
\hspace{5mm}- Retractives & abjure, correct, deny, disavow, disclaim, disown, recant, renounce, repudiate, retract, take back, withdraw \\
\hspace{5mm}- Assentives & accept, agree, assent, concur \\
\hspace{5mm}- Dissentives & differ, disagree, dissent, reject \\
\hspace{5mm}- Disputatives & demur, dispute, object, protest, question \\
\hspace{5mm}- Responsives & answer, reply, respond, retort \\
\hspace{5mm}- Suggestives & conjecture, guess, hypothesize, speculate, suggest \\
\hspace{5mm}- Suppositives & assume, hypothesize, postulate, stipulate, suppose, theorize \\
\midrule
\mbox{2. Directives} & \\
\hspace{5mm}- Requestives & ask, beg, beseech, implore, insist, invite, petition, plead, pray, request, solicit, summon, supplicate, tell, urge \\
\hspace{5mm}- Questions & ask, inquire, interrogate, query, question, quiz \\
\hspace{5mm}- Requirements & bid, charge, command, demand, dictate, direct, enjoin, instruct, order, prescribe, require \\
\hspace{5mm}- Prohibitives & enjoin, forbid, prohibit, proscribe, restrict \\
\hspace{5mm}- Permissives & agree to, allow, authorize, bless, consent to, dismiss, excuse, exempt, forgive, grant, license, pardon, release, sanction \\
\hspace{5mm}- Advisories & admonish, advise, caution, counsel, propose, recommend, suggest, urge, warn \\
\midrule
\mbox{3. Commissives} & \\
\hspace{5mm}- Promises & promise, swear, vow, contract, bet, guarantee, surrender, invite \\
\hspace{5mm}- Offers & offer, propose, volunteer, bid \\
\midrule
\mbox{4. Acknowledgements} & \\
\hspace{5mm}- Apologize & e.g.\ Sorry! \\
\hspace{5mm}- Condole & commiserate, condole \\
\hspace{5mm}- Congratulate & compliment, congratulate, felicitate \\
\hspace{5mm}- Greet & e.g.\ Hello! \\
\hspace{5mm}- Thank & e.g.\ Thanks! \\
\hspace{5mm}- Bid & bid, wish \\
\hspace{5mm}- Accept & e.g.\ Okay! \\
\hspace{5mm}- Reject & refuse, reject, spurn \\
\midrule
\mbox{5. Effectives} & By mutual belief, one can be hired, appointed, nominated, elected, promoted, naturalized, or ordained. \\
\midrule
\mbox{6. Verdictives} & By the right of authority, one can be acquitted, certified, or disqualified.\\
\end{longtable}

Since communicative illocutionary acts can be self-explained by the verbs exemplified, so I will not explain these further. Conventional illocutionary acts are a little confusing. As its name tells us, effectives effect changes in institutional state of affairs. They achieve their effects only because of mutually belief, for example, graduation of a student, veto of a bill, consecration of a site. Verdictives is very similar to effectives, but instead of only mutual belief they achieve the effects mainly by authority of the institution. For example, an umpire can call a player out of the field, a judge can declare a defendant guilty, or a tax assessor can assess a piece of property.

As you may realize, mistaking one illocutionary act for another can fail the communication. P\=ali, unfortunately, has a pretty large pitfall of this. For example, by the sentence alone ``\pali{gacch\=ami nagara\d m}'' can be an assertion (``I go to town''), a question (``Do I go to town?''), a request (``Let me go to town''), or a wish (``May I go to town'').\footnote{In P\=ali, first person conjugation of present tense and imperative mood share the same forms. And when a verb is put at the beginning, the sentence can be seen as a question.} Therefore, taking the context into consideration is essential to the understanding of the sentence. And if we can recognize the range of possibility of the intention of sentences we read, our translation will be more effective and accurate.
