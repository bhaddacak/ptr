\markboth{}{Abbreviations}
\clearpage
\phantomsection
\addcontentsline{toc}{chapter}{Abbreviations}
\setcounter{footnote}{0}
\chapter*{Abbreviations}

\section*{References to Literary Works}

Referencing scheme of P\=ali literature in this volume follows the structure of \emph{Cha\a{d}t\a{d}tha Sa\.ng\a{=}ayana Tipi\a{d}taka Restructured}, included in \textsc{P\a{=}ali\,Platform} 3. For extended explanations see PNL.

\begin{longtable}[c]{@{}>{\raggedright\arraybackslash}p{0.17\linewidth}>{\raggedright\arraybackslash}p{0.78\linewidth}@{}}
\toprule
\bfseries\upshape \mbox{Abbrev.} & \bfseries\upshape Description \\ \midrule
\endfirsthead
\toprule
\bfseries\upshape \mbox{Abbrev.} & \bfseries\upshape Description \\ \midrule
\endhead
\bottomrule
\ltblcontinuedbreak{2}
\endfoot
\bottomrule
\endlastfoot
%%
A3 & Tikanip\a{=}ata, A\a{.}nguttaranik\a{=}aya, Suttapi\a{d}taka \\
A4 & Catukkanip\a{=}ata, A\a{.}nguttaranik\a{=}aya, Suttapi\a{d}taka \\
A6 & Chakkanip\a{=}ata, A\a{.}nguttaranik\a{=}aya, Suttapi\a{d}taka \\
A7 & Sattakanip\a{=}ata, A\a{.}nguttaranik\a{=}aya, Suttapi\a{d}taka \\
A8 & A\a{d}t\a{d}thakanip\a{=}ata, A\a{.}nguttaranik\a{=}aya, Suttapi\a{d}taka \\
A9 & Navakanip\a{=}ata, A\a{.}nguttaranik\a{=}aya, Suttapi\a{d}taka \\
A10 & Dasakanip\a{=}ata, A\a{.}nguttaranik\a{=}aya, Suttapi\a{d}taka \\
Ap1 & Ther\a{=}apad\a{=}ana, Khuddakanik\a{=}aya, Suttapi\a{d}taka \\
Buv & Bhikkhuvibha\a{.}nga (Mah\a{=}avibha\a{.}nga), Vinayapi\a{d}taka \\
Cv & C\a{=}u\a{d}lavagga, Vinayapi\a{d}taka \\
D1 & S\a{=}ilakkhandhavagga, D\a{=}ighanik\a{=}aya, Suttapi\a{d}taka \\
D2 & Mah\a{=}avagga, D\a{=}ighanik\a{=}aya, Suttapi\a{d}taka \\
D3 & P\a{=}athikavagga, D\a{=}ighanik\a{=}aya, Suttapi\a{d}taka \\
Dhp & Dhammapada, Khuddakanik\a{=}aya, Suttapi\a{d}taka \\
Dhp-a\footnote{References to the commentary on Dhammapada use the same pattern as in Dhammapada, hence pointing to verse numbers, because the stories in the commentary have no reference point. This means the quoted text may appear either before (more likely) or after the verse cited, but still within its own story.} & Dhammapada-a\a{d}t\a{d}thakath\a{=}a \\
Dhs & Dhammasa\a{.}nga\a{d}n\a{=}i, Abhidhammapi\a{d}taka \\
DN & D\a{=}ighanik\a{=}aya (with sutta no.) \\
It & Itivuttaka, Khuddakanik\a{=}aya, Suttapi\a{d}taka \\
It-a & Itivuttaka-a\a{d}t\a{d}thakath\a{=}a \\
Ja & J\a{=}ataka, Khuddakanik\a{=}aya, Suttapi\a{d}taka \\
Ja-a & J\a{=}ataka-a\a{d}t\a{d}thakath\a{=}a \\
Kacc & Kacc\a{=}ayanaby\a{=}akara\a{d}na\a{d}m \\
M1 & M\a{=}ulapa\a{d}n\a{d}n\a{=}asa, Majjhimanik\a{=}aya, Suttapi\a{d}taka \\
M2 & Majjhimapa\a{d}n\a{d}n\a{=}asa, Majjhimanik\a{=}aya, Suttapi\a{d}taka \\
M3 & Uparipa\a{d}n\a{d}n\a{=}asa, Majjhimanik\a{=}aya, Suttapi\a{d}taka \\
Mil & Milindapa\a~nh\a{=}a, Khuddakanik\a{=}aya, Suttapi\a{d}taka \\
MN & Majjhimanik\a{=}aya (with sutta no.) \\
MN-a & Majjhimanik\a{=}aya-a\a{d}t\a{d}thakath\a{=}a (with sutta no.) \\
Mogg & Moggall\a{=}anaby\a{=}akara\a{d}na\a{d}m \\
Mv & Mah\a{=}avagga, Vinayapi\a{d}taka \\
Nidd1 & Mah\a{=}aniddesa, Khuddakanik\a{=}aya, Suttapi\a{d}taka \\
PNL & P\a{=}ali for New Learners\footnote{\url{bhaddacak.github.io/pnl}} \\
Pps & Papa\a~ncas\a{=}udan\a{=}i, Majjhimanik\a{=}aya-a\a{d}t\a{d}thakath\a{=}a \\
Psm & Pa\a{d}tisambhidh\=amagga, Khuddakanik\a{=}aya, Sut. \\
PTSD & \mbox{The Pali Text Society's Pali-English Dictionary}\footnote{\citealp{rhys:ptsd}} \\
Pv & Petavatthu, Khuddakanik\a{=}aya, Suttapi\a{d}taka \\
S1 & Sag\a{=}ath\a{=}avagga (SN\,1--11), Sa\a{d}myuttanik\a{=}aya, Sut. \\
S2 & Nid\a{=}anavagga (SN\,12--21), Sa\a{d}myuttanik\a{=}aya, Sut. \\
S3 & Khandhavagga (SN\,22--34), Sa\a{d}myuttanik\a{=}aya, Sut. \\
S4 & \mbox{Sa\a{d}l\a{=}ayatanavagga (SN\,35--44), Sa\a{d}myuttanik\a{=}aya, Sut.} \\
S5 & Mah\a{=}avagga (SN\,45--56), Sa\a{d}myuttanik\a{=}aya, Sut. \\
Sadd-Pad & Saddan\a{=}itipakara\a{d}na\a{d}m, Padam\a{=}al\a{=}a \\
Snp & Suttanip\a{=}ata, Khuddakanik\a{=}aya, Suttapi\a{d}taka \\
Snp-a & Suttanip\a{=}ata-a\a{d}t\a{d}thakath\a{=}a \\
Sp & Samantap\a{=}as\a{=}adik\a{=}a, Vinaya-a\a{d}t\a{d}thakath\a{=}a \\
Srp & S\a{=}aratthappak\a{=}asin\a{=}, Sa\a{d}myuttanik\=aya-a\a{d}t\a{d}thakath\a{=}a \\
Thag & Therag\a{=}ath\a{=}a, Khuddakanik\a{=}aya, Suttapi\a{d}taka \\
Thig & Ther\a{=}ig\a{=}ath\a{=}a, Khuddakanik\a{=}aya, Suttapi\a{d}taka \\
Vism & Visuddhimagga \\
\end{longtable}

\newpage
\section*{Grammatical Terms}
\begin{longtable}[c]{@{}>{\raggedright\arraybackslash}p{0.17\linewidth}>{\raggedright\arraybackslash}p{0.78\linewidth}@{}}
\toprule
\bfseries\upshape \mbox{Abbrev.} & \bfseries\upshape Description \\ \midrule
\endfirsthead
\toprule
\bfseries\upshape \mbox{Abbrev.} & \bfseries\upshape Description \\ \midrule
\endhead
\bottomrule
\ltblcontinuedbreak{2}
\endfoot
\bottomrule
\endlastfoot
%%
abl. & Ablative case (Pa\a~ncam\=i) \\
abs. & Absolutive \\
acc. & Accusative case (Dutiy\=a) \\
adj. & Adjective (Gu\a{d} nan\=ama) \\
adv. & Adverb \\
aor. & Aorist tense (Ajjatan\=i) \\
cond. & Conditional mood (K\=al\=atipatti) \\
dat. & Dative case (Catu\a{d}t\a{d}th\=i) \\
dict. & Dictionary form \\
f. & Feminine gender (Itth\=ili\a.nga) \\
f.p.p. & Future Passive Participle \\
fut. & Future tense (Bhavissanti) \\
g. & gender (Li\.nga) \\
gen. & Genitive case (Cha\a{d}t\a{d}th\=i) \\
imp. & Imperative mood (Pa\a~ncam\=i) \\
imperf. & Imperfect tense (Hiyyattan\=i) \\
ind. & Indeclinable (Avy\=aya) \\
inf. & Infinitive \\
ins. & Instrumental case (Tatiy\=a) \\
loc. & Locative case (Sattam\=i) \\
m. & Masculine gender (Pulli\a.nga) \\
n. & Noun (N\=ama) \\
nom. & Nominative case (Pa\a{d}tham\=a) \\
nt. & Neuter gender (Napu\a{d}msakali\a.nga) \\
num. & Number (Vacana) \\
opt. & Optative mood (Sattam\=i) \\
p.p. & Past Participle \\
perf. & Perfect tense (Parokkh\=a) \\
pl. & Plural (Bahuvacana) \\
pr.p. & Present Participle \\
pres. & Present tense (Vattam\=an\=a) \\
pron. & Pronoun (Sabban\=ama) \\
sg. & Singular (Ekavacana) \\
v. & Verb (\=Akhay\=ata) \\
v.i. & Intransitive verb \\
v.t. & Transitive verb \\
voc. & Vocative case (\=Al\=apana) \\
\end{longtable}
