\chapter{Commentaries to the first sermon}\label{chap:comcak}

As you may realize when studying the first sermon, traditional commentaries play a significant role helping us make sense of difficult terms. So, reading a commentary of the related text is advisable, even if you may not believe it in entirety. In this chapter I will show how to read the commentaries of \pali{Dhammacakkappavattanasutta}. In fact, I work on this chapter before I finish that of the first sermon, otherwise I cannot make decisions how to translate the sutta.

In some cases, if explanations in the commentary are not clear enough, I will consult the subcommentary and give some remarks in my explanation. If we are lucky, we will get a better elucidation. But sometimes, subcommentaries can mislead us and be likely to confuse us more.

\phantomsection
\addcontentsline{toc}{section}{Pre-reading introduction}
\section*{1.\ Pre-reading introduction}

\paragraph*{About the text} Since the main text has two instances in the canon, so do the commentaries. We will read them all. The commentary to the Vinaya, called \pali{Samantap\=as\=adik\=a} by name, has only a short explanation, because the author does not want to discuss the Suttanta in the Vinaya (see the translation below). However, subcommentaries on this part are quite numerous. On this account, the substantial commentary on \pali{Dhammacakkappavattanasutta} is only in \pali{S\=aratthapak\=asin\=i}, the Sa\d myuttanik\=aya's traditional commentary. In the P\=ali collection we mainly use, the names of commentaries are not mentioned. It follows the canon's arrangement, and simply known as \pali{a\d t\d thakath\=a} to the related parts. Hence, both are called \pali{Mah\=avagga-a\d t\d thakath\=a} coincidentally. The overall structure of this commentary follows the main text. So, we can identify a relevant part easily by paragraph numbers, hence Mv-a 1.13 and Srp5\,1081 (SN-a\,56) respectively. 

\paragraph*{About the author} It is quite indisputable that the author of both commentaries is Buddhaghosa, the great Therav\=ada commentator and the author of Visuddhimagga.

\paragraph*{About the audience} Commentaries are supposed to be used internally. So, the target audience of the text is Buddhist monks, particularly those of Mah\=avih\=ara school. However, Oskar von Hin\"uber has shown that, as stated in \pali{Samantap\=as\=adik\=a}, the commentaries were translated from Sinhala to P\=ali in order to make those who cannot read Sinhala understand the texts. Thus the commentaries, particularly to the Vinaya, may also have monks outside Mah\=avih\=ara as target audience. Consider this quotation:
\begin{quote}
The \pali{vinaya} commentary reached out to monks living outside Ceylon and sought to internationalize the Mah\=avih\=ara position, in contrast to the \pali{nik\=aya-} and \pali{abhidhamma}-commentaries, which were created together with the \pali{Visuddhimagga} in the first place for the Mah\=avih\=ara monks themselves to reassure them of the orthodoxy of their views.\footnote{\citealp[p.~123]{hinuber:translating}}
\end{quote}

\paragraph*{About time and place} Von Hin\"uber gives us a possible range of Buddhaghosa's date as AD 370 to 450.\footnote{\citealp[p.~103]{hinuber:literature}} So, the date of composition should not be far from that. The place is undoubtedly in Sri Lanka.

\paragraph*{About motives} To explain problematic points in the sutta is a visible motive. We can think further that to assert Mahavih\=ara position is another viable one.

\paragraph*{About text function} The text is supposed to do informative function, but as we read it, expressive and operative function seem to overshadow that. See more in concluding section.

\phantomsection
\addcontentsline{toc}{section}{Reading with a draft translation}
\section*{2.\ Reading with a draft translation}

For the translation done in this section, I try to keep the sentence structure as we have done in the main text. However, also to keep it readable some words are not translated directly. The learners should know this if they track every word carefully. In commentaries, some sentences are very long, so you will also see break-markers, e.g.\ \fbox{1}, here.

\subsection*{1)\ From \pali{Samantap\=as\=adik\=a}}

\ \par
\setcounter{sennum}{0}
\refstepcounter{sennum}\label{sen:cakkhu1}
\arabic{sennum}. \pali{\textbf{Cakkhukara\d n\=i}'ti pa\~n\~n\=acakkhu\d m sandh\=ay\=aha.}\\
$\triangleright$ [The Buddha] said that the wisdom eye is meant by `\pali{Cakkhukar\-a\d n\=i}.'\\
\begin{longtable}[c]{|p{0.9\linewidth}|}
\hline
\hspace{5mm}\small By the verb \pali{\=aha}, we have to figure out what is the subject of this sentence. The intended subject is supposedly the Buddha himself. But how did the commentator know that? Maybe, the speaker is someone else.\\
\hspace{5mm}\dag\ \small This can be seen as a use of omission technique or a presupposition (see Chapter \ref{chap:discourse}) to make the reader take something for granted.\\
\hline
\end{longtable}

\stepcounter{sennum}
\arabic{sennum}. \pali{Ito para\d m sabba\d m padatthato utt\=anameva.}\\
$\triangleright$ From here all other [terms are] just shallow because of [their] specific meaning.\\
\begin{longtable}[c]{|p{0.9\linewidth}|}
\hline
\hspace{5mm}\small This may mean that other terms are clear enough. No additional explanation is needed.\\
\hline
\end{longtable}

\stepcounter{sennum}
\arabic{sennum}. \pali{Adhipp\=ay\=anusandhiyojan\=adibhedato pana papa\~ncas\=udaniy\=a majjhima\d t\d thakath\=aya\d m vuttanayena veditabba\d m.}\\
$\triangleright$ However, conclusion, connection, and application, etc., [apart from here] should be known by the sense explained in the commentary to Majjhimanik\=aya, [called] Papa\~ncas\=udan\=i.\footnote{In Pps1\,33 (MN-a\,3), there is an explanation of `\pali{cakkhukara\d n\=i},' etc.}\\

\newpage
\begin{longtable}[c]{|p{0.9\linewidth}|}
\hline
\hspace{5mm}\small A long compound here is \pali{adhipp\=ay\=anusandhiyojan\=adibheda} (\pali{adhipp\=aya + anusandhi + yojana + \=adi + bheda}). Generally, \pali{bheda} means `breaking' or `separation,' as we find in \pali{sa\.nghabheda} (dissention of the Order), \pali{mittabheda} (breaking of friendship), \pali{s\=ilabheda} (breach of morality). Another compound that might confuse new students is \pali{vuttanaya} (\pali{vutta + naya}), the sense that was said.\\
\hline
\end{longtable}

\stepcounter{sennum}
\arabic{sennum}. \pali{Ito pa\d t\d th\=aya hi ativitth\=arabh\=irukassa mah\=ajanassa citta\d m anurakkhant\=a suttantakatha\d m ava\d n\d nayitv\=a vinayakatha\d myeva va\d n\-\d nayiss\=ama.}\\
$\triangleright$ So, from here on, [we], protecting the mind of the masses who are afraid of over-explanation, not explaining the Suttanta's matter, will explain only the Vinaya's matter.\\

\begin{longtable}[c]{|p{0.9\linewidth}|}
\hline
\hspace{5mm}\small The use of the plural verb \pali{va\d n\d nayiss\=ama} implies that this commentary was not produced by one person.\\
\hline
\end{longtable}

\subsection*{2)\ From \pali{S\=aratthapak\=asin\=i}}

\begin{center}
\textbf{\pali{Dhammacakkappavattanasuttava\d n\d nan\=a}}\par
$\triangleright$ A commentary to the discourse concerning\\ the forward moving of the wheel of Dhamma 
\end{center}

\stepcounter{sennum}
\arabic{sennum}. \pali{Dutiyassa pa\d thame \textbf{b\=ar\=a\d nasiya}n'ti eva\d mn\=amake nagare.}\\
$\triangleright$ In the first [sutta] of the second [group], `in B\=ar\=a\d nas\=i' [means] in the city called as such.\\

\begin{longtable}[c]{|p{0.9\linewidth}|}
\hline
\hspace{5mm}\small The part with boldface contains terms used in the main text and it is the subject of explanation. I insert a mark to separate \pali{iti} from the chunk. This can make new learners more comfortable. My \pali{iti} marking scheme is slightly different from that of the collection, particularly when \pali{-\d m} joins with \pali{iti}.\\
\hline
\end{longtable}

\stepcounter{sennum}
\arabic{sennum}. \pali{\textbf{Isipatane} migad\=aye'ti is\=ina\d m patanuppatanavasena eva\d mladdhan\=ame mig\=ana\d m abhayad\=anavasena dinnatt\=a migad\=ayasa\.nkh\=ate \=ar\=ame.}\\
$\triangleright$ [Concerning] `in \pali{isipatana migad\=aya},' by the influence of the flying and falling of seers, [its] name was given thus [as `\pali{isipatana}.'] [And] by the influence of the grant for deer's sanctuary, the park is so-called `\pali{migad\=aya}' because of the giving.\\
\begin{longtable}[c]{|p{0.9\linewidth}|}
\hline
\hspace{5mm}\small We will meet compounds ending with \pali{-vasa} a lot in due course, often in instrumental case (\pali{-vasena}). It means `by the power/influence/contribution of \ldots' Sometimes I omit this notion, if the implication is suggested by the context.\\
\hline
\end{longtable}

\stepcounter{sennum}
\arabic{sennum}. \pali{Ettha hi uppannuppann\=a sabba\~n\~nuisayo patanti, dhammacak\-kappavattanattha\d m nis\=idant\=i'ti attho.}\\
$\triangleright$ In this place, all-knowing seers, having arisen from time to time, fall---in the sense that [they] sit to make the wheel of Dhamma move.\\

\begin{longtable}[c]{|p{0.9\linewidth}|}
\hline
\hspace{5mm}\small I cannot make it clearer than this, because I do not understand it fully, and subcommentaries do not help much. There might be an attempt to connect the past events to the incident of the first sermon.\\
\hspace{5mm}\small An interesting word here is `\pali{uppannuppann\=a}' (\pali{uppanna + uppanna}). By its repetition, hence I render it as `arisen from time to time.'\\
\hline
\end{longtable}

\refstepcounter{sennum}\label{sen:isipatana}
\arabic{sennum}. \pali{Nandam\=ulakapabbh\=arato satt\=ahaccayena nirodhasam\=apattito vu\d t\d thit\=a anotattadahe katamukhadhovan\=adikicc\=a \=ak\=asena \=agantv\=a paccekabuddhaisayopettha otara\d navasena patanti, uposathattha\~nca anuposathattha\~nca sannipatanti, gandham\=adana\d m pa\d tigacchant\=api tatova uppatant\=i'ti imin\=a is\=ina\d m patanuppatanavasena ta\d m `isipatanan'ti vuccati.}\\
$\triangleright$ From the slope of mount Nandam\=ulaka, having emerged from a 7-day deep meditation, having done things such as washing the face at lake Anodatta, having come from the air, Paccekabuddha seers fall in this place by floating down, come together to do the Vinaya recitation, full and minor, [or] go to mount Gandham\=adana or fly from there. From this [account], that [park] is called `\pali{isipatana}' by the influence of the flying and falling of seers.\\

\newpage
\begin{longtable}[c]{|p{0.9\linewidth}|}
\hline
\hspace{5mm}\small Some words have to be explained additionally. To make it digestible, I put \pali{nirodhasam\=apatti} simply as `deep meditation.' By technical meaning of the term, it means like consciousness ceases there. Another technical term is \pali{paccekabuddha} meaning Buddhas who do not establish the religion. We may call them `lone Buddhas.'\\
\hspace{5mm}\small The next term is \pali{uposathattha} meaning the recitation of the P\=atimokkha, the main rules of monks. I think that `\pali{anuposathattha}' is similar but a smaller one. Maybe, it is an abridged recitation. It sounds strange that those `lone Buddhas' even do the recitation, because on one ever imposes any rule on them.\\
\hspace{5mm}\dag\ \small The anecdote in this sentence looks spectacular. It is supposed to be real, because no word suggests that this is just a speculation or imagination---no modal verb or particle is used. This is a discursive making in action. The main presupposition behind this is everything said in commentaries is true, no matter how strange it is. When \pali{sabba\~n\~n\=u} (all-knowing) is used in this context, it hints that somehow this magnificent event can be know miraculously. If not by the Buddha himself, other `seers' can also know it. So, it is not meant to be doubted.\\
\hline
\end{longtable}

\stepcounter{sennum}
\arabic{sennum}. \fbox{\upshape 1} \pali{\textbf{\=Amantes\=i}'ti d\=ipa\.nkarap\=adam\=ule kat\=abhin\=ih\=arato pa\d t\d th\=aya p\=aramiyo p\=urento anupubbena pacchimabhave kat\=abhinikkhamano anupubbena bodhima\d n\d da\d m patv\=a \ \fbox{\upshape 2} tattha apar\=ajitapalla\.nke nisi\-nno m\=arabala\d m bhinditv\=a pa\d thamay\=ame pubbeniv\=asa\d m anussaritv\=a majjhimay\=ame dibbacakkhu\d m visodhetv\=a pacchimay\=am\=avas\=ane dasasahassilokadh\=atu\d m unn\=adento sabba\~n\~nuta\d m patv\=a \fbox{\upshape 3} [satta] satt\=ah\=ani bodhima\d n\d de v\=itin\=ametv\=a mah\=abrahmun\=a \=ay\=acitadhammadesano buddhacakkhun\=a loka\d m voloketv\=a lok\=anuggahena b\=ar\=a\d na\-si\d m gantv\=a pa\~ncavaggiye sa\~n\~n\=apetv\=a dhammacakka\d m pavattetuk\=amo \=amantesi.}\\
$\triangleright$ \fbox{1} [Concerning] `\pali{\=amantesi}' ([He] called), [the explanation goes like this]: From the time of the miracle done at the feet of the Buddha D\=ipa\.nkara, [the Buddha] have been fulfilling the Perfections gradually, [until] in the latest being, [from] the renunciation done to the [enlightenment] attained under the Bo-tree successively. \fbox{2} In that place, [the Buddha], having sat the unconquered sitting, destroying the power of M\=ara; in the first watch having remembered [his] past lives; in the middle watch having purified [his] divined eye; [and] in the last watch, vibrating the 10,000-world-system, [he] attained the omniscience. \fbox{3} Having spent 7 days at the Bo-tree, preaching the Dhamma requested by god Mah\=abrahm\=a; having examined the world with the enlightened eye; by helping the world, having gone to B\=ar\=a\d nas\=i; having made the Five monks convinced, [having] a desire to move the wheel of Dhamma forward, [then the Buddha] called [them].\\
\begin{longtable}[c]{|p{0.9\linewidth}|}
\hline
\hspace{5mm}\small To tackle this long sentence, I cut it into three pieces, hence three English sentences. That means I have to change some P\=ali non-finite verbs to finite ones. This does a little harm to the structure but the meaning is intact. The unsaid subject of the whole sentence is \pali{buddho} or \pali{bhagav\=a}; using \pali{bodhisatto} at the time before enlightenment sounds more suitable. In the first cut, I change \pali{p\=urento} (pr.p.) to progressive past verb. In the second cut, \pali{patv\=a} is changed likewise. In the last cut, \pali{\=amantesi} is already the main verb. If you do not like my strategy, however, you may translate this in one stretch. In the text, I mark `\pali{satta}' out, because it seems redundant and out of place.\\
\hspace{5mm}\dag\ \small Why is only a short verb of calling (\pali{\=amantesi}) unpacked into the Buddha's life story? Is this an over-explanation? It seems that the word itself does not need any clarification at all. We can see this as a kind of framing.\\
\hspace{5mm}\dag\ \small The picture of the Buddha called the five ascetics in order to tell what he knew is depicted as important, buttressed by a significant backstory. Like when we see a flashback in a movie or a backstory in a novel, it means the related event is important in someway. A simple calling needs no backstory, but this one is not an ordinary calling.\\
\hspace{5mm}\dag\ \small Another point worth noting here is the use of \pali{k\=ama} in \pali{pavattetuk\=amo}. The compound modifies the Buddha, the subject of the sentence. This tells us that \pali{k\=ama} has neutral meaning, not completely negative as most Buddhists hold it. In psychological terms, it can be seen as `drive' or `motive.' It can have a positive meaning, as we find in \pali{dhammak\=amo} (One who loves goodness). So, I think it is natural to say that the Buddha has a certain desire. Otherwise we have no word to say that an arhant has a will to do something. This can bring us to the problem of demarcation of desire---how to tell a bad desire from a good, or a neutral one?\\
\hline
\end{longtable}

\stepcounter{sennum}
\arabic{sennum}. \pali{\textbf{Dveme, bhikkhave, ant\=a}'ti dve ime, bhikkhave, ko\d t\d th\=as\=a.}\\
$\triangleright$ [Concerning] `\pali{Dveme, bhikkhave, ant\=a},' [it means] ``Monks, these two portions.''\\

\refstepcounter{sennum}\label{sen:10000world}
\arabic{sennum}. \pali{Imassa pana padassa saha samud\=ah\=arena samud\=ah\=aranigghoso he\d t\d th\=a av\=ici\d m upari bhavagga\d m patv\=a dasasahassilokadh\=atu\d m pattharitv\=a a\d t\d th\=asi.}\\
$\triangleright$ The sound of this utterance, together with the conversation, having reached the lowest hell below [and] the highest existence above, having spread over the 10,000-world-system, remained.\\
\begin{longtable}[c]{|p{0.9\linewidth}|}
\hline
\hspace{5mm}\dag\ \small This is another exaggerated account used to extol the event. By the sentence's structure, the happening sounds real. We might think that the author did not expect it will be believed literally. It is just a part of poetic expression. If so, how about other strange accounts found here and there? Should we take them seriously? To me, this is like insinuation in effect. The readers are supposed to believe it at first. For skeptic readers, it is also alright if it is seen as a figurative speech.\\
\hline
\end{longtable}

\refstepcounter{sennum}\label{sen:180millions}
\arabic{sennum}. \pali{Tasmi\d myeva samaye a\d t\d th\=arasako\d tisa\.nkh\=a brahm\=ano sam\=agacchi\d msu, pacchimadis\=aya s\=uriyo atthameti, p\=ac\=inadis\=aya \=as\=a\d lha\-nakkhattena yutto pu\d n\d nacando uggacchati.}\\
$\triangleright$ In that occasion, 180 millions of god Brahm\=as came together. The sun sets in the west. The full moon, having engaged with the constellation of \=As\=a\d lha, rises up in the east.\\
\begin{longtable}[c]{|p{0.9\linewidth}|}
\hline
\hspace{5mm}\small \pali{\=As\=a\d lha} is two months after \pali{Vis\=akha}, the time of enlightenment. To the present day, it is around July. Specifying the time here is informative.\\
\hspace{5mm}\dag\ \small It does not matter how the number of gods is counted. Big numbers in P\=ali are not meant to take seriously. It just means `a lot.' When a great number of gods come together, the event looks very important. The account has extolling effect like other spectacular ones. This sentence also tells us that the event happened in a full-moon night, two months after the enlightenment. How exact of this occurrence is not worth doubting, even though it might be more realistic if the event happened in daytime. The point is that such an important event has to take place when the moon is full.\\
\hline
\end{longtable}

\stepcounter{sennum}
\arabic{sennum}. \pali{Tasmi\d m samaye bhagav\=a ima\d m dhammacakkappavattanasutta\d m \=arabhanto `dveme, bhikkhave, ant\=a'ti\=adim\=aha.}\\
$\triangleright$ In that time, the Blessed One, beginning this Dhammacakka discourse, thus said ``Monks, these two extremes, etc.''\\

\stepcounter{sennum}
\arabic{sennum}. \pali{Tattha \textbf{pabbajiten\=a}'ti gihisa\d myojana\d m chinditv\=a pabbajjupagatena.}\\
$\triangleright$ In that, `\pali{pabbajitena}' [means] by one who, having cut the fetter of household life, undertook the ascetic life.\\

\stepcounter{sennum}
\arabic{sennum}. \pali{\textbf{Na sevitabb\=a}'ti na va\d la\~njetabb\=a.}\\
$\triangleright$ `\pali{Na sevitabb\=a}' [means] [it] should not be used.\\

\stepcounter{sennum}
\arabic{sennum}. \pali{\textbf{Yo c\=aya\d m k\=amesu k\=amasukhallik\=anuyogo}'ti yo ca aya\d m vatthuk\=amesu kilesak\=amasukhassa anuyogo.}\\
$\triangleright$ '\pali{Yo c\=aya\d m k\=amesu k\=amasukhallik\=anuyogo}' [means] which this practice of enjoyment in objects of pleasure.\\
\begin{longtable}[c]{|p{0.9\linewidth}|}
\hline
\hspace{5mm}\small By its technical meaning, \pali{kilesak\=amasukha} means the enjoyment based on craving, or precisely defilement in the mind.\\
\hline
\end{longtable}

\stepcounter{sennum}
\arabic{sennum}. \pali{\textbf{H\=ino}'ti l\=amako. \textbf{Gammo}'ti g\=amav\=as\=ina\d m santako.}\\
$\triangleright$ `\pali{H\=ino}' [means] inferior. `\pali{Gammo}' [means] a property of those living in a village.\\

\stepcounter{sennum}
\arabic{sennum}. \pali{\textbf{Pothujjaniko}'ti andhab\=alajanena \=aci\d n\d no.}\\
$\triangleright$ `\pali{Pothujjaniko}' [means] [the deed] practiced by a foolish person.\\

\begin{longtable}[c]{|p{0.9\linewidth}|}
\hline
\hspace{5mm}\dag\ \small Equating a common person (\pali{pothujjanika} or \pali{puthujjana}) to a foolish or ignorant one (\pali{andhab\=alajana}) is clearly a discursive manipulation. The result of this framing is the common belief among Therav\=ada adherents that a renouncer's life is better, or is a wiser choice, than a household life. Ironically, even though people believe as such, few undertake the renouncer's path. An upshot of this is that the renouncer's life is regarded to have higher value, because it is harder to pursue. So, it is reasonable, in fact obligatory, to respect and support renouncers.\\
\hline
\end{longtable}

\stepcounter{sennum}
\arabic{sennum}. \pali{\textbf{Anariyo}'ti na ariyo na visuddho na uttamo na v\=a ariy\=ana\d m santako.}\\
$\triangleright$ `\pali{Anariyo}' [means] [it is] not noble, not clean, not excellent, or not a property of the noble.\\
\begin{longtable}[c]{|p{0.9\linewidth}|}
\hline
\hspace{5mm}\dag\ \small A synonym of \pali{pothujjanika} in this context is \pali{anariya}, the opposite of \pali{ariya} (noble, of the Aryan). The word has a racial connotation. This framing also suggests that it is better to be a gentleman rather than a common one. Or with a racial tint, it is better to act like the Aryan. It has a lot to discuss about the issue if we go deeper. Reading the entry of `\pali{ariya}' in PTSD is worthwhile.\\
\hline
\end{longtable}

\stepcounter{sennum}
\arabic{sennum}. \pali{\textbf{Anatthasa\d mhito}'ti na atthasa\d mhito, hitasukh\=avahak\=ara\d na\d m anissito'ti attho.}\\
$\triangleright$ `\pali{Anatthasa\d mhito}' [means] [it has] no benefit, in the sense that it cannot be depended on for obtaining welfare and happiness.\\

\stepcounter{sennum}
\arabic{sennum}. \pali{\textbf{Attakilamath\=anuyogo}'ti attano kilamathassa anuyogo, attano dukkhakara\d nan'ti attho.}\\
$\triangleright$ `\pali{Attakilamath\=anuyogo}' [means] the practice of making oneself exhausted, in the sense of making oneself suffer.\\

\stepcounter{sennum}
\arabic{sennum}. \pali{\textbf{Dukkho}'ti ka\d n\d tak\=apassayaseyy\=ad\=ihi attam\=ara\d nehi dukkh\=avaho.}\\
$\triangleright$ `\pali{Dukkho}' [means] [the action that] brings suffering by [actions] like lying upon [a bed of] thorns, etc., [resulting in] killing oneself.\\

\refstepcounter{sennum}\label{sen:cakkhu2}
\arabic{sennum}. \pali{Pa\~n\~n\=acakkhu\d m karot\=i'ti \textbf{cakkhukara\d n\=i}. Dutiyapada\d m tasseva vevacana\d m.}\\
$\triangleright$ [The Buddha] makes the wisdom eye [happen], thus `\pali{cakkhukara\d n\=i}.' The second term is just its synonym.\\

\stepcounter{sennum}
\arabic{sennum}. \pali{\textbf{Upasam\=ay\=a}'ti kiles\=upasamatth\=aya.}\\
$\triangleright$ `\pali{Upasam\=aya}' [means] for the benefit of pacifying defilements.\\

\stepcounter{sennum}
\arabic{sennum}. \pali{\textbf{Abhi\~n\~n\=ay\=a}'ti catunna\d m sacc\=ana\d m abhij\=ananatth\=aya.}\\
$\triangleright$ `\pali{Abhi\~n\~n\=aya}' [means] for the benefit of knowing fully the four [noble] Truths.\\

\stepcounter{sennum}
\arabic{sennum}. \pali{\textbf{Sambodh\=ay\=a}'ti tesa\d myeva sambujjhanatth\=aya.}\\
$\triangleright$ `\pali{Sambodh\=aya}' [means] for the benefit of understanding clearly those [four Truths].\\

\stepcounter{sennum}
\arabic{sennum}. \pali{\textbf{Nibb\=an\=ay\=a}'ti nibb\=anasacchikiriy\=aya.}\\
$\triangleright$ `\pali{Nibb\=an\=ay\=a}' [means] for experiencing nirvana.\\

\stepcounter{sennum}
\arabic{sennum}. \pali{Sesamettha ya\d m vattabba\d m siy\=a, ta\d m he\d t\d th\=a tattha tattha vuttameva.}\\
$\triangleright$ Which remainder in here worth discussing may exist, that [remainder] was discussed here and there below.\\

\stepcounter{sennum}
\arabic{sennum}. \pali{Saccakath\=api sabb\=ak\=areneva visuddhimagge vitth\=arit\=a.}\\
$\triangleright$ Even accounts concerning truth were explained completely in Visuddhimagga.\footnote{Vism\,16.529ff}\\

\stepcounter{sennum}
\arabic{sennum}. \pali{\textbf{Tipariva\d t\d ta}n'ti sacca\~n\=a\d nakicca\~n\=a\d nakata\~n\=a\d nasa\.nkh\=at\=ana\d m ti\d n\d na\d m pariva\d t\d t\=ana\d m vasena tipariva\d t\d ta\d m.}\\
$\triangleright$ `\pali{Tipariva\d t\d ta\d m}' [means] having three rounds by the influence of three rounds of the so-called `\pali{sacca\~n\=a\d na}', `\pali{kicca\~n\=a\d na}' and `\pali{kata\~n\=a\d na}.'\\

\stepcounter{sennum}
\arabic{sennum}. \pali{Ettha hi `ida\d m dukkha\d m ariyasacca\d m, ida\d m dukkhasamudayan'ti eva\d m cat\=usu saccesu yath\=abh\=uta\d m \~n\=a\d na\d m \textbf{sacca\~n\=a\d na\d m} n\=ama.}\\
$\triangleright$ Here, the insight of seeing things as they are in the four Truths thus, ``This [is] the noble Truth of suffering, this [is] the cause of suffering,'' is called `\pali{sacca\~n\=a\d na}.'\\

\stepcounter{sennum}
\arabic{sennum}. \pali{Tesuyeva `pari\~n\~neyya\d m pah\=atabban'ti eva\d m kattabbakiccaj\=anana\~n\=a\d na\d m \textbf{kicca\~n\=a\d na\d m} n\=ama.}\\
$\triangleright$ The insight of knowing what should be done in those [Truths] thus, ``[This] should be known, [this] should be discarded,'' is called `\pali{kicca\~n\=a\d na}.'\\

\stepcounter{sennum}
\arabic{sennum}. \pali{`Pari\~n\~n\=ata\d m pah\=inan'ti eva\d m tassa tassa kiccassa katabh\=avaj\=anana\~n\=a\d na\d m \textbf{kata\~n\=a\d na\d m} n\=ama.}\\
$\triangleright$ The insight of knowing the state of having done each of that work thus, ``[This] was known, [this] was discarded,'' is called `\pali{kata\~n\=a\d na}.'\\

\stepcounter{sennum}
\arabic{sennum}. \pali{\textbf{Dv\=adas\=ak\=ara}n'ti tesa\d myeva ekekasmi\d m sacce ti\d n\d na\d m ti\d n\-\d na\d m \=ak\=ar\=ana\d m vasena dv\=adas\=ak\=ara\d m.}\\
$\triangleright$ `\pali{Dv\=adas\=ak\=ara\d m}' [means] having twelve manners by the influence of three manners in each Truth of those [four].\\

\stepcounter{sennum}
\arabic{sennum}. \pali{\textbf{\~N\=a\d nadassana}n'ti etesa\d m tipariva\d t\d t\=ana\d m dv\=adasanna\d m \=ak\=ar\=ana\d m vasena uppanna\~n\=a\d nasa\.nkh\=ata\d m dassana\d m.}\\
$\triangleright$ `\pali{\~N\=a\d nadassana\d m}' [means] the insight having arisen from the influence of these three rounds, twelve manners.\\

\stepcounter{sennum}
\arabic{sennum}. \pali{\textbf{Dhammacakkhu}n'ti a\~n\~nattha tayo magg\=a t\=i\d ni ca phal\=ani dhammacakkhu n\=ama honti, idha pa\d thamamaggova.}\\
$\triangleright$ [For] `\pali{dhammacakkhu\d m},' in other place the three paths and three fruits are called the Dhamma eye, [but] here [it is] just the first path.\\
\begin{longtable}[c]{|p{0.9\linewidth}|}
\hline
\hspace{5mm}\small This may need some doctrinal explanation. According to the four stages of awakening, namely the stream-enterer, the once-returner, the non-returner, and the arhant, each stage can be divided further into two steps, \pali{magga} (on the way) and \pali{phala} (done). Hence, three \pali{magga}s and \pali{phala}s refer to the first three stages of awakening. Normally, those who achieve these possess the so-called Dhamma eye. According to the commentary, in this discourse the Dhamma eye refers only to the first awakening.\\
\hline
\end{longtable}

\stepcounter{sennum}
\arabic{sennum}. \pali{\textbf{Dhammacakke}'ti pa\d tivedha\~n\=a\d ne ceva desan\=a\~n\=a\d ne ca.}\\
$\triangleright$ `\pali{Dhammacakke}' [means] the insight of penetration and demonstration.\\

\stepcounter{sennum}
\arabic{sennum}. \pali{Bodhipalla\.nke nisinnassa hi cat\=usu saccesu uppanna\d m dv\=adas\=ak\=ara\d m pa\d tivedha\~n\=a\d nampi, isipatane nisinnassa dv\=adas\=ak\=ar\=aya saccadesan\=aya pavattita\d m desan\=a\~n\=a\d nampi dhammacakka\d m n\=ama.}\\
$\triangleright$ The insight of penetrating the twelve manners arising in the four Truths when [the Buddha] sat under the Bo-tree, and the insight of demonstration when [the Buddha] sat in the Isipatana [park] to teach the Truths by twelve manners [that had] happened, [are] called the wheel of Dhamma.\\

\stepcounter{sennum}
\arabic{sennum}. \pali{Ubhayampi heta\d m dasabalassa ure pavatta\~n\=a\d nameva.}\\
$\triangleright$ These two insights [were] this knowledge happening in the chest of the Ten-power [= Buddha]\\

\stepcounter{sennum}
\arabic{sennum}. \pali{Im\=aya desan\=aya pak\=asentena bhagavat\=a dhammacakka\d m pavattita\d m n\=ama.}\\
$\triangleright$ [It is] called the wheel of Dhamma moved forward by the Buddha who was illustrating with this teaching.\\

\stepcounter{sennum}
\arabic{sennum}. \pali{Ta\d m paneta\d m dhammacakka\d m y\=ava a\~n\~n\=asiko\d n\d da\~n\~natthero a\d t\d th\=arasahi brahmako\d t\=ihi saddhi\d m sot\=apattiphale pati\d t\d th\=ati, t\=ava na\d m bhagav\=a pavatteti n\=ama, pati\d t\d thite ca pavattita\d m n\=ama.}\\
$\triangleright$ [Concerning] this wheel of Dhamma, to which extent that the Venerable Ko\d n\d da\~n\~na who knew [it] together with 180 millions of god Brahm\=as attains the stream-enterer, to that extent it is called that [wheel] is set going by the Blessed One. Also it is regarded as `having moved forward' because [it has been] established.\\

\begin{longtable}[c]{|p{0.9\linewidth}|}
\hline
\hspace{5mm}\small This instance is a bit confusing. It seems that \pali{Ta\d m} at the beginning is out of place, because we already have \pali{eta\d m dhammacakka\d m}. I think it might be more sensible if it is \pali{Ya\d m} instead, to correlate with \pali{na\d m} in the latter part.\\
\hline
\end{longtable}

\stepcounter{sennum}
\arabic{sennum}. \pali{Ta\d m sandh\=aya \textbf{pavattite ca pana bhagavat\=a dhammacakke bhumm\=a dev\=a saddamanuss\=avesu}n'ti\=adi vutta\d m.}\\
$\triangleright$ With reference to that [moving of the wheel], ``\pali{pavattite ca pana bhagavat\=a dhammacakke \ldots},'' etc.\ was said.\\

\stepcounter{sennum}
\arabic{sennum}. \pali{Tattha \textbf{bhumm\=a}'ti bh\=uma\d t\d thakadevat\=a.}\\
$\triangleright$ In that [discourse], `\pali{bhumm\=a}' [means] gods living on the earth.\\
\begin{longtable}[c]{|p{0.9\linewidth}|}
\hline
\hspace{5mm}\small Here, \pali{bh\=uma\d t\d thakadevat\=a} can be broken down to \pali{bh\=uma + \d tha + devat\=a}. For \pali{\d tha}, see the entry in PTSD.\\
\hline
\end{longtable}

\stepcounter{sennum}
\arabic{sennum}. \pali{\textbf{Saddamanuss\=avesu}n'ti ekappah\=areneva s\=adhuk\=ara\d m datv\=a -- `eta\d m bhagavat\=a'ti\=ad\=ini vadant\=a anus\=avayi\d msu.}\\
$\triangleright$ `\pali{Saddamanuss\=avesu\d m}' [means] [the gods], having given the applause in unison, having said ``\pali{eta\d m bhagavat\=a},'' etc., announced.\\

\stepcounter{sennum}
\arabic{sennum}. \pali{\textbf{Obh\=aso}'ti sabba\~n\~nuta\~n\~n\=a\d nobh\=aso.}\\
$\triangleright$ `\pali{Obh\=aso}' [means] the light of the insight into omniscience.\\

\stepcounter{sennum}
\arabic{sennum}. \pali{So hi tad\=a dev\=ana\d m dev\=anubh\=ava\d m atikkamitv\=a virocittha.}\\
$\triangleright$ At that time, that [light] has shone, going beyond the divine power of gods.\\

\stepcounter{sennum}
\arabic{sennum}. \pali{\textbf{A\~n\~n\=asi vata, bho, ko\d n\d da\~n\~no}'ti imassapi ud\=anassa ud\=ah\=aranigghoso dasasahassilokadh\=atu\d m pharitv\=a a\d t\d th\=asi.}\\
$\triangleright$ [Concerning] ``Oh!, friends, Ko\d n\d da\~n\~na understood [it],'' the clamor of this exclamation, having spread to the 10,000-world-system, remained.\\

\phantomsection
\addcontentsline{toc}{section}{Conclusion and discussion}
\section*{3.\ Conclusion and discussion}

Apart from some parts attributed as over-explanations, the commentary, particularly from \pali{S\=aratthapak\=asin\=i}, gives us clarification of marked keywords, even though the insertion of figurative expressions can mislead the readers occasionally. The explanations are undoubtedly in line with the author's stance, but all Buddhist schools would accept these without big difficulty. Overall, it is quite readable, no super long compound or sentence to be tackled.

Let us wrap up the issue of text type or text function. Commentaries are supposed to provide additional information that clarifies the main text. So, it should do informative function chiefly. According to what we have read in this commentary, information given here seems not so useful. One reason, I think, is the main text itself is quite clear and self-explained. Giving a just so story about \pali{Isipatana} (Sentence \ref{sen:isipatana}) goes beyond mere information giving. The anecdote given sounds unreal and unverifiable by any way. Who did really know that? Why should we know that then? And should we take it seriously? If not, what else we should? On what ground should we base or judge?

The main reason why a spectacular account is given is to make the subject important in some way. The place where the first sermon was delivered has to be historically significant. Moreover, depicting the number of 180 millions of gods (e.g.\ Sentence \ref{sen:180millions}) and the 10,000-world-system (e.g.\ Sentence \ref{sen:10000world}) clearly increases the significance of the event. Therefore persuasive function in this text is obvious. This tactic of explanation is very common throughout all commentaries. And we can see it in the main texts as well.

We can think that this is the way religious commentators worked at the time. Apart from making the canon more understandable, commentaries have to do promoting function at the same time to compete with other schools or religions. In today context, which the manufacture of truth is a pressing issue, commenting texts in such a way undermines the reliability of the texts.

Another way to look at it is to see that, in fact, the commentary does expressive function, hence to express the author's loyalty towards the religion. In that mode of articulation, exaggeration is common. This can also be the case, as we often see the uses of figurative language in narrations. The drawback of this view is ``To what extent should we take the fact value from the text?,'' because imagination and facts look mixed-up in many places.

It would be better, I surmise, had Buddhaghosa not translated any commentary. If, instead, the Sangha had taken more rigorous effort to preserve the original Sinhala commentaries, today we would get much more realistic understanding of the religion. So, I see that Buddhaghosa's translations of the old commentaries was a big mistake in the history of Buddhism.

Or, we can see it another way that there are no such things as Sinhala commentaries. If there were really some, they could not be so comprehensive like their P\=ali counterparts we have today. So, the commentaries were mostly new works. The reference to the old commentaries is just a rhetoric device to increase credibility. By this view, Buddhaghosa really did a great contribution to the understanding of the canon. However, this understanding is based mainly on his own perspective at the time, little to do with the early tradition.
