\chapter{Mindfulness of breathing}\label{chap:mindfulness}

We have seen some of theoretical parts in the Buddhist doctrine. Now we will read a practical discourse. The general idea of Buddhist contemplative practice is called \emph{mindfulness meditation} or \pali{satipa\d t\d th\=ana}. And the most important method is \emph{mindfulness of breathing} or \pali{\=an\=ap\=anasati}. We will examine a short sutta, taken from Majjhimanik\=aya, concerning this meditation.

\phantomsection
\addcontentsline{toc}{section}{Pre-reading introduction}
\section*{1.\ Pre-reading introduction}

\paragraph*{About the text} Mindfulness of breathing has good treatments in the old layer of the canon. It is a part of the grand discourse on mindfulness meditation in D\=ighanik\=aya, \pali{Mah\=asatipa\d t\d th\=anasutta} (D2\,374, DN\,22). It appears four times in Majjhimanik\=aya: in \pali{Mah\=asatipa\d t\d th\=anasutta} (M1\,107, MN\,10), in \pali{Mah\=ar\=ahulov\=adasutta} (M2\,121, MN\,62), in \pali{\=An\=ap\=anassatisutta} (M3\,148--9, MN\,121), and in \pali{K\=ayagat\=asatisutta} (M3\,154, MN\,122). A dedicated collection to the subject is found in \=An\=ap\=anasa\d myutta of Sa\d myuttanik\=aya, M\=ah\=avagga (SN\,54). It also appears one time in \pali{Girim\=anandasutta} of A\.nguttaranik\=aya (A10\,60). In a later layer, it has a short explanation in \pali{Satok\=ari\~n\=a\d naniddesa} of Pa\d tisambhidh\=amagga (Psm\,1:63--65), and some elaboration after that. Undoubtedly, the topic is one of the most mentioned in subsequent literature concerning Buddhist meditation.

The sutta brought here is named directly \pali{\=An\=ap\=anassatisutta}. We will read the whole sutta (M3\,144--152). It is not too long and has several repetitions.

\paragraph*{About the author} The text sender of Majjhimanik\=aya was the Sangha of an early council, and the text producer can be the compilers and redactors at the time. In the story, the Buddha delivered the teaching in an assembly.

\paragraph*{About the audience} The audience of the text is Buddhists in general, and monks in particular. In the story, the assembly consisted of several well-known monks, such as, S\=ariputta, Moggall\=ana, Mah\=akassapa, Mah\=akacc\=ayana, among others.

\paragraph*{About time and place} For the text, the possible time and place of writing is in an early council in India before the text was written down. Even if the text underwent change after that, it was not altered much, I think. In the story, the assembly took place in S\=avatth\=i, in one full-moon night. It was also the \pali{pav\=ara\d n\=a} day (roughly, the end of the lent) of that rain retreat.

\paragraph*{About motives} To preserve the teaching is a straight motive. To formulate a practical guideline in a systematic way for practitioners can be a viable one.

\paragraph*{About text function} Informative function can be seen obviously. It has a low degree on expressive function. By the fact that the text encourages a meditative practice, operative function can be seen accordingly. The text also constitutes a belief towards meditation in Buddhism.

\phantomsection
\addcontentsline{toc}{section}{Reading with a draft translation}
\section*{2.\ Reading with a draft translation}

\bigskip
\begin{center}
\textbf{\pali{\=An\=ap\=anassatisutta\d m}}\par
$\triangleright$ A discourse concerning\\mindfulness of breathing
\end{center}

\setcounter{sennum}{0}
\pali{\fbox{\stepcounter{sennum}\arabic{sennum}} 144.\ Eva\d m me suta\d m -- eka\d m samaya\d m bhagav\=a s\=avatthiya\d m viharati pubb\=ar\=ame mig\=aram\=atup\=as\=ade sambahulehi abhi\~n\~n\=atehi abhi\~n\~n\=atehi therehi s\=avakehi saddhi\d m -- \=ayasmat\=a ca s\=ariputtena \=ayasmat\=a ca mah\=amoggall\=anena \=ayasmat\=a ca mah\=akassapena \=ayasmat\=a ca mah\=akacc\=ayanena \=ayasmat\=a ca mah\=ako\d t\d thikena \=ayasmat\=a ca mah\=akappinena \=ayasmat\=a ca mah\=acundena \=ayasmat\=a ca anuruddhena \=ayasmat\=a ca revatena \=ayasmat\=a ca \=anandena, a\~n\~nehi ca abhi\~n\~n\=atehi abhi\~n\~n\=atehi therehi s\=avakehi saddhi\d m.}

\addtocounter{sennum}{-1}
$\triangleright$ \fbox{\stepcounter{sennum}\arabic{sennum}} 144.\ As it was heard by me thus, in one occasion the Blessed One lives in S\=avatth\=i, in a building of Mig\=aram\=at\=a, Pubb\=ar\=ama, together with many [individually] well-known senior disciples, [namely] Ven.\,S\=ariputta, Ven.\,Mah\=amoggall\=ana, Ven.\,Mah\=a\-kassapa, Ven.\,Mah\=akacc\=ayana, Ven.\,Mah\=ako\d t\d thika, Ven.\,Mah\=akap\-pina, Ven.\,Mah\=acunda, Ven.\,Anuruddha, Ven.\,Revata, Ven.\,\=Ananda, and other [individually] well-known senior disciples.\\

\begin{longtable}[c]{|p{0.9\linewidth}|}
\hline
\hspace{5mm}\small To be precise, \pali{abhi\~n\~n\=atehi abhi\~n\~n\=atehi} means each of them is well-known. They are not well-known collectively. So, I put `individually' in the translation.\\
\hspace{5mm}\dag\ \small Mentioning several well-known monks here tells us that this event was significant, so was the teaching delivered.\\
\hline
\end{longtable}

\pali{\fbox{\stepcounter{sennum}\arabic{sennum}} Tena kho pana samayena ther\=a bhikkh\=u nave bhikkh\=u ovadanti anus\=asanti. \fbox{\stepcounter{sennum}\arabic{sennum}} Appekacce ther\=a bhikkh\=u dasapi bhikkh\=u ovadanti anus\=asanti, appekacce ther\=a bhikkh\=u v\=isampi bhikkh\=u ovadanti anus\=asanti, appekacce ther\=a bhikkh\=u ti\d msampi bhikkh\=u ovadanti anus\=asanti, appekacce ther\=a bhikkh\=u catt\=ar\=isampi bhikkh\=u ovadanti anus\=asanti. \fbox{\stepcounter{sennum}\arabic{sennum}} Te ca nav\=a bhikkh\=u therehi bhikkh\=uhi ovadiyam\=an\=a anus\=asiyam\=an\=a u\d l\=ara\d m pubben\=apara\d m visesa\d m j\=ananti.}

\addtocounter{sennum}{-3}
$\triangleright$ \fbox{\stepcounter{sennum}\arabic{sennum}} By that time, senior monks give advices to [and] instruct new monks. \fbox{\stepcounter{sennum}\arabic{sennum}} Some senior monks give advices to [and] instruct ten monks, some give to twenty monks, some thirty, [and] some forty. \fbox{\stepcounter{sennum}\arabic{sennum}} Those new monks, being given advices [and] instructed by senior monks, know the excellent, superb [knowledge] unknown formerly.\\

\pali{\fbox{\stepcounter{sennum}\arabic{sennum}} 145.\ Tena kho pana samayena bhagav\=a tadahuposathe pannarase pav\=ara\d n\=aya pu\d n\d n\=aya pu\d n\d nam\=aya rattiy\=a bhikkhusa\.nghaparivuto abbhok\=ase nisinno hoti. \fbox{\refstepcounter{sennum}\arabic{sennum}\label{sen:komudi}} Atha kho bhagav\=a tu\d nh\=ibh\=uta\d m tu\d nh\=ibh\=uta\d m bhikkhusa\.ngha\d m anuviloketv\=a bhikkh\=u \=amantesi -- `\=araddhosmi, bhikkhave, im\=aya pa\d tipad\=aya; \=araddhacittosmi, bhikkhave, im\=aya pa\d tipad\=aya. Tasm\=atiha, bhikkhave, bhiyyosomatt\=aya v\=iriya\d m \=arabhatha appattassa pattiy\=a, anadhigatassa \linebreak adhigam\=aya, asacchikatassa sacchikiriy\=aya. Idhev\=aha\d m s\=avatthi\-ya\d m komudi\d m c\=atum\=asini\d m \=agamess\=am\=i'ti. \fbox{\stepcounter{sennum}\arabic{sennum}} Assosu\d m kho j\=anapad\=a bhikkh\=u -- `bhagav\=a kira tattheva s\=avatthiya\d m komudi\d m c\=atum\=asini\d m \=agamessat\=i'ti. Te j\=anapad\=a bhikkh\=u s\=avatthi\d m osaranti bhagavanta\d m dassan\=aya. \fbox{\stepcounter{sennum}\arabic{sennum}} Te ca kho ther\=a bhikkh\=u bhiyyosomatt\=aya nave bhikkh\=u ovadanti anus\=asanti. Appekacce ther\=a bhikkh\=u dasapi bhikkh\=u ovadanti anus\=asanti, appekacce ther\=a bhikkh\=u v\=isampi bhikkh\=u ovadanti anus\=asanti, appekacce ther\=a bhikkh\=u ti\d msampi bhikkh\=u ovadanti anus\=asanti, appekacce ther\=a bhikkh\=u catt\=ar\=isampi bhikkh\=u ovadanti anus\=asanti. \fbox{\stepcounter{sennum}\arabic{sennum}} Te ca nav\=a bhikkh\=u therehi bhikkh\=uhi ovadiyam\=an\=a anus\=asiyam\=an\=a u\d l\=ara\d m pubben\=apara\d m visesa\d m j\=ananti.}

\addtocounter{sennum}{-5}
$\triangleright$ \fbox{\stepcounter{sennum}\arabic{sennum}} 145.\ By that time, the Blessed One sat in the open air, surrounded by a group of monks, in the Uposatha Day, the fifteenth [lunar day], the end of the rain retreat (\pali{pav\=ara\d n\=a}), a full-moon night. \fbox{\stepcounter{sennum}\arabic{sennum}} Then the Blessed One, having looked over the monks' assembly, each of them [was] in a state of silence, addressed the monks thus, ``Monks, I undertake [in] this line of pratice. Monks, I favor [in] this line of practice. Therefore, monks, you should undertake the effort in a higher degree, for the attainment of what was unattained, for the achievement of what was unachieved, for the realization of what was unrealized. Thus I will wait here in S\=avatth\=i until the four months [of the rainy season ends, when Komud\=i flowers bloom].'' \fbox{\stepcounter{sennum}\arabic{sennum}} Monks in the countryside have heard thus, ``The Blessed One will wait there in S\=avatth\=i until the end of rainy season.'' Those rural monks come into S\=avatth\=i to see the Blessed One. \fbox{\stepcounter{sennum}\arabic{sennum}} Hence, those senior monks give advices to [and] instruct new monks in a greater degree. Some give to ten monks, some twenty, some thirty, [and] some forty. \fbox{\stepcounter{sennum}\arabic{sennum}} Those new monks, being given advices [and] instructed by senior monks, know the excellent, superb [knowledge] unknown formerly.\\

\begin{longtable}[c]{|p{0.9\linewidth}|}
\hline
\hspace{5mm}\small In \fbox{\ref{sen:komudi}}, \pali{tasm\=atiha} is more or less equal to just \pali{tasm\=a}. For \pali{komud\=i c\=atum\=asin\=i}, see \pali{komud\=i} in PTSD. Exactly it means the full moon day in Kattik\=a (November). That means the Buddha would stay there for another month, after the three months of the rain retreat ended.\\
\hspace{5mm}\dag\ \small The use of `\pali{bhiyyosomatt\=aya v\=iriya\d m \=arabhatha}' reflects an attitude towards rigoruous meditative practice. That is to say, monks should practice meditation rigorously, strenuously more than one usually thinks he can. This picture is different from what we see in the early stage of propagation. Perhaps, the focus on rigorous meditative practice was developed later when the Sangha was formed as an institute, and meditation became a hallmark of the religion.\\
\hline
\end{longtable}

\pali{\fbox{\stepcounter{sennum}\arabic{sennum}} 146.\ Tena kho pana samayena bhagav\=a tadahuposathe pannarase komudiy\=a c\=atum\=asiniy\=a pu\d n\d n\=aya pu\d n\d nam\=aya rattiy\=a bhik\-khusa\.nghaparivuto abbhok\=ase nisinno hoti. \fbox{\stepcounter{sennum}\arabic{sennum}} Atha kho bhagav\=a tu\d nh\=ibh\=uta\d m tu\d nh\=ibh\=uta\d m bhikkhusa\.ngha\d m anuviloketv\=a bhikkh\=u \=amantesi -- `apal\=ap\=aya\d m, bhikkhave, paris\=a; nippal\=ap\=aya\d m, bhikkh\-ave, paris\=a; suddh\=a s\=are pati\d t\d thit\=a. \fbox{\stepcounter{sennum}\arabic{sennum}} Tath\=ar\=upo aya\d m, bhikkhave, bhikkhusa\.ngho; tath\=ar\=up\=a aya\d m, bhikkhave, paris\=a yath\=ar\=up\=a paris\=a \=ahuneyy\=a p\=ahuneyy\=a dakkhi\d neyy\=a a\~njalikara\d n\=iy\=a anuttara\d m pu\~n\~n\-akkhetta\d m lokassa. \fbox{\stepcounter{sennum}\arabic{sennum}} Tath\=ar\=upo aya\d m, bhikkhave, bhikkhusa\.ngho; tath\=ar\=up\=a aya\d m, bhikkhave, paris\=a yath\=ar\=up\=aya paris\=aya appa\d m dinna\d m bahu hoti, bahu dinna\d m bahutara\d m. Tath\=ar\=upo aya\d m, bhikkhave, bhikkhusa\.ngho; tath\=ar\=up\=a aya\d m, bhikkhave, paris\=a yath\=ar\=up\=a paris\=a dullabh\=a dassan\=aya lokassa. \fbox{\stepcounter{sennum}\arabic{sennum}} Tath\=ar\=upo aya\d m, bhikkhave, bhikkhusa\.ngho; tath\=ar\=up\=a aya\d m, bhikkhave, paris\=a yath\=ar\=upa\d m parisa\d m ala\d m yojanaga\d nan\=ani dassan\=aya gantu\d m pu\d t\-osen\=api'.}

\addtocounter{sennum}{-5}
$\triangleright$ \fbox{\stepcounter{sennum}\arabic{sennum}} 146.\ By that time, the Blessed One sat in the open air, surrounded by a group of monks, in the Uposatha Day, the fifteenth [lunar day], the end of rainy season when Komud\=i flowers bloom, a full-moon night. \fbox{\stepcounter{sennum}\arabic{sennum}} Then the Blessed One, having looked over the monks' assembly, each of them [was] in a state of silence, addressed the monks thus, ``Monks, this assembly [is] void of prattle. Monks, this assembly [is] void of idle chatter. [It] was purified, established in the essence [= no nonsense]. Monks, this group of monks as such; monks, this assembly as such is the assembly worthy of adoration, worthy of hospitality, worthy of offerings, worthy of salutation with joined palms, the world's unsurpassed field of merit. \fbox{\stepcounter{sennum}\arabic{sennum}} Monks, this group of monks as such; monks, this assembly as such is [that when] little offering was given by people, [the result of it is] plenty; [when] plenty of offering [was given], [the result is even] more than plenty. \fbox{\stepcounter{sennum}\arabic{sennum}} Monks, this group of monks as such; monks, this assembly as such is rare to be seen in the world. \fbox{\stepcounter{sennum}\arabic{sennum}} Monks, this group of monks as such; monks, this assembly as such is like the assembly suitable to go to see, even yojanas far [and] with a food supply [carried].''\\

\newpage
\begin{longtable}[c]{|p{0.9\linewidth}|}
\hline
\hspace{5mm}\small We can see \pali{ya-ta} structure here in the form of \pali{yath\=ar\=ura-tath\=ar\=upa}. To rearrange them into a familiar form, you have to shift the phrases with \pali{yath\=ar\=ura} to the front, and translate them as ``which assembly as such \ldots, that assembly as such is this assembly.''\\
\hspace{5mm}\small When a \pali{ya} part follows a \pali{ta} part, normally we can drop the markers, as shown in my translation.\\
\hspace{5mm}\dag\ \small The passage in the latter half is a discourse making. It constitutes and strengthens a belief that practitioners in certain condition are more favorable, worthy for veneration and offering. Even when people give them a little, the result of the giving is multiplied. Undoubtedly, the Sangha survives, and even thrives, through the history by this strong belief in giving endorsed by the text. My argument here is not about whether giving things to the Sangha really yields a multiplied result or not. Instead, when we do discourse analysis we have to be aware that certain understanding can shape the readers' conception of truth. And that is operative function of the text in action.\\
\hline
\end{longtable}

\pali{\fbox{\stepcounter{sennum}\arabic{sennum}} 147.\ Santi, bhikkhave, bhikkh\=u imasmi\d m bhikkhusa\.nghe arahanto kh\=i\d n\=asav\=a vusitavanto katakara\d n\=iy\=a ohitabh\=ar\=a anuppattasadatth\=a parikkh\=i\d nabhavasa\d myojan\=a sammada\~n\~n\=avimutt\=a -- evar\=up\=api, bhikkhave, santi bhikkh\=u imasmi\d m bhikkhusa\.nghe. \fbox{\stepcounter{sennum}\arabic{sennum}} Santi, bhikkhave, bhikkh\=u imasmi\d m bhikkhusa\.nghe pa\~ncanna\d m orambh\=agiy\=ana\d m sa\d myojan\=ana\d m parikkhay\=a opap\=atik\=a tattha pari\-nibb\=ayino an\=avattidhamm\=a tasm\=a lok\=a -- evar\=up\=api, bhikkhave, santi bhikkh\=u imasmi\d m bhikkhusa\.nghe. \fbox{\stepcounter{sennum}\arabic{sennum}} Santi, bhikkhave, bhikkh\=u imasmi\d m bhikkhusa\.nghe ti\d n\d na\d m sa\d myojan\=ana\d m parikkhay\=a r\=agadosamoh\=ana\d m tanutt\=a sakad\=ag\=amino sakideva ima\d m loka\d m \=agantv\=a dukkhassanta\d m karissanti -- evar\=up\=api, bhikkhave, santi bhikkh\=u imasmi\d m bhikkhusa\.nghe. \fbox{\stepcounter{sennum}\arabic{sennum}} Santi, bhikkhave, bhikkh\=u imasmi\d m bhikkhusa\.nghe ti\d n\d na\d m sa\d myojan\=ana\d m parikkhay\=a sot\=apann\=a avinip\=atadhamm\=a niyat\=a sambodhipar\=ayan\=a -- evar\=up\=api, bhikkhave, santi bhikkh\=u imasmi\d m bhikkhusa\.nghe.}

\addtocounter{sennum}{-4}
$\triangleright$ \fbox{\stepcounter{sennum}\arabic{sennum}} 147.\ Monks, there are monks in this group, [who were] arhants, free from mental obsessions, [had] fulfilled their living, done what should be done, laid down burdens, attained their own benefit, destroyed the fetters of rebirth, [and got] liberated [by] having known perfectly. Monks, there are such monks also in this group. \fbox{\stepcounter{sennum}\arabic{sennum}} Monks, there are monks in this group, [who are] those destroying the lower five fetters of rebirth, [will] be born spontaneously, reach the final release there, normally not return from that world. Monks, there are such monks also in this group. \fbox{\stepcounter{sennum}\arabic{sennum}} Monks, there are monks in this group, [who are] those destroying the [lower] three fetters of rebirth, those having the weakening of lust, anger, and delusion, those coming once---having come to this world only once, they will make the end of suffering. Monks, there are such monks also in this group. \fbox{\stepcounter{sennum}\arabic{sennum}} Monks, there are monks in this group, [who are] those destroying the [lower] three fetters of rebirth, those reaching the stream, normally not falling down, certain for enlightenment eventually. Monks, there are such monks also in this group.\\

\begin{longtable}[c]{|p{0.9\linewidth}|}
\hline
\hspace{5mm}\small The four enlightened qualities are described here. The account follows a standard pattern. By name, they are \pali{arahanta, an\=ag\=am\=i, sakad\=ag\=am\=i,} and \pali{sot\=apanna} respectively.\\
\hspace{5mm}\small The ten fetters of rebirth are also mentioned, because they are used as the gauge to measure those qualities. They are (1) personality-belief, (2) sceptical doubt, (3) clinging to mere rules and ritual, (4) sensuous craving, (5) ill-will, (6) craving for fine-material existence, (7) craving for immaterial existence, (8) conceit, (9) restlessness, and (10) ignorance.\footnote{\citealp[pp.~161--2]{nyanatiloka:dict}}\\
\hspace{5mm}\dag\ \small This measurement of religious achievement is only in theory or in history, because nowadays no one can tell what is going on in others' mind. Some might claim as such but it is still questionable that they really know directly whether certain fetters are present or not. In Vipassan\=a meditation circle, another measurement of mental phenomena in meditation is used verbally, i.e.\ by interview. Folk Buddhists yet judge the achievers by another tangible way. If a person can do a miracle, or even has some stories about it, he or she is deemed an achiever. And when that person die, if his or her ashes turn to glossy pebbles, he or she is undoubtedly an enlightened one.\\
\hspace{5mm}\dag\ \small Categorizing achievers in religions is common. Every religion has its own way to classify those who attain the goal of the religion. This is a discourse making, in the sense that certain identity and relationship are established. When some are deemed achievers, their relation to ordinary people will change. They will be marked as extraordinary and regarded as genuine disciples.\footnote{From a stock phrase, \pali{catt\=ari purisayug\=ani a\d t\d tha purisapuggal\=a esa bhagavato s\=avakasa\.ngho}, the four persons are the four achievers. Each of these has two steps, being on the path and having achieved, hence eigth in total.} They overwhelmingly get veneration and support.\\
\hspace{5mm}\dag\ \small In the past, only the Buddha could tell who are of which quality. Nowadays, as we might guess, immature practitioners tend to seek a title instead of the real achievement which has little to do with the classification. That is a visible effect of this discursive practice. Moreover, when practitioners target their goal to one of these qualities, they think that they have to be a certain kind of `person.' That goes against the very spirit of Buddhism, as far as I can tell.\\
\hline
\end{longtable}

\pali{\fbox{\stepcounter{sennum}\arabic{sennum}} Santi, bhikkhave, bhikkh\=u imasmi\d m bhikkhusa\.nghe catunna\d m satipa\d t\d th\=an\=ana\d m bh\=avan\=anuyogamanuyutt\=a viharanti -- evar\-\=up\=api, bhikkhave, santi bhikkh\=u imasmi\d m bhikkhusa\.nghe. Santi, bhikkhave, bhikkh\=u imasmi\d m bhikkhusa\.nghe catunna\d m sammappadh\=an\=ana\d m bh\=avan\=anuyogamanuyutt\=a viharanti\ldots pe\ldots catunna\d m \linebreak iddhip\=ad\=ana\d m\ldots pa\~ncanna\d m indriy\=ana\d m\ldots pa\~ncanna\d m bal\=ana\d m\ldots sattanna\d m bojjha\.ng\=ana\d m\ldots ariyassa a\d t\d tha\.ngikassa maggassa bh\=avan\=a\-nuyogamanuyutt\=a viharanti -- evar\=up\=api, bhikkhave, santi bhikkh\=u imasmi\d m bhikkhusa\.nghe. \fbox{\stepcounter{sennum}\arabic{sennum}} Santi, bhikkhave, bhikkh\=u imasmi\d m bhikkhusa\.nghe mett\=abh\=avan\=anuyogamanuyutt\=a viharanti\ldots karu\d n\=a\-bh\=avan\=anuyogamanuyutt\=a viharanti\ldots mudit\=abh\=avan\=anuyogamanu\-yutt\=a viharanti\ldots upekkh\=abh\=avan\=anuyogamanuyutt\=a viharanti\ldots asu\-bhabh\=avan\=anuyogamanuyutt\=a viharanti\ldots aniccasa\~n\~n\=abh\=avan\=anuyogamanuyutt\=a viharanti -- evar\=up\=api, bhikkhave, santi bhikkh\=u imasmi\d m bhikkhusa\.nghe. \fbox{\stepcounter{sennum}\arabic{sennum}} Santi, bhikkhave, bhikkh\=u imasmi\d m bhikkhusa\.nghe \=an\=ap\=anassatibh\=avan\=anuyogamanuyutt\=a viharanti. \fbox{\stepcounter{sennum}\arabic{sennum}} \=An\=ap\=anassati, bhikkhave, bh\=avit\=a bahul\=ikat\=a mahapphal\=a hoti mah\=anisa\d ms\=a. \=An\=ap\=anassati, bhikkhave, bh\=avit\=a bahul\=ikat\=a catt\=aro satipa\d t\d th\=ane parip\=ureti. Catt\=aro satipa\d t\d th\=an\=a bh\=avit\=a bahul\=ik\-at\=a satta bojjha\.nge parip\=urenti. Satta bojjha\.ng\=a bh\=avit\=a bahul\=ikat\=a vijj\=avimutti\d m parip\=urenti.}

\addtocounter{sennum}{-4}
$\triangleright$ \fbox{\stepcounter{sennum}\arabic{sennum}} Monks, there are monks in this group, [who] live, engaging in the development of the four foundations of mindfulness. Monks, there are such monks also in this group. Monks, there are monks in this group, [who] live, engaging in the development of the four right efforts. \ldots\ the four roads to power. \ldots\ the five spiritual faculties. \ldots\ the five spiritual powers. \ldots\ the seven factors of enlightenment. \ldots\ the noble eight-fold path. Monks, there are such monks also in this group. \fbox{\stepcounter{sennum}\arabic{sennum}} Monks, there are monks in this group, [who] live, engaging in the development of loving-kindness. \ldots\ compassion. \ldots\ sympathetic joy. \ldots\ equanimity. \ldots\ [the recognition] of loathsomeness [of the body]. \ldots\ the recognition of impermanence. Monks, there are such monks also in this group. \fbox{\stepcounter{sennum}\arabic{sennum}} Monks, there are monks in this group, [who] live, engaging in the development of mindfulness of breathing. \fbox{\stepcounter{sennum}\arabic{sennum}} Monks, the mindfulness of breathing, [when] having been cultivated and practiced frequently, is rich in result, rich in benefit. Monks, the mindfulness of breathing, [when] having been cultivated and practiced frequently, fulfills the four foundations of mindfulness. The four foundations of mindfulness, [when] having been cultivated and practiced frequently, fulfill the seven factors of enlightenment. The seven factors of enlightenment, [when] having been cultivated and practiced frequently, fulfill insight and liberation.\\

\begin{longtable}[c]{|p{0.9\linewidth}|}
\hline
\hspace{5mm}\small For those who are unfamiliar with suttas in the P\=ali canon, they may wonder what is the use of this tedious part. One possible answer is that the whole sutta here was used in recitation, possibly before or during meditative sessions. The informative function does not matter much in this regard. We find the 37 requisites of enlightenment (\pali{bodhipakkhiyadhamma})\footnote{\citealp[See][p.~35]{nyanatiloka:dict}.} here, together with some other meditative techniques.\\
\hspace{5mm}\dag\ \small By this instance, we can infer that what appears in a P\=ali sutta is not really the exact words delivered from the Buddha. Several parts are clearly insertions for some purpose by the compilers. That is why we should see the canon as the product of the Sangha rather than the Buddha himself. If we take this seriously enough, we should no longer say things like ``The Buddha said such and such things in the canon.'' Maybe, for accuracy we should just say, ``Such and such ideas of the Buddha are described in the canon.''\footnote{As the readers may realize, I try to set the Buddha apart from the canon. The main reason is that the product of the founder should not be seen as false. But we can find many flaws in the canon. So, it should not be seen as his product. However, by this approach I also realize that the Buddha becomes idolized as a divine being. And an undeniable fact is the Buddha himself is a construction right from the canon. So, the issue turns to be circular, and my attempt to separate the founder from the subsequent teachings is doomed at start.}\\
\hspace{5mm}\small The last part of this passage asserts the importance of this meditative method. It can lead directly to enlightenment, so to speak.\\
\hline
\end{longtable}

\pali{\fbox{\stepcounter{sennum}\arabic{sennum}} 148.\ Katha\d m bh\=avit\=a ca, bhikkhave, \=an\=ap\=anassati katha\d m bahul\=ikat\=a mahapphal\=a hoti mah\=anisa\d ms\=a? \fbox{\stepcounter{sennum}\arabic{sennum}} Idha, bhikkhave, bhikkhu ara\~n\~nagato v\=a rukkham\=ulagato v\=a su\~n\~n\=ag\=aragato v\=a nis\=idati palla\.nka\d m \=abhujitv\=a uju\d m k\=aya\d m pa\d nidh\=aya parimukha\d m sati\d m upa\d t\d thapetv\=a. So satova assasati satova passasati.}

\addtocounter{sennum}{-2}
$\triangleright$ \fbox{\stepcounter{sennum}\arabic{sennum}} 148.\ Monks, how was mindfulness of breathing cultivated? How is [it], having been practiced frequently, rich in result, rich in benefit? \fbox{\stepcounter{sennum}\arabic{sennum}} Monks, a monk in this [religion], having gone to a forest, to the root of a tree, or to an empty house, [then] sits, bending [the legs] to a cross-legged posture, maintaining the straight body, making the awareness alert. He breathes out with mindfulness, breathes in with mindfulness.\\

\begin{longtable}[c]{|p{0.9\linewidth}|}
\hline
\hspace{5mm}\small The essence of this sutta seems to start here. What does an empty house (\pali{su\~n\~n\=ag\=ara}) look like? There is no explanation in the commentary. In other places, nine locations are mentioned when a monk goes for meditation, namely, a forest, a tree's root, a mountain, a grotto on a mountain's slope, a cave in a mountain, a graveyard, a grove, an open-air area, and a heap of straw (\pali{ara\~n\~na\d m rukkham\=ula\d m pabbata\d m kandara\d m giriguha\d m sus\=ana\d m vanapattha\d m abbhok\=asa\d m pal\=alapu\~nja\d m}).\footnote{D2\,320 (DN\,19); Vibha\.nga 12.508 (Abhidhamma)} No \pali{su\~n\~n\=ag\=ara} is ever mentioned. So, we can infer that by empty house here it means one of these places, except a forest and a tree's root, because these two are already said separately.\\
\hspace{5mm}\small An commentary of another sutta suggests that \pali{assasati} means `to beathe in' and \pali{passasati} means `to breathe out.'\footnote{\pali{Ass\=asoti antopavisanan\=asikav\=ato. Pass\=asoti bahinikkhamanan\=asikav\=ato.} (Pps1\,305, MN-a\,28)} Whereas in the commentary to the Vinaya, the explanation is reversed.\footnote{\pali{Ass\=aso'ti bahi nikkhamanav\=ato. Pass\=aso'ti anto pavisanav\=ato.} (Sp1\,165)} It sounds natural to put `to breathe in' first, but some meditation teachers suggest you should start meditating with a relaxed breathing out.\\
\hline
\end{longtable}

\pali{\fbox{\stepcounter{sennum}\arabic{sennum}} (1) D\=igha\d m v\=a assasanto `d\=igha\d m assas\=am\=i'ti paj\=an\=ati, d\=igha\d m v\=a passasanto `d\=igha\d m passas\=am\=i'ti paj\=an\=ati; (2) rassa\d m v\=a assasanto `rassa\d m assas\=am\=i'ti paj\=an\=ati, rassa\d m v\=a passasanto `rassa\d m passas\=am\=i'ti paj\=an\=ati; (3) `sabbak\=ayapa\d tisa\d mved\=i assasiss\=am\=i'ti sikkhati, `sabbak\=ayapa\d tisa\d mved\=i passasiss\=am\=i'ti sikkhati; (4) `passambhaya\d m k\=ayasa\.nkh\=ara\d m assasiss\=am\=i'ti sikkhati, `passambhaya\d m k\=ayasa\.nkh\=ara\d m passasiss\=am\=i'ti sikkhati.}

\addtocounter{sennum}{-1}
$\triangleright$ \fbox{\stepcounter{sennum}\arabic{sennum}}\\
(1) [When] breathing in long, [one] knows clearly thus ``[I am] breathing in long.'' [When] breathing out long, [one] knows clearly thus ``[I am] breathing out long.''\\
(2) [When] breathing in short, [one] knows clearly thus ``[I am] breathing in short.'' [When] breathing out short, [one] knows clearly thus ``[I am] breathing out short.''\\
(3) [One] learns thus ``[I], being one who feels the whole body, will breathe in.'' [One] learns thus ``[I], being one who feels the whole body, will breathe out.''\\
(4) [One] learns thus ``[I], [when] making the body's conditions calm down, will breathe in.'' [One] learns thus ``[I], [when] making the body's conditions calm down, will breathe out.''\\

\begin{longtable}[c]{|p{0.9\linewidth}|}
\hline
\hspace{5mm}\small In the four foundations of mindfulness, these first four steps are the first foundation, the contemplation of the body. In (3), \pali{sabbak\=ayapa\d tisa\d mved\=i} is an adjective, so it modifies `I.'\footnote{However, by the way I put it in English, I treat it as a noun. So, it is an apposition to the subject.} In (4), \pali{passambhaya\d m} is present participle and causative.\footnote{See \pali{passambhati} in PTSD.}\\
\hspace{5mm}\small As we shall see in \fbox{\ref{sen:kaayekaayaa}} below, \pali{k\=aya} means the breathing itself. So, `feeling the whole body' can mean `feeling the all happenings of the breath.'' And when the body is calmed down, it means the breath itself is calmed down. This suggests that when practicing, one can observe only the breath, not the whole body.\\
\hline
\end{longtable}

\pali{\fbox{\stepcounter{sennum}\arabic{sennum}} (5) `P\=itipa\d tisa\d mved\=i assasiss\=am\=i'ti sikkhati, `p\=itipa\d tisa\d mved\=i passasiss\=am\=i'ti sikkhati; (6) `sukhapa\d tisa\d mved\=i assasiss\=am\=i'ti sikkh\-ati, `sukhapa\d tisa\d mved\=i passasiss\=am\=i'ti sikkhati; (7) `cittasa\.nkh\=arapa\d tisa\d mved\=i assasiss\=am\=i'ti sikkhati, `cittasa\.nkh\=arapa\d tisa\d mved\=i passasiss\=am\=i'ti sikkhati; (8) `passambhaya\d m cittasa\.nkh\=ara\d m assasiss\=am\=i'ti sikkhati, `passambhaya\d m cittasa\.nkh\=ara\d m passasiss\=am\=i'ti sikkhati.}

\addtocounter{sennum}{-1}
$\triangleright$ \fbox{\stepcounter{sennum}\arabic{sennum}}\\
(5) [One] learns thus ``[I], being one who feels rapture, will breathe in.'' [One] learns thus ``[I], being one who feels rapture, will breathe out.''\\
(6) [One] learns thus ``[I], being one who feels joy, will breathe in.'' [One] learns thus ``[I], being one who feels joy, will breathe out.''\\
(7) [One] learns thus ``[I], being one who feels the mind's conditions, will breathe in.'' [One] learns thus ``[I], being one who feels the mind's conditions, will breathe out.''\\
(8) [One] learns thus ``[I], [when] making the mind's conditions calm down, will breathe in.'' [One] learns thus ``[I], [when] making the mind's conditions calm down, will breathe out.''\\

\begin{longtable}[c]{|p{0.9\linewidth}|}
\hline
\hspace{5mm}\small These four are the second foundation, the contemplation of feeling. The attention is shifted now from the body or the breath to feelings happening during the practice. Rapture is a burst of blissful feeling, whereas joy is more subtle and stable. Negative feeling, like pain, can happen as well. In practice, whatever happens one just observes it attentively, no more no less.\\
\hline
\end{longtable}

\pali{\fbox{\stepcounter{sennum}\arabic{sennum}} (9) `Cittapa\d tisa\d mved\=i assasiss\=am\=i'ti sikkhati, `cittapa\d tisa\d m\-ved\=i passasiss\=am\=i'ti sikkhati; (10) `abhippamodaya\d m citta\d m assasiss\=am\=i'ti sikkhati, `abhippamodaya\d m citta\d m passasiss\=am\=i'ti sik\-khati; (11) `sam\=adaha\d m citta\d m assasiss\=am\=i'ti sikkhati, `sam\=adaha\d m citta\d m passasiss\=am\=i'ti sikkhati; (12) `vimocaya\d m citta\d m assasiss\=am\=i'ti sikkhati, `vimocaya\d m citta\d m passasiss\=am\=i'ti sikkhati.}

\addtocounter{sennum}{-1}
$\triangleright$ \fbox{\stepcounter{sennum}\arabic{sennum}}\\
(9) [One] learns thus ``[I], being one who feels the mind, will breathe in.'' [One] learns thus ``[I], being one who feels the mind, will breathe out.''\\
(10) [One] learns thus ``[I], [when] making the mind rejoiced, will breathe in.'' [One] learns thus ``[I], [when] making the mind rejoiced, will breathe out.''\\
(11) [One] learns thus ``[I], [when] concentrating the mind, will breathe in.'' [One] learns thus ``[I], [when] concentrating the mind, will breathe out.''\\
(12) [One] learns thus ``[I], [when] making the mind free, will breathe in.'' [One] learns thus ``[I], [when] making the mind free, will breathe out.''\\

\begin{longtable}[c]{|p{0.9\linewidth}|}
\hline
\hspace{5mm}\small These four are the third foundation, the contemplation of the mind. Now the attention is shifted to the mind itself. Mostly, the practitioners will experience recurring thoughts and certain visualization. For \pali{sam\=adaha\d m}, see \pali{sam\=adahati} in PTSD. And \pali{vimocaya\d m} is present participle and causative form of \pali{vimuccati}. The meaning of `free' is vague here. I think it means ``free from the disturbance of recurring thoughts and visualization.'' Thoughts may occur but not disturbing.\\
\hline
\end{longtable}

\pali{\fbox{\stepcounter{sennum}\arabic{sennum}} (13) `Anicc\=anupass\=i assasiss\=am\=i'ti sikkhati, `anicc\=anupass\=i passasiss\=am\=i'ti sikkhati; (14) `vir\=ag\=anupass\=i assasiss\=am\=i'ti sikkhati, `vir\=ag\=anupass\=i passasiss\=am\=i'ti sikkhati; (15) `nirodh\=anupass\=i assasiss\=am\=i'ti sikkhati, `nirodh\=anupass\=i passasiss\=am\=i'ti sikkhati; (16) `pa\d tinissagg\=anupass\=i assasiss\=am\=i'ti sikkhati, `pa\d tinissagg\=anupass\=i passasiss\=am\=i'ti sikkhati. \fbox{\stepcounter{sennum}\arabic{sennum}} Eva\d m bh\=avit\=a kho, bhikkhave, \=an\=ap\=anassati eva\d m bahul\=ikat\=a mahapphal\=a hoti mah\=anisa\d ms\=a.}

\addtocounter{sennum}{-2}
$\triangleright$ \fbox{\stepcounter{sennum}\arabic{sennum}}\\
(13) [One] learns thus ``[I], being one who sees impermanence, will breathe in.'' [One] learns thus ``[I], being one who sees impermanence, will breathe out.''\\
(14) [One] learns thus ``[I], being one who sees dispassionateness, will breathe in.'' [One] learns thus ``[I], being one who sees dispassionateness, will breathe out.''\\
(15) [One] learns thus ``[I], being one who sees cessation, will breathe in.'' [One] learns thus ``[I], being one who sees cessation, will breathe out.''\\
(16) [One] learns thus ``[I], being one who sees abandonment, will breathe in.'' [One] learns thus ``[I], being one who sees abandonment, will breathe out.''\\
\fbox{\stepcounter{sennum}\arabic{sennum}} In that way, monks, mindfulness of breathing [was] cultivated; in that way [it], having been practiced frequently, is rich in result, rich in benefit?\\

\newpage
\begin{longtable}[c]{|p{0.9\linewidth}|}
\hline
\hspace{5mm}\small These last four are the forth foundation, the contemplation on dhammas. The term is always ambiguous. In this context it probably means mind-objects. So, now the attention is shifted from the mind to the objects of it. One can see, for example, thoughts come and go. Then the attachment of those thoughts gradually subsides. The thoughts cease eventually, hence abandoned. Other kinds of mental phenomena can happen here, such as doubt, restlessness, drowsiness, or boredom.\\
\hline
\end{longtable}

\pali{\fbox{\stepcounter{sennum}\arabic{sennum}} 149.\ Katha\d m bh\=avit\=a ca, bhikkhave, \=an\=ap\=anassati katha\d m bahul\=ikat\=a catt\=aro satipa\d t\d th\=ane parip\=ureti? \fbox{\stepcounter{sennum}\arabic{sennum}} Yasmi\d m samaye, bhikkhave, bhikkhu (1) d\=igha\d m v\=a assasanto `d\=igha\d m assas\=am\=i'ti paj\=an\=ati, d\=igha\d m v\=a passasanto `d\=igha\d m passas\=am\=i'ti paj\=an\=ati; (2) rassa\d m v\=a assasanto `rassa\d m assas\=am\=i'ti paj\=an\=ati, rassa\d m v\=a passasanto `rassa\d m passas\=am\=i'ti paj\=an\=ati; (3) `sabbak\=ayapa\d tisa\d mved\=i assasiss\=am\=i'ti sikkhati, `sabbak\=ayapa\d tisa\d mved\=i passasiss\=am\=i'ti sik\-khati; (4) `passambhaya\d m k\=ayasa\.nkh\=ara\d m assasiss\=am\=i'ti sikkhati, `passambhaya\d m k\=ayasa\.nkh\=ara\d m passasiss\=am\=i'ti sikkhati; \fbox{\stepcounter{sennum}\arabic{sennum}} k\=aye k\=ay\=anupass\=i, bhikkhave, tasmi\d m samaye bhikkhu viharati \=at\=ap\=i sampaj\=ano satim\=a vineyya loke abhijjh\=adomanassa\d m. \fbox{\refstepcounter{sennum}\arabic{sennum}\label{sen:kaayekaayaa}} K\=ayesu k\=aya\~n\~natar\=aha\d m, bhikkhave, eva\d m vad\=ami yadida\d m -- ass\=asapass\=as\=a. Tasm\=atiha, bhikkhave, k\=aye k\=ay\=anupass\=i tasmi\d m samaye bhikkhu viharati \=at\=ap\=i sampaj\=ano satim\=a vineyya loke abhijjh\=adomanassa\d m.}

\addtocounter{sennum}{-4}
$\triangleright$ \fbox{\stepcounter{sennum}\arabic{sennum}} 149.\ Monks, how was mindfulness of breathing cultivated? How does [it], [when] having been practiced frequently, fulfill the four foundations of mindfulness? \fbox{\stepcounter{sennum}\arabic{sennum}} In which occasion, monks, a monk, (1) [when] breathing in long, knows clearly thus ``[I am] breathing in long.'' [When] breathing out long, [he] knows clearly thus ``[I am] breathing out long.'' (2) [When] breathing in short, [he] knows clearly thus ``[I am] breathing in short.'' [When] breathing out short, [he] knows clearly thus ``[I am] breathing out short.'' (3) [He] learns thus ``[I], being one who feels the whole body, will breathe in.'' [He] learns thus ``[I], being one who feels the whole body, will breathe out.'' (4) [He] learns thus ``[I], [when] making the body's conditions calm down, will breathe in.'' [He] learns thus ``[I], [when] making the body's conditions calm down, will breathe out.'' \fbox{\stepcounter{sennum}\arabic{sennum}} In that occasion, monks, [that] monk lives, being one who see a body in the body, strenuous, thoughtful, mindful, removing avarice and grief in the world. \fbox{\stepcounter{sennum}\arabic{sennum}} Monks, I call, namely, breathing in and breathing out, thus a certain body in bodies. Therefore, monks, in that occasion [that] monk lives, being one who see a body in the body, strenuous, thoughtful, mindful, removing avarice and grief in the world.\\

\pali{\fbox{\stepcounter{sennum}\arabic{sennum}} Yasmi\d m samaye, bhikkhave, bhikkhu (5) `p\=itipa\d tisa\d mved\=i assasiss\=am\=i'ti sikkhati, `p\=itipa\d tisa\d mved\=i passasiss\=am\=i'ti sikkhati; (6) `sukhapa\d tisa\d mved\=i assasiss\=am\=i'ti sikkhati, `sukhapa\d tisa\d mved\=i passasiss\=am\=i'ti sikkhati; (7) `cittasa\.nkh\=arapa\d tisa\d mved\=i assasiss\=am\=i'ti sikkhati, `cittasa\.nkh\=arapa\d tisa\d mved\=i passasiss\=am\=i'ti sikkhati; (8) `passambhaya\d m cittasa\.nkh\=ara\d m assasiss\=am\=i'ti sikkhati, `passambhaya\d m cittasa\.nkh\=ara\d m passasiss\=am\=i'ti sikkhati; \fbox{\stepcounter{sennum}\arabic{sennum}} vedan\=asu vedan\=anupass\=i, bhikkhave, tasmi\d m samaye bhikkhu viharati \=at\=ap\=i sampaj\=ano satim\=a vineyya loke abhijjh\=adomanassa\d m. \fbox{\stepcounter{sennum}\arabic{sennum}} Vedan\-\=asu vedan\=a\~n\~natar\=aha\d m, bhikkhave, eva\d m vad\=ami yadida\d m -- ass\=asapass\=as\=ana\d m s\=adhuka\d m manasik\=ara\d m. Tasm\=atiha, bhikkhave, vedan\=asu vedan\=anupass\=i tasmi\d m samaye bhikkhu viharati \=at\=ap\=i sampaj\=ano satim\=a vineyya loke abhijjh\=adomanassa\d m.}

\addtocounter{sennum}{-3}
$\triangleright$ \fbox{\stepcounter{sennum}\arabic{sennum}} In which occasion, monks, a monk (5) learns thus ``[I], being one who feels rapture, will breathe in.'' [He] learns thus ``[I], being one who feels rapture, will breathe out.'' (6) [He] learns thus ``[I], being one who feels joy, will breathe in.'' [He] learns thus ``[I], being one who feels joy, will breathe out.'' (7) [He] learns thus ``[I], being one who feels the mind's conditions, will breathe in.'' [He] learns thus ``[I], being one who feels the mind's conditions, will breathe out.'' (8) [He] learns thus ``[I], [when] making the mind's conditions calm down, will breathe in.'' [He] learns thus ``[I], [when] making the mind's conditions calm down, will breathe out.'' \fbox{\stepcounter{sennum}\arabic{sennum}} In that occasion, monks, [that] monk lives, being one who see a feeling in feelings, strenuous, thoughtful, mindful, removing avarice and grief in the world. \fbox{\stepcounter{sennum}\arabic{sennum}} Monks, I call, namely, the careful consideration of breathing in and breathing out, thus a certain feeling in feelings. Therefore, monks, in that occasion [that] monk lives, being one who see a feeling in feelings, strenuous, thoughtful, mindful, removing avarice and grief in the world.\\

\pali{\fbox{\stepcounter{sennum}\arabic{sennum}} Yasmi\d m samaye, bhikkhave, bhikkhu (9) `cittapa\d tisa\d mved\=i assasiss\=am\=i'ti sikkhati, `cittapa\d tisa\d mved\=i passasiss\=am\=i'ti sikkhati; (10) `abhippamodaya\d m citta\d m assasiss\=am\=i'ti sikkhati, `abhippamodaya\d m citta\d m passasiss\=am\=i'ti sikkhati; (11) `sam\=adaha\d m citta\d m assasiss\=am\=i'ti sikkhati, `sam\=adaha\d m citta\d m passasiss\=am\=i'ti sikkhati; (12) `vimocaya\d m citta\d m assasiss\=am\=i'ti sikkhati, `vimocaya\d m citta\d m passasiss\=am\=i'ti sikkhati; \fbox{\stepcounter{sennum}\arabic{sennum}} citte citt\=anupass\=i, bhikkhave, tasmi\d m samaye bhikkhu viharati \=at\=ap\=i sampaj\=ano satim\=a vineyya loke abhijjh\=adomanassa\d m. \fbox{\stepcounter{sennum}\arabic{sennum}} N\=aha\d m, bhikkhave, mu\d t\d thassatissa asampaj\=anassa \=an\=ap\=anassati\d m vad\=ami. Tasm\=atiha, bhikkhave, citte citt\=anupass\=i tasmi\d m samaye bhikkhu viharati \=at\=ap\=i sampaj\=ano satim\=a vineyya loke abhijjh\=adomanassa\d m.}

\addtocounter{sennum}{-3}
$\triangleright$ \fbox{\stepcounter{sennum}\arabic{sennum}} In which occasion, monks, a monk (9) learns thus ``[I], being one who feels the mind, will breathe in.'' [He] learns thus ``[I], being one who feels the mind, will breathe out.'' (10) [He] learns thus ``[I], [when] making the mind rejoiced, will breathe in.'' [He] learns thus ``[I], [when] making the mind rejoiced, will breathe out.'' (11) [He] learns thus ``[I], [when] concentrating the mind, will breathe in.'' [He] learns thus ``[I], [when] concentrating the mind, will breathe out.'' (12) [He] learns thus ``[I], [when] making the mind free, will breathe in.'' [He] learns thus ``[I], [when] making the mind free, will breathe out.'' \fbox{\stepcounter{sennum}\arabic{sennum}} In that occasion, monks, [that] monk lives, being one who see a mind in the mind, strenuous, thoughtful, mindful, removing avarice and grief in the world. \fbox{\stepcounter{sennum}\arabic{sennum}} Monks, I do not tell the mindfulness of breathing to one forgetful [and] thoughtless. Therefore, monks, in that occasion [that] monk lives, being one who see a mind in the mind, strenuous, thoughtful, mindful, removing avarice and grief in the world.\\

\begin{longtable}[c]{|p{0.9\linewidth}|}
\hline
\hspace{5mm}\small Unlike other groups which have plural forms of the objects of attention, i.e.\ \pali{k\=ayesu, veden\=asu,} and \pali{dhammesu} below, here only singular form is found, \pali{citte citt\=anupass\=i}.\\
\hspace{5mm}\small This implies that the mind is meant to be only one thing, or one faculty. What does it mean by `to see a mind in the mind' then? The commentary does not clarify this point. It can be understood that even though the mind is counted as one entity, it arises momentarily. So, `to see a mind in the mind' can mean `to see a momentary mind in the mind.' That is my shoehorned interpretation. This in fact is the point the Abhidhammic movement tries to make.\\
\hline
\end{longtable}

\pali{\fbox{\stepcounter{sennum}\arabic{sennum}} Yasmi\d m samaye, bhikkhave, bhikkhu (13) `anicc\=anupass\=i assasiss\=am\=i'ti sikkhati, `anicc\=anupass\=i passasiss\=am\=i'ti sikkhati; (14) `vir\=ag\=anupass\=i assasiss\=am\=i'ti sikkhati, `vir\=ag\=anupass\=i passasiss\=am\=i'ti sikkhati; (15) `nirodh\=anupass\=i assasiss\=am\=i'ti sikkhati, `nirodh\=anupass\=i passasiss\=am\=i'ti sikkhati; \ (16) `pa\d tinissagg\=anupass\=i assasiss\=am\=i'ti sikkhati, `pa\d tinissagg\=anupass\=i passasiss\=am\=i'ti sikkhati; \fbox{\stepcounter{sennum}\arabic{sennum}} dhammesu dhamm\=anupass\=i, bhikkhave, tasmi\d m samaye bhikkhu viharati \=at\=ap\=i sampaj\=ano satim\=a vineyya loke abhijjh\=adomanassa\d m. \fbox{\stepcounter{sennum}\arabic{sennum}} So ya\d m ta\d m abhijjh\=adomanass\=ana\d m pah\=ana\d m ta\d m pa\~n\~n\=aya disv\=a s\=adhuka\d m ajjhupekkhit\=a hoti. Tasm\=atiha, bhikkhave, dhammesu dhamm\=anupass\=i tasmi\d m samaye bhikkhu viharati \=at\=ap\=i sampaj\=ano satim\=a vineyya loke abhijjh\=adomanassa\d m. \fbox{\stepcounter{sennum}\arabic{sennum}} Eva\d m bh\=avit\=a kho, bhikkhave, \=an\=ap\=anassati eva\d m bahul\=ikat\=a catt\=aro satipa\d t\d th\=ane parip\=ureti.}

\addtocounter{sennum}{-4}
$\triangleright$ \fbox{\stepcounter{sennum}\arabic{sennum}} In which occasion, monks, a monk (13) learns thus ``[I], being one who sees impermanence, will breathe in.'' [He] learns thus ``[I], being one who sees impermanence, will breathe out.'' (14) [He] learns thus ``[I], being one who sees dispassionateness, will breathe in.'' [He] learns thus ``[I], being one who sees dispassionateness, will breathe out.'' (15) [He] learns thus ``[I], being one who sees cessation, will breathe in.'' [He] learns thus ``[I], being one who sees cessation, will breathe out.'' (16) [He] learns thus ``[I], being one who sees abandonment, will breathe in.'' [He] learns thus ``[I], being one who sees abandonment, will breathe out.'' \fbox{\stepcounter{sennum}\arabic{sennum}} In that occasion, monks, [that] monk lives, being one who see a dhamma in dhammas, strenuous, thoughtful, mindful, removing avarice and grief in the world. \fbox{\stepcounter{sennum}\arabic{sennum}} He, having seen by wisdom that removal of avarice and grief, becomes indifferent properly. Therefore, monks, in that occasion [that] monk lives, being one who see a dhamma in dhammas, strenuous, thoughtful, mindful, removing avarice and grief in the world. \fbox{\stepcounter{sennum}\arabic{sennum}} In that way, monks, mindfulness of breathing, cultivated, in that way, having been practiced frequently, fulfills the four foundations of mindfulness.\\

\newpage
\begin{longtable}[c]{|p{0.9\linewidth}|}
\hline
\hspace{5mm}\small In the third cut, we find a \pali{ya-ta} pattern. The second \pali{ta\d m} correlates to \pali{ya\d m}. We can translate it precisely as ``Which removal of that avarice-and-grief [exists], he, having seen that [removal] by wisdom, becomes indifferent properly.'' And I think \pali{ajjhupekkhit\=a} should be \pali{ajjhupekkhito} instead, to be agreeable with the subject `\pali{So}.'\\
\hline
\end{longtable}

\pali{\fbox{\stepcounter{sennum}\arabic{sennum}} 150.\ Katha\d m bh\=avit\=a ca, bhikkhave, catt\=aro satipa\d t\d th\=an\=a katha\d m bahul\=ikat\=a satta bojjha\.nge parip\=urenti? \fbox{\stepcounter{sennum}\arabic{sennum}} Yasmi\d m sam\-aye, bhikkhave, bhikkhu k\=aye k\=ay\=anupass\=i viharati \=at\=ap\=i sampaj\=ano satim\=a vineyya loke abhijjh\=adomanassa\d m, upa\d t\d thit\=assa tasmi\d m samaye sati hoti asammu\d t\d th\=a. \fbox{\stepcounter{sennum}\arabic{sennum}} Yasmi\d m samaye, bhikkh\-ave, bhikkhuno upa\d t\d thit\=a sati hoti asammu\d t\d th\=a, satisambojjha\.ngo tasmi\d m samaye bhikkhuno \=araddho hoti. \fbox{\stepcounter{sennum}\arabic{sennum}} Satisambojjha\.nga\d m tasmi\d m samaye bhikkhu bh\=aveti, satisambojjha\.ngo tasmi\d m samaye bhikkhuno bh\=avan\=ap\=arip\=uri\d m gacchati.}

\addtocounter{sennum}{-4}
$\triangleright$ \fbox{\stepcounter{sennum}\arabic{sennum}} 150.\ Monks, how were the four foundations of mindfulness cultivated? How do [they], having been practiced frequently, fulfill the seven factors of enlightenment? \fbox{\stepcounter{sennum}\arabic{sennum}} In which occasion, monks, a monk lives, being one who see a body in the body, strenuous, thoughtful, mindful, removing avarice and grief in the world, in that occasion [his] established mindfulness is not forgetful. \fbox{\stepcounter{sennum}\arabic{sennum}} In which occasion, monks, the monk's established mindfulness is not forgetful, in that occasion the monk's mindfulness-factor of the enlightenment gets started. \fbox{\stepcounter{sennum}\arabic{sennum}} In that occasion, the monk cultivates the mindfulness-factor of the enlightenment. In that occasion, the monk's mindfulness-factor of the enlightenment becomes fully developed.\\

\begin{longtable}[c]{|p{0.9\linewidth}|}
\hline
\hspace{5mm}\small From here towards the end, the seven fators of enlightenment are related, and described repetitively. Here is a quick list of the factors:\\
(1) mindfulness (\pali{sati})\\
(2) investigation of dhammas (\pali{dhammavicaya})\\
(3) perseverance, energy (\pali{viriya})\\
(4) rapture (\pali{p\=iti})\\
(5) tranquility (\pali{passaddhi})\\
(6) concentration (\pali{sam\=adhi})\\
(7) equanimity (\pali{upekkh\=a})\\
\hline
\end{longtable}

\pali{\fbox{\stepcounter{sennum}\arabic{sennum}} So tath\=asato viharanto ta\d m dhamma\d m pa\~n\~n\=aya pavicinati pavicayati pariv\=ima\d msa\d m \=apajjati. \fbox{\stepcounter{sennum}\arabic{sennum}} Yasmi\d m samaye, bhikkhave, bhikkhu tath\=asato viharanto ta\d m dhamma\d m pa\~n\~n\=aya pavicinati pavicayati pariv\=ima\d msa\d m \=apajjati, dhammavicayasambojjha\.ngo tasmi\d m samaye bhikkhuno \=araddho hoti, dhammavicayasambojjha\.nga\d m tasmi\d m samaye bhikkhu bh\=aveti, dhammavicayasambojjha\.ngo tasmi\d m samaye bhikkhuno bh\=avan\=ap\=arip\=uri\d m gacchati.}

\addtocounter{sennum}{-2}
$\triangleright$ \fbox{\stepcounter{sennum}\arabic{sennum}} That [monk], [when] living mindfully, examines, investigates, undertakes a thorough examination that dhamma by wisdom. \fbox{\stepcounter{sennum}\arabic{sennum}} In which occasion, monks, the monk, living mindfully, examines, investigates, undertakes a thorough examination that dhamma by wisdom, in that occasion the monk's dhamma-investigation-factor of the enlightenment gets started. In that occasion, the monk cultivates the dhamma-investigation-factor of the enlightenment. In that occasion, the monk's dhamma-investigation-factor of the enlightenment becomes fully developed.\\

\begin{longtable}[c]{|p{0.9\linewidth}|}
\hline
\hspace{5mm}\small The commentary explains that ``\pali{Pavicinat\=i'ti anicc\=adivasena pavicinati. Itara\d m padadvaya\d m etasseva vevacana\d m.}'' So, the investigation targets the three characteristics, i.e.\ impermanence, unbearableness, and insubstantiality. Other two terms are just its synonyms. In other editions, \pali{pavicayati} becomes \pali{pavicarati}.\\
\hspace{5mm}\small As I often noted, \pali{dhamma} in P\=ali is always ambiguous. In the explanation above, it probably means the phenomenon in the field of awareness that is subject to the three characteristics.\\
\hline
\end{longtable}

\pali{\fbox{\stepcounter{sennum}\arabic{sennum}} Tassa ta\d m dhamma\d m pa\~n\~n\=aya pavicinato pavicayato pariv\=ima\d msa\d m \=apajjato \=araddha\d m hoti v\=iriya\d m asall\=ina\d m. \fbox{\stepcounter{sennum}\arabic{sennum}} Yasmi\d m samaye, bhikkhave, bhikkhuno ta\d m dhamma\d m pa\~n\~n\=aya pavicinato pavicayato pariv\=ima\d msa\d m \=apajjato \=araddha\d m hoti v\=iriya\d m asall\=ina\d m, v\=iriyasambojjha\.ngo tasmi\d m samaye bhikkhuno \=araddho hoti, v\=iriyasambojjha\.nga\d m tasmi\d m samaye bhikkhu bh\=aveti, v\=iriyasambojjha\.ngo tasmi\d m samaye bhikkhuno bh\=avan\=ap\=arip\=uri\d m gacchati.}

\addtocounter{sennum}{-2}
$\triangleright$ \fbox{\stepcounter{sennum}\arabic{sennum}} When that dhamma was examined, investigated, undertaken a thorough examination by wisdom, his perseverance gets started actively. \fbox{\stepcounter{sennum}\arabic{sennum}} In which occasion, monks, when that dhamma was examined, investigated, undertaken a thorough examination by wisdom, [and] the monk's perseverance gets started actively, in that occasion the monk's perseverance-factor of enlightenment gets started. In that occasion, the monk cultivates the perseverance-factor of the enlightenment. In that occasion, the monk's perseverance-factor of the enlightenment becomes fully developed.\\

\pali{\fbox{\stepcounter{sennum}\arabic{sennum}} \=Araddhav\=iriyassa uppajjati p\=iti nir\=amis\=a. \fbox{\stepcounter{sennum}\arabic{sennum}} Yasmi\d m samaye, bhikkhave, bhikkhuno \=araddhav\=iriyassa uppajjati p\=iti nir\=a\-mis\=a, p\=itisambojjha\.ngo tasmi\d m samaye bhikkhuno \=araddho hoti, p\=itisambojjha\.nga\d m tasmi\d m samaye bhikkhu bh\=aveti, p\=itisambojjha\.ngo tasmi\d m samaye bhikkhuno bh\=avan\=ap\=arip\=uri\d m gacchati.}

\addtocounter{sennum}{-2}
$\triangleright$ \fbox{\stepcounter{sennum}\arabic{sennum}} When the perseverance got started, a rapture without defilement arises. \fbox{\stepcounter{sennum}\arabic{sennum}} In which occasion, monks, when the perseverance got started, [and] the monk's rapture without defilement arises, in that occasion the monk's rapture-factor of enlightenment gets started. In that occasion, the monk cultivates the rapture-factor of the enlightenment. In that occasion, the monk's rapture-factor of the enlightenment becomes fully developed.\\

\begin{longtable}[c]{|p{0.9\linewidth}|}
\hline
\hspace{5mm}\small As an absolute construction, we can separate \pali{\=araddhav\=iriyassa} as \pali{\=araddhassa v\=iriyassa}. This is in passive voice. In the next occurrence of this, I take \pali{bhikkhuno} as a genitive related to \pali{p\=iti}, so it does not relate to the compound. Also, \pali{bhikkhuno} can be dative, thus ``the rapture arises to the monk.'' \\
\hspace{5mm}\small In the commentary, \pali{nir\=amis\=a} means `without defilement' (\pali{Nir\=amis\=a'ti nikkiles\=a}).\\
\hline
\end{longtable}

\pali{\fbox{\stepcounter{sennum}\arabic{sennum}} P\=itimanassa k\=ayopi passambhati, cittampi passambhati. \fbox{\stepcounter{sennum}\arabic{sennum}} Yasmi\d m samaye, bhikkhave, bhikkhuno p\=itimanassa k\=ayopi passambhati, cittampi passambhati, passaddhisambojjha\.ngo tasmi\d m samaye bhikkhuno \=araddho hoti, passaddhisambojjha\.nga\d m tasmi\d m samaye bhikkhu bh\=aveti, passaddhisambojjha\.ngo tasmi\d m samaye bhikkhuno bh\=avan\=ap\=arip\=uri\d m gacchati.}

\addtocounter{sennum}{-2}
$\triangleright$ \fbox{\stepcounter{sennum}\arabic{sennum}} When rapture in the mind [happens], the body calms down, and the mind calms down. \fbox{\stepcounter{sennum}\arabic{sennum}} In which occasion, monks, when rapture in the mind [happens], the monk's body and mind calms down, in that occasion the tranquility-factor of enlightenment gets started. In that occasion, the monk cultivates the tranquility-factor of the enlightenment. In that occasion, the monk's tranquility-factor of the enlightenment becomes fully developed.\\

\pali{\fbox{\stepcounter{sennum}\arabic{sennum}} Passaddhak\=ayassa sukhino citta\d m sam\=adhiyati. \fbox{\stepcounter{sennum}\arabic{sennum}} Yasmi\d m samaye, bhikkhave, bhikkhuno passaddhak\=ayassa sukhino citta\d m sam\=adhiyati, sam\=adhisambojjha\.ngo tasmi\d m samaye bhikkhuno \=araddho hoti, sam\=adhisambojjha\.nga\d m tasmi\d m samaye bhikkhu bh\=aveti, sam\=adhisambojjha\.ngo tasmi\d m samaye bhikkhuno bh\=avan\=ap\=arip\=uri\d m gacchati.}

\addtocounter{sennum}{-2}
$\triangleright$ \fbox{\stepcounter{sennum}\arabic{sennum}} When the body calmed down, [and the monk is] happy, the mind gets concentrated. \fbox{\stepcounter{sennum}\arabic{sennum}} In which occasion, monks, when the body calmed down, [and the monk is] happy, the monk's mind gets concentrated, in that occasion concentration-factor of enlightenment gets started. In that occasion, the monk cultivates the concentration-factor of the enlightenment. In that occasion, the monk's concentration-factor of the enlightenment becomes fully developed.\\

\pali{\fbox{\stepcounter{sennum}\arabic{sennum}} So tath\=asam\=ahita\d m citta\d m s\=adhuka\d m ajjhupekkhit\=a hoti. \fbox{\stepcounter{sennum}\arabic{sennum}} Yasmi\d m samaye, bhikkhave, bhikkhu tath\=asam\=ahita\d m citta\d m s\=adhuka\d m ajjhupekkhit\=a hoti, upekkh\=asambojjha\.ngo tasmi\d m samaye bhikkhuno \=araddho hoti, upekkh\=asambojjha\.nga\d m tasmi\d m samaye bhikkhu bh\=aveti, upekkh\=asambojjha\.ngo tasmi\d m samaye bhikkhuno bh\=avan\=ap\=arip\=uri\d m gacchati.}

\addtocounter{sennum}{-2}
$\triangleright$ \fbox{\stepcounter{sennum}\arabic{sennum}} When the mind got concentrated as such, that [monk] becomes indifferent properly. \fbox{\stepcounter{sennum}\arabic{sennum}} In which occasion, monks, when the mind got concentrated as such, [and] the monk becomes indifferent properly, in that occasion equanimity-factor of enlightenment gets started. In that occasion, the monk cultivates the equanimity-factor of the enlightenment. In that occasion, the monk's equanimity-factor of the enlightenment becomes fully developed.\\

\pali{\fbox{\stepcounter{sennum}\arabic{sennum}} 151.\ Yasmi\d m samaye, bhikkhave, bhikkhu vedan\=asu\ldots pe\ldots \linebreak citte\ldots dhammesu dhamm\=anupass\=i viharati \=at\=ap\=i sampaj\=ano satim\=a vineyya loke abhijjh\=adomanassa\d m, upa\d t\d thit\=assa tasmi\d m samaye sati hoti asammu\d t\d th\=a. Yasmi\d m samaye, bhikkhave, bhikkhuno upa\d t\d thit\=a sati hoti asammu\d t\d th\=a, satisambojjha\.ngo tasmi\d m samaye bhikkhuno \=araddho hoti, satisambojjha\.nga\d m tasmi\d m samaye bhik\-khu bh\=aveti, satisambojjha\.ngo tasmi\d m samaye bhikkhuno bh\=avan\=ap\=arip\=uri\d m gacchati.}\par
\pali{So tath\=asato viharanto ta\d m dhamma\d m pa\~n\~n\=aya pavicinati pavic\-ayati pariv\=ima\d msa\d m \=apajjati. Yasmi\d m samaye, bhikkhave, bhikkhu tath\=asato viharanto ta\d m dhamma\d m pa\~n\~n\=aya pavicinati pavicayati pariv\=ima\d msa\d m \=apajjati, dhammavicayasambojjha\.ngo tasmi\d m samaye bhikkhuno \=araddho hoti, dhammavicayasambojjha\.nga\d m tasmi\d m samaye bhikkhu bh\=aveti, dhammavicayasambojjha\.ngo tasmi\d m samaye bhikkhuno bh\=avan\=ap\=arip\=uri\d m gacchati.}\par
\pali{Tassa ta\d m dhamma\d m pa\~n\~n\=aya pavicinato pavicayato pariv\=ima\d msa\d m \=apajjato \=araddha\d m hoti v\=iriya\d m asall\=ina\d m. Yasmi\d m samaye, bhikkhave, bhikkhuno ta\d m dhamma\d m pa\~n\~n\=aya pavicinato pavicayato pariv\=ima\d msa\d m \=apajjato \=araddha\d m hoti v\=iriya\d m asall\=ina\d m, v\=iriyasambojjha\.ngo tasmi\d m samaye bhikkhuno \=araddho hoti, v\=iriyasambojjha\.nga\d m tasmi\d m samaye bhikkhu bh\=aveti, v\=iriyasambojjha\.ngo tasmi\d m samaye bhikkhuno bh\=avan\=ap\=arip\=uri\d m gacchati.}\par
\pali{\=Araddhav\=iriyassa uppajjati p\=iti nir\=amis\=a. Yasmi\d m samaye, bhikkhave, bhikkhuno \=araddhav\=iriyassa uppajjati p\=iti nir\=amis\=a, p\=itisambojjha\.ngo tasmi\d m samaye bhikkhuno \=araddho hoti, p\=itisambojjha\.nga\d m tasmi\d m samaye bhikkhu bh\=aveti, p\=itisambojjha\.ngo tasmi\d m samaye bhikkhuno bh\=avan\=ap\=arip\=uri\d m gacchati.}\par
\pali{P\=itimanassa k\=ayopi passambhati, cittampi passambhati. Yasmi\d m samaye, bhikkhave, bhikkhuno p\=itimanassa k\=ayopi passambhati, cittampi passambhati, passaddhisambojjha\.ngo tasmi\d m samaye bhikkhuno \=araddho hoti, passaddhisambojjha\.nga\d m tasmi\d m samaye bhikkhu bh\=aveti, passaddhisambojjha\.ngo tasmi\d m samaye bhikkhuno bh\=avan\=ap\=arip\=uri\d m gacchati.}\par
\pali{Passaddhak\=ayassa sukhino citta\d m sam\=adhiyati. Yasmi\d m samaye, bhikkhave, bhikkhuno passaddhak\=ayassa sukhino citta\d m sam\=adhiyati, sam\=adhisambojjha\.ngo tasmi\d m samaye bhikkhuno \=araddho hoti, sam\=adhisambojjha\.nga\d m tasmi\d m samaye bhikkhu bh\=aveti, sam\-\=adhisambojjha\.ngo tasmi\d m samaye bhikkhuno bh\=avan\=ap\=arip\=uri\d m gacchati.}\par
\pali{So tath\=asam\=ahita\d m citta\d m s\=adhuka\d m ajjhupekkhit\=a hoti. Yasmi\d m samaye, bhikkhave, bhikkhu tath\=asam\=ahita\d m citta\d m s\=adhuka\d m ajjhupekkhit\=a hoti, upekkh\=asambojjha\.ngo tasmi\d m samaye bhikkhuno \=araddho hoti, upekkh\=asambojjha\.nga\d m tasmi\d m samaye bhikkhu bh\=aveti, upekkh\=asambojjha\.ngo tasmi\d m samaye bhikkhuno bh\=avan\=ap\=arip\=uri\d m gacchati. \fbox{\stepcounter{sennum}\arabic{sennum}} Eva\d m bh\=avit\=a kho, bhikkhave, catt\=aro satipa\d t\d th\=an\=a eva\d m bahul\=ikat\=a satta sambojjha\.nge parip\=urenti.}

\addtocounter{sennum}{-2}
$\triangleright$ \fbox{\stepcounter{sennum}\arabic{sennum}} 151.\ In which occasion, monks, a monk lives, being one who see a feeling in feeling, \ldots a mind in the mind, \ldots a dhamma in dhammas, strenuous, thoughtful, mindful, removing avarice and grief in the world, in that occasion [his] established mindfulness is not forgetful. \ldots\ In that occasion, the monk's equanimity-factor of the enlightenment becomes fully developed. \fbox{\stepcounter{sennum}\arabic{sennum}} In that way, monks, the four foundations of mindfulness was cultivated. In that way, having been practiced frequently, [they] fulfill the seven factors of enlightenment.\\

\pali{\fbox{\stepcounter{sennum}\arabic{sennum}} 152.\ Katha\d m bh\=avit\=a ca, bhikkhave, satta bojjha\.ng\=a katha\d m bahul\=ikat\=a vijj\=avimutti\d m parip\=urenti? \fbox{\stepcounter{sennum}\arabic{sennum}} Idha, bhikkhave, bhikkhu satisambojjha\.nga\d m bh\=aveti vivekanissita\d m vir\=aganissita\d m niro\-dhanissita\d m vossaggapari\d n\=ami\d m. Dhammavicayasambojjha\.nga\d m bh\=aveti\ldots pe\ldots v\=iriyasambojjha\.nga\d m bh\=aveti\ldots p\=itisambojjha\.nga\d m bh\=a\-veti\ldots passaddhisambojjha\.nga\d m bh\=aveti\ldots sam\=adhisambojjha\.nga\d m \linebreak bh\=aveti\ldots upekkh\=asambojjha\.nga\d m bh\=aveti vivekanissita\d m vir\=aganissita\d m nirodhanissita\d m vossaggapari\d n\=ami\d m. \fbox{\stepcounter{sennum}\arabic{sennum}} Eva\d m bh\=avit\=a kho, bhikkhave, satta bojjha\.ng\=a eva\d m bahul\=ikat\=a vijj\=avimutti\d m parip\=urent\=i'ti.}

\addtocounter{sennum}{-3}
$\triangleright$ \fbox{\stepcounter{sennum}\arabic{sennum}} 152.\ Monks, how were the seven factors of enlightenment cultivated? How do [they], being practiced frequently, fulfill insight and liberation? \fbox{\stepcounter{sennum}\arabic{sennum}} Monks, a monk in this [religion] cultivates the mindfulness-factor, the dhamma-investigation-factor, the perseverance-factor, the rapture-factor, the tranquility-factor, the concentration-factor, [and] the equanimity-factor of enlightenment, by means of seclusion, dispassionateness, [and] cessasion, resulting in relinquishment. \fbox{\stepcounter{sennum}\arabic{sennum}} In that way, monks, the seven factors of enlightenment was cultivated. In that way [they], having been practiced frequently, fulfill insight and liberation.\\

\pali{\fbox{\stepcounter{sennum}\arabic{sennum}} Idamavoca bhagav\=a. Attaman\=a te bhikkh\=u bhagavato bh\=asita\d m abhinandun'ti.}

\addtocounter{sennum}{-1}
$\triangleright$ \fbox{\stepcounter{sennum}\arabic{sennum}} [When] the Blessed One said this [teaching], those delighted monks rejoiced in the saying of the Blessed One.\\

\pali{\fbox{\stepcounter{sennum}\arabic{sennum}} \=An\=ap\=anassatisutta\d m ni\d t\d thita\d m a\d t\d thama\d m.}

\addtocounter{sennum}{-1}
$\triangleright$ \fbox{\stepcounter{sennum}\arabic{sennum}} A discourse concerning mindfulness of breathing, the eight [sutta], was finished.\\

\phantomsection
\addcontentsline{toc}{section}{Conclusion and discussion}
\section*{3.\ Conclusion and discussion}

After reading this sutta, we now have an idea how an old sutta was constructed. The substantial part of this sutta is relatively small. The extra tedious repetitions, related someway to the gist, seem to be insertions for certain purposes, e.g.\ to make meditation aid or to valorize the discourse. The sutta by and large is readable. Some difficulties about technicality can be tracked down by an effort.

For those who look for a solid meditation guide, this information is helpful, because it likely came from real experiences. The instruction provides a reliable map when one gets on the practice. Some adaptation is to be applied, however, because in real life mental phenomena are far richer than those mentioned.

One result of this discursive practice is the assertion of the role of meditation to liberation. By the account, practicing mindfulness leads directly to enlightenment, unlike what we have seen in the first and second sermon. In those suttas, the monks got understanding and liberation after just listening to the teaching. No meditation was mentioned there. In many early suttas, the picture is also depicted as such.

This topic is worth digging into if it is of interest to some readers. Scholarly literature on the issue is not hard to find.\footnote{See \citealp{wynne:origin}, for example.} To my view, despite my preference towards rationality, I hold that meditation is essential to our healthy living. Meditation can regulate our emotions in the way that reasoning alone cannot. We can strip all ideologies laden in the practice, and use it outside the religion's context. Only when meditation makes the practitioners too obsessed, it can be futile or even harmful. How about the transcendental goal of meditation then? That depends on how `transcendental' is defined. Whatever that means, a moderate and healthy meditation is always a good practice.
