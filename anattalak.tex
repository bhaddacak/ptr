\chapter{The second sermon}\label{chap:anattalak}

After we have read the first sermon of the Buddha, it is advisable to read the second sermon as well, for a couple of reasons. First, the second sermon, \pali{Anattalakkha\d nasutta} by name, is one of the most important discourses in the whole canon, philosophically speaking. If you want to know what Buddhism is all about in a nutshell, read the second sermon. It is the first sermon that gave the Dhamma eye to Ko\d n\d da\~n\~na, but the second sermon made him and all his friends liberated. This sutta represents the core of Buddhism that can distinguish it from other religions philosophically.

Second, the second sermon is really easy to read, easier even than the first one, and pretty short. So, we can go through it without any help from commentaries.

To encourage the learners to make an effort to read this sutta by themselves first, I change the format of presentation as we shall see below.

\phantomsection
\addcontentsline{toc}{section}{Pre-reading introduction}
\section*{1.\ Pre-reading introduction}

\paragraph*{About the text} Like the first sermon, \pali{Anattalakkha\d nasutta} has two instances in the P\=ali collection: in the Vinaya (Mv\,1.20--24) and in Sa\d myuttanik\=aya (S3\,59, SN\,22). We will use the text mainly from the latter. In Khandhavagga of Sa\d myuttanik\=aya, the sutta is the seventh (of 10) in the sixth (of 15) group of the first (of 13) collection. The repetitive form of the text suggests that it is a remnant of the oral handover.

\paragraph*{About the author} Like the first sermon, the apparent sender of this text was the Buddha, but the effective sender was probably the Sangha at the time of an early compilation. The text producer was the compilers at the council.

\paragraph*{About the audience} The sutta is supposed to be used within the religion. So, the audience are mainly Buddhist monks and adherents. This sutta is also used in ceremonial contexts, particularly in funerals, by Buddhists nowadays. Some may argue that the audience of the sutta was the group of five monks. That is not the case when we consider the textual form. It is a retelling of that event, so the five monks did not hear what we see in the content.

\paragraph*{About time and place} The venue of this event is the same as the first sermon, hence the Isipatana Deer Park. The time was not clear, but it seems not long from the first delivery, perhaps a few days or weeks; but unlikely to be the same day, because it took time when other monks, apart from Ko\d n\d da\~n\~na, learned the first sermon and understood it.\footnote{See Mv\,1.18--19.}

\paragraph*{About motives} To preserve and assert the doctrine of nonself (\pali{anatt\=a}) seems to be the primary motive of the production of this text.

\paragraph*{About text function} Chiefly, the text perform informative function, both in historical and doctrinal dimension. When the text is used in rituals, it performs ceremonial function by which the meaning of the sutta is hardly taken into consideration.

\phantomsection
\addcontentsline{toc}{section}{Reading with a draft translation}
\section*{2.\ Reading with a draft translation}

This time I will separate P\=ali text from its translation. Sentences are broken down and marked with numbers to ease the navigation.

\bigskip
\begin{center}
\textbf{\pali{Anattalakkha\d nasutta\d m}}\par
\end{center}

\setcounter{sennum}{0}
\pali{\fbox{\stepcounter{sennum}\arabic{sennum}} Eka\d m samaya\d m bhagav\=a b\=ar\=a\d nasiya\d m viharati isipatane migad\=aye. \fbox{\stepcounter{sennum}\arabic{sennum}} Tatra kho bhagav\=a pa\~ncavaggiye bhikkh\=u \=amantesi -- `bhikkhavo'ti. \fbox{\stepcounter{sennum}\arabic{sennum}} `Bhadante'ti te bhikkh\=u bhagavato paccassosu\d m. Bhagav\=a etadavoca --}

\pali{\fbox{\stepcounter{sennum}\arabic{sennum}} R\=upa\d m, bhikkhave, anatt\=a. \fbox{\stepcounter{sennum}\arabic{sennum}} R\=upa\~nca hida\d m, bhikkhave, att\=a abhavissa, nayida\d m r\=upa\d m \=ab\=adh\=aya sa\d mvatteyya, labbhetha ca r\=upe -- `eva\d m me r\=upa\d m hotu, eva\d m me r\=upa\d m m\=a ahos\=i'ti. \fbox{\stepcounter{sennum}\arabic{sennum}} Yasm\=a ca kho, bhikkhave, r\=upa\d m anatt\=a, tasm\=a r\=upa\d m \=ab\=adh\=aya sa\d mvattati, na ca labbhati r\=upe -- `eva\d m me r\=upa\d m hotu, eva\d m me r\=upa\d m m\=a ahos\=i'ti.}

\pali{\fbox{\stepcounter{sennum}\arabic{sennum}} Vedan\=a anatt\=a. Vedan\=a ca hida\d m, bhikkhave, att\=a abhavissa, nayida\d m vedan\=a \=ab\=adh\=aya sa\d mvatteyya, labbhetha ca vedan\=aya -- `eva\d m me vedan\=a hotu, eva\d m me vedan\=a m\=a ahos\=i'ti. Yasm\=a ca kho, bhikkhave, vedan\=a anatt\=a, tasm\=a vedan\=a \=ab\=adh\=aya sa\d mvattati, na ca labbhati vedan\=aya -- `eva\d m me vedan\=a hotu, eva\d m me vedan\=a m\=a ahos\=i'ti.}

\pali{\fbox{\stepcounter{sennum}\arabic{sennum}} Sa\~n\~n\=a anatt\=a. Sa\~n\~n\=a ca hida\d m, bhikkhave, att\=a abhavissa, nayida\d m sa\~n\~n\=a \=ab\=adh\=aya sa\d mvatteyya, labbhetha ca sa\~n\~n\=aya -- `eva\d m me sa\~n\~n\=a hotu, eva\d m me sa\~n\~n\=a m\=a ahos\=i'ti. Yasm\=a ca kho, bhikkhave, sa\~n\~n\=a anatt\=a, tasm\=a sa\~n\~n\=a \=ab\=adh\=aya sa\d mvattati, na ca labbhati sa\~n\~n\=aya -- `eva\d m me sa\~n\~n\=a hotu, eva\d m me sa\~n\~n\=a m\=a ahos\=i'ti.}

\pali{\fbox{\stepcounter{sennum}\arabic{sennum}} Sa\.nkh\=ar\=a anatt\=a. Sa\.nkh\=ar\=a ca hida\d m, bhikkhave, att\=a abhavissa\d msu, nayida\d m sa\.nkh\=ar\=a \=ab\=adh\=aya sa\d mvatteyyu\d m, labbhetha ca sa\.nkh\=aresu -- `eva\d m me sa\.nkh\=ar\=a hontu, eva\d m me sa\.nkh\=ar\=a m\=a ahesun'ti. Yasm\=a ca kho, bhikkhave, sa\.nkh\=ar\=a anatt\=a, tasm\=a sa\.nkh\=ar\=a \=ab\=adh\=aya sa\d mvattanti, na ca labbhati sa\.nkh\=aresu -- `eva\d m me sa\.nkh\=ar\=a hontu, eva\d m me sa\.nkh\=ar\=a m\=a ahesun'ti.}

\pali{\fbox{\stepcounter{sennum}\arabic{sennum}} Vi\~n\~n\=a\d na\d m anatt\=a. Vi\~n\~n\=a\d na\~nca hida\d m, bhikkhave, att\=a abhavissa, nayida\d m vi\~n\~n\=a\d na\d m \=ab\=adh\=aya sa\d mvatteyya, labbhetha ca vi\~n\~n\=a\d ne -- `eva\d m me vi\~n\~n\=a\d na\d m hotu, eva\d m me vi\~n\~n\=a\d na\d m m\=a ahos\=i'ti. Yasm\=a ca kho, bhikkhave, vi\~n\~n\=a\d na\d m anatt\=a, tasm\=a vi\~n\~n\=a\d na\d m \=ab\=adh\=aya sa\d mvattati, na ca labbhati vi\~n\~n\=a\d ne -- `eva\d m me vi\~n\~n\=a\d na\d m hotu, eva\d m me vi\~n\~n\=a\d na\d m m\=a ahos\=i'ti.}

\pali{\fbox{\stepcounter{sennum}\arabic{sennum}} Ta\d m ki\d m ma\~n\~natha, bhikkhave, r\=upa\d m nicca\d m v\=a anicca\d m v\=a'ti? \fbox{\stepcounter{sennum}\arabic{sennum}} Anicca\d m, bhante. \fbox{\stepcounter{sennum}\arabic{sennum}} Ya\d m pan\=anicca\d m dukkha\d m v\=a ta\d m sukha\d m v\=a'ti? \fbox{\stepcounter{sennum}\arabic{sennum}} Dukkha\d m, bhante. \fbox{\stepcounter{sennum}\arabic{sennum}} Ya\d m pan\=anicca\d m dukkha\d m vipari\d n\=amadhamma\d m, kalla\d m nu ta\d m samanupassitu\d m -- `eta\d m mama, esohamasmi, eso me att\=a'ti? \fbox{\stepcounter{sennum}\arabic{sennum}} No heta\d m, bhante.}

\pali{\fbox{\stepcounter{sennum}\arabic{sennum}} Ta\d m ki\d m ma\~n\~natha, bhikkhave, vedan\=a nicc\=a v\=a anicc\=a v\=a'ti? Anicc\=a, bhante. Ya\d m pan\=anicca\d m dukkha\d m v\=a ta\d m sukha\d m v\=a'ti? Dukkha\d m, bhante. Ya\d m pan\=anicca\d m dukkha\d m vipari\d n\=amadhamma\d m, kalla\d m nu ta\d m samanupassitu\d m -- `eta\d m mama, esohamasmi, eso me att\=a'ti? No heta\d m, bhante.}

\pali{\fbox{\stepcounter{sennum}\arabic{sennum}} Ta\d m ki\d m ma\~n\~natha, bhikkhave, sa\~n\~n\=a nicc\=a v\=a anicc\=a v\=a'ti? Anicc\=a, bhante. Ya\d m pan\=anicca\d m dukkha\d m v\=a ta\d m sukha\d m v\=a'ti? Dukkha\d m, bhante. Ya\d m pan\=anicca\d m dukkha\d m vipari\d n\=amadhamma\d m, kalla\d m nu ta\d m samanupassitu\d m -- `eta\d m mama, esohamasmi, eso me att\=a'ti? No heta\d m, bhante.}

\pali{\fbox{\stepcounter{sennum}\arabic{sennum}} Ta\d m ki\d m ma\~n\~natha, bhikkhave, sa\.nkh\=ar\=a nicc\=a v\=a anicc\=a v\=a'ti? Anicc\=a, bhante. Ya\d m pan\=anicca\d m dukkha\d m v\=a ta\d m sukha\d m v\=a'ti? Dukkha\d m, bhante. Ya\d m pan\=anicca\d m dukkha\d m vipari\d n\=amadhamma\d m, kalla\d m nu ta\d m samanupassitu\d m -- `eta\d m mama, esohamasmi, eso me att\=a'ti? No heta\d m, bhante.}

\pali{\fbox{\stepcounter{sennum}\arabic{sennum}} Ta\d m ki\d m ma\~n\~natha, bhikkhave, vi\~n\~n\=a\d na\d m nicca\d m v\=a anicca\d m v\=a'ti?  Anicca\d m, bhante. Ya\d m pan\=anicca\d m dukkha\d m v\=a ta\d m sukha\d m v\=a'ti? Dukkha\d m, bhante. Ya\d m pan\=anicca\d m dukkha\d m vipari\d n\=amadhamma\d m, kalla\d m nu ta\d m samanupassitu\d m -- `eta\d m mama, esohamasmi, eso me att\=a'ti? No heta\d m, bhante.}

\pali{\fbox{\stepcounter{sennum}\arabic{sennum}} Tasm\=atiha, bhikkhave, ya\d m ki\~nci r\=upa\d m at\=it\=an\=agatapaccuppanna\d m ajjhatta\d m v\=a bahiddh\=a v\=a o\d l\=arika\d m v\=a sukhuma\d m v\=a h\=ina\d m v\=a pa\d n\=ita\d m v\=a ya\d m d\=ure santike v\=a, sabba\d m r\=upa\d m -- `neta\d m mama, nesohamasmi, na meso att\=a'ti evameta\d m yath\=abh\=uta\d m sammappa\~n\~n\=aya da\d t\d thabba\d m.}

\pali{\fbox{\stepcounter{sennum}\arabic{sennum}} Y\=a k\=aci vedan\=a at\=it\=an\=agatapaccuppann\=a ajjhatt\=a v\=a bahiddh\=a v\=a o\d l\=arik\=a v\=a sukhum\=a v\=a h\=in\=a v\=a pa\d n\=it\=a v\=a y\=a d\=ure santike v\=a, sabb\=a vedan\=a -- `neta\d m mama, nesohamasmi, na meso att\=a'ti evameta\d m yath\=abh\=uta\d m sammappa\~n\~n\=aya da\d t\d thabba\d m.}

\pali{\fbox{\stepcounter{sennum}\arabic{sennum}} Y\=a k\=aci sa\~n\~n\=a at\=it\=an\=agatapaccuppann\=a ajjhatt\=a v\=a bahiddh\=a v\=a o\d l\=arik\=a v\=a sukhum\=a v\=a h\=in\=a v\=a pa\d n\=it\=a v\=a y\=a d\=ure santike v\=a, sabb\=a sa\~n\~n\=a -- `neta\d m mama, nesohamasmi, na meso att\=a'ti evameta\d m yath\=abh\=uta\d m sammappa\~n\~n\=aya da\d t\d thabba\d m.}

\pali{\fbox{\stepcounter{sennum}\arabic{sennum}} Ye keci sa\.nkh\=ar\=a at\=it\=an\=agatapaccuppann\=a ajjhatt\=a v\=a bahiddh\=a v\=a o\d l\=arik\=a v\=a sukhum\=a v\=a h\=in\=a v\=a pa\d n\=it\=a v\=a ye d\=ure santike v\=a, sabbe sa\~n\~n\=a -- `neta\d m mama, nesohamasmi, na meso att\=a'ti evameta\d m yath\=abh\=uta\d m sammappa\~n\~n\=aya da\d t\d thabba\d m.}

\pali{\fbox{\stepcounter{sennum}\arabic{sennum}} Ya\d m ki\~nci vi\~n\~n\=a\d na\d m at\=it\=an\=agatapaccuppanna\d m ajjhatta\d m v\=a bahiddh\=a v\=a o\d l\=arika\d m v\=a sukhuma\d m v\=a h\=ina\d m v\=a pa\d n\=ita\d m v\=a ya\d m d\=ure santike v\=a, sabba\d m vi\~n\~n\=a\d na\d m -- `neta\d m mama, nesohamasmi, na meso att\=a'ti evameta\d m yath\=abh\=uta\d m sammappa\~n\~n\=aya da\d t\d thabba\d m.}

\pali{\fbox{\stepcounter{sennum}\arabic{sennum}} Eva\d m passa\d m, bhikkhave, sutav\=a ariyas\=avako r\=upasmimpi nibbindati, vedan\=ayapi nibbindati, sa\~n\~n\=ayapi nibbindati, sa\.nkh\=aresupi nibbindati, vi\~n\~n\=a\d nasmimpi nibbindati. \fbox{\stepcounter{sennum}\arabic{sennum}} Nibbinda\d m virajjati; vir\=ag\=a vimuccati. \fbox{\stepcounter{sennum}\arabic{sennum}} Vimuttasmi\d m vimuttamiti \~n\=a\d na\d m hoti. \fbox{\stepcounter{sennum}\arabic{sennum}} `Kh\=i\d n\=a j\=ati, vusita\d m brahmacariya\d m, kata\d m kara\d n\=iya\d m, n\=apara\d m itthatt\=ay\=a'ti paj\=an\=at\=i'ti.}

\pali{\fbox{\stepcounter{sennum}\arabic{sennum}} Idamavoca bhagav\=a. Attaman\=a pa\~ncavaggiy\=a bhikkh\=u bhagavato bh\=asita\d m abhinandu\d m. \fbox{\stepcounter{sennum}\arabic{sennum}} Imasmi\~nca pana veyy\=akara\d nasmi\d m bha\~n\~nam\=ane pa\~ncavaggiy\=ana\d m bhikkh\=una\d m anup\=ad\=aya \=asav\-ehi citt\=ani vimucci\d ms\=u'ti.}

\pali{\fbox{\stepcounter{sennum}\arabic{sennum}} Sattama\d m.}

\bigskip
\begin{center}
\textbf{A discourse concerning\\the characteristic of nonself}
\end{center}

\setcounter{sennum}{0}
\fbox{\stepcounter{sennum}\arabic{sennum}} In one occasion, the Blessed One lives in Isipatana Deer Park, B\=ar\=a\d nas\=i. \fbox{\stepcounter{sennum}\arabic{sennum}} In that place, the Blessed One called the group of five monks, ``Monks!'' \fbox{\stepcounter{sennum}\arabic{sennum}} Those monks responded to the Blessed One, ``Sir!'' [Then] the Blessed One said this:

\fbox{\stepcounter{sennum}\arabic{sennum}} Form, monks, [is] nonself. \fbox{\stepcounter{sennum}\arabic{sennum}} If this form, monks, was the self, this form would not exist for illness, and [you] might obtain the form thus, ``May my form be in this way, may my form not have been in this way.'' \fbox{\stepcounter{sennum}\arabic{sennum}} Because, monks, form [is] nonself, therefore form exists for illness, and [one] does not obtain the form thus, ``May my form be in this way, may my form not have been in this way.''

\begin{longtable}[c]{|p{0.9\linewidth}|}
\hline
\hspace{5mm}\small Form in this context means the corporeal or physical body. Philosophically sensitive, \pali{anatt\=a} is hard to translate exactly. I use `nonself' because it retains the compound form of the word. It possibly means `not the self.' This has slightly different meaning when we put it as `void of self/soul/ego/personhood.'\\
\hspace{5mm}\small It is better not to over-translate a highly sensitive word. Try to keep the word structure, and if an explanation is needed, put it as a side text, or commentary. However, doing as such is really difficult, according to our cognitive inclination. We automatically over-translate things based on our beliefs and preferences. So, be aware cautiously.\\
\hspace{5mm}\dag\ \small We should take a look at this key sentence closely, ``\pali{r\=upa\d m ca hi ida\d m att\=a abhavissa, na ida\d m r\=upa\d m \=ab\=adh\=aya sa\d mvatteyya}.'' By its structure, this looks like a speculation or hypothesis, marked by conditional and optative mood.\footnote{According to speech act theory, as we have seen in Chapter \ref{chap:howmisunder}, this looks more like \emph{suggestive} or \emph{suppositive} act rather than \emph{assertive} act.} But the tradition does not allow that interpretation. Everything the Buddha said, or in the canon, is a truth claim.\\
\hspace{5mm}\dag\ \small Put it in logical form, this can be read, ``If form is the self, then it does not fall ill'' (if p then not q). This is equivalent to, ``If form falls ill, then it is nonself'' (if q then not p).\footnote{The full form of this argument can be as follows (\textit{modus tollens}):\\(1) If form is the self, then it does not fall ill. (if p then not q)\\(2) Form indeed falls ill. (q)\\(3) Therefore Form is not the self. (therefore not p)} Form is subject to illness because it is uncontrollable at will.\\
\hspace{5mm}\dag\ \small Then the Buddha's claim is this: ``Whatever is uncontrollable is nonself'' or ``What is counted as the self has to be controllable.'' See more discussion in the concluding part below.\\
\hspace{5mm}\dag\ \small Another point worth a note is that it is grammatical correct when \pali{ida\d m} (nt.) is used with \pali{r\=upa\d m} and \pali{vi\~n\~n\=ana\d m}, but not with \pali{vedan\=a} (f.), \pali{sa\~n\~n\=a} (f.), and \pali{sa\.nkh\=ar\=a} (pl.\,m.). This tells us that the textual form we see is a result of an attempt to put things into format by using a template. We will see things like this throughout the canon, particularly in the early layer of texts. Most of the time the arrangement looks agreeable, but sometimes mistakes are seen. In this case, it looks much like a copy-paste error in modern word processor.\\
\hspace{5mm}\small The use of \pali{labbhetha r\=upe} here looks a little strange. I take it as imperative mood of second person. It should be of third person to match \pali{labhati} in the following sentence. Perhaps, \pali{labbhetha} is a causative verb.\\
\hspace{5mm}\small However, I think, \pali{labbhati} and its variation here means much like ``[One] is able to get/make \ldots''\\
\hspace{5mm}\small Locative by form, \pali{r\=upe} is used as object or in accusative sense.\\
\hspace{5mm}\small The next four paragraphs are repetitive, so I do not make full translation of them.\\
\hline
\end{longtable}

\fbox{\stepcounter{sennum}\arabic{sennum}} Feeling, monks, [is] nonself. \ldots

\fbox{\stepcounter{sennum}\arabic{sennum}} Perception, monks, [is] nonself. \ldots

\fbox{\stepcounter{sennum}\arabic{sennum}} Mental formations, monks, [are] nonself. \ldots

\fbox{\stepcounter{sennum}\arabic{sennum}} Consciousness, monks, [is] nonself. \ldots

\fbox{\stepcounter{sennum}\arabic{sennum}} What do you think [about] that, monks, form is permanent or impermanent? \fbox{\stepcounter{sennum}\arabic{sennum}} Impermanent, sir. \fbox{\stepcounter{sennum}\arabic{sennum}} Then, which [is] impermanent, is that suffering or happiness? \fbox{\stepcounter{sennum}\arabic{sennum}} Suffering, sir. \fbox{\stepcounter{sennum}\arabic{sennum}} Which [is] impermanent, suffering, normally changing, is it suitable to see that thus, ``This [is] mine; I am this; This is my self''? \fbox{\stepcounter{sennum}\arabic{sennum}} No, it isn't, sir.

\fbox{\stepcounter{sennum}\arabic{sennum}} What do you think [about] that, monks, feeling is permanent or impermanent? \ldots

\fbox{\stepcounter{sennum}\arabic{sennum}} What do you think [about] that, monks, perception is permanent or impermanent? \ldots

\fbox{\stepcounter{sennum}\arabic{sennum}} What do you think [about] that, monks, mental formations are permanent or impermanent? \ldots

\fbox{\stepcounter{sennum}\arabic{sennum}} What do you think [about] that, monks, consciousness is permanent or impermanent? \ldots

\fbox{\stepcounter{sennum}\arabic{sennum}} Therefore, monks, which any form, in the past, future or present, internal or external, coarse or subtle, low or excellent, which [is] far or near, all that form should be seen with right knowledge as they really are thus, ``This is not mine. I am not this. This is not my self.''

\begin{longtable}[c]{|p{0.9\linewidth}|}
\hline
\hspace{5mm}\small The explanations are not always clear to us. The commentary of this sutta does not help much. It is the subcommentary of Mah\=avagga in the Vinaya that has further explanation.\\
\hspace{5mm}\small For example, concerning far vs.\ near, the subcommentary says thus, ``\pali{ya\d m sukhuma\d m, tadeva duppa\d tivijjhasabh\=avatt\=a d\=ure, itara\d m suppa\d tivijjhasabh\=avatt\=a santike}''\footnote{S\=aratthad\=ipan\=i, Mah\=avagga-\d t\=ik\=a 1.22} (Which [aggregate] is subtle, that condition difficult to understand is far, the other easy to understand is near).\\
\hspace{5mm}\small An attempt to explain this in the subcommentary does not help much either.\\
\hline
\end{longtable}

\fbox{\stepcounter{sennum}\arabic{sennum}} Which any feeling \ldots

\fbox{\stepcounter{sennum}\arabic{sennum}} Which any perception \ldots

\fbox{\stepcounter{sennum}\arabic{sennum}} Which any mental formations \ldots

\fbox{\stepcounter{sennum}\arabic{sennum}} Which any consciousness \ldots

\fbox{\stepcounter{sennum}\arabic{sennum}} Monks, a noble disciple, having heard, seeing as such, gets wearied of form, feeling, perception, mental formations, and consciousness. \fbox{\stepcounter{sennum}\arabic{sennum}} [When] getting wearied of [those], [he/she] detaches himself/herself [from those]. Because of the absence of desire, he/she is liberated. \fbox{\stepcounter{sennum}\arabic{sennum}} When one was liberated, there is knowledge thus, ``[One was] liberated.'' \fbox{\stepcounter{sennum}\arabic{sennum}} [One] knows clearly thus, ``[Further] birth was exhausted, the religious life was fulfilled, what should be done was done, [and] there is no other [work] for being here [on earth again]''

\begin{longtable}[c]{|p{0.9\linewidth}|}
\hline
\hspace{5mm}\small As an irregular noun, \pali{passanta} (one who is seeing) declines like \pali{gacchanta} (one who is going), hence \pali{passa\d m} for nom. See Appendix \externalref{B.4} of PNL.\\
\hline
\end{longtable}

\fbox{\stepcounter{sennum}\arabic{sennum}} The Blessed One said this. The delighted five monks rejoiced [in] the speech of the Blessed One. \fbox{\stepcounter{sennum}\arabic{sennum}} While this exposition was being said, the minds of the five monks were liberated from intoxicants because of detachment.''

\fbox{\stepcounter{sennum}\arabic{sennum}} The seventh [was finished].

\phantomsection
\addcontentsline{toc}{section}{Conclusion and discussion}
\section*{3.\ Conclusion and discussion}

At textual level, this text is easy to read. There is no difficult words to tackle with great effort. New learners should be happy with this instance. Only one problem for very new students is how to recognize and break down joining words. For example, \pali{nesohamasmi} is \pali{na + eso + aha\d m + asmi}. To sharpen this skill, you only have to go through many of texts. The skill is accumulative. Reviewing Appendix \externalref{D} of PNL can be helpful if someone is still baffled with this.

At conceptual level, this text can be a point of argumentation. Some interpret the notion of nonself just for negating the substantiality or realness of the five aggregates. That is to say, form is not our `real' self, so are consciousness and other mental states. By this interpretation, other `real' self is yet to be defined. This means there is possibly other kind of self apart from the five aggregates. This is not the orthodox view, though.

This problem of interpretation is perennial, perhaps from the Buddha's time. I do not want to go into this in detail. I just want to remind students of P\=ali and Buddhism that when you master the language to some degree, textual manipulation is at your fingertips. Please use it with responsibility. 

And one caveat for truth seekers, we easily get lost in a textual labyrinth. So, be aware clearly what you are looking for. Think it carefully when you try to find out what the absolute meaning of \pali{anatt\=a} is.

Although this is not a good place to discuss a philosophical issue, but the notion of \pali{anatt\=a} is so important that we should ponder upon this after we read the text. Many scholars often point out that the Buddha proposed the \pali{anatt\=a} thesis in order to response to \pali{\=atman} in Upanisadic tradition. As we have read the sutta, we find that the main idea why the five aggregates cannot be counted as self is they are not under one's control. We cannot order the five aggregates as we wish, thus, for example, ``\pali{eva\d m me r\=upa\d m hotu, eva\d m me r\=upa\d m m\=a ahosi}'' as we have seen in the text. 

For this very reason, the Buddha asserted that all five aggregates are void of self or not one's self, and do not belong to anyone. If the self refused by the Buddha has a marked characteristic of controllableness, it is not the Upanisadic \pali{\=atman} he argued about. Because controllableness is not a marked characteristic of \pali{\=atman} as well. That is to say, \pali{\=atman} indeed cannot set itself free and control its destiny. If the Buddha attacked this point, he was committing straw man fallacy (attacking a wrong, weaker point), if not red herring (misleading to an irrelevant issue). The main characteristic of \pali{\=atman} is invariability, not absolute autonomy.

However, later the Buddha related the five aggregates to impermanence and suffering, and then he concluded that the five aggregates does not belong to anyone, one are not those entities, and those are not anyone's self (from \pali{neta\d m mama, nesohamasmi, na meso att\=a}).\footnote{The full form of this argument can be put as follows (\textit{modus ponens}):\\(1) If form is impermanent, then it is nonself. (if p then q)\\(2) Form is indeed impermanent. (p)\\(3) Therefore form is nonself. (therefore q)}

Reading this sutta closely, I find that the Buddha's idea here does not directly conflict with Upanisad at all, because the five aggregates are not counted as \pali{\=atman} either. Doing some logic can help us clarify this. The Buddha's claim is that if \pali{att\=a} exists, it is able to control its state (if p then q). So, if the state is not under control, then \pali{att\=a} does not exist (if not q then not p). Consequently, the five aggregates are not under control, hence there is no \pali{att\=a}. Next, if something is impermanent, it is not \pali{att\=a} either (if not r then not p). This is equivalent to if it is \pali{att\=a}, it has to be permanent (if p then r). That confirms Upanisadic logic in turn. If we look for the Buddha's negation of \pali{\=atman}, we have to look elsewhere, not in this sutta.

Concerning controllableness, there might be a confusion between ownership of things and power to control those things at will. Ownership and power are not the same thing. A king can exert power over his subjects, but he does not own the subjects. Likewise, a government has judicial power over its citizens, but government does not own the citizens. Contrastingly, someone can own a car, but cannot make it run perfectly all the time. Or children by nature is owned by their parents, but the parents cannot control everything in their offspring's life.

As we may see, it is not ownership or power the Buddha was talking about. The Buddha denied the existence of the entity, or the controller, exerting that power. The five aggregates cannot be controlled, not because they are not owned by someone or the owner is powerless, but because there is in fact no controller at all, hence nonself. Does this sound better? Unfortunately, this argument is begging the question.

The Buddha's claim in this sutta, as mentioned above, is ``If form falls ill, then it is nonself.'' If we read `falling ill' as `uncontrollable,' then `no controller' in turn, the claim becomes ``If form has no controller, then it is nonself.'' This statement says nothing because `no controller' and `nonself' mean the same thing (if p then p).

As we have seen, defending nonself by the lack of control is unlikely to succeed, or survive close scrutiny. And defending from impermanence is just a confirmation of Upanisadic tenet. In fact, the Buddha could deliver his message without mentioning the self at all. He just said that ``Look!, this bundle of body and mind falls ill and dies. It does not belong to anyone. It falls ill an dies by its own nature. You should not attach to this uncontrollable thing. Be careful, and live your life wisely. Period.'' By this way of expounding, the Buddha could stay away from metaphysical issue of self, and put more focus on practical guidance. To put it another way, our suffering has nothing to do with metaphysical self.\footnote{This is the reason why, for practitioners' perspective, whether one approaches by seeking the true self, as Vedantic adherents do, or one approaches by seeking nonself, as Buddhists do, he or she can reach a kind of suffering-free, liberating state, often referred as non-duality. For a relevant reading, see \citealp{davis:advaita}.} It is empirical self that counts.

However, I think the Buddha had to bring \pali{att\=a} into discussion because he indeed wanted to demolish the idea of the permanent self. His logic might work at the time, but it is unconvincing by modern standard. Buddhists can argue that the Buddha was not interested in logic or in making an airtight argument. That is true, I believe so. The main point of studying the teaching is to put it into practice, not trying to belittle or debunk it. 

Still, we have to understand the things by reasoning, not just accept anything by mere faith. The idea of nonself is in fact perceivable by reasoning, no extraordinary knowledge is needed. David Hume\footnote{in \emph{A Treatise of Human Nature} (1739), \url{https://davidhume.org/texts/t/1/4/6}}, or more recently Derek Parfit\footnote{in \emph{Reasons and Persons} (1984)}, could do that with reason, why can't we? It is even easier to see as such when we openly listen to what cognitive psychology says today.

Some Buddhists still want to keep the idea of nonself esoteric, by insisting that only the Buddha and enlightened disciples can know it by exceptional wisdom eyes. My objection of the view is no one can see non-existing entity, even if he or she has super eyes capable of seeing galaxies light years away. There is no nonself to be seen by eyes.\footnote{This reminds me to Alice in \emph{Through the Looking-Glass}. The White King mistakes \emph{nobody} as an entity that can be seen, as shown by this dialogue.\\[1mm]``I see nobody on the road,'' said Alice.\\``I only wish \textit{I} had such eyes,'' the King remarked in a fretful tone. ``To be able to see Nobody! And at that distance too! Why, it's as much as \textit{I} can do to see real people, by this light!''} We can know the existence of nonself, or better the non-existence of self, only by reasoning.\footnote{This can be attributed as the wisdom eye (\pali{pa\~n\~n\=acakkhu}).} And I maintain that the Buddha really used reason to deny the permanent self, albeit not so sound, as we have seen in the sutta.
