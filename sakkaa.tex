\chapter{\pali{Sakk\=a, ala\d m, labbh\=a}}\label{chap:sakkaa}

P\=ali has a handful of magic words. They are magic in the sense that they defy a fixed classification, so that they can do things unexpectedly. Among those, \pali{sakk\=a} (able to, possible to), \pali{ala\d m} (enough, suitable), and \pali{labbh\=a} (possible, allowable, obtained) are frequently seen. By word class, these are grouped as indeclinables or particles. This means they retains their form regardless of how they are used.

\section*{\fbox{\pali{Sakk\=a}}}

The word is not crucially indispensable. In conversation, we can use verb \pali{sakkoti} to express the meaning in its all possibilities. However, \pali{sakk\=a} is used here and there, in the canon, and more frequently in post canonical texts. The main reason, I think, is it is very handy to use. You do not need to care what tense, what person, what number is being used, as well as, whether it is in active or passive form. We just use it as it is. That is the good side of handiness.

In reading texts, however, dealing with highly flexible terms can cause a headache, because we have to read them carefully. Sometimes they can be picked by the wrong side, then the very point is missed in the end. But this term is really magic to me, because, among a few P\=ali words, \pali{sakk\=a} has very specific meaning and use that makes it quite easy to deal with. Normally, \pali{sakk\=a} is used together with infinitives, verbs in \pali{tu\d m} form. Yet there are some variation in sentence structure that we should be aware of, as described below.

\paragraph*{1.\ \pali{Sakk\=a} as subject} When a sentence with \pali{sakk\=a} has no subject, but has the main verb of 3rd person in active form, \pali{sakk\=a} is treated as the subject. So, it is translated as noun, hence `ability' or `possibility,' for example:\par
- \pali{Tattha na\d m \=agata\d m gahetu\d m sakk\=a bhavissati}\footnote{Dhp-a\,2.21} (There will be a possibility to catch that [king] who has come in that [place].)\footnote{Thai P\=ali teachers see the sentence a kind of passive structure, and suggest that an instrumental actor should be added to make the sentence clearer. So, in this case, \pali{tay\=a} (by you) is added resulting in ``\pali{\ldots\ gahetu\d m [tay\=a] sakk\=a \ldots}'' (a possibility to catch that [by you]). However, Thais do not translate this sentence in passive voice. It goes active in meaning, something like ``That you can catch that [king] who has come in that [place] exists.'' That is a reason, I think, why learning P\=ali in Thai is rather difficult and confusing.}\par
- \pali{yattakena oloketu\d m sakk\=a hoti, tattaka\d m chidda\d m katv\=a \ldots}\footnote{Dhp-a\,2.21} (By which much the ability to look [through a gap] exists, [they], having made a gap by that much, \ldots)\par
- \pali{Ta\d m pana tena tena up\=ayena ma\~ncapa\d tip\=adaka\d m v\=a p\=adakathalika\d m v\=a phalakap\=i\d tha\d m v\=a k\=atu\d m sakk\=api bhaveyya.}\footnote{Dhp-a\,3.41} (There might be a possibility to make that piece of wood, by anyone with certain means, a bed support or a foot support or a sitting plank.)\par

\paragraph*{2.\ \pali{Sakk\=a} as passive verb} If the sentence has subject in nominative case which corresponds to the infinitive, it is in passive structure for sure. The sentence can be translated like other passive sentences, but with \pali{sakk\=a} as a (helping) verb, for example:\par
- \pali{na sakk\=a so ag\=aramajjhe vasantena [may\=a] p\=uretu\d m}\footnote{Dhp-a\,1.1} (That [practice] is not able to be fulfilled by [me] living in the house.)\par
- \pali{sama\d nadhammo n\=ama sar\=ira\d m y\=apentena sakk\=a [tay\=a] k\=atu\d m}\footnote{Dhp-a\,1.1} (Such duty of monks is able to be done [by you] with the body being kept up.)\par
- \pali{Kammavip\=ako n\=ama na sakk\=a kenaci pa\d tib\=ahitu\d m.}\footnote{Dhp-a\,1.15} (Such fruit of an action is not able to be prevented by anyone.)\par

\paragraph*{3.\ \pali{Sakk\=a} as impersonal passive verb} If the infinitive in the sentence related to an accusative noun, or the verb is intransitive (no noun at all), the sentence is in impersonal passive structure (\pali{bh\=avav\=acaka}). See examples below.\par
- \pali{na sakk\=a [may\=a] etena saddhi\d m ekato bhavitu\d m}\footnote{Dhp-a\,1.17} (To be united with this [king] [by me] is not possible)\par
- \pali{na sakk\=a [may\=a] vih\=ara\d m tuccha\d m k\=atu\d m}\footnote{Dhp-a\,7.91} (To make the house empty [by me] is not capable.)\par

\paragraph*{4.\ \pali{Sakk\=a} as subject complement} If the sentence has nominative subject with verb `to be' (finite or non-finite), \pali{sakk\=a} is treated as subject complement, for example:\par
- \pali{sace ete r\=upino hutv\=a katthaci pakkhipitu\d m [kenaci] sakk\=a bhaveyyu\d m}\footnote{Dhp-a\,3.39} (If these [defilements] have form, they should be able to be put in anywhere [by anyone].)\par

\section*{\fbox{\pali{Ala\d m}}}

Basically, this term means `enough' or `suitable.' It is used as a particle, so its form is never changed. To English users, \pali{ala\d m} works like an adverb. As explored by Thai P\=ali teachers, \pali{ala\d m} has three uses described as follows:

\paragraph*{1.\ \pali{Ala\d m} in prohibition} It is pretty much like we say ``Enough!'' or ``Stop!'' in English. In a milder sense, it means ``It is not suitable'' or in a suggestive tone, ``You shouldn't do that.'' Let us see some examples:\par
- \pali{Satth\=a ``ala\d m ettakena cittappas\=adena imass\=a''ti [cintetv\=a] pakk\=ami}\footnote{Dhp-a\,1.2} (The Buddha left, [having thought], thus ``That's enough for this [man] by this amount of faithful mental state.'')\footnote{In this sentence, \pali{ala\d m} is seen, by Thai P\=ali teachers, as subject. It is very unusual to see as such, even in Thai language. Moreover, it is said that when \pali{ala\d m} is the subject of a sentence, no verb is needed. To force translating \pali{ala\d m} as subject, we can go like this, ``The adequate amount for this [man] by this much of faithful mental state is done.''}\par
- \pali{Ala\d m, ayya, m\=at\=a me tajjessati}\footnote{Dhp-a\,4.50} (It is not suitable, master, my mother will threaten.)\par
- \pali{ala\d m, bhikkhave, m\=a cintayittha}\footnote{Dhp-a\,22.315} (Enough!, monks, do not think.)\par
- \pali{Adhiv\=asetv\=a ca atha kho bhagav\=a vera\~nja\d m br\=ahma\d na\d m ``ala\d m ghar\=av\=asapalibodhacint\=ay\=a''ti sa\~n\~n\=apetv\=a}\footnote{Sp1\,22} (Having accepted [the invitation], then the Blessed One, having made brahman Vera\~nja know thus, ``It is not suitable for thinking and worrying about the household life,'' \ldots)\par

\paragraph*{2.\ \pali{Ala\d m} in positive meaning} It is like we say ``That's enough/ sufficient'' or ``That's suitable'' in a positive manner. We can add \pali{na} (not) to make the meaning negative. So, it sounds like the first use. Here are some examples:\par
- \pali{mama dh\=itu ettak\=a g\=avo ala\d m}\footnote{Dhp-a\,4.53} (Cows [by] this amount is enough for my daughter.)\par
- \pali{Navahi, bhikkhave, a\.ngehi samann\=agata\d m kula\d m anupagantv\=a v\=a n\=ala\d m upagantu\d m, upagantv\=a v\=a n\=ala\d m nis\=iditu\d m}\footnote{A9\,17} (Monks, a family possessed with nine characteristics is not suitable to approach, [when] having not approached; and not suitable to sit in, [when] having approached.)\par

\paragraph*{3.\ \pali{Ala\d m} as `adornment'} This meaning happens when the term is combined with `to do' (\pali{kara}), e.g.\ \pali{ala\.nkaroti, ala\.nkato,} or \pali{ala\.nkaritv\=a}. This explanation sounds unusual. You can argue that it has nothing to do with \pali{ala\d m} above. But some traditional teachers see it in this way. Here is an example:\par
- \pali{etha tumhe, vadhuyo, yena ala\.nk\=arena ala\.nkat\=a pubbe ra\d t\d thap\-\=alassa kulaputtassa piy\=a hotha man\=ap\=a tena ala\.nk\=arena ala\.nkarotha}\footnote{M2\,300 (MN\,82)} (Come!, Girls. By which ornament you adorned yourselves earlier for Ra\d t\d thap\=ala to be pleased, you adorn by that ornament [again].)\par

\section*{\fbox{\pali{Labbh\=a}}}

By its root, this term means `obtained.' But in some contexts, it can also mean `possible' or `allowable.' That means its meaning overlaps with that of \pali{sakk\=a}. There are two kinds of \pali{labbh\=a} used in the texts:

\paragraph*{1.\ \pali{Labbh\=a} as derivative} The term is formed by primary derivation with root \pali{labha} plus \pali{\d naya} operation, resulting in \pali{labbha}. The term is used as noun in three genders, hence \pali{labbho} (m.), \pali{labbh\=a} (f.), and \pali{labbha\d m} (nt.). Here is an example:\par
- \pali{ra\~n\~no \=a\d n\=a ak\=atu\d m na labbh\=a}\footnote{Dhp-a\,2.24} (Command of the king is not possible to ignore [= not to do].)\footnote{In fact, \pali{labbh\=a} as derivative is very rare to find. In this instance, it can also be seen as particle anyway.}\par

\paragraph*{2.\ \pali{Labbh\=a} as particle} This use can be found more often. As particle, the term does not need to agree with related words. So, it is easier to use. Sentences with \pali{labh\=a}, as well as \pali{sakk\=a} and \pali{ala\d m} above, are seen by the tradition as passive structure. So, an instrumental actor is suggested to be added for the sake of clarity. Here are some examples:\par
- \pali{pave\d nirajja\d m n\=ama, t\=ata, ida\d m [tay\=a] na labbh\=a eva\d m k\=atu\d m}\footnote{Dhp-a\,4.47} (Such traditional kingship, son, is not obtained [by you] here to do in that way.)\par
- \pali{yamaha\d m tumh\=aka\d m dadeyya\d m, ta\d m kuto [tumhehi] labbh\=a}\footnote{Sp1\,22} (Which [gift] I should give to them, where is that [gift] obtained [by them]?)\par
- \pali{tva\d m m\=at\=apit\=una\d m ekaputtako, na labbh\=a tay\=a pabbajitu\d m}\footnote{Sp1\,25} (You [are] the only son of the parents, [so] to go forth by you is not suitable.)\par
