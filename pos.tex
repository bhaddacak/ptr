\chapter{Part of speech analysis}\label{chap:pos}

By composing a sentence, not just taking words from a dictionary and stringing them together, we can make an utterance. To make a meaningful statement, we have to put words in relation to one another, by order like in English mostly, or by inflection like in P\=ali. To know the function of each word in a sentence is essential for decoding the message. We normally call the word function \emph{part of speech} or \emph{word class}.

P\=ali's word categories are less than English's. In traditional courses, there are only three main groups to learn when we start to read texts: \emph{n\=ama} (including nouns, pronouns, and adjectives), \emph{verb} (including finite and non-finite kind of verbs), and \emph{nip\=ata} (non-inflectional particles).

\phantomsection
\addcontentsline{toc}{section}{N\=ama}
\section*{1.\ N\=ama}

This supergroup consists of three word classes: noun, pronoun, and adjective. All these use the same rules of declension with slightly different variation.

\phantomsection
\addcontentsline{toc}{subsection}{Noun (\pali{n\=aman\=ama})}
\subsection*{Noun (\pali{n\=aman\=ama})}

This class is the majority of words. The real knack for telling what is noun comes from you have to see words a lot. By principle, we identify a noun by its ending (\pali{vibhatti}). There is vast variation of noun endings. They can tell whether a word is of what gender, singular or plural, and in which case. Some endings are easy to recognized. For example, words with \pali{-o} ending are almost surely singular masculine noun with nominative case (they can be pronoun and adjective as well, see below). 

Some case endings can be a good clue for noun. For example, words with \pali{-ena, -in\=a, -un\=a} are mostly singular masculine or neuter noun in instrumental case, comparing to \pali{-iy\=a, -uy\=a} of feminine noun in various cases. Words with \pali{-ssa} are mostly singular masculine or neuter noun in dative or genitive case. Words with \pali{-sm\=a, -mh\=a} are almost certainly singular masculine or neuter noun in ablative case. Likewise, words with \pali{-smi\d m, -mhi} are almost certainly singular masculine or neuter noun in locative case.

Words ending with \pali{-na\d m} is likely plural noun (of three genders) in dative or genitive case. Words with \pali{-su} ending is almost certainly plural noun (of three genders) in locative case. Plural instrumental and ablative case which have \pali{-hi} ending are hard to say, because imperative verbs of second person end with \pali{-hi} as well. But verbs with \pali{-hi} ending are found occasionally, so it is quite safe to take such an ending as noun. You can tell plural instrumental and ablative case by \pali{-bhi} ending, however, but this is really rare to find.

That is to say, if you master all declensions of nouns, it will be easy to tell what is noun. And if you can remember words in other classes which are far less than nouns, you can recognize nouns even more easily.

Recognizing a noun by its form is a good starter for analysis, but it is far from enough. We have to recognize by its function as well. This task is more important and difficult. We have to know how each case works grammatically. I can only summarize some of main ideas here. For more information, see Chapter \externalref{38} of PNL.

\paragraph*{Noun in nominative case} Being the subject of sentences is the main function of nominative case. This includes the following:
\begin{compactenum}[(1)]
\item Logical and grammatical subject of active structure:\par
- \pali{\textbf{d\=arako} ti\d t\d thati} (A boy stands.)\par
- \pali{\textbf{d\=arak\=a} p\=upa\d m bhu\~njati} (A girl eats cake.)\par
\item Logical and grammatical subject of causative structure:\par
-\pali{\textbf{pit\=a} putte p\=athas\=ala\d m gacch\=apeti} (A father makes children go to school.)\par
\item Grammatical subject of passive structure:\par
- \pali{\textbf{p\=upo} d\=arak\=aya bhu\~nj\=iyati} (Cake is eaten by a girl.)\par
\item Grammatical subject of causal passive structure:\par
- \pali{pitar\=a puttehi \textbf{p\=athas\=al\=a} gacch\=apiyati} (The school is made gone to by children, by [order of] a father.)\par
\item Subject of sentences with verb `to be' omitted:\par
- \pali{\textbf{rukkho} [hoti]} ([There is] a tree.)\par
\item In simile with verb `to be' omitted:\par
- \pali{Tato na\d m dukkhamanveti, \textbf{cakka\d m}va vahato pada\d m}\footnote{Dhp\,1.1} (From that [action], suffering follows [him/her] like the wheel [follows] the foot-step carrying [= of the cow carrying the cart].)\footnote{For new students, I have a quick explanation of some points. First, \pali{dukkhamanveti} (\pali{dukkha\d m + anveti}) means ``suffering follows.'' In nominative case, \pali{dukkha\d m} (nt.) is the subject. Second, \pali{cakka\d mva} (\pali{cakka\d m + iva}) means ``like a wheel.'' Also in nominative case, \pali{cakka\d m} (nt.) is used in simile marked by \pali{iva}. And third, \pali{vahato} is past participle of \pali{vahati} (to carry). This can be seen as an adjective modifying \pali{pada\d m}, thus the animal which carried the cart (but I translate it as present participle anyway). It will be more sensible if \pali{hoti} is put in the last part.}\par
\item Subject complement:\par
- \pali{pit\=a \textbf{\=acariyo} hoti} (Father is a teacher.)\par
- \pali{Appam\=ado amata\textbf{pada\d m}, pam\=ado maccuno \textbf{pada\d m}}\footnote{Dhp\,2.21} (Carefulness [is] a deathless path, carelessness [is] a path of death.)\par
- \pali{avijj\=a parama\d m \textbf{mala\d m}}\footnote{Dhp\,18.243} (Negligence [is] the ultimate stain.)\\[1.5mm]
Technically, this is called \pali{vikatikatt\=a} (see also the adjective section below). When verb `to be' is omitted, it can be seen as apposition (two nouns refer to the same thing). It is worth noting that a noun in P\=ali cannot modify other noun like English, e.g.\ school teacher. But a noun can modify other noun by applying an appropriate case, e.g.\ \pali{p\=a\d thas\=al\=aya \=acariyo} (school's teacher), or more handily as a compound, e.g.\ \pali{p\=a\d thas\=al\=acariyo}. If two nouns with nom.\ sit together, it is supposed to be subject complement or apposition. Hence, \pali{p\=a\d thas\=al\=a \=acariyo} is illogical because school is not teacher. It is also worth noting that the main noun and the complement are not necessary to be in the same gender, but case and number has to be conforming.\\
\item Object complement of verb `to do/make' in passive structure:\par
- \pali{kattabba\d m \textbf{kusala\d m} bahu\d m}\footnote{Dhp\,4.53} (Many wholesome [actions] should be done.)\\[1.5mm]
This is a special case of grammatical subject of passive structure mentioned above. In grammatical textbook, this is called \pali{vikatikamma} (adjective included). In passive sentences with verb `to do' (\pali{kattabba\d m}\footnote{For future passive participle, see Chapter \externalref{32} of PNL.} in this case), the object of `to do' takes nominative case (the omitted \pali{kamma\d m} in this case). This function can also be seen in active structure with accusative case (see below).
\item Object complement of verb `to say/see/know' in passive structure:\par
- \pali{\textbf{amata\d m} vuccati nibb\=ana\d m}\footnote{Dhp-a\,2.23} (Nibb\=ana is said as the deathless [nature].)\\[1.5mm]
Technically, this is called \pali{sambh\=avana}. In English, it is like we say ``Something is said/seen/known as \ldots'' In this instance, \pali{nibb\=ana\d m} (nom.\ nt.) is subject of the sentence. The verb is \pali{vuccati} (to be said/called), which is in passive form. Thus, \pali{amata\d m} completes the verb by modifying the meaning of \pali{nibb\=ana\d m}. The use is much like in `to do/make' case above, but the tradition sees them as different application. See also in accusative case below.
\end{compactenum}

\paragraph*{Noun in accusative case} The main function of this case is to mark direct object. It can do other things as well. Here are what we should know:
\begin{compactenum}[(1)]
\item Object of verbs in general:\par
- \pali{\textbf{kamma\d m} karoti} ([One] does an action.)\par
\item Destination of verb to go, etc.:\par
- \pali{\textbf{geha\d m} gacchati} ([One] goes home.)\par
- \pali{\textbf{s\=ala\d m} pavisati} ([One] enters the hall.)\par
\item Object of command in causative structure:\par
- \pali{s\=am\=i \textbf{purisa\d m} kamma\d m karoti} (The master makes a person do work.)\par
\item Continuation of space and time:\par
- \pali{\textbf{tiyojana\d m} gacchati} ([One] goes by three yojanas.)\par
- \pali{\textbf{tem\=asa\d m} vasati} ([One] lives for three months.)\par
\item Person or thing one talks to:\par
- \pali{s\=adh\=u'ti \=acariyo \textbf{sissa\d m} bh\=asati} ('Good!', the teacher says to a student.)\par
\item Adverbial\footnote{Normally, when adjectives take accusative case, they work as adverbial. When nouns work likewise, generally they take instrumental case, for example, \pali{sukhena karoti} ([One] does with ease = does easily). When the line between noun and adjective is blurry, a noun in acc.\ can be adverbial in some contexts.}:\par
- \pali{\textbf{sukha\d m} seti} ([One] sleeps comfortably.)\par
\item Object complement of verb `to do/make' in active structure:\par
- \pali{A\d t\d th\=ina\d m \textbf{nagara\d m} kata\d m}\footnote{Dhp\,11.150} ([This body] was made the city of bones.)\par
- \pali{Dhamma\d m care \textbf{sucarita\d m}, na na\d m \textbf{duccarita\d m} care}\footnote{Dhp\,13.169. Sometimes, verb `to practice' (\pali{carati}) can take a complement like in this instance. However, I see \pali{dhamma\d m} and \pali{sucarita\d m} as synonym. So, they look more like apposition to me.} ([One] should practice the Dhamma as right conduct. [One] should not practice that as bad conduct.)\par
\item Object complement of verb `to say/see/know' in active structure:\par
- \pali{buddha\d m \textbf{sara\d na\d m} gacch\=ami} ([I] go to [= regard] the Buddha as the refuge.)\par
- \pali{tamaha\d m br\=umi \textbf{br\=ahma\d na\d m}}\footnote{Dhp\,26.385} (I call that [person] brahman.)\par
- \pali{P\=apopi passati \textbf{bhadra\d m}, y\=ava p\=apa\d m na paccati}\footnote{Dhp\,9.119} (As long as an evil [one] sees [an evil action] as good, the evil [result] is not ripened.)\par
- \pali{Att\=ana\~nce \textbf{piya\d m} ja\~n\~n\=a}\footnote{Dhp\,12.157. Here, \pali{ja\~n\~n\=a} is irregular form of \pali{j\=aneyya} (optative). This sentence is a speculation.} (If [one] may know the self as beloved, \ldots)\par
\end{compactenum}

\paragraph*{Noun in instrumental case} As the name tells us, this case mainly marks the means one uses to do an action, like we use `with' or `by' in English. It also does other important grammatical functions. Here is a quick summary:
\begin{compactenum}[(1)]
\item Instrument:\par
- \pali{\textbf{p\=adena} gacchati} ([One] goes by foot [= walks].)\par
\item Adverbial:\par
- \pali{mama \textbf{vacanena} eva\d m vadehi} (Say this by my word [= Repeat this after me].)\par
\item Agent of passive structure:\par
- \pali{p\=upo \textbf{d\=arak\=aya} bhu\~nj\=iyati} (Cake is eaten by a girl.)\par
\item Cause:\par
- \pali{s\=a \textbf{icch\=aya} p\=upa\d m bhu\~njati} (She eats cake by [= because of] desire.)\par
\item Together with someone or something:\par
- \pali{s\=a \textbf{sah\=ayehi} saddhi\d m p\=upa\d m bhu\~njati} (She eats cake with friends.)\footnote{This use normally comes with particle \pali{saha} or \pali{saddhi\d m}. In negative sense, \pali{vin\=a} (without) can be used. For some more information, see Chapter \externalref{12} and Appendix \externalref{F} of PNL.}\par
\item Having certain quality or things (\pali{itthambh\=uta}):\par
- \pali{s\=a \textbf{assumukhena} p\=upa\d m bhu\~njati} (She eats cake with a tearful face.)\footnote{We can see a nuance here. A tearful face is not instrument. It is supposed to be the mouth by which one eats. The sentence just says the girl eats while weeping. In Thai tradition, this sentence is suggested to be rendered as ``Having a tearful face, she eats cake.'' However, you can see the cry as instrument if the context shows that the girl uses her tears as the tool for allowing her eating.}\par
- \pali{\textbf{Manas\=a} ce \textbf{padu\d t\d thena}, bh\=asati v\=a karoti v\=a}\footnote{Dhp\,1.1} (With/ Having corrupted mind, if [one] say or do, \ldots)\par
- \pali{ta\d mkha\d na\~n\~neva r\=aj\=a \textbf{ukk\=ahi dh\=ariyam\=an\=ahi} tattha gantv\=a}\footnote{Dhp-a\,1.3} (In just that moment, the king, with torches holding [by attendants], having gone in that [place], \ldots)\par
- \pali{So rodam\=ano tattheva \d thatv\=a satthari cakkhupatha\d m vijahante \textbf{hadayena phalitena} k\=ala\d m katv\=a}\footnote{Dhp-a\,1.6} (That [elephant], having stood crying there, when the Buddha disappeared [from] its eyes, having died with a broken heart, \ldots)\par
- \pali{Atha so issaro \textbf{yath\=adhoteneva pattena} \=agacchanta\d m paccekabuddha\d m disv\=a}\footnote{Dhp-a\,2.21} (Then, that master, having seen a Pecceka Buddha coming with a bowl cleaned as such, \ldots)\par
- \pali{S\=a \textbf{thanehi kh\=ira\d m mu\~ncantehi} u\d t\d thahitv\=a allac\=ivar\=a [hutv\=a] gantv\=a thera\d m ga\d nhi}\footnote{Dhp-a\,12.160} (She, with breasts emitting milk, having risen up, having had a wet robe, having gone, seized the senior monk.)\par
\end{compactenum}

\paragraph*{Noun in dative case} The main function of this case is to mark indirect object. It also has specific uses with some terms.\footnote{For more information, see Chapter \externalref{13} and \externalref{38} of PNL.}
\begin{compactenum}[(1)]
\item Indirect object:\par
- \pali{\=acariyo \textbf{d\=arakassa} potthaka\d m dad\=ati} (A teacher gives a book to a boy.)\par
\item With `to satisfy' or `to delight':\par
- \pali{p\=upo \textbf{d\=arak\=aya} ruccati} (Cake satisfies the girl.)\par
\item With `suitable' or `capable':\par
- \pali{\=at\=ap\=i bhabbo \textbf{sambhodh\=aya} [hoti]} (A strenuous person is suitable to/capable of enlightenment.)\par
\item With `enough':\par
- \pali{ala\d m tass\=a \textbf{bhu\~njan\=aya}} ([It is] enough for her eating.)\footnote{Or we can also see \pali{tass\=a} as dat., thus ``It is enough for her to eat [for eating].'' Apart from neutral meaning, it can be used as a blame. Hence, the sentence implies ``She should stop eating.''}\par
\end{compactenum}

\paragraph*{Noun in ablative case} This case marks `sources,' much like we use `from' in English. It also has some other uses.\footnote{See Chapter \externalref{11} of PNL, and for more details, see Chapter \externalref{38}.}
\begin{compactenum}[(1)]
\item Source:\par
- \pali{\textbf{ag\=arasm\=a} anag\=ariya\d m pabbajanti}\footnote{Mv\,1.12} ([They] go forth from the house to the homelessness.)\par
\item Cause:\par
- \pali{\textbf{avijj\=apaccay\=a} sa\.nkh\=ar\=a}\footnote{S2\,1 (SN\,12)} (From [= because of] ignorance, conditioned [things arise].)\par
\item In comparison:\par
- \pali{hatth\=i \textbf{s\=ukar\=a} mahanto hoti} (An elephant is bigger from [= than] a pig.)\par
\item With `fear':\par
- \pali{m\=usik\=a \textbf{bi\d l\=alasm\=a} bh\=ayati} (A mouse fears [from] a cat.)\par
\end{compactenum}

\paragraph*{Noun in genitive case} Since this case shares mostly the same forms as dative case, sometimes it can be confusing. If the meaning is about possession, it is surely genitive case. This case also has some other grammatical functions.\footnote{For much more details, see Chapter \externalref{38} of PNL.}
\begin{compactenum}[(1)]
\item Possession:\par
- \pali{tassa \textbf{\=acariyassa} putto pa\~n\~nav\=a hoti} (A son of that teacher is wise.)\par
\item Singling out:\par
- \pali{\textbf{ti\d n\d na\d m putt\=ana\d m} kani\d t\d tho pa\~n\~nav\=a hoti} (Of those three sons, the youngest one is wise.)\footnote{For the irregular \pali{kani\d t\d tha}, see Chapter \externalref{18} of PNL.}\par
\item Subject of absolute construction:\par
- \pali{\textbf{purisassa} dh\=avantassa `upakara!' iti ugghoseti} (While the man is running, [he] shouts ``Help!'')\par
- \pali{cundas\=ukarikassa gehadv\=ara\d m pidahitv\=a \textbf{s\=ukar\=ana\d m} m\=ariy\-am\=an\=ana\d m ajja sattamo divaso}\footnote{Dhp-a\,1.15. In the collection, it is \pali{cundas\=ukaritassa} which is incorrect. The structure of this sentence is unusual. The very subject is \pali{ajja} with verb `to be' left out. This makes \pali{pidahitv\=a} isolated, so it should be seen as an independent clause. Hence this \pali{tv\=a} verb works like a finite verb (see also the section of \pali{tv\=a} below).} (When pigs were being killed, the house's door of Cundas\=ukarika was closed; today is the seventh day.)\par
\item Object of verbs (occasionally):\par
- \pali{Dh\=iro p\=urati \textbf{pu\~n\~nassa}, thoka\d m thokampi \=acina\d m}\footnote{Dhp\,9.122} (A wise person is full of merit, little by little accumulatively.)\par
\end{compactenum}

\paragraph*{Noun in locative case} The main function of this case relates to space and time. It can also do some similar things like the genitive and other cases.
\begin{compactenum}[(1)]
\item As `in' (concealment):\par
- \pali{amh\=aka\d m \textbf{gehe} bahudhana\d m} (plenty of wealth in our house)\par
\item As `in' (mixture):\par
- \pali{\textbf{tilesu} tela\d m} (oil in sesame seeds)\par
- \pali{\textbf{ucch\=usu} raso} (taste in sugar canes)\par
\item As `in' (abiding):\par
- \pali{\textbf{jale} macch\=a} (fish in water)\par
- \pali{nanu, j\=ivaka, \textbf{vih\=are} bhikkh\=u atthi}\footnote{Dhp-a\,2.25} (J\=ivaka, aren't monks in the building?)\par
\item As `in' (relation to certain nouns):\par
- \pali{bhante, \textbf{imasmi\d m s\=asane} kati dhur\=an\=i}\footnote{Dhp-a\,1.1} (Sir, how many practices are there in this religion?)\par
\item As `in' (relation to certain verbs):\par
- \pali{buddharatana\d m \textbf{loke} uppanna\d m}\footnote{Dhp-a\,26.416} (The Buddha-jewel has been arisen in the world.)\par
\item As `on':\par
- \pali{\textbf{\=asane} nis\=idi} ([One] sat on the seat.)\par
\item As `over':\par
- \pali{t\=ani gahetv\=a pa\~ncasat\=a hatth\=i pa\~ncanna\d m bhikkhusat\=ana\d m \textbf{matthake} dh\=arayam\=an\=a \d thassanti}\footnote{Dhp-a\,13.177} (Having held those [parasols] over the head of 500 monks, the 500 elephants will stand.)\par
\item As `to':\par
- \pali{r\=ajamuddika\d m n\=iharitv\=a attano \textbf{a\.nguliya\d m} pilandhi}\footnote{Dhp-a\,2.21} (Having taken out the king's ring, [she] adorned it to her own finger.)\par
\item As `nearby':\par
- \pali{\textbf{nagaradv\=are} g\=am\=a} (villages nearby the city's gate)\par
\item Time marking:\par
- \pali{\textbf{Tasmi\d m} kho pana \textbf{samaye} dhamm\=a honti, khandh\=a honti}\footnote{Dhs\,3:121} (In that occasion dhammas exist, so do the aggregates.)\par
- \pali{\textbf{divase divase} d\=ana\d m datv\=a s\=ila\d m rakkhati}\footnote{Dhp-a\,9.119} (Everyday, having given alms, [he] observes the precept.)\footnote{For repetition, see Chapter \externalref{28} of PNL.}\par
\item Cause:\par
- \pali{ku\~njaro \textbf{dantesu} ha\~n\~nate}\footnote{Kacc\,310} (An elephant is killed because of tusks.)\par
\item Singling out:\par
- \pali{\textbf{t\=isu puttesu} kani\d t\d tho pa\~n\~nav\=a hoti} (In those three sons, the youngest one is wise.)\par
\item Subject (rare):\par
- \pali{asukag\=amato \textbf{asukag\=amagamana\d t\d th\=ane} sama\d m [hoti]}\footnote{Dhp-a\,4.44} (The going to village over there, from [another] village over there, [is] smooth.)\par
\item Subject of absolute construction:\par
\end{compactenum}

\begin{quote}
\pali{Yath\=a attha\.ngate \textbf{s\=uriye}, honti satt\=a tamogat\=a;}\\
\pali{Eva\d m \textbf{buddhe} anuppanne, hoti loko tamogato.}\footnote{Ap1\,5:8}\\[1.5mm]
``In which way when the sun set down, beings were in darkness;''\\
``In that way when the Buddha did not yet arise, the world was in darkness.''\footnote{For correlative sentences, see Chapter \externalref{16} of PNL. Normally, \pali{yath\=a} should pair with \pali{tath\=a}. Sometimes, it can pair with \pali{eva\d m} like this instance.}
\end{quote}

\paragraph*{Noun in vocative case} The only use of this case is to address the participant of the conversation (interlocutor), for example:\par
- \pali{ki\d m vadi, \textbf{\=acariya}} (Whay did you say, teacher?)\par
- \pali{tena hi, \textbf{bhikkhave}, bhikkh\=una\d m sikkh\=apada\d m pa\~n\~napess\=ami}\footnote{Buv1\,39} (Therefore, monks, [I] will declare religious rules for monks.)\par

\phantomsection
\addcontentsline{toc}{subsection}{Adjective (\pali{gu\d nan\=ama})}
\subsection*{Adjective (\pali{gu\d nan\=ama})} 

Since adjectives are subsumed under \pali{n\=ama} category, together with nouns and pronouns, they share declensional forms. This means telling a noun from an adjective is difficult sometimes. In P\=ali, adjectives can stand alone and function themselves as nouns. For example, \pali{ucco} (high) can be seen without a noun nearby, so it means ``a high person or thing.'' Some may argue that the gender of adjectives implies what is left out. In \pali{ucco} case, a masculine thing might be omitted. If we can infer from the context, it should be treated as an adjective with a noun unsaid.\footnote{Traditionally speaking, noun and adjective are always different. Noun has gender by its own, but adjective does not. If an adjective stands alone, an appropriate noun has to be added when it is translated. That is a common practice in the classroom.}

A vital rule concerning adjectives is when they modify a noun, they have to take the same gender, number, and case as that noun. Here are a summary of adjective's possible functions:
\begin{compactenum}[(1)]
\item Modifier (\pali{visesana}):\par
- \pali{\textbf{th\=ulo} bi\d l\=alo} ([There is] a fat cat.)\par
\item Subject complement:\par
- \pali{aya\d m bi\d l\=alo \textbf{th\=ulo} hoti} (This cat is fat.)\par
- \pali{up\=asak\=a maya\d m \textbf{samagg\=a} j\=at\=a, tumhepi no purimasadis\=a hotha}\footnote{Dhp-a\,1.6} (Lay devotees, we became united, and may you be like our former time.)\par
- \pali{so t\=aya saddhi\d m s\=ilavipatti\d m \textbf{patto} bhavissati}\footnote{Dhp-a\,1.1. As past participle, \pali{patto} can be treated as an adjective. The future verb \pali{bhavissati} expresses a speculation, so using `must' is suitable here.} (He must be [the one who has] transgressed the precept together with her.)\par
\item Object of `to do':\par
-\pali{s\=a mukha\d m \textbf{p\=aka\d ta\d m} karoti} (She makes the face visible [= She uncloses her face].)\footnote{This function is close to English when we use `to make' with an adjective. This \pali{vikatikamma} can be found in both active and passive structure. In active sentences, it takes acc.\ like this example. See also, noun in nom.\ and acc.\ above.}\par
\item Adverbial (\pali{kiriy\=avisesana}, with acc.):\par
- \pali{Tena hi, br\=ahma\d na, su\d nohi; \textbf{s\=adhuka\d m} manasikarohi; bh\=asiss\=ami}\footnote{D1\,318 (DN\,4)} (Then, brahman, listen, think it over carefully; I will say.)\par
\end{compactenum}

\phantomsection
\addcontentsline{toc}{subsection}{Pronoun (\pali{sabban\=ama})}
\subsection*{Pronoun (\pali{sabban\=ama})} 

Because of its definite number, pronoun is easily recognizable word class. There are only 27 pronouns listed by Aggava\d msa as follows:

\begin{quote}
\pali{sabba katara katama ubhaya itara a\~n\~na a\~n\~natara \\a\~n\~natama pubba para apara dakkhi\d na uttara adhara ya ta eta ima amu ki\d m eka ubha dvi ti catu tumha amha}\footnote{Sadd-Pad Ch.\,12}
\end{quote}

Remembering all these words and their declensions is indispensable for P\=ali learners, so it should be no problem with these. Even though pronouns use the same declesional rules as nouns, some of them have specific ways of inflection, particularly personal pronouns. Therefore keeping all forms of pronouns in mind is very important.

It is worth noting that pronouns can function as pronominal adjective at any time, particularly when accompanied with other nouns. Much like English, for example, \pali{sabba} (all) can be used as pronoun like \pali{sabbe maranti} (all die), and as modifier like \pali{sabbe sa\.nkh\=ar\=a} (all conditioned things). This can also cause ambiguity. For example, if \pali{sabbe} is treated as pronoun in \pali{sabbe sa\.nkh\=ar\=a}, it can mean ``all [are] conditioned things'' (with verb `to be' left out). Ambiguity and vagueness are a powerful tool for textual manipulation, so be careful with this.

Pronoun can be used in various functions as noun, except vocative case, so the quick summaries described above can also be applied to pronoun mostly. There are many things to know about pronoun, please consult PNL if needed.

\phantomsection
\addcontentsline{toc}{section}{Verb}
\section*{2.\ Verb}

In a sentence, verb is the most important part. We can tell whether a string of words is a sentence or not by finding a verb. Words functioning as verb are far less than nouns, and they undergo a distinct formation process. So, it is quite easy to tell verbs apart from other kinds of word. What we call verb in P\=ali is roughly divided into two groups: (a) \pali{\=akhy\=ata}, and (b) verbal \pali{kita}. The former is called \emph{finite} verb in English grammar. It is the main verb that finishes each sentence. And the latter is more or less equivalent to \emph{non-finite} verb, like infinitive and participle.

Apart from those two groups, there are some particles (indeclinables) seen as verb equivalent, for example, \pali{atthi/natthi}\footnote{The term has two uses, as a normal verb which assumes various forms, and as a particle which stays unchanged.} (to exist/not exist), \pali{sakk\=a} (be able), \pali{ala\d m} (be suitable), and \pali{labbh\=a} (be possible, may be obtained). All these will be discussed in the \pali{nip\=ata} section.

\subsection*{Main verbs (\pali{\=akhy\=ata})}

In P\=ali grammar books, there are eight verb classes in total. In a sentence, the main verb can be either of simple tense, past tense (aorist, perfect, or imperfect), future tense, imperative mood, optative mood, or conditional mood. Recognizing a main verb is quite easy because verb conjugations have distinct forms. The hard part is to master verb formation is rather challenging to new learners.

That is to say, apart from recognizing its form when we encounter a verb, we have to know these also: Which person (1st, 2nd, or 3rd) is that verb for? Is it singular or plural? Is it in active or passive structure? Is it in causative structure? All these entail a lot to explain. So, I cannot make a quick summary of verb recognition. For more information, please see PNL, particularly Chapter \externalref{36, 37}, several chapters before those, and Appendix \externalref{C}.

\subsection*{Derivative verbs (verbal \pali{kita})}

Verbs in participial clauses are used extensively in P\=ali. Despite having root as their basic part like \pali{\=akhy\=ata}, these verbs are formed by different means. Technically, they are called \emph{primary derivation} (\pali{kita}). The most important verbs in this group are present participle (\pali{-anta, -m\=ana}), past participle (\pali{-ta}), future passive participle (\pali{-an\=iya, -tabba}), infinitive \mbox{(\pali{-tu\d m})}, and absolutive (\pali{-tv\=a}. Recognizing these verbs is also easy, even though several irregular forms have to be remembered.

One thing should be noted about P\=ali non-finite verbs is some of them can finish a sentence like finite verbs, for example, verbs in \pali{-ta, -tabba,} and \pali{-an\=iya} form.\footnote{Some might still argue that they cannot, because those sentences have verb `to be' left out. I see this as pedagogic strategy. It is up to what your teachers say.} Another point is, like English, participles in P\=ali can also function as adjectives (or eventually nouns). This can make a definite judgement is not easy to reach. So, you should be aware of this possibility and open to alternative translations.

Now I will summarize functions of these verbs as follows:

\phantomsection
\addcontentsline{toc}{subsection}{\pali{Anta, m\=ana}}
\subsection*{\fbox{\pali{Anta, m\=ana}}} 

These two forms work as present participle. Much like English, these occur only in subordinate clauses, or just modifiers. The two forms can be used interchangeably, except one key difference: \pali{Anta} form can be used only in active and causative structure, whereas \pali{m\=ana} form can be used in all structures (active, causative, passive, impersonal passive, and casual passive). I wrote about present participle in Chapter \externalref{30} of PNL. The marked functions of these are listed below:

\begin{compactenum}[(1)]
\item Modifier:\par
- \pali{id\=ani \textbf{kariyam\=ano} uposatho} (the recitation of disciplines being done now)\par
- \pali{So sot\=apannopi \textbf{sam\=ano} se\d t\d th\=i}\footnote{Dhp-a\,1.18. Here, \pali{sam\=ano} is verb `to be' (\pali{anta} form of \pali{atthi}). So, \pali{sot\=apanno} is subject complement, not object of the verb.} (that rich person even being a Stream Enterer)\par
- \pali{Bhattu\d t\d th\=ana\d t\d th\=anampi \textbf{aj\=ananto} kulaputto}\footnote{Dhp-a\,1.17. The verb can take an object like this one.} (the son of the family unknowing even the source of food)\par
- \pali{may\=a pacchato \textbf{nikkhamam\=an\=a} taru\d naitth\=i ida\d m n\=ama \textbf{karont\=i} di\d t\d th\=a}\footnote{Dhp-a\,10.133. This sentence is in passive structure with two modifying participles, one before the noun, and another after.} (The young woman getting out from behind, doing this [kind of action], was seen by me.)\par

\item Verb of participial clauses:\par
- \pali{So sot\=apannopi sam\=ano se\d t\d thidh\=itari uppannasoka\d m adhiv\=asetu\d m \textbf{asakkonto}}\footnote{Dhp-a\,1.18. In the collection, \pali{se\d t\d th\=i} joins with \pali{dh\=itari} as \pali{se\d t\d thidh\=itari}, not a compound. The two words have to be read separately.} (That rich person even being a Stream Enterer, being unable to endure the grief arisen over the daughter)\footnote{As you may realize, modifier and participial clause are really close. If you are not in a classroom where you have to please your teacher, you can merge these two functions together, I think.}\par
- \pali{Vis\=akh\=a sasura\d m \textbf{b\=ijayam\=an\=a} \d thit\=a}\footnote{Dhp-a\,4.53} (Vis\=akh\=a, having stood, fanning [her] father-in-law)\par
- \pali{So eva\d m \textbf{cintento}va \d thatv\=a \textbf{b\=ijayam\=ano} therassa s\=ise t\=alav\-a\d n\d tena pahari}\footnote{Dhp-a\,3.37} (That [monk], having stood, thinking as such, [while] fanning, hit the head of the senior monk with the fan's handle.)\par

\item Subject complement (in nom.), a special case of modifier with verb `to be,' for example:\par
- \pali{So pana b\=alo thera\d m disv\=api \textbf{apassanto} viya hutv\=a adhomukho bhu\~njateva}\footnote{Dhp-a\,4.53} \ \ (That foolish [father-in-law], even having seen the senior monk, [acting] like not-seeing, only eats [food] with the face cast down.)\footnote{Here, \pali{apassanto} (not-seeing) is subject complement, as well as, \pali{adhomukho}. The verb `to be' is \pali{hutv\=a} (absolutive form). The main verb is in present tense with emphatic particle, \pali{bhu\~njateva} (\pali{bhu\~njati + eva}).}\par

\item Object complement (in acc.):\par
- \pali{Satth\=a tassa ajjh\=asaya\d m viditv\=a attano sa\.ngh\=a\d ti\d m sil\=asana\d m \textbf{pa\d ticch\=adayam\=ana\d m} [katv\=a] khipi}\footnote{Dhp-a\,14.181} (The Buddha, having known his [= Sakka's] intention, threw his own outer robe, [making] it cover the stone.)\footnote{In the text, there is no \pali{katv\=a}, but it is inserted to make the sense clearer.}\par

\item Verb of absolute construction (in gen.\ or loc.):\par
- \pali{therassa nidda\d m \textbf{anokkamantassa}}\footnote{Dhp-a\,1.1. See the full sentence below in \pali{ta} section.} (While the senior monk did not fall into sleep)\par
- \pali{Tassa nicca\d m satthu santika\d m \textbf{gacchantassa} pabbajj\=aya citta\d m nami}\footnote{Dhp-a\,2.25} (When he went to the master's place frequently, [his] mind inclined for going forth.)\par
- \pali{Bhikkh\=usu pana attano pattac\=ivar\=ad\=ini \textbf{karontesu} mah\=akassapattheropi c\=ivar\=ani dhovi}\footnote{Dhp-a\,7.91} (While monks were doing their own bowls and robes, etc., even the elderly Mah\=akassapa washed [his] robes.)\par
- \pali{So bhu\~njitv\=a \=ah\=ara\d m j\=ir\=apetu\d m asakkonto aru\d ne \textbf{uggacchante} k\=alamak\=asi}\footnote{Dhp-a\,2.21} (Having eaten food, unable to digest it, when the dawn rose, he died.)\\[1.5mm]
In the traditional view, absolute construction has two kinds: \pali{lakkha\d na} and \pali{an\=adara}. The former means like a sign. Hence, a subordinate clause signals the main action by providing a timing clue or a parallel event. The examples above are of this type. In the latter kind, the relation of the main and subordinate clause is different. It sounds like the main action relentlessly happens regardless of the subordinate action. Here are some examples:\footnote{In the official P\=ali course of Thai tradition, \pali{an\=adara} kind of sentences uses only genitive case. But it can either go with gen.\ or loc.}\par
- \pali{Ta\d m \textbf{karontass}evassa rogo balav\=a ahosi}\footnote{Dhp-a\,1.2} (Even when [the doctor] did with that medicine, his illness became intense.)\par
- \pali{idheva me nipajjitv\=a \textbf{marantass\=a}pi apar\=apara\d m \textbf{parivattantass\=a}pi tay\=a saddhi\d m gamana\d m n\=ama natthi}\footnote{Dhp-a\,1.1} (Even when I, lying down, roll from side to side [and] die here, there is no such thing like going with you.)\par
- \pali{so kulaputto m\=at\=apit\=una\d m \textbf{rodant\=ana\d m} geh\=a nikkhamitv\=a pabbaji} (That son of a family, while the parents were crying, having gone out from the house, went forth.)\par
\end{compactenum}

\phantomsection
\addcontentsline{toc}{subsection}{\pali{Ta}}
\subsection*{\fbox{\pali{Ta}}} 

This is P\=ali past participle. It works much like \pali{anta} and \pali{m\=ana} form, except it has past meaning. We can find \pali{ta} form quite often, because it can be used in various structures. This verb form is easily recognizable, albeit some irregular forms have to be remembered. In rare cases, past participle can be alternatively in \pali{tavantu} and \pali{t\=av\=i} form. If the use of \pali{anta} and \pali{m\=ana} is understandable, there should not be any problem with \pali{ta}. The key difference between \pali{anta/m\=ana} and \pali{ta} form is the latter can finish sentences like finite verbs. In PNL, you can read about past participle mainly in Chapter \externalref{31}, also Chapter \externalref{32, 33} in lesser extent. Important functions of this verb form is shown below:

\begin{compactenum}[(1)]
\item Modifier:\par
- \pali{So \textbf{uppanna\d m} l\=abha\d m anurujjhati}\footnote{A8\,6} (He is pleased with the gain that happened.)\par
- \pali{may\=a \textbf{khitto} saro silampi vinivijjhitv\=a gacchati}\footnote{Dhp-a\,2.21} (The arrow shot by me goes, piercing through even a stone.)\par

\item Subject complement:\par
- \pali{k\=ayena \textbf{sa\d mvuto} siy\=a}\footnote{Dhp\,17.231. This sentence has active meaning.} ([A monk] should restrain [himself] with the body.)\par
- \pali{Ajja amh\=aka\d m r\=ajabh\=avo tumhehi \textbf{\~n\=ato} bhavissati}\footnote{Dhp-a\,4.51. This sentence has passive meaning.} (Today our royal status will be [a status] known by you [all].)\par
- \pali{m\=aga\d n\d diy\=aya \textbf{k\=arita\d m} bhavissati}\footnote{Dhp-a\,2.21. This sentence has casual passive meaning.} ([The action] will be made done by M\=aga\d n\d diy\=a.)\par

\item Object complement:\par
- \pali{\textbf{appamatta\d m} pana \textbf{pamatta\d m} karosi}\footnote{Dhp-a\,2.29} ([You] make a careless [one] careful.)\par

\item Adverbial (in acc.):\par
- \pali{Satth\=a bhaddiyanagare \textbf{yath\=abhiranta\d m} viharitv\=a pakk\=ami}\footnote{Dhp-a\,4.53} (The Buddha, having stayed in Bhaddiya city as long as he liked, [then] went away.)\par

\item As finite verb:\par
- \pali{Appam\=adena maghav\=a, dev\=ana\d m se\d t\d thata\d m \textbf{gato}}\footnote{Dhp\,2.30. This sentence is in active structure.} (The king of gods became the best of gods because of carefulness.)\par
- \pali{Satth\=a ``na, bhikkhave, id\=aneva, pubbepesa tumh\=aka\d m antar\=ayamak\=asiyev\=a''ti vatv\=a tehi \textbf{y\=acito}}\footnote{Dhp-a\,12.159. This sentence is in passive structure.} (The Buddha was asked by those [monks], having said ``Not only in this time, monks, [but] also in the past this [person] made you in danger.'')\par
- \pali{imasmi\d m v\=are te \textbf{cir\=ayita\d m}}\footnote{Dhp-a\,3.38. This sentence is impersonal passive.} (In this time, delaying was done by you.)\par
- \pali{nand\=aya pokkhara\d n\=i \textbf{k\=arit\=a}}\footnote{Dhp-a\,2.30. This sentence is causal passive.} (A pond was made built by Nand\=a.)\par

\item Verb of absolute construction:\par
- \pali{Atha therassa nidda\d m anokkamantassa pa\d thamam\=ase \textbf{atikk\-ante} majjhimam\=ase \textbf{sampatte} akkhirogo uppajji}\footnote{Dhp-a\,1.1} (Then, while the senior monk did not fall into sleep, when the first month passed, the second month was reached, the eye illness arose.)
\end{compactenum}

\phantomsection
\addcontentsline{toc}{subsection}{\pali{An\=iya, tabba}}
\subsection*{\fbox{\pali{An\=iya, tabba}}}

This group of verbal \pali{kita} is called future passive participle by P\=ali scholars. It has nothing to do with future tense, but rather it has imperative/optative meaning, which does not indicate time. A marked characteristic of these verb forms is they are used only in passive structure. Moreover, like verbs in \pali{ta} form, these verbs can finish sentences like a finite verb. There are a number of uses concerning these verbs described as the following (see some more information in Chapter \externalref{32, 33} in PNL):

\begin{compactenum}[(1)]
\item Modifier (also subject and object complement):\par
- \pali{uttarimpi \textbf{kattabba\d m} [pu\~n\~na\d m] atthi}\footnote{Dhp-a\,3.36} (There is [merit] that should be done even more [than this].)\par
- \pali{\textbf{rama\d n\=iya\d m} \d th\=ana\d m n\=ama sabbesa\d m piya\d m}\footnote{Dhp-a\,2.30} (a kind of delightful place, beloved to all)\par

\item Subject:\par
- \pali{eva\d m ariyamagga\~n\=a\d naggin\=api mahant\=ani ca khuddak\=ani ca sa\d myojan\=ani \d dahantena \textbf{gantabba\d m} bhavissati}\footnote{Dhp-a\,2.31} (The going of hindrances, big and small, by burning with fire of wisdom of the noble path in such a manner, will happen.)\par
- \pali{Manuss\=ana\d m ma\.ngal\=ama\.ngala\d t\d th\=anesu bhikkh\=uhi \textbf{gantabba\d m} hoti}\footnote{Dhp-a\,7.91} (There is the going by monks to ceremonies of people, [both] festive and mortuary.)\par
- \pali{\=ayuva\d d\d dhanakum\=arena kira sattame divase \textbf{maritabba\d m} abhavissa}\footnote{Dhp-a\,8.109} (It is said that there might be the dying by \=Ayuva\d d\d dhanakum\=ara in the seventh day.)\par

\item As finite verb:\par
- \pali{k\=ara\d nenettha \textbf{bhavitabba\d m}}\footnote{Dhp-a\,2.30} (It should be a cause in this.)\footnote{This impersonal passive sentence is difficult to translate. It can go more literally as ``By a cause in this (\pali{k\=ara\d nena + ettha}), existing [of the cause] should be the case.''}\par
- \pali{\textbf{kattabba\d m} kusala\d m bahu\d m}\footnote{Dhp\,4.53} (Many wholesome [actions] should be done.)\\[1.5mm]
- \pali{\textbf{Kara\d n\=iya}matthakusalena}\footnote{Snp\,143} ([The action] should be done by [one who is] wise in usefulness.)\footnote{The sentence can be broken down to \pali{[kamma\d m] kara\d n\=iya\d m atthakusalena [puggalena]}.}\par
- \pali{buddhagu\d na\d m \textbf{j\=an\=apetabba\d m}}\footnote{Dhp-a\,4.49} (The quality of the Buddha should be made known.)\par
- \pali{sace me gata\d t\d th\=ane dh\=itu doso uppajjati, tumhehi \textbf{sodhetabbo}}\footnote{Dhp-a\,4.53} (If a mistake of my daughter arises upon the place [she has] gone, [it] should be corrected by you.)\par

\item Verb of absolute construction:\par
-\pali{a\~n\~nasmi\d m \textbf{kathetabbe} a\~n\~nameva katheti}\footnote{Dhp-a\,11.152} (When other words should be said, [he] says yet other words.)\par
\end{compactenum}

\clearpage
\phantomsection
\addcontentsline{toc}{subsection}{\pali{Tu\d m}}
\subsection*{\fbox{\pali{Tu\d m}}}

This is roughly equivalent to infinitive in English. Another rare alternative form of this is \pali{tave} or \pali{tve}. Verbs in \pali{tu\d m} form is easy to recognize, as well as easy to handle, comparing to other \pali{kita} verbs. For more information about its use, see Chapter \externalref{34} of PNL. Important functions of this verb form can be listed as follows:

\begin{compactenum}[(1)]
\item As dative case:\par
- \pali{Paripphandatida\d m citta\d m, m\=aradheyya\d m \textbf{pah\=atave}}\footnote{Dhp\,3.34} (This mind trembles to avoid the realm of Death.)\par
- \pali{m\=atu abhidhammapi\d taka\d m \textbf{desetu\d m} gato}\footnote{Dhp-a\,14.181} ([The Buddha] went [to T\=avati\d msa] to preach the Abhidhamma to [his] mother.)\par

\item Subject (as nt.):\par
- \pali{bhikkhun\=a n\=ama k\=ay\=ad\=ini \textbf{rakkhitu\d m} va\d t\d tati}\footnote{Dhp-a\,17.231. The \pali{tu\d m} clause is mostly in passive structure. The instrumental actor (\pali{bhikkhun\=a}) is a clue. This use of \pali{va\d t\d tati} with \pali{-tu\d m} is mainly found in post-canonical literature.} (To protect the body, etc., by a monk is suitable.)\par
- \pali{Ayutta\d m tumh\=aka\d m appamattata\d m udaka\d m niss\=aya anagghe khattiye \textbf{n\=asetu\d m}}\footnote{Dhp-a\,15.197. This sentence can be seen as having verb `to be' left out. And \pali{(a)yutta\d m} works as adjective. This idiomatic use is also mostly found in post-canonical texts.} (To destroy priceless warriors by [just] the reason of your little water was not suitable.)\par
- \pali{sa\.ng\=ama\d m oti\d n\d nahatthino hi cat\=uhi dis\=ahi \=agate sare \textbf{sahitu\d m} bh\=aro}\footnote{Dhp-a\,2.21. This sentence has verb `to be' omitted. Please note on genders used here. As the subject, \pali{sahitu\d m} is nt.\ equivalent, but \pali{bh\=aro}, the subject complement, is m.} (To endure arrows having come from the four directions [is] the burden of an elephant having gone down to the battle.)\par
- \pali{may\=a im\=asa\d m dvinna\d m sampatt\=ina\d m nipph\=adaka\d m kamma\d m \textbf{k\=atu\d m} va\d t\d tati}\footnote{Dhp-a\,1.9} (To do the action for producing these two fortunes by me is suitable.)\par

\item Object:\par
- \pali{Bhante, aha\d m mahallakak\=ale pabbajito ganthadhura\d m \textbf{p\=uretu\d m} na sakkhiss\=ami}\footnote{Dhp-a\,1.1} (Sir, having gone forth in old age, I will not be able to fulfill the study of scriptures.)\par
- \pali{aha\d m pana s\=uriyassa \textbf{uggantu\d m} na dass\=ami}\footnote{Pps2\,65 (MN-a\,56). Here, \pali{uggantu\d m} functions like an action noun.} (I will not give the sun's rising.)\par
- \pali{s\=uro hutv\=a patta\d m \textbf{gahetu\d m} avisahanto}\footnote{Dhp-a\,2.21. The verb can be used in subordinate clause like this one.} (Having been courageous, [he], being unable to hold the bowl, \ldots)\par

\item Modifier:\par
- \pali{ima\d m sa\d mvacchara\d m idheva vasitv\=a bha\d n\d da\d m \textbf{vikki\d nitu\d m} cittamak\=asi}\footnote{Dhp-a\,20.286. This can be seen as object complement (\pali{cittamak\=asi = citta\d m + ak\=asi}).} ([He], having lived here this year, made [his] mind for selling good.)\par

\item In compounds (mostly with \pali{k\=ama}):\footnote{See also an example in \pali{tv\=a} part below concering action nouns.}\par
- \pali{So buddhappamukhassa sa\.nghassa s\=aligabbhad\=ana\d m \textbf{d\=atuk\-\=amo} hutv\=a je\d t\d thabh\=atika\d m upasa\.nkamitv\=a}\footnote{Dhp-a\,1.11} (He, having been one desiring to give ripening young rice to the Sangha with the Buddha as the head, approaching the eldest brother, \ldots)\par
\end{compactenum}

\phantomsection
\addcontentsline{toc}{subsection}{\pali{Tv\=a}}
\subsection*{\fbox{\pali{Tv\=a}}}

This is called `absolutive' and other names by scholars. There is really no English grammatical term suitable for this. We use `absolutive' nonetheless, for it just makes a good distinction. This verb form can be alternatively in \pali{tv\=ana} and \pali{tuna} or \pali{t\=una}, but far rarer. I have talked about verbs in \pali{tv\=a} form in Chapter \externalref{31} of PNL. Basically, this verb form appears only in subordinate clauses (but see blow). Many new students think this verb has only past meaning. That is not the case because its meaning depends on the context, as described below:

\begin{compactenum}[(1)]
\item Verb of a previous action of the same subject:\par
- \pali{r\=aj\=a \textbf{\~natv\=a} amhe n\=asessati}\footnote{Dhp-a\,2.21} (The king, having known, will destroy us.)\par
- \pali{Tassa sakaladivasa\d m kamma\d m \textbf{katv\=a} ch\=atassa sar\=ire v\=at\=a kuppi\d msu}\footnote{Dhp-a\,2.21} (Having done work all day, the [internal] wind agitated in his hungry body.)\par
- \pali{udake \textbf{nimujjitv\=a} a\~n\~nena \d th\=anena \textbf{uttaritv\=a} gaccheyy\=asi}\footnote{Dhp-a\,1.3} ([You], having dived into the water, should go, [by] having come out by other place.)\par
- \pali{ki\d m me janassa saddh\=adeyya\d m \textbf{niv\=asetv\=a} vicara\d nena, attano pilotikameva niv\=asess\=ami}\footnote{Dhp-a\,10.143} (What [is the use] to me by going about, having put on [the cloth] given by people's faith? [I] will just wear my own old rag.)\par
- \pali{sabbepi sa\.nkh\=ar\=a \textbf{hutv\=a} abh\=ava\d t\d thena anicc\=a}\footnote{Dhp-a\,20.277} (Even all conditioned things [are] impermanent, by the sense [that they], having appeared, [then] disappear.)\footnote{The main clause here is ``\pali{sabbepi sa\.nkh\=ar\=a anicc\=a [honti]}.'' Hence, \pali{hutv\=a} relates to the first part of compound \pali{abh\=ava\d t\d thena} (\pali{abh\=ava + attha}). For \pali{attha/a\d t\d tha}, see \pali{attha} in PTSD.}\par
- \pali{ajjeva may\=a \textbf{pal\=ayitv\=a} pabbajitu\d m va\d t\d tati}\footnote{Dhp-a\,7.98. In this instance, the \pali{tv\=a} action is prior to a \pali{tu\d m} verb. This sentence has passive structure marked by \pali{may\=a} (ins.). A more precise translation can be, ``Today, having escaped, it is suitable to go forth by me.''} (Today escaping to go forth by me is suitable.)\par
- \pali{bhikkhun\=a n\=ama attan\=a paccaye \textbf{labhitv\=a} a\~n\~na\d m \textbf{anoloketv\=a} sayameva paribhu\~njitu\d m va\d t\d tati}\footnote{Dhp-a\,8.101.} ([When a monk] normally gets food by his own, it is suitable to eat alone, not looking at others.)\par

\item Verb of the same action repeated (to stress that the action is complete):\par
- \pali{Sopi nikkhamitv\=a pabbaji, \textbf{pabbajitv\=a} ca pana na cirasseva arahatta\d m p\=apu\d ni}\footnote{Dhp-a\,6.84. Here, \pali{pabbajitv\=a} repeats \pali{pabbaji}, whereas \pali{nikkhamitv\=a} is a previous action.} (He [indeed], having left [home], went forth. Having gone forth, in not a long time [he] attained the arahantship.)\par
- \pali{yena bhagav\=a tenupasa\.nkami\d msu, \textbf{upasa\.nkamitv\=a} bhagavanta\d m abhiv\=adetv\=a ekamanta\d m nis\=idi\d msu, ekamanta\d m nisinn\=a kho te bhikkh\=u bhagavanta\d m etadavocu\d m}\footnote{Dhp-a\,2.21. Here, \pali{upasa\.nkamitv\=a} repeats \pali{upasa\.nkami\d msu}, whereas \pali{abhiv\=adetv\=a} is the subsequent action prior to \pali{nis\=idi\d msu}. When several \pali{tv\=a} verbs run in succession, they show a sequence of actions. It should be noted that past participle can do the same job, as \pali{nisinn\=a} is used here in the last part.} \\(Which place the Buddha [stayed], monks approached that place. Having approached [him], having bowed down to the Buddha, [they] sat on one side. Having sat on one side, those monks said this to the Blessed One, \ldots)\par
- \pali{tatr\=aha\d m, \=avuso, eka\d m peta\d m addasa\d m, tassa evar\=upo n\=ama attabh\=avo, aha\d m ta\d m \textbf{disv\=a}}\footnote{Dhp-a\,10.136. Here, \pali{disv\=a} repeats \pali{addasa\d m}.} (Friend, I have seen one ghost in that [mountain]; its character [is] such that. I, having seen that \ldots)\par

\item Verb of a simultaneous action:\par
- \pali{\textbf{ugghosetv\=a} vicaranta\d m}\footnote{Dhp-a\,10.142. This use is equivalent to present participle, as shown nearby in the same text, \pali{ugghosento vicar\=ami}. In this instance, \pali{vicaranta\d m} functions as an adjective. This shows that \pali{tv\=a} verbs can be concurrent to any form of verbs, and they even look like adjectives sometimes (see below).} ([I who was] traveling, announcing, \ldots)\par
- \pali{tattha jh\=ana\d m \textbf{sam\=apajjitv\=a} nis\=idi}\footnote{Dhp-a\,2.21} ([They] sat, engaging in deep meditation.)\par
- \pali{satt\=aha\d m ekapalla\.nkena nisinno hoti a\~n\~natara\d m sam\=adhi\d m \textbf{sam\=apajjitv\=a}}\footnote{Dhp-a\,4.56} ([Ven.\ Mah\=akassapa] sat by one stretch during seven days, engaging in certain meditation.)\par
- \pali{Bhikkh\=u bhagavanta\d m \textbf{\=ad\=aya} j\=ivakambavana\d m agama\d msu}\footnote{Dhp-a\,7.90} (Monks, carrying the Blessed One, went to J\=ivaka's mango grove.)\par
- \pali{ima\d m sa\d mvacchara\d m idha \textbf{vasitv\=a} bha\d n\d da\d m vikki\d nitv\=a gamiss\=ami}\footnote{Dhp-a\,20.286} (During this year [I], living here, selling good, [then] will go.)\footnote{This is a little tricky. The first \pali{tv\=a} verb, \pali{vasitv\=a} happens at the same time as \pali{vikki\d nitv\=a} (not \pali{gamiss\=ami}). But \pali{vikki\d nitv\=a} itself is a previous action of \pali{gamiss\=ami}. Unfortunately, English translation cannot show this nuance.}\par

\item Verb of a subsequent action of the same subject:\par
- \pali{gaccha, t\=asa\d m \textbf{datv\=a}}\footnote{Dhp-a\,2.21} (Go, giving [the chickens] to those [women].)\\[1.5mm]
Using \pali{tv\=a} verbs for a subsequent action is quite rare to find. One reason, I think, is ambiguity. Many instances suitable to put in this meaning can be seen as simultaneous events as well.

\item Verb of a previous action of action nouns: Derivative nouns, mostly with \pali{yu} (Kacc/Sadd) or \pali{ana} (Mogg) operation, can have a previous action marked by \pali{tv\=a} verbs.\par
- \pali{Attano pa\~n\~n\=anur\=upena eka\d m v\=a dve v\=a nik\=aye sakala\d m v\=a pana tepi\d taka\d m buddhavacana\d m \textbf{ugga\d nhitv\=a} tassa \textbf{dh\=ara\d na\d m}, \textbf{kathana\d m}, \textbf{v\=acanan}ti ida\d m ganthadhura\d m n\=a\-ma}\footnote{Dhp-a\,1.1} (When [one] has learned the Buddha's word, the Tipi\d taka, one or two collections or all of them, suitably to one's own wisdom; the remembering, saying [and] teaching of that is called this scriptural study.)\par
- \pali{aya\d m m\=atu thana\d m \textbf{chinditv\=a} pitu v\=a galalohita\d m \textbf{n\=iharitv\=a} \textbf{kh\=adana}samattho atikakkha\d lo}\footnote{Dhp-a\,8.100. Sometimes, the action noun is a part of compounds.} (This [man] [is] very cruel, having ability in eating mother's breast bitten and [eating] blood taken from father's throat.)\par
- \pali{ukkhepakehi v\=ariyam\=an\=anampi ca tesa\d m ta\d m \textbf{anupariv\=are\-tv\=a} \textbf{vicara\d na}bh\=ava\~nca \=arocesi}\footnote{Dhp-a\,1.6. For compounds with \pali{bh\=ava}, see also Chapter \ref{chap:minors}.} ([One monk] also told [the Buddha] about the state of traveling of those even being ostracised by the punishers, surrounding that \mbox{[Dhamma preacher]}.)\par
- \pali{dvinna\d m geh\=ana\d m antare mahanta\d m \=av\=a\d ta\d m \textbf{kha\d n\=apetv\=a} as\=itisaka\d tamatt\=ani khadirad\=ar\=uni \textbf{\=ahar\=apetv\=a p\=ur\=apetv\=a} aggi\d m \textbf{datv\=a} amhe a\.ng\=ara\=av\=a\d te \textbf{p\=atetv\=a nigga\d nhituk\=amo}}\footnote{Dhp-a\,4.58. It is sensible that this form of verb also works with compounds of \pali{tu\d m} verbs like this one.} ([Garahadinna], having made a big pit dug between two houses, having made 80 cartloads of acacia wood fetched and put [into the pit], having given fire [to it], wants to assault me, [by] making me fall into the burning pit.)\par

\item As finite verb: There are certain conditions that make this inevitable as described below.\footnote{The main idea is the subject of \pali{tv\=a} clause is not exactly the same as the main clause. So, it should be seen as a separated sentence finished by \pali{tv\=a} verbs.}\par
\begin{compactenum}[(a)]
\item Subject of \pali{tv\=a} clause is collective:\par
- \pali{Te ubhopi ekamaggen\=api \textbf{agantv\=a} eko pacchimadv\=arena magga\d m ga\d nhi, eko puratthimadv\=arena}\footnote{Dhp-a\,20.281} \\(Those two [monks] did not even go by the same way. One took the path by the west gate, one by the east gate.)\par
- \pali{Eva\d m \textbf{va\d d\d dhetv\=a} nagarav\=as\=i a\d t\d tha ko\d tiyo ad\=asi.}\footnote{Dhp-a\,9.124. In the text we have, \pali{va\d d\d dhetv\=a} is repeated twice. I remove the redundant one according to the Thai edition.} ([The two rich men] made [their] wealth flourish as such. The one who lives in the city gave eighty millions.)\par
- \pali{He\d t\d th\=aga\.ng\=ayampi dve itthiyo nh\=ayam\=an\=a ta\d m bh\=ajana\d m udaken\=ahariyam\=ana\d m \textbf{disv\=a} ek\=a ``mayheta\d m bh\=ajana''nti \=aha.}\footnote{Dhp-a\,26.416} (Two women, bathing in the lower part of the Ganges, saw that bowl floating in the water. One [woman] said thus, ``That bowl is mine.'')\par
\item Subject of \pali{tv\=a} clause is distibutive:\par
- \pali{ekameka\d m bhikkhusata\d m nisinnanisinna\d t\d th\=aneyeva saha pa\d tisambhid\=ahi arahatta\d m patv\=a veh\=asa\d m \textbf{abbhuggantv\=a} sabbepi te bhikkh\=u \ldots\ tath\=agatassa suva\d n\d nava\d n\d na\d m sar\=ira\d m va\d n\d nent\=a thoment\=a p\=ade vandi\d msu}\footnote{Dhp-a\,25.376} (Each of one hundred monks, having attained the arhantship with discriminating knowledge in their own seat, rose up to the air. All monks, \ldots, having praised, paid homage to the golden body of the Buddha at the foot.)\footnote{This P\=ali sentence can be cut into two English sentences. The first has \pali{abbhuggantv\=a} as the main verb. The subject of this is the individual monk in the group of 100. The key word that makes us decide as such is \pali{ekameka\d m} (eka\d m eka\d m) which mean `each of them.' So, \pali{abbhuggantv\=a} is the verb of the distributive individuals. In the latter part, the subject who do the homage is the whole group of 100 monks. The two parts therefore do not exactly have the same subject.}\par
\item Subject of \pali{tv\=a} clause is of the same kind as the nearby clauses:\par
- \pali{j\=ati jar\=aya santika\d m, jar\=a by\=adhino santika\d m, by\=adhi mara\d nassa santika\d m \textbf{pesetv\=a} mara\d na\d m ku\d th\=ariy\=a chindant\=a viya j\=ivita\d m chindati}\footnote{Dhp-a\,10.135. In Thai edition, \pali{pesetv\=a} is moved to the first part, hence ``\pali{j\=ati jar\=aya santika\d m pesetv\=a, \ldots}''} (Birth sent [beings] to the presence of decay. Decay sent [beings] to the presence of sickness. Sickness sent [beings] to the presence of death. Death cut the life like cutting [a tree] with a hatchet.)\par
- \pali{mama sar\=ira\d m majjhe \textbf{bhijjitv\=a} eko bh\=ago orimat\=ire patatu}\footnote{Dhp-a\,5.75} (My body was broken in the middle. May one part fall into this side of the riverbank.)\par
\end{compactenum}

\item Modifier:\par
- \pali{ima\d m g\=ama\d m \textbf{niss\=aya} koci \=ara\~n\~nako vih\=aro atthi}\footnote{Dhp-a\,1.1} (Is there any forest monastery depending on this village?)\par
- \pali{so g\=amato \textbf{nikkhamitv\=a} ara\~n\~ne g\=ita\d m \textbf{g\=ayitv\=a} d\=ar\=uni \linebreak uddharantiy\=a ekiss\=a itthiy\=a g\=itasadda\d m sutv\=a sare nimitta\d m ga\d nhi}\footnote{Dhp-a\,1.1} (Having listened to the singing of one woman who has left the village, has sung the song in the forest, has taken [some] wood, he took the sign of/in [that] sound.)\footnote{This is a bit complicated. The main idea of the sentence is the boy was stunned by the sound of singing. Other parts are used to describe the girl, the singer. So, \pali{nikkhamitv\=a} and \pali{g\=ayitv\=a} modify \pali{itthiy\=a}.}\par
- \pali{Devadhammaj\=ananake \textbf{\d thapetv\=a} avasese labh\=ami}\footnote{Dhp-a\,10.141} ([I] get the remainders [of people] excluding those who know the divine virtue.)\footnote{This idiomatic use of \pali{\d thapetv\=a} is often seen. It means like `except' or `excluding' but it is not used as preposition like in English. We have to think it as a verb. Other words that can be used in this purpose are \pali{vin\=a} and \pali{a\~n\~natra}. Both are indeclinable particles.}\par
- \pali{Bhante, d\=asakammakare \textbf{up\=ad\=aya} sabbe akkosanti}\footnote{Dhp-a\,2.21} (Sir, all, who took slaves and workers, scold.)\par

\item Adverbial:\par
- \pali{Sabbe deve \textbf{atikkamma}, sambuddhova virocati}\footnote{Pv\,319} (The Omniscient One shines beyond all gods.)\par
- \pali{Ima\d m dukanip\=ate al\=inacittaj\=ataka\d m \textbf{vitth\=aretv\=a} kathesi}\footnote{Dhp-a\,6.76} ([The Buddha?] told the story of Al\=inacitta in this twofold section in detail.)\footnote{For Al\=inacitta, see Ja\,2:11.}\par
- \pali{satthu sayanassa upari vit\=ana\d m \textbf{katv\=a} bandhi}\footnote{Dhp-a\,9.116} ([He] tied [the cloth] by making it the canopy over the Buddha's bed.)\footnote{This instance can also be seen as two simultaneous actions.}\par

\item Cause (when the action has a different subject):\par
- \pali{Ima\~nca \textbf{pitv\=ana} rasa\d m pa\d n\=ita\d m, mado na sa\~nj\=ayati sindhav\=ana\d m}\footnote{Ja\,2:65} (Intoxication is not arisen to Sindh horses because of drinking this delicious taste.)\par
- \pali{Bhante, ettak\=ani \textbf{katv\=a}pi ayy\=ana\d m dinn\=ani na nassanti}\footnote{Dhp-a\,2.21} (Sir, [the cloths] given to masters do not perish because of using even this much.)\par
- \pali{dvinna\d m kul\=ana\d m gu\d namahattata\d m \textbf{pa\d ticca} s\=avatthi\d m niss\=aya pa\~ncav\=isativass\=ani vass\=av\=asa\d m vasi}\footnote{Dhp-a\,1.1} (Because of two greatly beneficial families, [the Buddha] has lived in S\=avatth\=i for 25 rains [= years].)\par
- \pali{Ta\d m \textbf{sutv\=a} se\d t\d thino ``ya\d m k\=aremi, ta\d m na hoti; ya\d m na k\=aremi, tadeva hot\=i''ti mahanta\d m domanassa\d m uppajji.}\footnote{Dhp-a\,2.21} (Because of hearing that, the great grief happened to the rich man, [by realizing that] ``Which [action] I have [someone] do it, that is not done. Which I do not, that is done [instead].'')\par
\end{compactenum}

\bigskip
To make us easier grasp the differences among derivative verbs, I summarize them in Table \ref{tab:kitasum}. In the table, common verbal \pali{kita}s are listed. I also include two rare past participles, \pali{tavantu} and \pali{t\=av\=i} form, because they do not function fully like \pali{ta} form. And remember that when \pali{tv\=a} form is mentioned, it always includes \pali{tv\=ana} and \pali{tuna/t\=una}.

In the table, we can see that whether the verbs are used in present tense or past tense. Some verb forms have no tense at all, i.e.\ \pali{an\=iya, tabba,} and \pali{tu\d m}. In fact, we can see \pali{tv\=a} form in the same way, because its tense does not depend on the verb itself, but on the main verb in the sentence. But it is widely held that \pali{tv\=a} form has past meaning, so I put it in that way, but also remind us that sometimes it can have present meaning.

Also, we can see that whether the verbs are used in active (including causative) or passive (including impersonal and causal passive) structure. Some forms can do both functions.

In the last column, it shows that whether the verbs can finish sentences like finite verbs. Normally, \pali{tv\=a} verbs can not do this. But in some conditions it can (see above).

\begin{table}[!hbt]
\centering
\caption{Derivative verbs summarized}
\label{tab:kitasum}
\bigskip
\begin{tabular}{@{}*{2}l*{5}c@{}} \toprule
\multicolumn{2}{c}{\bfseries Verb form} & \bfseries Pres. & \bfseries Past & \bfseries Act. & \bfseries Pass. & \bfseries Finite \\
\midrule
\pali{anta} & (pr.p.) & \checkmark & & \checkmark & & \\
\pali{m\=ana} & (pr.p.) & \checkmark & & \checkmark & \checkmark & \\
\pali{ta} & (p.p.) & & \checkmark & \checkmark & \checkmark & \checkmark \\
\pali{tavantu} & (p.p.) & & \checkmark & \checkmark & & \\
\pali{t\=av\=i} & (p.p.) & & \checkmark & \checkmark & & \\
\pali{an\=iya} & (f.p.p.) & & & & \checkmark & \checkmark \\
\pali{tabba} & (f.p.p.) & & & & \checkmark & \checkmark \\
\pali{tu\d m} & (inf.) & & & \checkmark & \checkmark & \\
\pali{tv\=a} & (abs.) & (\checkmark) & \checkmark & \checkmark & \checkmark & (\checkmark) \\
\bottomrule
\end{tabular}
\end{table}

\phantomsection
\addcontentsline{toc}{section}{Nip\=ata}
\section*{3.\ Nip\=ata}

P\=ali has extensive uses of particles. Some of them are essential to the meaning, like \pali{na} (not), some are not quite so, like \pali{pana} (but, etc.), and some mean nothing at all, like \pali{pana} in some contexts. I roughly divide particles into two big groups: (a) sentence particles, and (b) word-related particles. 

As the name implies, sentence particles relate to the whole sentence. I divide this group further into two subgroups: connectors and adverbials. Connectors are those such as \pali{hi, ca, pana, tu}.\footnote{See also \emph{opening particles} in Chapter \ref{chap:sentence}.} This subgroup of particles adds little to the meaning of a sentence. So, largely we can neglect these particles without a substantial loss when we read texts.

Adverbial particles are meaningful by its own right, for example, \pali{na} (not), \pali{m\=a} ([do] not), \pali{addh\=a} (surely), \pali{avassa\d m} (certainly), \pali{puna} (again), \pali{micch\=a} (wrongly). So, taking their meaning into account is indispensable.

Word-related particles also have distinct meaning, for example, \pali{ca} (and), \pali{v\=a} (or), \pali{viya} (like), \pali{saha} (together with). Considering the meaning of these particles with their related word, therefore, is important to gain an accurate translation.

In this section, I will not list all particles and explain their uses. You can review that content in Appendix \externalref{F} of PNL. Instead, I will show only some of them that can be difficult to translate sometimes.\footnote{In Thai traditional approach, the nuances of meaning are taken very seriously in order to preserve every P\=ali word. That makes meaning assigned to some particles confusing, or downright senseless in some cases. That is the cost we pay when every bit of P\=ali sentences is preserved through translation. In English translation, however, we can take an easier stance by ignoring trivial particles, especially the meaningless ones. That is not because English translation is better or worse, but rather we should always go back to the P\=ali source. So, any translation is provisional and not worthy to take seriously. Why don't make it easy to learn first, I always wonder?}

\paragraph*{\pali{Hi, ca, pana} (also, because, but, however)} This group of particles\footnote{In Thai courses, this group consists of \pali{hi, ca, pana,} and \pali{tu}. I drop \pali{tu} here to make this easier, because uses of the term is really hard to find. In the canon, \pali{tu} is found in verses mostly, and it is virtually meaningless.} can confuse new students a lot, because of their wide range of meaning. When encountering these, it is better to ignore them and try to read the whole sentence first. Then you will realize which meaning is the best fit. If no meaning rings good, you can drop the particles altogether. See these examples:\par
- \pali{Satth\=ar\=a \textbf{hi} atisa\d nhasukhuma\d m tilakkha\d na\d m \=aropetv\=a \=adimajjhapariyos\=anakaly\=a\d no dhammo desito, na sakk\=a so ag\=aramajjhe vasantena p\=uretu\d m, pabbajiss\=ami, t\=ata}\footnote{Dhp-a\,1.1} (Because the teaching pointed out by the Buddha [is] beautiful at the beginning, the middle, and the end, leading to the subtle Three Characteristics. One is unable to fulfill that by living in the house. I will go forth, brother.)\par
- \pali{Therassa \textbf{ca} tattha neva \~n\=at\=i, na s\=alohit\=a atthi, tena saddhi\d m manteyya?}\footnote{Dhp-a\,1.1. In Thai edition, it is ``\pali{kena saddhi\d m manteyya}.''} (But the senior monk has no relative here, with whom should he consult?)\par
- \pali{Nivesanesu \textbf{pana} tesa\d m dvinna\d m dvinna\d m bhikkhusahass\=ana\d m nicca\d m pa\~n\~natt\=asan\=aneva honti.}\footnote{Dhp-a\,1.1} (Also, in their houses there are always seats prepared for 2,000 monks [in each house].)\par
- \pali{mahallakassa \textbf{hi} attano hatthap\=ad\=api anassav\=a honti}\footnote{Dhp-a\,1.1} (Even an old man's own hands and feet are disobedient.)\footnote{In this instance, \pali{hi} can be dropped without any impact.}\par
- \pali{Te vu\d t\d thavass\=a \textbf{ca pana} satth\=ara\d m da\d t\d thuk\=am\=a hutv\=a theram\=aha\d msu, ``bhante, satth\=ara\d m da\d t\d thuk\=amamh\=a''ti.}\footnote{Dhp-a\,1.1} \\(Those [monks], having been ones desiring to see the Buddha, said to the senior monk thus, ``Sir, we want to see the Buddha.'')\par

\paragraph*{\pali{Hi, ca} (in fact)} In some contexts, these two particles mean like ``Normally, \ldots'' or ``In fact, \ldots'' or ``It is true that \ldots''\par
- \pali{Itthisaddo viya \textbf{hi} a\~n\~no saddo puris\=ana\d m sakalasar\=ira\d m pharitv\=a \d th\=atu\d m samattho n\=ama natthi.}\footnote{Dhp-a\,1.1. In Thai edition, it is \pali{itth\=isaddo}.} (It is true that there is no other sound [called to be] capable to keep pervading the man's whole body like a woman's voice.)\footnote{As a particle, \pali{n\=ama}, if it is not meant `name,' is awkward to render into English. I sometimes ignore it. Here, `called to be' sounds reasonable, but redundant nonetheless. And sometimes, I translate it as `such' (see the following example).}\par
- \pali{buddh\=a \textbf{ca} n\=ama na sakk\=a pam\=adena \=ar\=adhetu\d m}\footnote{Dhp-a\,1.1. In That edition, \pali{pam\=adena} becomes \pali{sa\d thena}.} (Normally, such the Buddha is unable to be pleased by a careless [person].)\par

\paragraph*{\pali{Hi} (as the story goes)} This is like a signal to an account, in the sense of `to illustrate' or `as we put it in detail.'\par
- \pali{Ekasmi\~n\textbf{hi} samaye titthiy\=a sannipatitv\=a mantesu\d m}\footnote{Dhp-a\,10.137} (As the story goes, in one occasion adherents of other religion, having come together, thought, \ldots)\par

\paragraph*{\pali{Ma\~n\~ne} (it seems)} This denotes a speculation.\par
- \pali{kh\=i\d n\=asav\=api \textbf{ma\~n\~ne} k\=amasukha\d m s\=adiyanti}\footnote{Dhp-a\,5.69} (It seems that even arhants enjoy the sensual pleasure.)\par
- \pali{j\=an\=ati \textbf{ma\~n\~ne} satth\=a may\=a uttaritara\d m p\=a\d tih\=ariya\d m k\=atu\d m samattha\d m}\footnote{Dhp-a\,14.181} (It seems that the Buddha knows [someone] having more ability than me to make a miracle.)\par

\paragraph*{\pali{Tena hi} (if it is so)} This is like we use `in that case' or simply `then.'\footnote{See also \pali{eva\d m sante} in Chapter \ref{chap:minors}.} \par
- \pali{R\=aj\=a satthu \=arocesi ``na ga\d nhanti, bhante''ti. ``\textbf{Tena hi}, mah\=ar\=aja, aggha\d m oh\=areh\=i''ti.}\footnote{Dhp-a\,11.147} (The king told the Buddha, ``[People] do not accept, sir.'' [The Buddha replied,] ``If it is so, Your Majesty, reduce the price.'')\par
- \pali{``tumheyeva, t\=ata, r\=aj\=ana\d m y\=acath\=a''ti vatv\=a ``\textbf{tena hi}, t\=ata, ma\d m gahetv\=a y\=ah\=i''ti vutto}\footnote{Dhp-a\,11.152} (``Father, we should request to the king'' said [the son]. ``If it is so, son, leading me, go'' said [the father].)\footnote{When used for addressing, \pali{ta\=ta} can mean either `father' or `son' or `brother.'}\par

\paragraph*{\pali{Atha ca pana} (then why?)} In a way, this is close to `\pali{tena hi},' but it implies a question.\par
- \pali{bhadde, tva\d m ito pubbe amhehi na di\d t\d thapubb\=a, \textbf{atha ca pana} no mahanta\d m sakk\=ara\d m karosi, j\=an\=asi tva\d m amhe}\footnote{Dhp-a\,3.43} (Madam, you was never seen by us before, then why [you] pay great honor to us, do you know us?)\par 
- \pali{Vi\d ta\d t\=ubho disv\=a ``nanu bha\d ne `s\=akiy\=a asattagh\=atak\=amh\=a'ti vada\-nti, \textbf{atha ca pana} me purise n\=asent\=i''ti.}\footnote{Dhp-a\,4.47. We can break down \pali{asattagh\=atak\=amh\=a} to \pali{na + satta + gh\=ataka + amha}. Since this is a compound, it funcions as subject complement. So, precisely, in the inner quote, it is read ``The S\=akyas [are] we-are-no-killer-of-beings.''} (Vi\d ta\d t\=ubha, having seen [that and said], ``Man, don't the S\=akyas say that `We do not kill beings'? Then why [they] kill my people?'')\par
- \pali{\textbf{Atha ca pana} loka\d m va\~ncento ``aha\d m v\=atabhakkho, ekap\=adena ti\d t\d th\=ami, na nis\=id\=ami, na nipajj\=am\=i''ti vadesi, mamampi va\~ncetuk\=amosi}\footnote{Dhp-a\,5.70} (Then [you] said, cheating the world, ``I feed on wind, stand on one foot, not sit, not sleep.'' Do you want to cheat even me?)\par

\paragraph*{\pali{Atha v\=a, api ca, v\=a} (in addition)} These particles denote ``the following condition is also the case.''\par
- \pali{kulava\d msa\d m \d thapess\=ami, d\=ayajja\d m pa\d tipajj\=ami, \textbf{atha v\=a} pana pet\=ana\d m k\=ala\.nkat\=ana\d m dakkhi\d na\d m anuppadass\=ami}\footnote{D3\,267 (DN\,31)} ([I] will retain the family line. [I will] behave as a [good] heir. Also when [the parents] die, [I] will dedicate merit [to them].)\par
- \pali{\textbf{atha v\=a} panassa eva\d m di\d t\d thi hoti ``yo me aya\d m att\=a vado vedeyyo \ldots''}\footnote{M1\,19 (MN\,2)} (In addition, a view happens to that [person] thus, ``Which this my self, the speaker, the knower, \ldots'')\par 
- \pali{\textbf{api ca} kho pana tumhe na id\=aneva pa\~ncasu \d th\=anesu asa\d mvut\=a, pubbepi asa\d mvut\=a}\footnote{Dhp-a\,25.360} (In addition, you [all] did not restrain the five senses only in this time, even in the past [you were also] unrestraint.)\footnote{Rendering \pali{api ca} as `but' here sounds equally sensible.}\par 
- \pali{\textbf{api ca} kho pana br\=ahma\d nassa arahantesu adhimatta\d m pema\d m}\footnote{Dhp-a\,26.383} (In addition, the brahman's love in the arhants [is] too much.)\par

\medskip
Interestingly, in the canon `\pali{api ca}' tends to mean `but,' for example:\par
- \pali{n\=aha\d m, bhante, bhagavanta\d m dha\d msemi; \textbf{api ca}, eva\d m vad\=ami \ldots}\footnote{D1\,482 (DN\,11)} (Sir, I do not assault the Blessed One, but [I mean to] say thus \ldots)\par 

\medskip
However, in some context both `but' and `also' are equally awkward. So, it might be better to be left out, as we see in this instance.\par
- \pali{Te sabhiyena paribb\=ajakena pa\~nhe pu\d t\d th\=a na samp\=ayanti; asamp\=ayant\=a kopa\~nca dosa\~nca appaccaya\~nca p\=atukaronti. \textbf{Api ca} sabhiya\d m yeva paribb\=ajaka\d m pa\d tipucchanti.}\footnote{Snp\,515} (They are not able to answer the questions asked by wanderer Subhiya. Being unable to answer [those questions], [they] show anger, ill temper, and sulkiness. [Also/but] [they] just counter-question wanderer Subhiya.)\par

\paragraph*{\pali{Ki\~nc\=api} (although)} This unit is often used in concessive clauses (see Chapter \externalref{29} of PNL). It is normally paired with \pali{pi, ca,} or \pali{pana}.\par
- \pali{br\=ahma\d na, \textbf{ki\~nc\=api} tva\d m eva\d m vadesi, ida\d m \textbf{pana} nikkilesasseva pada\d m}\footnote{Dhp-a\,14.179} (Brahman, although you said that, [but] this footprint [is] of the one free from defilement.)\par

\paragraph*{\pali{Vin\=a, a\~n\~natra} (without)} Like \pali{saha} and \pali{saddhi\d m} (see Chapter \externalref{12} of PNL), \pali{vin\=a} is also accompanied with terms in instrumental case. Likewise, \pali{a\~n\~natra}\footnote{Do not be confused with \pali{a\~n\~natara} (certain, some, yet other).}, normally means `elsewhere,' can work in the same manner. Sometimes, it sounds more like `except.' Moreover, \pali{\d thapetv\=a} can be used in the same purpose but differently (see \pali{tv\=a} part above).\par
- \pali{ayye, bhagav\=a nandakum\=ara\d m gahetv\=a gato, tumhehi ta\d m \textbf{vin\=a} karissati}\footnote{Dhp-a\,1.13} (Madam, the Blessed One, having taken Prince Nanda, went [away]. He will make him without you [= separate both of you].)\par
- \pali{ajjatagged\=an\=aha\d m, \=avuso \=ananda, \textbf{a\~n\~natr}eva bhagavat\=a \textbf{a\~n\~natra} bhikkhusa\.ngh\=a uposatha\d m kariss\=ami}\footnote{Dhp-a\,1.17. Here, \pali{a\~n\~natra bhikkhusa\.ngh\=a} shows that the term can also be used with ablative case.} (From today on, Ven.\ \=Ananda, I will do the Uposatha [= recite the Vinaya] without the Blessed One, without the Sangha.)\par
- \pali{imin\=a katakamma\d m \textbf{a\~n\~natra} satth\=ar\=a ko j\=anissati}\footnote{Dhp-a\,9.127} \\(Without the Buddha, who will know the action done [in the past] by this [action]?)\par
- \pali{na me ima\d m soka\d m a\~n\~ne nibb\=apetu\d m sakkhissanti \textbf{a\~n\~natra} tath\=agatena}\footnote{Dhp-a\,10.142} (No one will be able to do away with my grief except the Blessed One.)\par

\paragraph*{\pali{Aho} (oh/wow, alas)} This exclamation marker has two poles of meaning, positive (surprise) and negative (pity).\par
- \pali{\textbf{aho} buddh\=ana\d m kath\=a n\=ama acchariy\=a}\footnote{Dhp-a\,13.174} \\(Oh!, such speech of the buddhas [is] wonderful.)\par 
- \pali{\textbf{aho} maya\d m mah\=aup\=asik\=a bh\=ariya\d m kamma\d m ak\=asi}\footnote{Dhp-a\,3.35} (Alas!, this great lay devotee has done a grave action.)\par 

\section*{Concluding remarks}

Comparing to English, part of speech analysis for P\=ali translation is relatively easy because there are less category to worry about. We do not have to concern about preposition, conjunction, and other word classes. But by the fact that P\=ali is a highly inflectional language, knowing only word classes used in P\=ali sentences is not enough to know the meaning. We have to know the relation among terms in sentences, as I have demonstrated in Chapter \ref{chap:sentence}. However, to master P\=ali translation, all knowledge about grammar and usages has to be put together, as well as the theoretical background we have discussed from the beginning. 

In a way, learning to read P\=ali is more difficult to just say it, as we have learned in PNL. When we say a thing, we choose our own wording to make it easy to understand. But when we read a written text, we have to follow the writer's style, which sometimes is difficult to decode. That is to say, if the learners follow my sequence of learning strictly by trying to make sense of the general usages first, then tackle the real texts later, they will find that reading P\=ali is not so difficult.
