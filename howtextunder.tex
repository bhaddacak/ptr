\chapter{How is text understood?}\label{chap:howtextunder}

As we have learned so far, from the nature of our cognition to the complex relation between signs and meaning, obtaining original understanding from reading a text is not easy, if not possible. If it is so easy, we have no need to learn all of these. Because of its difficulty, we have to be careful and address it in various perspectives. In this section, I will lead the readers to another area concerning how we understand text. It is called \emph{hermeneutics}.\footnote{The theory and methodology of interpretation, especially of scriptural text. (\url{https://www.ahdictionary.com/word/search.html?q=hermeneutics})}

\phantomsection
\addcontentsline{toc}{section}{Introduction to hermeneutics}
\section*{Introduction to hermeneutics}

Unlike in the study of biblical texts, hermeneutics is rarely mentioned in the study of Buddhist scriptures. That does not mean Buddhist scholars have no methodology in reading texts, but rather the methodology was rigidly set by former exegetes as we see in the commentaries and grammatical books. We just have to follow the tradition. Discussion on the part of method is hardly found.

To grasp what it is all about, we have to know the subject historically. In the modern era, it is said that Friedrich Schleiermacher (1768--1834) established the field. To him, hermeneutics is the art of understanding, in a nutshell, ``the reexperiencing of the mental processes of the text's author.''\footnote{\citealp[p.~86]{palmer:hermeneutics}} That is the direct way when we think what a reader should get from a reading. By the best of our knowledge nowadays, reproducing the same mental states from one person to another is not (yet) possible. The author indeed has something in mind, but once the idea is converted into words its clarity goes, because ambiguities are rife at every level of language processing, as well as other factors we have seen previously. So, the ideal principle of Schleiermacher is untenable in practice.

Before we go to other theorists of hermeneutics, discussing the role of the author is worthwhile here. I have a question: ``Does the intention of the author really matter in text reading?'', and the following one: ``Isn't it better to put aside the author and read the text as it is presented to us?'' The shift of significance from the author to the text itself has a great impact in literary studies. As W.\,K.\ Wimsatt Jr.\ and M.\,C.\ Beardsley put it, ``the design or intention of the author is neither available nor desirable as a standard for judging the success of a work of literary art.''\footnote{\citealp[p.~468]{wimsatt:fallacy}} By this view, the intention of the author, even the life of the author, is irrelevant to text reading. The readers should get the best out of the text by their own right. Does this sound alright?

Some scholars say `No,' because text cannot mean whatever we want it to mean. If text ``means what it says, then it means nothing in particular.''\footnote{\citealp[p.~13]{hirsch:validity}} Even though text can be read in a number of ways, there must be only one true meaning, which the author intends. This argument sounds reasonable, because most of the time we think we get the author right. If it is not so, literary criticism will be clueless. However, seeing that the author is a unified person who consciously has a clear, consistent intention in producing the whole work is quite a mistake, as far as cognitive psychology can tell us. In a reading religious text which the authors are really unknown or long dead, the readers have to reconstruct the original intention by their own understanding. That sounds equal to that the readers control the meaning by themselves. In practice, religious authority play a major role to determine the text's intention. Keep this in mind for a moment. This point is important as we will discuss more later when we talk about `discourse.'

Another scholar who `kills' the author is Roland Bathes (1915--80). He declares ``the death of the author.''\footnote{\citealp[pp.~142--8]{barthes:image}} To Barthes the person who produce the text does not exist in the first place. It is the subject `I' posited by language that tells the story. The author is therefore a constructed entity. Barthes holds such a view because he sees text as follows:

\begin{quote}
[A] text is not a line of words releasing a single `theological' meaning (the `message' of the Author-God) but a multi-dimensional space in which a variety of writings, none of them original, blend and clash. The text is a tissue of quotations drawn from the innumerable centres of culture.\footnote{\citealp[p.~146]{barthes:image}}
\end{quote}

So, a text is in fact an amalgam of other previous texts. We call this nowadays \emph{intertextuality}.\footnote{coined by Julia Kristeva} To say that text has only one intention to be read is therefore improbable, because every part has its underlying intention. So, intentions inevitably blend and clash as Barthes puts it. That means the only control moves to the reader. Thus at the cost of the death of the author, it is ``the birth of the reader.''\footnote{\citealp[p.~148]{barthes:image}}

Shifting focus from the author or the intention of text to the reader sounds unfamiliar to religious mindsets, because religious adherents are supposed to obey religious authority. The dominating reading is preferable to an individual reading, so to speak. When an uncommon interpretation appears, it is likely to be seen as heretical. We have often seen this phenomenon throughout our history. This belief stands on a mythical view that there is only one true reading from text with some narrow leeway, maintained by authority. Other radical readings are all false. This position is promoted not because of its reasonableness, but rather because the power structure prefers certain reading over others. We will come to this more later.

Now we will return to hermeneutics in a more general sense---science of interpretation. This field of study addresses the problem of how to interpret a text in the right way. I will not make a sharp distinction between `understanding' and `interpretation' and `reading.' I use these more or less as synonyms. Science of interpretation sounds parochial, because there is no universal principle of interpretation. You cannot use methods of biblical interpretation to read the P\=ali canon, so to speak. However, there is a conceptual idea central to the problem of interpretation called \emph{hermeneutic circle}.

Simply put, in making sense out of a text, the parts make the whole understood, and the whole makes the parts understood. To understand the text we have to understand its constituent parts first, i.e.\ words, sentences, paragraphs, and so on. But to understand the lower elements correctly, we have to know that what the higher parts is all about. So, the process runs in circle. Understanding of the parts enhances understanding of the whole and vice versa. Even though this sounds endless and impossible to reach perfect understanding, but our understanding grows better in each run.

\begin{quote}
Interpretation never comes to an end -- or, at least, any ending to which interpretation comes is always temporary, always contingent, always open to revision.\footnote{\citealp[p.~3]{malpas:intro}}
\end{quote}

As noted above, we always can get a better understanding. The existing interpretation is not necessary the best one. When the context changes, our understanding should be changed accordingly.

Another figure next to Schleiermacher as a theorist of hermeneutics is Wilhelm Dilthey (1833--1911). Dilthey's idea is not far from Schleiermacher in believing that we somehow can reexperience the author's life through his or her works. A key difference in Dilthey's idea is not just immediate experience when one produces a work, but the whole life of the author that is to be experienced. For him, hermeneutic circle is more or less the circle of life of the author, as he expresses thus:

\begin{quote}
This circle repeats itself in the relation between an individual work and the development and spiritual tendencies of its author, and it returns again in the relation between an individual work and its literary genre.\footnote{\citealp[p.~249]{dilthey:hermeneutics}}
\end{quote}

As you have seen, hermeneutic circle is not limited to one work, the relation between the whole work and its parts, but rather it is the relation between the work and its environment. To understand one's work is even more difficult in this case, because we have to understand, not just the work itself, but the place where the work is situated, and perhaps the whole life of the author as well. Echoing the discussion above, Dilthey admits that ``understanding always remains partial and can never be completed.''\footnote{\citealp[p.~249]{dilthey:hermeneutics}} 

How could we understand anything when we cannot fully do it anyway? This is a critical turn, as we shall see in subsequent thinkers. Understanding is not mainly about text anymore, it is about life. A key thinker who developed this idea further, the most difficult and profound one, is Martin Heidegger (1889--1976).

By applying phenomenology to hermeneutics, Heidegger sees understanding as ``a mode or constituent element of being-in-the-world.''\footnote{\citealp[p.~131]{palmer:hermeneutics}} Understanding is not something one can possess, as Richard Palmer explains:

\begin{quote}
For Heidegger, understanding is the power to grasp one's own possibilities for being, within the context of the life world in which one exists. It is not a special capacity or gift for feeling into the situation of another person, nor is it the power to grasp the meaning of some `expression of life' on a deeper level.\footnote{\citealp[p.~131]{palmer:hermeneutics}}
\end{quote}

That is to say, the center of understanding is not other's mind or something significant in the world, but rather \emph{being}\footnote{In Heidegger's terms, it is \emph{Dasein}, `there-being' or `being-there.'} itself. This religionish explanation ends up in philosophy of existentialism. The task of hermeneutics is to interpret \emph{Dasein} to itself.\footnote{\citealp[p.~55]{schmidt:hermeneutics}} Heidegger's idea is radical to hermeneutics because it is no longer theory of interpretation, but rather the way of understanding oneself, as he puts it thus:

\begin{quote}
In hermeneutics what is developed for Dasein is a possibility of its becoming and being for itself in the manner of an \emph{understanding} of itself.\footnote{\citealp[p.~11]{heidegger:ontology}}
\end{quote}

To put it simpler, when we read a text, our interpretation of it is related directly to our being. It does not matter much what the text says, for we cannot be absolutely sure. What counts most is why we should understand in a certain way. Again, this idea echoes the shift of focus from the author or the text itself to the reader mentioned above. That is to say, when certain understanding occurs, it must affect the life of the reader somehow. And it is never finished, because life changes, so does significance from the reading.

Here comes hermeneutic circle seen by Heidegger. For him, ``[i]nterpretation is never a presuppositionless grasping of something previously given.''\footnote{\citealp[p.~146]{heidegger:being}} That means we inevitably have something in mind before we start the part-whole interactive circle. A kind of hermeneutic kick-start, so to speak. This echoes what I call existing knowledge structure previously. Heidegger calls this \emph{fore-structure} of understanding.\footnote{Precisely, it has three parts: \emph{fore-having}, \emph{foresight}, and \emph{fore-conception}. For explanation, see \emph{Being and Time}, \citealp[pp.~145--6]{heidegger:being}.} This pre-knowledge prevents us from seeing text as it is. What we see is ``nothing other than the self-evident, undiscussed \emph{prejudice} of the interpreter.''\footnote{\citealp[p.~146]{heidegger:being}, emphasis added} The notion of prejudice is important to the later development done by Hans-Georg Gadamer (1900--2002), one of his student.

Following Heidegger, Gadamer sees understanding as ``the original form of the realization of \emph{Dasein}.''\footnote{\citealp[p.~250]{gadamer:truth}} Understanding text thus becomes understanding life itself, as Gadamer puts it:

\begin{quote}
[F]or understanding the expressions of life or of texts---it still remains true that all such understanding is ultimately self-understanding.\footnote{\citealp[p.~251]{gadamer:truth}}
\end{quote}

Before we can understand something, we must be equipped with prerequisite knowledge with some values laden. From such prejudices, all understanding begins.\footnote{What Heidegger calls fore-structure of understanding is called `prejudice' by Gadamer. In German, the term neutrally means \emph{prejudgement} without negative connotation.} There is no way for all of us to escape this condition. To overcome hermeneutic circle is not to escape it, for it cannot be done, but rather to find ``the legitimacy of prejudices.''\footnote{\citealp[p.~278]{gadamer:truth}} One answer of this is the legitimization from authority and tradition. This leads us to the notion of \emph{horizon}.

In Gadamer's view, prejudices constitute ``the horizon of a particular present, for they represent that beyond which it is impossible to see.''\footnote{\citealp[p.~305]{gadamer:truth}} Let me explain in this way. We know we have prejudices, so does authority of the tradition. When a text is read, we apply our prejudices to gain some bounded understanding, like at the point we stand we cannot see beyond the skyline. We also know that the tradition also has its prejudices, hence its horizon. To gain better understanding, we have to enter the horizon of the tradition, maybe by suspending some of our prejudices. Therefore understanding is the fusion of different horizons.\footnote{\citealp[p.~305]{gadamer:truth}} As Gadamer puts it, ``[t]he understanding of something written is not a repetition of something past but the sharing of a present meaning.''\footnote{\citealp[p.~394]{gadamer:truth}}

Some questions come to my mind, can we really know the horizon of the past? Because it is we who read all the things from the text. Isn't other's horizon constructed from our very horizon? This can render that fusing the horizons is just an illusion. Or maybe it is like a reconciliation between other voices and our own. What if we cannot really believe what the tradition says because it is downright false according to our current knowledge? This suggests that we always read texts selectively, and the criteria of selection are preconditioned in our mind.

Before we finish this chapter, let us wrap up the answer of ``How is text understood?'' Seeing the issue from various perspectives, we can tell that the answer is not simple, like you know words' meaning, then you combine them into sentences' meaning, paragraphs' meaning, and so on, and you get the text's meaning at the end. There are many things to be taken into consideration. The main point is that the process of reading is not passive. It is in essence the interaction between subject, the one who read, and object, the text.

When the original context of the text is difficult to be reconstructed, i.e.\ we cannot know for sure what is important to the context\footnote{There is no necessity, no evidence, concerning what is important and what is unimportant. The judgment of importance is itself a guess. \citep[p.~77]{ricoeur:interpretation}}, it does not means we should follow our gut feeling and read texts in whatever way we want. This makes us lapse into relativism. And when anything goes, discussing about it turns useless. A healthy treatment of this is we have to assess the reading by \emph{logic of probability}\footnote{To show that an interpretation is more probable in the light of that we know is something other than showing that a conclusion is true. \citep[p.~78]{ricoeur:interpretation}} instead of logic of empirical verification (because this is out of reach). That is to say, from what we know we can tell that which interpretation is more probable than others. The method is argumentative discipline comparable to ``juridical procedures used in legal interpretation, a logic of uncertainty and of qualitative probability.''\footnote{\citealp[p.~78]{ricoeur:interpretation}}

At the heart of the problem, we can see that the relation between text and meaning is not one-to-one. The quality of understanding reproduced is a matter of degree of its likelihood. This is not the end of our story, but the starting point to what is called `discourse' as we will discuss later on.
