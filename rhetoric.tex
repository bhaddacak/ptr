\chapter{The rhetoric of the Buddha's omniscience}\label{chap:rhetoric}

Unlike the preceding chapters, this one will be less theoretical. It is an application. The idea presented here is crucial to our attitude towards P\=ali scriptures, so I put it as a fundamental prerequisite. Some Buddhists may feel uneasy with this, but we cannot simply evade the problem and take it for granted. We have to face it straightforwardly, rationally, critically, and honestly.

By `rhetoric' used here, I means simply ``an effective and persuasive use of language.'' The term has a close relation to `discourse' discussed earlier in Chapter \ref{chap:howunderman}. As far as verbal communication is concerned, they can mean roughly the same thing. A marked difference between the two is `rhetoric' puts more focus on form or style, whereas `discourse' concerns less on style but more on purpose.\footnote{A communication can be mere, or empty, rhetoric that is meaningless at large. It is just something sounds good but nonsensical. But when communication is a discourse, certain signification can always be read off.} However, I do not take the nuance seriously. Therefore, replacing `rhetoric' with `discourse' in the title makes no difference here. I stress on rhetoric because I want to underline effectiveness and persuasiveness of this discourse.

Why is the idea of omniscience important to the status of P\=ali canon? Comparing with other religions' canon, you can easily see that the sacred teaching has to come from an exceptional being. And omniscience, quality of being all-knowing, is indispensable to that being, otherwise the canon will be groundless. We will address the problem of omniscience of the Buddha first.

How do we know that the Buddha knows everything in present, past, and future? The bare fact is only Buddhist scriptures say that.\footnote{\pali{sama\d no gotamo sabba\~n\~n\=u sabbadass\=av\=i, aparisesa\d m \~n\=a\d nadassana\d m pa\d tij\=an\=ati} (M2\,185, MN\,71). \pali{At\=ita\d m \ldots An\=agata\d m \ldots Paccuppanna\d m sabba\d m j\=an\=at\=iti} (Psm\,1:120).} If there is such an extraordinary person who lived as long as 80 years old, there must be some historical record outside the religion. But as we know so far, such evidence is hopelessly scanty.

Some may say the canon itself can attest the Buddha's all-knowing. If we neglect direct statements that claim as such and assess truth-value of the content, we can roughly divide the result into three cases: (1) physical reality that can be verified scientifically, (2) unfalsifiable claims that cannot be verified, and (3) mental states that can be verified by everyone.

In the first case, most of physical reality described in the canon is outdated, inaccurate, if not downright false. For example, the structure of the universe depicted in the canon is a flat world-system having earth at the center with the sun and the moon revolving around.\footnote{Ati 8.81} This type of account is not many, comparing to the later cases.

In the second case, most assertions in the canon are unverifiable in nature. Many of them are stories of the past and intangible existence. Some are future events prophesied. The past events told in the P\=ali canon are rarely, if not never, confirmed by other contemporary sources, let alone events happened eons ago. Other spheres of existence outside human and animal realm are never proved positively by scientific means. And prophecies by their vague character cannot be proved wrong. There must be some hits in some way in some place and time. This part of the canon is contributive to most of the Buddhist worldview. It is the ideal tool for textual manipulation, thanks to unfalsifiability.

To me, the most interesting prophecy given by the Buddha, as told in the Vinaya, is the prediction that if women were allowed to go forth, the religious life (\pali{brahmacariya} = the religion at large) would last 500 years, otherwise 1000 years.\footnote{\pali{Sace, \=ananda, n\=alabhissa m\=atug\=amo tath\=agatappavedite dhammavinaye ag\=arasm\=a anag\=ariya\d m pabbajja\d m, cira\d t\d thitika\d m, \=ananda, brahmacariya\d m abhavissa, vassasahassa\d m saddhammo ti\d t\d theyya. Yato ca kho, \=ananda, m\=atug\=amo tath\=agatappavedite dhammavinaye ag\=arasm\=a anag\=ariya\d m pabbajito, na d\=ani, \=ananda, brahmacariya\d m cira\d t\d thitika\d m bhavissati. Pa\~nceva d\=ani, \=ananda, vassasat\=ani saddhammo \d thassati.} (Cv\,10.403)} Considering this instance you face a dilemma: if you accept this as true, what you hold as true doctrine may be false; but if you reject this as false, you are questioning the Buddha's insight. How to deal with this problem? If you understand what I try to demonstrate, you will have no trouble with it. If you still feel uneasy, just read on.

The third case is psychological in nature. It is the most useful part of the doctrine as long as liberation is concerned. Buddhist psychological worldview is fittingly in line with modern psychotherapy. Can this really prove that the Buddha is all-knowing? I think it is not so. We accept the Buddha's insight on this issue because modern psychology has been proved as such, not the other way around.\footnote{If there any conflict between certain tenets in Buddhism and modern well-established knowledge. Buddhism tends to give way to that scientific knowledge. Dalai Lama also agrees with this: ``if scientific analysis were conclusively to demonstrate certain claims in Buddhism to be false, then we must accept the findings of science and abandon those claims.'' \citep[pp.~2--3]{dalai:universe}}

Let me conclude what we have seen so far. From textual evidence, there is no convincing reason to believe that the Buddha knows better than modern scientists about physical worlds. And it is undeniable that he knows how to cope with our predicaments in life effectively. That is enough to make him the Buddha. Other things surrounding this crux, which are mostly unverifiable, can be seen as stratagem.

If you agree with my conclusion, you can leave this chapter at this point happily. If you need more treatment, read on.

To tackle the problem of omniscience more carefully, let me start with this question: ``How can anyone know things beyond his or her senses?'' We all know that we use our senses to gain information of the world, and we use our mind to process that information to gain knowledge. By and large, we know the world by the interplay between perceiving and reasoning. We can also know beyond our direct senses by enhancing instruments like a telescope, and by accepting knowledge of others like we do in learning.

Can knowledge be put directly to our mind like in movie the Matrix? This question can be applied to religion-based sources of knowledge, such as revelation and intuition. To understand this, we have to accept that knowledge is a kind of information, meaningful information. And information is physical entity that has to be located somewhere, despite its intangibleness. If no one tells you that knowledge, or there is no the Matrix-like downloading method available yet, how do you get that?

There are only two options: (1) you know by your own capability of mind (intuition case), and (2) you know by superhuman help (revelation case). I rule out the second option because it is not in the Buddhist doctrine. Then how do you know something completely outside your mind? For example, how do you know events of distant past? Let us think like a child. There is a kind of cosmic Wikipedia holding all information in the universe, if any. To access to that database, you may have to develop you mental ability to connected to that network. This might be the picture most Buddhists think how the Buddha knows things.

Think it seriously, we accept that cosmic information storage is possible, in principle at least, because past events might be recorded by the light bouncing off the earth at that time. However, making our mind to access that information is unthinkable scientifically. Even our best instruments cannot do that. How is the mind capable of that? How does the mind go beyond our bodily boundary? Most Buddhists may think it can be once your mind is fully developed. Where is the evidence of that? I have to say there is none, except in hope and imagination. You cannot prove scientific hypotheses by mere anecdotes told in texts or hearsay.\footnote{Many Buddhists still cannot differentiate fable or allegory from historical account. Not because people lack intelligence, but rather the religion's maintainers have to do it that way. Even nowadays many preachers tell stories from the scriptures as if they are historical fact. In my childhood, I was often told that in the past animals could talk like in the J\=ataka, and I really believed that for some time, poor me!}

How does our mind know directly what happen on the other side of the world? This is the case of present knowledge beyond senses. One possible answer is you can see with your mind-eye. The mind-eye is no boundary. If it is developed properly, it can go anywhere faster than light. I think many of Buddhists still believe like this. In parapsychology, this kind of ESP (Extra-Sensory Perception) has never been proved positively, rigorously, and unquestionably in scientific manner. You might have heard stories here and there. But they are not testimony. If the Buddha really knew what happened in China, Europe, or the Middle East at that time, maybe you can add other planets in the list, we might have many more spectacular stories in the canon.\footnote{Buddhists might protest that the Buddha does not talk things useless and irrelevant to suffering. But there are many of them in the canon as I know so far.}

Another case is the knowledge of future events. How does anyone know what does not yet happen? You may think in terms of calculation. Once you know a lot of factors contributing to the happening, you can predict the outcome with high accuracy.\footnote{Pierre Simon marquis de Laplace (1749--1827), a great mathematician, once thought in this way.} That is true in practical engineering world, but not in our quantum universe, because the basis at quantum level is uncertainty. Even a simple simulation like \emph{Game of Life}\footnote{For more information, search ``Conway's Game of Life.'' To dig deeper into the field, try `complex system' or `complexity.'} is unpredictable, let alone our real life system. If Buddhists take impermanence seriously, they will admit that an exact prediction of the future is impossible, even those done by the Buddha himself.

I hold that when the Buddha gave certain prediction, he just made a remark or an allusion. If future events can be predicted, the law of karma turns pointless, because you cannot exert your intention to do a new karma. Searching for liberation itself also turns unappealing, because you cannot really do anything if you are destined to be liberated in a specific time and place. The real wisdom of the Buddha is that, I assert, he knows how to deal with this chaotic world healthily.

Another ability that is often related to omniscience is telepathy, to know what others think. From textual evidence, the Buddha has this ability unquestionably. In fact, all of us have a simple version of this ability known as Theory of Mind mentioned earlier in Chapter \ref{chap:howeach}. Can anyone know exactly what you are thinking? If so, what does it look like? By our flickering and ever changing thought, in what manner others know our thought, I wander. Even I am sometimes not sure what kind of thought happens to me. How can others be sure about that? From our best evidence, telepathy does not pass any rigorous scientific test. On the other hand, stage mentalists can do the same feat convincingly.\footnote{Seeing some shows of Derren Brown on Youtube, you may get the idea.} That makes me uncertain we do really need that ability to know the others' thought.\footnote{In contemporary context, I have heard a lot about extraordinary persons who are said to have that ability. I do not rule out this possibility, because many weird things can happen in our quantum world. But rather I think it is unfalsifiable, because mundane tricks can also do the same job with high accuracy. So, unless we have better evidence, believing it as inauthentic is safe in our age of religion as commodity.}

As far as I elaborate the point, I want to show that, scientifically speaking, the claim of omniscience is untenable. However, the tenet has to be kept as such for reliability of the canon. If you accept and understand like this, you understand discourse in action, and realize that sometimes certain reality has to be postulated to preserve more important things. I do not say using rhetoric like this one is bad. We have to consider its intention. My only concern to raise this issue is every learner should be aware of it as such. That will prevent us from falling victim to unhealthy textual manipulation.

To conclude this chapter in a straight manner, if the Buddha really has unlimited knowledge, he would have never asked any question or made any mistake. It seems that the idea is an fictitious eulogizing of later development. The early text suggests that the Buddha himself made no such claim.\footnote{[N]either did the Buddha claim omniscience nor was omniscience claimed of the Buddha until the very latest stratum in the P\=ali Canon and that is even after most of the books of the Abhidhamma had been completed. \citep[p.~381]{jayatilleke:knowledge}} We can see psychological need for this, because religious adherents need an exceptionally powerful leader to rely on. And this can add credibility to the doctrine, the scripture, and the religion as a whole.

If you are still unclear to the issue and want to research further, reading Ven.\,An\=alayo's ``The Buddha and Omniscience''\footnote{\citealp{analayo:omniscience}} first is very helpful.
