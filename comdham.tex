\chapter{A commentary to Dhammapada}\label{chap:comdham}

After the readers went through Chapter \ref{chap:comja}, they will know here why I have chosen that story. There is a connection between Myna Bird story and this one. Like commentaries to J\=atak\=a, commentaries to Dhammapada have a story as their main part, more elaborate than those of the J\=atakas. I hope that after reading this example the readers will feel more confident to read this kind of text by their own.

\phantomsection
\addcontentsline{toc}{section}{Pre-reading introduction}
\section*{1.\ Pre-reading introduction}

\paragraph*{About the text} This commentary of Dhammapada is the ninth (of 12) story (\pali{vatthu}) of the ninth (of 26) group (\pali{vagga}). It contains one stanza of Dhammapada (Dhp\,9.125) and related short stories with a commentary part at the end. The verse mentioned here is the same as we have seen previously in J\=ataka (Ja\,5:94), but with different story line. The same verse is also found in other places, i.e.\ Snp\,667, S1\,22 (SN\,1), and S1\,190 (SN\,7). Of these, there are no elaborate story in their commentaries like this one. This tells us something: The verse is really old and it was held and transmitted by several branches of maintainers, and it has several instances of commentary.

\paragraph*{About the author} Like the commentaries to J\=ataka, the author of this, held by the tradition, is also Buddhaghosa, but some modern scholars question this assumption (see Chapter \ref{chap:comja} on this account).

\paragraph*{About the audience} Although the form of J\=ataka's and Dhammapada's commentaries look alike, stories told in both groups are quite different. In Dhammapada's commentaries like this one, stories look more historical, even though multiple past lives are sometimes brought into account. Stories are more complex by their structure and some of them are very long, suitable for lay adults or older children than young ones. Like stories in J\=ataka, monks often use Dhammapada's stories in preaching.  Even though many stories sound like history, they do not target historians, and they should not be referred as historical facts. Still, many monks often buttress the genuineness of the stories.
 
\paragraph*{About time and place} The exact time and place are unknown. If the great Buddhaghosa counts, the time and place can be around that. As we shall see when reading the text, the story here looks like a development of that we read in Chapter \ref{chap:comja}. So, it is probable that commentaries on Dhammapada made use of J\=ataka materials\footnote{``about 60 stories common to both'' (\citealp[p.~133]{hinuber:literature})}, so they were possibly written later. The collection of stories may have been accumulated over time and have several stages of development.\footnote{Oskar von Hin\"uber wrote, ``[A]n old independent Dhp-commentarial tradition has been modernized under the influence of the Ja-commentary, when the `new' Dhp-a was created. The old Dhp-commentary contained most probably only short Vatthus \ldots'' (\citealp[p.~134]{hinuber:literature}).}

\paragraph*{About motives} To make a collection of stories suitable for preaching can be a viable motive. Even though many stories can be found in the canon, but they are documentary-like and not dramatic enough to interest ordinary listeners. Stories are an excellent tool for incorporating certain message. So, persuasion, or even proselytization, can also be seen as a motive. 

\paragraph*{About text function} For it is dubious to be a historical account, informative function should be ruled out. A visible function is expressive, comparable to that of a historical fiction. Operative function is also woven seamlessly inside.

\phantomsection
\addcontentsline{toc}{section}{Reading with a draft translation}
\section*{2.\ Reading with a draft translation}

\begin{center}
\textbf{9. \pali{Kokasunakhaluddakavatthu}}\par
$\triangleright$ The story of\\Koka the hunter and his dogs
\end{center}

\setcounter{sennum}{0}
\pali{\fbox{\stepcounter{sennum}\arabic{sennum}} Yo appadu\d t\d thass\=a'ti ima\d m dhammadesana\d m satth\=a jetavane viharanto koka\d m n\=ama sunakhaluddaka\d m \=arabbha kathesi.}

\addtocounter{sennum}{-1}
$\triangleright$ \fbox{\stepcounter{sennum}\arabic{sennum}} Having mentioned a hunter (and his dogs) named Koka, the Buddha, living in Jetavana, delivered this teaching thus, ``\pali{Yo appadu\d t\d thassa} [and so on].'' \\

\pali{\fbox{\stepcounter{sennum}\arabic{sennum}} So kira ekadivasa\d m pubba\d nhasamaye dhanu\d m \=ad\=aya sunakhaparivuto ara\~n\~na\d m gacchanto antar\=amagge eka\d m pi\d n\d d\=aya pavisanta\d m bhikkhu\d m disv\=a kujjhitv\=a `k\=a\d laka\d n\d ni me di\d t\d tho, ajja ki\~nci na labhiss\=am\=i'ti cintetv\=a pakk\=ami. \fbox{\stepcounter{sennum}\arabic{sennum}} Theropi g\=ame pi\d n\d d\=aya caritv\=a katabhattakicco puna vih\=ara\d m p\=ay\=asi. \fbox{\stepcounter{sennum}\arabic{sennum}} Itaropi ara\~n\~ne vicaritv\=a ki\~nci alabhitv\=a pacc\=agacchanto puna thera\d m disv\=a `ajj\=aha\d m ima\d m k\=a\d laka\d n\d ni\d m disv\=a ara\~n\~na\d m gato ki\~nci na labhi\d m, id\=ani me punapi abhimukho j\=ato, sunakhehi na\d m kh\=ad\=apess\=am\=i'ti sa\~n\~na\d m datv\=a sunakhe vissajjesi. \fbox{\stepcounter{sennum}\arabic{sennum}} Theropi `m\=a eva\d m kari up\=asak\=a'ti y\=aci. \fbox{\stepcounter{sennum}\arabic{sennum}} So `ajj\=aha\d m tava sammukh\=ibh\=utatt\=a ki\~nci n\=alattha\d m, punapi me sammukh\=ibh\=avam\=agatosi, kh\=ad\=apess\=ameva tan'ti vatv\=a sunakhe uyyojesi. \fbox{\stepcounter{sennum}\arabic{sennum}} Thero vegena eka\d m rukkha\d m abhiruhitv\=a purisappam\=a\d ne \d th\=ane nis\=idi. \fbox{\stepcounter{sennum}\arabic{sennum}} Sunakh\=a rukkha\d m pariv\=aresu\d m. Luddako gantv\=a `rukkha\d m abhiruhatopi te mokkho natth\=i'ti ta\d m saratu\d n\d dena p\=adatale vijjhi. \fbox{\stepcounter{sennum}\arabic{sennum}} Thero `m\=a eva\d m karoh\=i'ti ta\d m y\=aciyeva. \fbox{\stepcounter{sennum}\arabic{sennum}} Itaro tassa y\=acana\d m an\=adiyitv\=a punappuna\d m vijjhiyeva. \fbox{\stepcounter{sennum}\arabic{sennum}} Thero ekasmi\d m p\=adatale vijjhiyam\=ane ta\d m ukkhipitv\=a dutiya\d m p\=ada\d m olambitv\=a tasmi\d m vijjhiyam\=ane tampi ukkhip\-ati, evamassa so y\=acana\d m an\=adiyitv\=ava dvepi p\=adatal\=ani vijjhiyeva. \fbox{\stepcounter{sennum}\arabic{sennum}} Therassa sar\=ira\d m uk\=ahi \=aditta\d m viya ahosi. \fbox{\stepcounter{sennum}\arabic{sennum}} So vedan\=anuvattiko hutv\=a sati\d m paccupa\d t\d th\=apetu\d m n\=asakkhi, p\=arutac\=ivara\d m bhassantampi na sallakkhesi. \fbox{\stepcounter{sennum}\arabic{sennum}} Ta\d m patam\=ana\d m koka\d m s\=isato pa\d t\d th\=aya parikkhipantameva pati. \fbox{\stepcounter{sennum}\arabic{sennum}} Sunakh\=a `thero patito'ti sa\~n\~n\=aya c\=ivarantara\d m pavisitv\=a attano s\=amika\d m lu\~njitv\=a kh\=adant\=a a\d t\d thimatt\=avasesa\d m kari\d msu. \fbox{\stepcounter{sennum}\arabic{sennum}} Sunakh\=a c\=ivarantarato nikkhamitv\=a bahi a\d t\d tha\d msu. \fbox{\stepcounter{sennum}\arabic{sennum}} Atha nesa\d m thero eka\d m sukkhad\-a\d n\d daka\d m bha\~njitv\=a khipi. \fbox{\stepcounter{sennum}\arabic{sennum}} Sunakh\=a thera\d m disv\=a `s\=amikova amhehi kh\=adito'ti \~natv\=a ara\~n\~na\d m pavisi\d msu.}

\addtocounter{sennum}{-17}
$\triangleright$ \fbox{\stepcounter{sennum}\arabic{sennum}} It is said thus one day in the morning that [hunter], having taken a bow, surrounded by dogs, going to the forest, on the way having seen a monk going for alms, having been angry, thinking thus ``An unfortunate one was seen by me. Today [I] will get nothing,'' [then] went away. \fbox{\stepcounter{sennum}\arabic{sennum}} The elderly monk, having walked for alms in the village, [having been] the one who finished the eating, [then] went [back] to the dwelling again. \fbox{\stepcounter{sennum}\arabic{sennum}} Next, having wandered in the forest, having got nothing, [while] coming back, having seen the elderly monk again, [the hunter thought] thus ``Today having seen this unlucky man, having gone to the forest, I got nothing. This time meeting face to face [with him] happened to me even again. [I] will make dogs bite him.'' Having given a gesture, [the hunter] sent off the dogs. \fbox{\stepcounter{sennum}\arabic{sennum}} The elderly monk begged [him] thus ``Don't do that, mister.'' \fbox{\stepcounter{sennum}\arabic{sennum}} Having said thus ``Today, because of the face-to-face meeting with you, I got nothing. The face-to-face meeting came to me even again. [I] will make [the dogs] bite you.'', [the hunter] sent off the dogs. \fbox{\stepcounter{sennum}\arabic{sennum}} The monk, having climbed up a tree quickly, sat on a place at a man's height. \fbox{\stepcounter{sennum}\arabic{sennum}} The dogs [then] enclosed the tree. The hunter, having said thus ``Even climbing onto the tree, there is no salvation for you.'', [then] stabbed him with an arrow in the foot. \fbox{\stepcounter{sennum}\arabic{sennum}} The monk only begged him thus ``Don't do that.'' \fbox{\stepcounter{sennum}\arabic{sennum}} Then, not having taken his request, [the hunter] only stabbed again and again. \fbox{\stepcounter{sennum}\arabic{sennum}} The monk, being stabbed in one foot, having raised that [foot up], [then] having lowered down the second foot, being stabbed in that [second foot], raised up also that [foot]. In this way, that [hunter], not having taken [the monk's] request, stabbed [his] both feet. \fbox{\stepcounter{sennum}\arabic{sennum}} The monk's body became like being burned with torches. \fbox{\stepcounter{sennum}\arabic{sennum}} That [monk], having been one who experieces pain, was unable to keep his mindful state. [He] was not aware of the falling robe put on. \fbox{\stepcounter{sennum}\arabic{sennum}} That [robe], [while] falling down, fell onto Koka from the head to the whole body. \fbox{\stepcounter{sennum}\arabic{sennum}} The dogs, by thinking that ``The monk has fallen,'' having went to the center of the robe, having pulled up the owner [of the robe], biting, made only [his] bone remain. \fbox{\stepcounter{sennum}\arabic{sennum}} The dogs, having left the robe, stood outside. \fbox{\stepcounter{sennum}\arabic{sennum}} Then the monk, having broken a branch [to make a stick for driving the dogs away], threw it at them. \fbox{\stepcounter{sennum}\arabic{sennum}} The dogs, having seen the monk, [thinking thus] ``[Our] master was indeed bitten by us,'' having known [that], went away to the forest.\\

\pali{\fbox{\stepcounter{sennum}\arabic{sennum}} Thero kukkucca\d m upp\=adesi `mama c\=ivarantara\d m pavisitv\=a esa na\d t\d tho, aroga\d m nu kho me s\=ilan'ti. \fbox{\stepcounter{sennum}\arabic{sennum}} So rukkh\=a otaritv\=a satthu santika\d m gantv\=a \=adito pa\d t\d th\=aya sabba\d m ta\d m pavatti\d m \=arocetv\=a -- `bhante, mama c\=ivara\d m niss\=aya so up\=asako na\d t\d tho, kacci me aroga\d m s\=ila\d m, atthi me sama\d nabh\=avo'ti pucchi. \fbox{\stepcounter{sennum}\arabic{sennum}} Satth\=a tassa vacana\d m sutv\=a `bhikkhu aroga\d m te s\=ila\d m, atthi te sama\d nabh\=avo, so appadu\d t\d thassa padussitv\=a vin\=asa\d m patto, na kevala\~nca id\=aneva, at\=itepi appadu\d t\d th\=ana\d m padussitv\=a vin\=asa\d m pattoyev\=a'ti vatv\=a tamattha\d m pak\=asento at\=ita\d m \=ahari --}

\addtocounter{sennum}{-3}
$\triangleright$ \fbox{\stepcounter{sennum}\arabic{sennum}} The elderly monk raised a worry, [thinking] thus ``This [man], having enterred to my robe, has perished. Is my precept [still] unharmed?'' \fbox{\stepcounter{sennum}\arabic{sennum}} He, having climbed down the tree, having gone to the Buddha's place, having told the whole event from the beginning, [then] asked [the Buddha] thus, ``Sir, by a support from my robe, that man has perished. Is my precept [still] unharmed? Does my monkhood [still] exist?'' \fbox{\stepcounter{sennum}\arabic{sennum}} The Buddha, having listened to his words, said thus ``Monk, your precept is [still] unharmed. Your monkhood [still] exists. That [man], having done harm to one who does no harm [to others], fell into destruction. Not only in this time, but also in the past, did [he], having done harm to one who does no harm, fell into destruction.'' Illustrating that point, [the Buddha] brought a past [story]:\\

\pali{\fbox{\stepcounter{sennum}\arabic{sennum}} At\=ite kireko vejjo vejjakammatth\=aya g\=ama\d m vicaritv\=a ki\~nci kamma\d m alabhitv\=a ch\=atajjhatto nikkhamitv\=a g\=amadv\=are sambahule kum\=arake k\=i\d lante disv\=a `ime sappena \d da\d ms\=apetv\=a tikicchitv\=a \=ah\=ara\d m labhiss\=am\=i'ti ekasmi\d m rukkhabile s\=isa\d m niharitv\=a nipanna\d m sappa\d m dassetv\=a, `ambho, kum\=arak\=a eso s\=a\d likapotako, ga\d nhatha nan'ti \=aha. \fbox{\stepcounter{sennum}\arabic{sennum}} Atheko kum\=arako sappa\d m g\=iv\=aya\d m da\d lha\d m gahetv\=a n\=iharitv\=a tassa sappabh\=ava\d m \~natv\=a viravanto avid\=ure \d thitassa vejjassa matthake khipi. \fbox{\refstepcounter{sennum}\arabic{sennum}\label{sen:khandhatthika}} Sappo vejjassa khandha\d t\d thika\d m parikkhipitv\=a da\d lha\d m \d da\d msitv\=a tattheva j\=ivitakkhaya\d m p\=apesi, evamesa koko sunakhaluddako pubbepi appadu\d t\d thassa padussitv\=a vin\=asa\d m pattoyev\=a'ti.}

\addtocounter{sennum}{-3}
$\triangleright$ \fbox{\stepcounter{sennum}\arabic{sennum}} It is said thus in the past a physician, having wandered to a village for medical service, having got no work, being hungry, having left [the village], having seen several children playing at the village's gate, [thought] thus ``Having made [a child] bitten by this snake, having healed [the child], I will get [some] food.'' Having slept [and] stuck out the head off a tree hole, a snake was seen. [Then the physician] said thus ``Kids, that is a young myna bird. Catch it!'' \fbox{\stepcounter{sennum}\arabic{sennum}} Then one child, having seized the snake by the neck firmly, having taken [it] out, having known its snake-hood, shouting, [then] threw [it] upon the head of the physician standing nearby. \fbox{\stepcounter{sennum}\arabic{sennum}} The snake, having encircled the physician's neck, having bitten [him] strongly, made [the physician] die in that place. In this way, this Koka the hunter, in the past having done harm to one doing no harm, fell to destruction.\\

\pali{\fbox{\stepcounter{sennum}\arabic{sennum}} Satth\=a ima\d m at\=ita\d m \=aharitv\=a anusandhi\d m gha\d tetv\=a dhamma\d m desento ima\d m g\=atham\=aha --}

\addtocounter{sennum}{-1}
$\triangleright$ \fbox{\stepcounter{sennum}\arabic{sennum}} The Buddha, having brought this past [event], having related the conclusion, teaching the Dhamma, said this stanza:\\

\pali{125.\\
\fbox{\stepcounter{sennum}\arabic{sennum}} Yo appadu\d t\d thassa narassa dussati, suddhassa posassa ana\.nga\d nassa;\\
\fbox{\stepcounter{sennum}\arabic{sennum}} Tameva b\=ala\d m pacceti p\=apa\d m, sukhumo rajo pa\d tiv\=ata\d mva khitto'ti.}

\addtocounter{sennum}{-2}
$\triangleright$ \fbox{\stepcounter{sennum}\arabic{sennum}} Which [person] does harm to a person who does no harm, [who is] innocent and doing no wrong; \fbox{\stepcounter{sennum}\arabic{sennum}} the bad [result then] falls back to that fool, like fine dust being thrown against the wind.\\

\pali{\fbox{\stepcounter{sennum}\arabic{sennum}} Tattha appadu\d t\d thass\=a'ti attano v\=a sabbasatt\=ana\d m v\=a adu\d t\d th\-assa. \fbox{\stepcounter{sennum}\arabic{sennum}} Narass\=a'ti sattassa. \fbox{\stepcounter{sennum}\arabic{sennum}} Dussat\=i'ti aparajjhati. \fbox{\stepcounter{sennum}\arabic{sennum}} Suddhass\=a'ti nirapar\=adhasseva. \fbox{\stepcounter{sennum}\arabic{sennum}} Posass\=a'ti idampi aparen\=ak\=a\-rena satt\=adhivacanameva. \fbox{\stepcounter{sennum}\arabic{sennum}} Ana\.nga\d nass\=a'ti nikkilesassa. \fbox{\stepcounter{sennum}\arabic{sennum}} Paccet\=i'ti patieti. \fbox{\stepcounter{sennum}\arabic{sennum}} Pa\d tiv\=atan'ti yath\=a ekena purisena pa\d tiv\=ate \d thita\d m paharituk\=amat\=aya khitto sukhumo rajo'ti tameva purisa\d m pacceti, tasseva upari patati, evameva yo puggalo apadu\d t\d thassa purisassa p\=a\d nippahar\=ad\=ini dadanto padussati, tameva b\=ala\d m di\d t\d th\-eva dhamme, sampar\=aye v\=a niray\=ad\=isu vipaccam\=ana\d m ta\d m p\=apa\d m vip\=akadukkhavasena paccet\=i'ti attho.}

\addtocounter{sennum}{-8}
$\triangleright$ \fbox{\stepcounter{sennum}\arabic{sennum}} In that [stanza], \pali{appadu\d t\d thassa} means ``to the one who does not do harm to one's own self or to all beings.'' \fbox{\stepcounter{sennum}\arabic{sennum}} \pali{Narassa} means `being.' \fbox{\stepcounter{sennum}\arabic{sennum}} \pali{Dussati} means `to do some crime.' \fbox{\stepcounter{sennum}\arabic{sennum}} \pali{Suddhassa} means `innocent.' \fbox{\stepcounter{sennum}\arabic{sennum}} For \pali{posassa}, this is [also] a term for `being' by another condition. \fbox{\stepcounter{sennum}\arabic{sennum}} \pali{Ana\.nga\d nassa} means `without defilement [of mind].' \fbox{\stepcounter{sennum}\arabic{sennum}} \pali{Pacceti} means `to go back to' [\pali{pa\d ti + eti}]. \fbox{\stepcounter{sennum}\arabic{sennum}} For the meaning of \pali{pa\d tiv\=ata\d m}, [etc.,] just like fine dust thrown by a person to injure [another person] standing upwind, [it] goes back to that person, falling over him. In the same way, when giving a strike by hand, for example, to a person who does no harm, which person [in such a manner] does harm; that evil [action] goes back to that foolish [person] by the power of a painful result, ripening in this world or the next world such as in the hell.\\

\pali{\fbox{\stepcounter{sennum}\arabic{sennum}} Desan\=avas\=ane so bhikkhu arahatte pati\d t\d thahi, sampattaparis\=ayapi s\=atthik\=a dhammadesan\=a ahos\=i'ti.}

\addtocounter{sennum}{-1}
$\triangleright$ \fbox{\stepcounter{sennum}\arabic{sennum}} At the end of the teaching, that monk stood firmly in [= attained] the arhantship. The teaching was [also] beneficial to the assembly.\\

\pali{\fbox{\stepcounter{sennum}\arabic{sennum}} Kokasunakhaluddakavatthu navama\d m.}

\addtocounter{sennum}{-1}
$\triangleright$ \fbox{\stepcounter{sennum}\arabic{sennum}} The story of Koka the hunter and his dogs, the ninth, [was finished].\\

\phantomsection
\addcontentsline{toc}{section}{Conclusion and discussion}
\section*{3.\ Conclusion and discussion}

If the readers read this commentary successfully by their own effort, or with little help from my guidance, and they feel enjoyable and want to read more of them, my aim of writing this book is fulfilled. In the jungle of P\=ali texts, difficulties lie ahead to meet you all. Only when we read them with enjoyment and perseverance, the learning process can yield a good result. I insist that what have been learned so far, including from PNL, is enough to tackle any problem met along the way. If you cannot solve it in the moment, try reading commentary or subcommentary on the text, if any. If this does not help much, find some guideline from an existing translation. If nothing works, you can make your own judgement reasonably based on what you know. No one can say you are wrong because no one knows the right answer. That is the way all P\=ali learners go.

In the part of this content, I do not want to make a long discussion. I leave it to the learners to exercise their reasoning skill. There are some points I want to highlight though.

\begin{compactenum}[(1)]
\item The verse in this Dhammapada is word-for-word identical to that in the J\=ataka we have read earlier. This indicates that it came from one source.
\item The story of the hunter looks comical than real. And the way the monk approaches the Buddha is typical, conforming to a general template.
\item The story of the physician is similar to that J\=ataka, but slightly different in detail. This indicates that the outline of this story precedes the formation of the text. The story might be well-known and widespread among people, like Aesop's fables.
\item The commentary part of this Dhammapada and that J\=ataka are different. This means they were composed by different persons. And both of them understood some terms in different ways.
\item Discrepancy of the account on past lives shows that they are all made-up.
\end{compactenum}

