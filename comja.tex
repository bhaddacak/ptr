\chapter{A commentary to J\=ataka}\label{chap:comja}

In this example, we will read a commentary which has a parable as its main content. We will deal with a commentary to J\=ataka here. J\=ataka is the collection of stories of the Buddha in the past, often used in Dhamma talks. The most well-known and the longest one is the story of Vessantara, the last past life of the Buddha as a human being. We will not look into that though. I select one short commentary to study here, so that we can see its entire form. 

The main text of J\=ataka is in verses, and some of them are very good sayings suitable to repeat in preaching. So, it is sensible to add a story telling that why or when those verses were delivered. Then we have many stories related to verses in J\=ataka. There are 550 stories for about 2,500 verses.\footnote{\citealp[pp.~54--5]{hinuber:literature}. But 547 stories survive.}

\phantomsection
\addcontentsline{toc}{section}{Pre-reading introduction}
\section*{1.\ Pre-reading introduction}

\paragraph*{About the text} The story chosen here is called \pali{S\=a\d liyaj\=ataka} (the story of myna bird), No.\,367. It covers five stanzas (Ja\,5:90--94). We treat the text medium of this as written type, although some parts of it, the verses and the story sketch, might have been committed to memory long before the commentary.

\paragraph*{About the author} The tradition attributes this text to Buddhaghosa, the great commentator. Modern scholars tend to question this assumption. For example, K.\,R.\ Norman wrote, ``It is sometimes argued that the commentaries on the Dhammapada and the Jataka may have been written by another Buddhaghosa because their style is so different from his other commentaries.''\footnote{\citealp[p.~127]{norman:literature}}

\paragraph*{About the audience} Because the text is close to narrative literature, like folktales, its main target audience are probably lay devotees, particularly children. The text effectively helps strengthen the faith in the religion.
 
\paragraph*{About time and place} The time of this composition is unknown. If the great Buddhaghosa wrote it, it was around his time. But some materials in the J\=ataka commentaries might be older than that.\footnote{See \citealp[pp.~131--2]{hinuber:literature}.} The place of composing was possibly in Mah\=avih\=ara because the name is mentioned once in the introduction of the J\=ataka commentaries (\pali{ganth\=arambhakath\=a}).

\paragraph*{About motives} To teach morality based on Buddhist teachings can be a viable motive, than to reconstruct the biography of the Buddha in the past.

\paragraph*{About text function} The prosodic part has clear expressive function, whereas the story serves more in operative function.

\phantomsection
\addcontentsline{toc}{section}{Reading with a draft translation}
\section*{2.\ Reading with a draft translation}

\begin{center}
\textbf{[367] 7. \pali{S\=a\d liyaj\=atakava\d n\d nan\=a}}\par
$\triangleright$ A commentary to\\the story of Myna Bird
\end{center}

\setcounter{sennum}{0}
\pali{\fbox{\stepcounter{sennum}\arabic{sennum}} Yv\=aya\d m s\=a\d liyach\=apot\=i'ti ida\d m satth\=a ve\d luvane viharanto `\=avuso, devadatto t\=asak\=arakopi bhavitu\d m n\=asakkh\=i'ti vacana\d m \=arab\-bha kathesi. \fbox{\stepcounter{sennum}\arabic{sennum}} Tad\=a hi satth\=a `na, bhikkhave, id\=aneva, pubbepesa mama t\=asak\=arakopi bhavitu\d m n\=asakkh\=i'ti vatv\=a at\=ita\d m \=ahari.}

\addtocounter{sennum}{-2}
$\triangleright$ \fbox{\stepcounter{sennum}\arabic{sennum}} Mentioning [this] expression thus, ``Monks, Devadatta was unable even to be the one who makes [me] tremble,'' the Buddha, living in the Bamboo Grove, said thus, ``\pali{Yv\=aya\d m s\=a\d liyach\=apoti}'' [and so on]. \fbox{\stepcounter{sennum}\arabic{sennum}} In that time, the Buddha, having said thus, ``Not only in this time, monks, but also in the past, was this [Devadatta] unable even to be the threatening one to me,'' [then] brought [this] past [story].\\

\begin{longtable}[c]{|p{0.9\linewidth}|}
\hline
\hspace{5mm}\small Note that I translate \pali{t\=asak\=arakopi} (\pali{t\=asa + k\=araka + pi}) in two different ways. The term functions as a modifier of \pali{devadatto} in the first instance, but in the second it is treated as a noun. The ending \pali{pi} is just an emphatic particle, so I use `even' here. And \pali{pubbepesa} is \pali{pubbe + pi + esa}. This shows that sometimes \pali{esa} is used instead of \pali{eso}.\\
\hline
\end{longtable}

\pali{\fbox{\stepcounter{sennum}\arabic{sennum}} At\=ite b\=ar\=a\d nasiya\d m brahmadatte rajja\d m k\=arente bodhisatto g\=amake ku\d tumbikakule nibbattitv\=a taru\d nak\=ale pa\d msuk\=i\d lakehi d\=arak\-ehi saddhi\d m g\=amadv\=are nigrodharukkham\=ule k\=i\d lati. \fbox{\stepcounter{sennum}\arabic{sennum}} Tad\=a eko dubbalavejjo g\=ame ki\~nci alabhitv\=a nikkhamanto ta\d m \d th\=ana\d m patv\=a eka\d m sappa\d m vi\d tapabbhantarena s\=isa\d m n\=iharitv\=a nidd\=ayanta\d m dis\-v\=a `may\=a g\=ame ki\~nci na laddha\d m, ime d\=arake va\~ncetv\=a sappena \d da\d ms\=apetv\=a tikicchitv\=a ki\~ncideva ga\d nhiss\=am\=i'ti cintetv\=a bodhisatta\d m \=aha `sace s\=a\d liyach\=apa\d m passeyy\=asi, ga\d nheyy\=as\=i'ti. \fbox{\stepcounter{sennum}\arabic{sennum}} `\=Ama, ga\d nheyyan'ti. \fbox{\refstepcounter{sennum}\arabic{sennum}\label{sen:passeso}} `Passeso vi\d tapabbhantare sayito'ti. \fbox{\stepcounter{sennum}\arabic{sennum}} So tassa sappabh\=ava\d m aj\=ananto rukkha\d m \=aruyha ta\d m g\=iv\=aya\d m gahetv\=a `sap\-po'ti \~natv\=a nivattitu\d m adento suggahita\d m gahetv\=a vegena khipi. \fbox{\refstepcounter{sennum}\arabic{sennum}\label{sen:karakara}} So gantv\=a vejjassa g\=iv\=aya\d m patito g\=iva\d m palive\d thetv\=a `kara kar\=a'ti \d da\d msitv\=a tattheva na\d m p\=atetv\=a pal\=ayi.}

\addtocounter{sennum}{-6}
$\triangleright$ \fbox{\stepcounter{sennum}\arabic{sennum}} In the past, when Brahmadatta was made king in B\=ar\=a\d nas\=i, the Bodhisatta, having been born in a small village in a householder's family, at the time he was young, he plays with [other] children who plays dirt under a banyan tree near the village's gate. \fbox{\refstepcounter{sennum}\arabic{sennum}\label{sen:dubbalavejjo}} At that time, a disabled physician, not having any [client] in the village, [then] leaving, having reached at that place, having seen a snake sleeping inside a fork of a tree [and] stretching [its] head out, having thought thus ``Nothing was obtained by me in the village; having deceived these children [by] making them bitten by the snake [then] healing [them], I will get something,'' said to the Boddhisatta thus ``If [you] see a young myna bird, will [you] catch it?'' \fbox{\stepcounter{sennum}\arabic{sennum}} ``Yes, [I] will catch [it].'' \fbox{\stepcounter{sennum}\arabic{sennum}} ``Look!, that [is it] sleeping in the fork.'' \fbox{\stepcounter{sennum}\arabic{sennum}} He [the Boddhisatta], not having known the existence of that snake, having climbed up the tree, having seized it by the neck, having known that ``[It is] a snake,'' making [it] unable to recoil, having grasped [it] firmly, [then] threw away [the snake] quickly. \fbox{\stepcounter{sennum}\arabic{sennum}} That [snake], having gone, falling onto the physician's neck, having wrapped [his] neck, having bitten [him] thus ``kara kara,'' having made that [physician] fall down, [then] crawled away.\\

\begin{longtable}[c]{|p{0.9\linewidth}|}
\hline
\hspace{5mm}\small The king of B\=ar\=a\d nas\=i in the past was always named Brahmadatta. The point is trivial to the story, so do not take this seriously.\\
\hspace{5mm}\dag\ \small A question may arise that why a king and his capital is mentioned at all. As a good narrative skill, doing so makes the story sound real because of familiarity and intimacy. One rule of good novel writing is ``Make it tangible and specific, not general.'' This rule works like a spell. If we think further, it can be an attempt to normalize the idea that absolute monarchy is the only political system. If the authors did not intend this, they were instead a very product of this mindset.\\
\hspace{5mm}\small Left untranslated, \pali{bodhisatta} means one who will become the Buddha in the future.\\
\hspace{5mm}\small Even though the story is in the past, we often see present tense in use, like \pali{k\=i\d lati} here. So, I keep my translation agreeable to P\=ali tenses.\\
\hspace{5mm}\small A long sentence in \fbox{\ref{sen:dubbalavejjo}} is a good instance to learn. We can only find this kind of complex sentence in narrative accounts, not in normal commentaries. That is why stories are a very good resource for studying the language.\footnote{In official P\=ali curriculum, Thai monks and novices learn first to translate stories, and it is quite enjoyable (only if you master the grammar to some degree, but unfortunately teaching P\=ali grammar in Thai system is very boring and needlessly difficult).} I keep its structure by translating it in one stretch.\\
\hspace{5mm}\small The subject of the sentence is \pali{dubbalavejjo} and the main verb is \pali{\=aha} (past tense). A point worth studying carefully is how present participle (e.g.\ \pali{nikkhamanto}) works together with absolutive \pali{tv\=a} verbs (e.g.\ \pali{patv\=a}). These two kinds of verb denote different aspects. That is to say, present participle shows an ongoing action relative to a completed action expressed by \pali{tv\=a} verbs. Moreover, using \pali{tv\=a} verbs can show a chain of successive events. If some new learners are still confused by this brief explanation, a meticulous review of P\=ali verb system is urgently needed.\\
\hspace{5mm}\small In \fbox{\ref{sen:passeso}}, \pali{passeso} is \pali{passa} (imp.) and \pali{eso} (this/that). A more precise translation can be thus, if we take past participle seriously, ``Look!, that [is it]. [It] slept in the fork.''\\
\hspace{5mm}\small In \fbox{\ref{sen:karakara}}, `\pali{kara kara}' is onomatopoeia. It is the sound when the snake is biting. I have no idea what it is like, and I even do not know a snake makes sound when it bites. However, that is not the point. By this use, it adds comical picture to the story, hence making it spectacular when being told. Most children love sound effects.\\
\hline
\end{longtable}

\pali{\fbox{\stepcounter{sennum}\arabic{sennum}} Manuss\=a pariv\=arayi\d msu. Mah\=asatto sampattaparis\=aya dhamma\d m desento im\=a g\=ath\=a abh\=asi --}

\addtocounter{sennum}{-1}
$\triangleright$ \fbox{\stepcounter{sennum}\arabic{sennum}} People [then] surrounded [the Buddha]. The Great Being, preaching the Dhamma in the assembly, said these verses:\\

\pali{ 90.\\
\fbox{\stepcounter{sennum}\arabic{sennum}} Yv\=aya\d m s\=a\d liyach\=apo'ti, ka\d nhasappa\d m ag\=ahayi;\\
\fbox{\stepcounter{sennum}\arabic{sennum}} Tena sappenaya\d m [so] da\d t\d tho, hato p\=ap\=anus\=asako.}

\addtocounter{sennum}{-2}
$\triangleright$ \fbox{\stepcounter{sennum}\arabic{sennum}} Which person made [me] grasp a cobra [by deceiving that] ``This [is] a young myna bird.'' \fbox{\stepcounter{sennum}\arabic{sennum}} [That] this [person] was bitten by that snake. Died the evil adviser.\\

\begin{longtable}[c]{|p{0.9\linewidth}|}
\hline
\hspace{5mm}\small We can see \pali{ya-ta} pattern in use here. It is \pali{yo} in \pali{yv\=aya\d m} (\pali{yo + aya\d m}). And I add \pali{so} myself (see \fbox{\ref{sen:soayadm}} below). The idiom of \pali{so aya\d m} (that this) sounds odd to us, but this use is common. The two words have different function. \pali{So} (that) correlates to \pali{yo} (which), whereas \pali{aya\d m} points to this person, not other ones else.\\
\hline
\end{longtable}

\pali{ 91.\\
\fbox{\stepcounter{sennum}\arabic{sennum}} Ahant\=aramahant\=ara\d m, yo naro hantumicchati;\\
\fbox{\stepcounter{sennum}\arabic{sennum}} Eva\d m so nihato seti, yath\=aya\d m puriso hato.}

\addtocounter{sennum}{-2}
$\triangleright$ \fbox{\stepcounter{sennum}\arabic{sennum}} Which person wants to kill one who does no beat and does not kill. \fbox{\stepcounter{sennum}\arabic{sennum}} That person, having been destroyed, lies down as such, just like this man who was killed.\\

\newpage
\begin{longtable}[c]{|p{0.9\linewidth}|}
\hline
\hspace{5mm}\small A difficult compound here is \pali{ahant\=aramahant\=ara\d m} (\pali{ahant\=ara\d m + ahant\=ara\d m}). The dictionary form of the term is \pali{ahantu} or \pali{ahantar} (one who does not kill/beat), and \pali{ahant\=ara\d m} is its accusative case. The compound looks like a repetition, which makes no sense here. The commentary suggests that the second means `not killing' (see \fbox{\ref{sen:amaarenta}}).\\
\hline
\end{longtable}

\pali{ 92.\\
\fbox{\stepcounter{sennum}\arabic{sennum}} Ahant\=aramagh\=atenta\d m, yo naro hantumicchati;\\
\fbox{\stepcounter{sennum}\arabic{sennum}} Eva\d m so nihato seti, yath\=aya\d m puriso hato.}

\addtocounter{sennum}{-2}
$\triangleright$ \fbox{\stepcounter{sennum}\arabic{sennum}} Which person wants to kill one who does not kill and does not make others kill. \fbox{\stepcounter{sennum}\arabic{sennum}} That person, having been destroyed, lies down as such, just like this man who was killed.\\

\begin{longtable}[c]{|p{0.9\linewidth}|}
\hline
\hspace{5mm}\small The intention to kill expressed in this verse is at odds with the story. The physician did not mean to kill the child. So, the story does not fit well to this verse.\\
\hline
\end{longtable}

\pali{ 93.\\
\fbox{\stepcounter{sennum}\arabic{sennum}} Yath\=a pa\d msumu\d t\d thi\d m puriso, pa\d tiv\=ata\d m pa\d tikkhipe;\\
\fbox{\stepcounter{sennum}\arabic{sennum}} Tameva so rajo hanti, tath\=aya\d m puriso hato.}

\addtocounter{sennum}{-2}
$\triangleright$ \fbox{\stepcounter{sennum}\arabic{sennum}} Like a person throws away a fistful of dust against the wind, \fbox{\stepcounter{sennum}\arabic{sennum}} [then] that dust injures him [in return], so did this man who was killed.\\

\begin{longtable}[c]{|p{0.9\linewidth}|}
\hline
\hspace{5mm}\small Even though \pali{pa\d tikkhipe(yya)} is in optative form, we see it as a simple present verb. If we take optative meaning seriously, this sentence can be a speculation. By this instance, we can see how meter restriction in prosody ruins the clarity of the message. It may look trivial in this stanza, but there is a good chance we will encounter a more obscure or archaic case. So, keep in mind that it is not always grammatically correct in verses. We have to work around sometimes to get things clearer.\\
\hline
\end{longtable}

\pali{ 94.\\
\fbox{\stepcounter{sennum}\arabic{sennum}} Yo appadu\d t\d thassa narassa dussati, suddhassa posassa ana\.nga\d nassa;\\
\fbox{\stepcounter{sennum}\arabic{sennum}} Tameva b\=ala\d m pacceti p\=apa\d m, sukhumo rajo pa\d tiv\=ata\d mva khitto'ti.}

\addtocounter{sennum}{-2}
$\triangleright$ \fbox{\stepcounter{sennum}\arabic{sennum}} Which [person] does harm to a person who does no harm, [who is] innocent and doing no wrong; \fbox{\stepcounter{sennum}\arabic{sennum}} the bad [result then] falls back to that fool, like fine dust being thrown against the wind.\\

\begin{longtable}[c]{|p{0.9\linewidth}|}
\hline
\hspace{5mm}\small Genitive case is used here in accusative meaning. And I leave \pali{possasa} untranslated because it is just a synonym of \pali{narassa}. This verse is the highlight and summary. We will meet this again in Chapter \ref{chap:comdham}, so keep this in mind.\\
\hline
\end{longtable}

\pali{\fbox{\stepcounter{sennum}\arabic{sennum}} Tattha yv\=ayan'ti yo aya\d m, ayameva v\=a p\=a\d tho. \fbox{\refstepcounter{sennum}\arabic{sennum}\label{sen:soayadm}} Sappenayan'ti so aya\d m tena sappena da\d t\d tho. \fbox{\stepcounter{sennum}\arabic{sennum}} P\=ap\=anus\=asako'ti p\=apaka\d m anus\=asako.}

\addtocounter{sennum}{-3}
$\triangleright$ \fbox{\stepcounter{sennum}\arabic{sennum}} In those [verses], \pali{yv\=aya\d m} is \pali{yo aya\d m}. This [form also appears in] the text [= the canon]. \fbox{\stepcounter{sennum}\arabic{sennum}} \pali{Sappenaya\d m} means ``that this [person] bitten by that snake.'' \fbox{\stepcounter{sennum}\arabic{sennum}} \pali{P\=ap\=anus\=asako} means an advisor [who advises] to do bad things.\\

\pali{\fbox{\stepcounter{sennum}\arabic{sennum}} Ahant\=aran'ti apaharanta\d m. \fbox{\refstepcounter{sennum}\arabic{sennum}\label{sen:amaarenta}} Ahant\=aran'ti am\=arenta\d m. \fbox{\stepcounter{sennum}\arabic{sennum}} Set\=i'ti matasayana\d m sayati. \fbox{\stepcounter{sennum}\arabic{sennum}} Agh\=atentan'ti am\=arenta\d m. \fbox{\stepcounter{sennum}\arabic{sennum}} Suddhass\=a'ti nirapar\=adhassa. \ \fbox{\stepcounter{sennum}\arabic{sennum}} Posass\=a'ti sattassa. \ \fbox{\stepcounter{sennum}\arabic{sennum}} Ana\.nga\d nass\=a'ti idampi nirapar\=adhabh\=ava\~n\~neva sandh\=aya vutta\d m. \fbox{\stepcounter{sennum}\arabic{sennum}} Paccet\=i'ti kammasarikkhaka\d m hutv\=a patieti.}

\addtocounter{sennum}{-8}
$\triangleright$ \fbox{\stepcounter{sennum}\arabic{sennum}} [The first] \pali{ahant\=ara\d m} means `not beating.' \fbox{\stepcounter{sennum}\arabic{sennum}} [The second] \pali{ahant\=ara\d m} means `not killing.' \fbox{\stepcounter{sennum}\arabic{sennum}} \pali{Seti} means `to lie down [because of] death.' \fbox{\stepcounter{sennum}\arabic{sennum}} \pali{Agh\=atenta\d m} means `not killing.' \fbox{\stepcounter{sennum}\arabic{sennum}} \pali{Suddhassa} means `innocent.' \fbox{\stepcounter{sennum}\arabic{sennum}} \pali{Posassa} means `being.' \fbox{\stepcounter{sennum}\arabic{sennum}} `\pali{Ana\.nga\d nassa}' is said to mean also the state of this innocence. \fbox{\stepcounter{sennum}\arabic{sennum}} \pali{Pacceti} means [the result], being worthy of the action, comes back.\\

\pali{\fbox{\stepcounter{sennum}\arabic{sennum}} Satth\=a ima\d m dhammadesana\d m \=aharitv\=a j\=ataka\d m samodh\=anesi -- `tad\=a dubbalavejjo devadatto ahosi, pa\d n\d ditad\=arako pana ahameva ahosin'ti.}

\addtocounter{sennum}{-1}
$\triangleright$ \fbox{\stepcounter{sennum}\arabic{sennum}} The Buddha, having brought this teaching, put together the birth story thus, ``In that time, the diabled physician was Devadatta, and I was the wise child.\\

\begin{longtable}[c]{|p{0.9\linewidth}|}
\hline
\hspace{5mm}\dag\ \small This account is clearly a fictional theme of hero versus villain. Poor Devadatta is almost always set up to be the chief villain in Buddhist stories. It is dispensable in any good story. There must be a conflict, otherwise no one will listen or read your story. And the hero has to win in the end.\\
\hspace{5mm}\dag\ \small We can see the technique of framing (see Chapter \ref{chap:discourse}) in use here. That is to say, Devadatta is framed to be seen only his bad side, like a flat character in popular novels. And the Buddha is seen as the impeccable hero. The practice is ubiquitous in literature.\\
\hline
\end{longtable}

\pali{\fbox{\stepcounter{sennum}\arabic{sennum}} S\=a\d liyaj\=atakava\d n\d nan\=a sattam\=a.}

\addtocounter{sennum}{-1}
$\triangleright$ \fbox{\stepcounter{sennum}\arabic{sennum}} A commentary to the story of Myna Bird, the seventh, [was finished].\\

\phantomsection
\addcontentsline{toc}{section}{Conclusion and discussion}
\section*{3.\ Conclusion and discussion}

Even though this text has a few long and complex sentences, it is easy to read in general. The commentary part is also clear, suitable for new readers. The story is appealing to most children, particular the young ones, despite the absence of the bird mentioned. Also, the verses are gripping and deserving a repeat. In sum, this is an excellent material for moral teaching. Even if the Buddhist related account is removed, the story still makes sense for all systems of belief.

That is the best, healthy, and accurate way to understand the text. However, many Buddhists still hold this story as a historical fact, and believe that the Buddha-to-be indeed was born in that event. Perhaps, this is what the author of the text really intended to make. The story can strengthen the belief in karma and rebirth, good for the survival of the Sangha, and strengthen the ethical concern of people, good for the survival of the state and society. So, on one dares to question its authenticity.

Evaluating this text on factual ground seems not fair, because it might not be meant to be accurate. No matter how fictional and improbable it looks, the moral value of the text is undeniable.
