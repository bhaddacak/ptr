\markboth{}{Afterword}
\cleardoublepage
\phantomsection
\addcontentsline{toc}{chapter}{Afterword}
\chapter*{Afterword}\label{chap:afterword}

As the readers have seen after finishing all materials I have presented to this point, this book on reading P\=ali is different from other books of the same subject, in a radical way. To my view, reading a text, particular an ancient one, is not about the language alone that constitutes the text, but the whole context in which the language operates. So, just knowing grammar is not enough to penetrate the hidden meaning, intended or not, in the text. That is the main reason I have to bring everything I can think of, related to a successful text reading, to our consideration.

No surprise, traditional students will find this book overwhelming with irrelevant materials. I insist that nothing is irrelevant here, even though some parts have little use in practice. The related knowledge presented here is the minimum requirement for tackling text reading in the modern time. As you might realize, texts can be manipulated in many ways, some are healthy, some are not. Knowing how texts are exploited can liberate us from an unhealthy obsession. Furthermore, we can direct the use of texts in a more constructive way.

After the readers understood everything explained in this book, also in PNL, they are supposed to be capable of reading any kind of P\=ali literature. All materials essential to the language are exposed. The next thing to do is to get familiar with the variety we may meet in the coming readings.

My hope is that more and more people are able to access to P\=ali literature in the future, and treat it with a critical, yet respectful attitude. When they argue about Buddhism in any topic, it will be effective if they understand the point by its original form. We will use less translators as mediator, so, less intervention and exploitation, because we can read it ourselves and understand it thoroughly.

That is a remark on text study. For those who are not scholarly type, and are afraid of word army, but want to get some benefit from Buddhism. If you understand my points expressed here and there, reading just what I show you in this Book is enough, I think. Because, for me, Buddhism is not a talkative religion. There is really little to know, much to practice. I do not suggest either that you should burn all books and go to live in the forest, meditating. Just live your life healthily and wisely, in the manner that suits you.

That does not mean anything counts, because in my view healthiness and wisdom are objective states. They are true for all of us, regardless of what we believe. We may come up with different ways of life individually, but they are not far either, because we all know what destructive life styles look like, as well as what is counted as healthy. The use of wise reasoning is an indispensable skill, as well as the ability to control our impulses.
