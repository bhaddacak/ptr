\chapter{How does text have meaning?}\label{chap:howtextmean}

Now let us move to how meaning is attached to text. This issue is really important but few in P\=ali or Buddhist studies think about it seriously, and many are not aware of its significance. Most people hold a naive view that words have meaning, then text is just a combination of words' meaning. It is more complicated than that. Let us start with `text' first.

Technically, what we mean by `text' is ``a composite structure consisting of smaller sign elements.''\footnote{\citealp[p.~98]{danesi:quest}} That is simple and straightforward. Not only literary works are counted as text, but also musical notations, mathematical equations, conversations, or even rituals. By `sign' we simply mean ``something that stands for something else in some way.''\footnote{\citealp[p.~29]{danesi:quest}} When a sign is used, we say it has certain `meaning,' the association between the sign body and that `something else.' We will talk more about `meaning' below.

Just gathering signs in one place does not make text, but signs have to be arranged in a coherent way according to \emph{code} suggested by \emph{context} (see Chapter \ref{chap:howeach} for code and context). We call the encoded meaning that is contained in text `message.'\footnote{\citealp[pp.~97--8]{danesi:quest}} To understand text, hence we have to decode the message in the right way. A difficulty of this is that one message can convey multiple meanings, and one meaning can be derived from different messages.\footnote{\citealp[p.~16]{danesi:messages}}

What is `meaning' after all? Here is a definition from dictionary: ``Something that is conveyed or signified; sense or significance.''\footnote{The American Heritage Dictionary, \url{https://www.ahdictionary.com/word/search.html?q=meaning}} When someone understand something and he or she wants to express the idea into a textual form, he or she has to convert it into words by orchestrating an amount of meaning to represent the idea. Meaning of the idea as a whole can be roughly a combination of meaning of every single word. That means meaning can be of low level (i.e.\ word), and higher levels (i.e.\ sentence, paragraph, discourse). Now we will focus on low-level meaning first.

\phantomsection
\addcontentsline{toc}{section}{Introduction to semiotics}
\section*{Introduction to semiotics}

To understand how a word holds certain meaning, a field of study has to be introduced. This leads us to \emph{semiotics}---the study of signs.\footnote{\citealp[p.~2]{chandler:semiotics}} The reason why this matters to our concern can be explained as follows:

\begin{quote}
[A]t the heart of semiotics is the realization that the whole of human experience, without exception, is an interpretive structure mediated and sustained by signs.\footnote{\citealp[p.~5]{deely:basics}}
\end{quote}

In modern era, systematic study of signs began with two towering figures: Ferdinand de Saussure (1857--1913) and Charles Sanders Peirce (1839--1914). I will not explain the subject in detail, but just enough for us to move on.\footnote{The best introduction to semiotics I have ever read is Daniel Chandler's \emph{Semiotics: The Basics}. To enthusiastic learners, reading that work is highly recommended.}

Before we go further, I have to clarify that what we call `meaning' can mean two things: \emph{sense} and \emph{reference}. Confusion in this distinction can end up with a pointless argument. `Sense' or designation is conceptual meaning, whereas `reference' or denotation is referential meaning.

\begin{quote}
[T]he \emph{sense} is a specific meaning in the mind (a concept) and the \emph{reference} is something in the outside world (an \emph{object} or \emph{referent}).\footnote{\citealp[p.~11]{chandler:semiotics}}
\end{quote}

Seeing an example is the best explanation here. When I say `Buddha' to mean `enlightened being,' this meaning is sense, thus `Buddha' is a concept. When I say (the) `Buddha' to mean `Gotama Buddha,' this meaning is reference, thus `Buddha' refers to a particular person. When a sign is used, such as `Buddha' in this case, it can represent both sense and reference depending on the user's intention.\footnote{In Saussurean terms, for example, word `Buddha' is called \emph{signifier}, and what it means, the sense of `enlightened being,' is called \emph{signified}. Saussure himself asserts that ``[t]he linguistic sign unites, not a thing and a name, but a concept and a sound-image'' (\citealp[p.~66]{saussure:course}). Hence only sense, not reference, is represented by sign in this view. To make things simpler, we will not follow this strict definition.} Another point related to this matter is words with different senses can point to the same referent, for example, the Enlightened One (\pali{buddho}), the Blessed One (\pali{bhagav\=a}), the Thus Gone (\pali{tath\=agato}), and the Master (\pali{satth\=a}) all have Gotama Buddha as their referent.

In Saussure's view, ``the linguistic sign is arbitrary.''\footnote{\citealp[p.~67]{saussure:course}} This means there is no intrinsic relation between word and concept. For example, in English we call hairy, barking thing `dog,' but in P\=ali, it is `\pali{sunakha}.' There is no \emph{dog}-substance or \pali{sunakha}-substance in that shaggy, noisy thing. The words used to represent the same concept are arbitrarily chosen by cultures. We cannot say that calling it `dog' is right, but `\pali{sunakha}' is wrong, or vice versa. It totally depends on context. However, not all words are strictly arbitrary, for example, onomatopoeia like `cuckoo' or `ding dong,' and interjections like `ouch' or `ah.' As we shall discuss more later, choosing a word for certain concept is not entirely random. There must be some reason behind that. Please keep this in mind for a moment.

I have a question right now: ``Where does meaning reside, in the mind or in text?'' If you can definitely answer this, you do not yet understand the issue well enough. Just read on.

As we have seen earlier, meaning in text is also determined by code and context apart from concepts held by signs. It is not simple as when you have something to say, you put it into words, then you say that to me in the hope that I can understand it by the message itself. That is not really what happens. We can understand each other only when we share the same code and context. They are not just a bunch of definitions, grammar rules, and technical usages that constitute code. They are also social values and expectations.

The idea that value determines meaning has been already suggested by Saussure. He gives us an analogy of a chess game that has lost a chess piece, say, a knight. We can use other object, like a bottle cap, instead to play in that position. We can replace one thing with another because they have the same value.\footnote{\citealp[p.~110]{saussure:course}} The meaning thus is not really in the chess pieces, but in the matrix of convention. Seeing this in terms of language, a word is a convenient token that separates one meaning from others, just for making a distinction. The meaning is derived from the interaction between words and the matrix. That is to say, meaning in nature is culture-bounded.

The consequence of this is disturbing. Meaning turns to be not the representation of the real out there. It is just a relation within the system of signs. Think about a dictionary, for example, one definition always links to other definitions. And meanings are always subject to change. Old dictionaries are not useful anymore. Now we reach at the border of a forbidden area. We touch the line of relativism, a dangerous zone I do not want to lead you to.\footnote{Relativism, aka ``anything goes,'' is a no-no area in intellectual world. It ruins any attempt to establish the common truth. If it is real at any rate, argumentation is a waste of time, because everyone is right in one's own way. That sounds nice, but it will be no progress intellectually. So, as a good scholar, we should do our best to reach the common truth. In religious practice, however, I think some dose of relativism can bring a healthy living, sympathy, and detachment.} However, sometimes defending an idea objectively is challenging. We have to do it nonetheless.

Meaning is vulnerable to change because it ``is not an absolute, static concept to be found neatly parcelled up in the message. Meaning
is an active process.''\footnote{\citealp[p.~46]{fiske:communication}} Meaning is not stable, not only because we have active mechanism in cognition, not only because the way we use signs is changing over time, but also the codes in our cultural matrix are changing too. As Umberto Eco puts it: ``In exchanging messages and texts, judgments and mentions, people contribute to the \emph{changing} of codes.''\footnote{\citealp[p.~152]{eco:semiotics}}
