\chapter{Translation techniques}\label{chap:techniques}

Talking about techniques used in translation is an endless endeavor. Since this book is not directly about translation, and neglecting the issue is not a good idea either, so I will address this topic minimally. The content in this chapter is modeled after Jean-Paul Vinay and Jean Darbelnet's idea.\footnote{\citealp[pp.~31--40]{vinaydarbelnet:comparative}, first published in 1958}

Regarding translation techniques, Vinay and Darbelnet differentiates \emph{strategy} from \emph{procedure}. The former denotes an overall orientation of the translator.\footnote{As shown in \citealp[p.~23]{munday:translation}, several translation strategies are listed, namely free translation, idiomatic translation, functional translation, literal translation, source-oriented translation, target-oriented translation, foreignizing, exoticizing, naturalization, localization, domestication, and so on. As we have seen in Chapter \ref{chap:ethics}, domestication and foreignization are also counted as strategies.} In Vinay and Darbelnet's terms, they are \emph{direct} translation and \emph{oblique} translation. The latter is a specific technique used by the translator at a certain point in a text. According to Vinay and Darbelnet, there are seven procedures. The first three belong to direct translation, the rest to oblique translation.

\section*{Techniques for direct translation}

We use direct translation, also called literal translation, when word-for-word rendition can be done from SL to TL. There are three procedures mentioned as follows:

\paragraph*{Borrowing} When there is no parallel concept in TL, importing terms used from SL is the simplest method. For example, many Buddhist technical terms are borrowed, such as \pali{dhamma, kamma, nibb\=ana,} and so on.

\paragraph*{Calque} This is a special kind of borrowing. It happens at the level of phrase (words) or morpheme (sub-word).\footnote{Vinay and Darbelnet themselves do not consider the use at the morphemic level \citep[p.~35]{fawcett:translation}. But see also \citealp[p.~182]{munday:companion}. In fact, the definition of calque is quite inconsistent. In \citealp[pp.~17--8]{shuttleworth:dictionary}, this means `loan' translation which the individual elements of an SL item are translated literally to produce a TL equivalent. Thus, \pali{vijjubala} (electric power) can be counted as a calque. To make it clear, I hold that calque and literal translation are different in the way that a calque has imported sub-words or words, but literal translation does not.} When P\=ali is taken into consideration, it can happen in compounds. For example, \pali{buddhadhamma} can be translated as the `Buddhist Dhamma.'

\paragraph*{Literal translation} This is a direct transfer of ST into TT word-for-word. Mostly, word classes are maintained, and if possible, word order in a sentence. Some examples of terms translated literally are \pali{vijjubala} (electric power), \pali{ga\d naka(yanta)} (computer), and \pali{d\=urasadda(yanta)} (telephone).

\section*{Techniques for oblique translation}

When word-for-word translation does not work or is not impossible, sense-for-sense translation can be applied. This can also be called free translation, which the translation can go beyond word level. Thus, the unit of translation can be a phrase, clause, sentence, or even sentences. This strategy may be demanded for a smooth, readable translation. There are four procedures mentioned:

\paragraph*{Transposition} This happens when word class of an SL item is changed in TL. For example, my translation of impersonal passive structure uses an action/verbal noun for a passive verb, e.g.\ \pali{tena \d th\=iyate} (Standing is done by him).

\paragraph*{Modulation} Whereas transposition involves word-class manipulation, modulation manipulates the mental image instead. Possible methods for modulation can be using abstract for concrete, using part for whole (or a specific object for a general one), or using negation of the opposite. Here are examples:

\begin{quote}
[abstract for concrete]\\
\pali{udarassa k\=ara\d n\=a}\footnote{Mv\,1.73} (from the reason of stomach)\\
= from the reason of living/sustenance.\\[1.5mm]
[part for whole]\\
\pali{odana\d m s\=adhay\=ami}\footnote{Thig\,414} ([I] prepare boiled rice)\\
= [I] prepare food.\\[1.5mm]
[negation of the opposite]\\
\pali{as\=igha\d m carati} ([He/She] walks not-quickly)\\
= [He/She] walks slowly.\\[1.5mm]
\end{quote}

\paragraph*{\'Equivalence} This is defined by Vinay and Darbelnet as the rendition that ``replicates the same situation as in the original, whilst using completely different wording.''\footnote{\citealp[p.~342]{vinaydarbelnet:comparative}} This method is mostly used in idiomatic expressions. For example, we render \pali{k\=ala\d m karoti} as ``[He/She] dies'' not ``[He/She] makes time'' (nonsensical).\footnote{In Thai, \textit{(Kra)tham Kaala} (to die) is calqued on \pali{k\=ala\d m karoti}. The term is used only in scriptural context.} And it is natural to translate \pali{p\=adena gacchati} simply as ``He/She walks'' rather than literal ``He/She goes by the foot.''

\paragraph*{Adaptation} The result in TL of this method can be very different from SL, because unfamiliar situations in the source culture are adapted into the target culture. For example, \pali{pi\d n\d dap\=ata} (fall of a lump of food) is a technical term that is alien to non-Buddhist cultures. It can be rendered precisely as `morning alms collection with a silent presence.' Nowadays this is quite a well-known concept, so we can simply translate it as `collection of alms' or just `alms-round.'

\section*{Concluding remarks}

There is no definite solution what strategy or procedure should be used in a particular situation. In this chapter, just a rough picture is drawn to make you see the basic idea when translation techniques are taken into consideration.\footnote{To learn more on translation techniques, see, for example, \citealp{newmark:textbook} and \citealp{fawcett:translation}.} Since the main purpose of our course is to read texts for understanding, not to produce a fine translation, I suggest that we should try to render a text word-for-word as much as possible. Because understanding P\=ali text cannot be done by separating meaning from its grammatical structure. Furthermore, this can prevent us, to some extent, from imposing our ideological stance to the translation.

