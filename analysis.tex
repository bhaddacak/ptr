\chapter{Source text analysis}\label{chap:analysis}

Before we produce any translation work, we have to know the source text well enough. So, in the translation process, analyzing source texts is a precondition of the translation. I will talk about translation process and show you a simple model in Chapter \ref{chap:principle}. For now, I will introduce the readers to the principle of text analysis. The topic is quite detailed and technical. Because I do not want to overload P\=ali learners with unfamiliar subject, I water down the content to just-enough level (but still heavy to read though).

Text analysis is studied extensively by Christiane Nord.\footnote{By theoretical camp, Nord belongs to functional (skopos) school.} We will mainly follow her guideline here (with many things left out). According to Nord, there are two sets of factors contributing to text analysis: extratextual and intratextual factors.\footnote{\citealp[p.~41]{nord:analysis}} Normally, the former do not belong to the text itself. They are in the text's environment, such as the title, author, date of publication, etc. The latter are information provided by the text, e.g.\ the subject matter, content, etc.

The whole process of analysis incorporates `top-down' or extratextual analysis and `bottom-up' or intratextual analysis. When the extratextual information is unknown or uncertain, analyzing internal features can yield some external information. So, the process of analysis can run recursively. However, obtaining accurate information from texts of the distant past like the P\=ali canon can be very difficult.

\phantomsection
\addcontentsline{toc}{section}{Extratextual factors}
\section*{Extratextual factors}

\paragraph*{1.\ Sender} Broadly speaking, this means the person or institution who uses a text in order to convey a certain message to somebody else and/or to produce a certain effect.\footnote{\citealp[p.~48]{nord:analysis}} Sometimes sender can be confused with \emph{text producer} who is the person who writes the text according to the instructions of the sender, and makes it comply with the rules and norms of text production valid in the respective language and culture.\footnote{\citealp[p.~48]{nord:analysis}} If only the author's name of the work is given, normally it is assumed to be the text producer. The translator is counted as text producer by the same token. However, when the author is also the initiator, it can be the same person as the sender.

To be more precise, the person who designs and formats the text may be another expert. And the one who presents the text, e.g.\ a news reader or announcer, may be yet another person. When the text bears the name of both sender and text producer, the sender usually plays the primary role, because the communicative intention mainly comes from the sender, not the producer, let alone the designer and presenter.\footnote{However, as cognitive psychology tells us, every party involving in text production can affect the meaning one way or another.}

In sender analysis, we have to identify the sender of the text from textual environment or from the text itself, if possible. Then we try to figure out whether the sender is the same as the text producer. The sender may be absent in the context and only the text producer can be specified. We may have multiple answers to this factor ranked by their likelihood.

\paragraph*{2.\ Sender's intention} This means what function of the text that the sender intends it to be, and what effect on the receiver will be achieved.\footnote{According to speech act theory, the `function' is comparable to illocutionary act of the communication, and `effect' is comparable to percutionary act.} To be precise, there are three different perspectives of this factor, namely intention, function, and effect.\footnote{\citealp[p.~53]{nord:analysis}} The \emph{intention} is defined in terms of the sender who sets the purpose of the text. The \emph{effect} of the text happens only when the text is understood in a particular way. So, it is defined from the viewpoint of the receiver. And the \emph{function} of the text is defined externally before the reading, so it is viewed from the text itself.\footnote{In fact, a text cannot has a view by its own. The function therefore might be postulated by the community or authority that uses that text, if the information from the sender is not available. This means a text can perform multiple functions at a time.} Ideally, the three aspects are congruent: the receiver experiences what the sender intends. But it is not always so that the intention guarantees the corresponding effect, as put by Nord:

\begin{quote}
It is the receiver who `completes' the communicative action by receiving (i.e.\ using) the text in a certain function, which is the result of the configuration or constellation of all the situational factors (including the intention of the sender and the receiver's own expectations based on his/her knowledge of the situation).\footnote{\citealp[p.~53]{nord:analysis}}
\end{quote}

Practically speaking, sender's intention is difficult to identify, even when the sender is clearly known, or worse, even it is stated explicitly in the text, let alone when the real sender is totally unknown. Reasons from the nature of human mind are: the sender may have multiple intentions, sometimes contradictory ones; the sender may not be able to maintain a unified intention throughout the text; the text might be influenced by unconscious intention; and the sender misleads the receiver by an acceptable explicit intention but in fact hidden agendas are laid behind that.

So, analyzing the sender's intention is not an easy task. In our application to P\=ali text, which the exact sender is hard to pin down, let alone its intention, this analysis is done instead by text function analysis (see below), which is easier to figure out.

\paragraph*{3.\ Audience} This may be the most important factor in the whole translation process. Without the addressee of the text, the translation turns pointless. For modern literature, the audience is easily to be determined by considering the genre. Some works address specific audience, such as child fiction or romance. Some target a wider range of audience, like science fiction or comedy. For religious texts, the audience seems to be obvious---the adherents of that religion. However, if the texts are ancient, ST audience is not the same as TT audience because of cultural and linguistic differences.

Therefore, the translator must analyze both sides of audience: (1) ST addressees and their relationship to the source text, and (2) TT receivers whose presuppositions, expectations, knowledge, and communicative role will shape the target text.

A distinction should be made between addressee and chance receiver. The former is the target audience, while the latter is one who happens to read the text indirectly. Scholars who read the text without any commitment to the religion fall into the latter case. The production of TT can be shaped by what kind of receiver the translator has in mind. In my case, all of my translations of P\=ali in \pali{P\=ali for New Learners} and this volume, target primarily to non-adherent scholars and secondarily to learned Buddhists. This can explain why sometimes my translation is not easy to read.

In our course, however, as I often tell that our goal is to read P\=ali texts, not to produce a good translation, the audience in this case turns to be the readers or translators themselves. We read or translate texts here for our own understanding. So, the only audience to be analyzed here is the ST addressees. 

\paragraph*{4.\ Medium} This factor is referred to the means, or `channel,' by which the text is conveyed to the receiver. In general, it can be either in \emph{speech}, a face-to-face communication (such as an interview or a real conversation), or in \emph{writing} (such as books). Different types of medium determine how the TT is organized in terms of level of explicitness, arrangement of arguments, choice of sentence types, features of cohesion, use of non-verbal elements, etc. The distinction between speech and writing is not hard and fast, because both can be changed to one another. So, what should be considered are specific features of the medium, such as ``coincidence or discontinuity of text production and reception, indirect or direct form of communication, spontaneity of text production, opportunities for feedback operations, one-way communication, etc.''\footnote{\citealp[p.~63]{nord:analysis}} Another point the translator should keep in mind is that the same media may function differently in different cultures.

In our concern here, this factor is trivial because we mostly treat P\=ali texts as written type, even thought some of them once underwent oral transmission. After the canon was written down, it was no longer of the oral type. Even before that when the teaching was formulated by the councils, the form of oral communication was made rigid unnaturally. By such a form, the canon is closer to written texts than oral ones.

\paragraph*{5.\ Place of communication} This is about the location of text. It can be divided into \emph{place of text production} and \emph{place of text reception}. This does not mean only the linguistic aspect, but also cultural and political conditions. By a shift of place, the reading sometimes cannot be done directly, as noted by Nord:

\begin{quote}
A text published in a country where literature is censored must be read `in another light' than a text whose author has not been subject to any restrictions, since authors under censorship often write `between the lines.'\footnote{\citealp[p.~68]{nord:analysis}}
\end{quote}

This issue does little with our P\=ali text reading. However, being aware of differences in place when the text was produced can remind us that the text indeed does not belong to our cultural context.

\paragraph*{6.\ Time of communication} This is about the period of time when the text is produced and translated. When there is a considerable time lag, we have to take the historical change into consideration. Nord explains this clearly as follows:

\begin{quote}
Depending on the age of the text, the receiver/translator may have totally different expectations as to the typical features of the text type in question. S/he may even expect obsolete forms that are not used any more.\footnote{\citealp[p.~70]{nord:analysis}}
\end{quote}

When the dimension of time is concerned, we have choices of translating texts. It can be \emph{conservative} or \emph{re-creative} translation. The former is comparable to foreignization, to make the source cultural expectations visible. And the latter is comparable to domestication, to adopt the target cultural expectations in translation (see also Chapter \ref{chap:ethics}).

\paragraph*{7.\ Motive of communication} This factor is about the reason why the text is produced. In some cases, it is easy to see the reason behind the text production, for example, a news report is written because something notable has happened. We should not be confused motive with intention. Let me illustrate in this way. Someone may write a poem from the reason (motive) that he or she falls in love, with the intention to make his or her lover know that feeling and feel likewise. Still, sometimes it is hard to tell motive from intention. And more importantly, we cannot see others' motive directly. We at best can only make a speculation about it.

\paragraph*{8.\ Text function} As we have seen earlier, text function is closely related to the sender's intention, but they are not exactly the same thing. There are several ways to classify text function. I will show you here a general idea of three text functions: informative, expressive, and appellative/operative.\footnote{This reminds us to performative act of language use (see Chapter \ref{chap:howmisunder}). We can see text function as performative act at the level of the text as a whole.} Characteristics of these functions are summarized in Table \ref{tab:texttypes}.\footnote{adapted from \citealp[p.~115]{munday:translation}} Then I will add another text function which I see it can be the case if we take religious text into consideration.

\begin{table}[!hbt]
\centering
\caption{Functional characteristics of text types}
\label{tab:texttypes}
\bigskip
\footnotesize
\begin{tabular}{@{}%
	>{\raggedright\arraybackslash\bfseries}p{0.18\linewidth}%
	>{\raggedright\arraybackslash}p{0.22\linewidth}%
	>{\raggedright\arraybackslash}p{0.22\linewidth}%
	>{\raggedright\arraybackslash}p{0.22\linewidth}@{}%
} \toprule
& \bfseries Informative & \bfseries Expressive & \bfseries Operative \\
\midrule
Language function & Informative (representing objects and facts) & Expressive (expressing sender's attitude) & Appellative (making an appeal to text receiver) \\
& & & \\
Language dimension & Logical & Aesthetic & Dialogic \\
& & & \\
Text focus & Content-focused & Form-focused & Appellative-focused \\
& & & \\
TT should \ldots & Transmit referential content & Transmit aesthetic form & Elicit desired response \\
& & & \\
Translation method & `Plain prose', explicitation as required & `Identifying' method, adopt perspective of ST author & `Adaptive', equivalent effect \\
\bottomrule
\end{tabular}
\end{table}

Informative text type offers facts. This is the main function of communication, to give certain information. An appliance's user manual and a news report are mostly of this kind. Expressive text type is creative composition. The author uses artistic aspect of language to express certain feelings rather than just giving facts. Poetry is a marked example of this. And operative text type induces behavioral responses. It persuades the receiver to act in a certain way. An obvious example is advertisement.

Practically, a text can have all these three characteristics, despite one dominating function. For example, an ad in fact gives us some information in a beautiful form, but first and foremost it is used to persuade us to buy something. Sometimes text function is far from clear. Religious texts, for example, can serve multiple functions at a time. They give information in a persuasive form to attract the listener. The main function of religious texts can vary upon situations. Sometimes it can be a matter of dispute. For example, when the Buddha mentioned the four castes (in \pali{Agga\~n\~na Sutta}, DN 27, for example), the tradition grasps it by informative function, while Richard Gombrich sees it as a joke, hence expressive function.\footnote{\citealp[pp.~81--2]{gombrich:how}; \citealp[pp.~79--80]{gombrich:theravada}}

Moreover, text function can be changed from its original intention. For example, at first Jonathan Swift's \emph{Gulliver's Travels} (1726) was written as a satire to attack the government (operative function), but nowadays it is mostly read as an entertaining fiction (expressive function).

Those three basic functions of text can be attributed fittingly to literary text in general. To religious text, these seem not enough because there is another context that a portion of religious text is often used. That is in ritualistic performances or in religious ceremonies. In this kind of situation, the meaning of text is rarely taken into consideration. 

The text is often recited in the original language, and it makes things done by that recitation, not by the meaning it possesses. We can see a parallel picture by considering magical incantation. But religious recitation does more than making a miraculous effect. It can make a ceremony formal and authoritative. I call this flatly \emph{ceremonial}\footnote{In fact, I would prefer `\emph{performative}' to this term, if it has not already had a specific meaning in speech act theory. To avoid a confusion, so I use this blunt term.} function. So, in my analysis, as we shall see in due course, this function will be mentioned as well.

\phantomsection
\addcontentsline{toc}{section}{Intratextual factors}
\section*{Intratextual factors}

\paragraph*{1.\ Subject matter} Basically, the answer to the question ``What does the sender talk about?'' is about subject matter. In literature, it is called \emph{theme}. Normally, a fiction has one central theme, about love, about justice, or whatever, for example. We have to read through a novel to know its theme. Sometimes extratextual factors have to be taken into consideration as well. In non-fiction writing, the subject matter can be easily seen from the topic given, particularly in scientific articles. However, a text can hold multiple subject matters, and some subjects are culture-bound.

\paragraph*{2.\ Content} This is the answer to the question ``What does the text mean?'' It is about meaning or sense of the text from a reading. As we have seen from the beginning when we talked about understanding, meaning is a complex matter. So, it is not easy to obtain the `true' meaning of the text. Sometimes we have to take stylistic elements and connotations into account. In other words, we also have to read between the lines.

\paragraph*{3.\ Presuppositions} Nord provides a good explanation of this, so it is better to have the full quote here:

\begin{quote}
[P]resuppositions comprise all the information that the sender expects (= presupposes) to be part of the receiver's horizon. Since the sender wants the utterance to be understood, it seems logical that s/he will only presuppose information which the receiver can be expected to be able to `reconstruct.'\footnote{\citealp[p.~106]{nord:analysis}}
\end{quote}

That is to say, presuppositions are not part of the text. They are left untold because they are supposed to be known in advance. So, identifying presuppositions is not straightforward. It can be analyzed from the content. For an easy example, a presupposition of a ghost story is the possibility that something supernatural exists. This does not confirm whether the supernatural exists or not, but rather it provides the logical ground essential to understand the story told. Likewise, in the large part of the P\=ali canon, a marked presupposition is that the afterlife exists.

\paragraph*{4.\ Text composition} There are two levels of text composition: macrostructure and microstructure. The former is about how the text is organized from textual units. For example, a text can be divided into parts, chapters, sections, and paragraphs. It can be composed of the beginning, ending, footnotes, quotations, and so on. The latter comprises sub-units that make things sensible, i.e.\ information units, stages of a plot, logical relations, thematic structure, and so on.

\paragraph*{5.\ Non-verbal elements} These ``comprise the paralinguistic elements of face-to-face communication (e.g.\ facial expressions, gestures, voice quality, etc.) as well as the non-linguistic elements belonging to a written text (photos, illustrations, logos, special types of print, etc.).''\footnote{\citealp[p.~118]{nord:analysis}} For ancient texts like the P\=ali canon, these elements are rarely presented.

\paragraph*{6.\ Lexis} This is simply about words used in the text. Extratextual factors can affect lexis by determining regional or social dialects, choice of register, and technical terminology used. It can be conditioned by stylistic interest of the sender, e.g.\ stylistic markers, connotations, rhetorical figures of speech such as metaphors and similes, individual word coinages, puns. The choice of words can reflect somehow the sender's intention, as noted by Nord:

\begin{quote}
In order to elicit the sender's intention it seems advisable to analyse the `degree of originality' of the lexis used in the text. This is common practice with similes and metaphors.\footnote{\citealp[p.~125]{nord:analysis}}
\end{quote}

\paragraph*{7.\ Sentence structure} This is about construction and complexity of sentences. They can be short, long, simple, or complex. Other syntactic features can also affect sentence structure, such as word order, parallelism, rhetorical question, parenthesis, ellipsis, and so on.

\paragraph*{8.\ Suprasegmental features}\label{par:suprasegment} These are specific non-verbal elements, such as intonational features, pauses, etc. They can be done by graphical devices, like punctuation (the use of quotation marks, dashes and parentheses, for example), capitalization, italicization, and so on. In spoken texts, these can be signaled by acoustic means, e.g.\ tonicity, modulation, variations in pitch and loudness. Analyzing these features can have some benefits noted by Nord as follows:

\begin{quote}
The analysis of suprasegmental features often yields information about the content (e.g.\ irony) and the subject matter (e.g.\ the `solemn' tone of a funeral address), as well as presuppositions (e.g.\ an interruption of the intonation contour in allusions) and composition (e.g.\ pauses, stress on the rhematic parts of the utterance).\footnote{\citealp[p.~137]{nord:analysis}}
\end{quote}

\subsection*{Concluding remarks}

It is quite logical to say that you cannot do any translation if you do not understand the source text. This does not mean you have to know the text thoroughly, but rather you should know its limitation and variety of its options. You may stipulate certain sender's intention and text type to make a goal-specific translation, for instance.

As pointed by Nord, every factor does not stand alone. They interrelate to each other. So, the analyzing process is not a one-time task. We have to do it recursively. To understand the extratextual factors, we have to get some hints from intratexual factors, and vice versa. For example, to some extent, the sender's intention can be suggested by word choices used in ST.

At last, we can say that text analysis is difficult to do, and thorough text analysis is impossible. As we shall see in my demonstrations in Part III, I take only some factors to be analyzed before the reading just enough to be familiar with the text. And during the reading, only some marked issues are raised to be analyzed. That is to say, not all of these can be applied. We have to choose them accordingly to the text we read.
