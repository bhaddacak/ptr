\chapter{The first sermon}\label{chap:dhammacak}

Our first full-blown example of text reading is no better than the first sermon, \pali{Dhammacakkappavattanasutta}. I choose this for a number of reasons. First, It is quite well-known to Buddhists and those interested in Buddhism, and we should make it known more and understood. Knowing what the Buddha taught at the very beginning is crucial to the understanding of the religion in the early stage. Therefore, it is sensible to take it seriously and read it carefully.

Second, the construction of the text looks old. It was intended to be recited and memorized, as many Buddhists still do today. We will have a taste what the oldest layer of the canon looks like. And third, the text is not too long and quite easy to read, so less burdensome to new learners.

For the model of reading process, I will follow roughly what were given as a guideline in Chapter \ref{chap:principle}. I say `roughly' because the process is difficult to be separated as clear steps. For example, reading and intra-textual analysis, and even some of extra-textual analysis, practically have to be done at the same time mutually. And, (critical) discourse analysis, as described in Chapter \ref{chap:discourse}, is also applied to the analysis simultaneously. Moreover, not everything we have discussed will be applied here.

Hence, the process below will be divided into three big phases: pre-reading introduction, reading with a draft translation, and post-reading conclusion and discussion.

To gain more understanding of the text, I also bring the commentaries to the sutta to our study in the next example (Chapter \ref{chap:comcak}). Therefore this and the next chapter should be studied in tandem.

\phantomsection
\addcontentsline{toc}{section}{Pre-reading introduction}
\section*{1.\ Pre-reading introduction}

This step is about textual environment, the information we know before the text is being thoroughly read. This information, if we are lucky, can be obtained additionally from works of scholars who have studied the text formerly. Here are what we get so far.

\paragraph*{About the text} According to the current collection of P\=ali texts, the first sermon appears in two places: first, in the Vinaya, Mah\=avagga (Mv\,1.13--17), and second, in Sa\d myuttanik\=aya, Mah\=avagga (S5\,1081, SN\,56). Beside different opening, the two texts are identical. However, they are treated as different entities, having their own editorial notes, commentaries and subcommentaries. We will use the main text from Sa\d myuttanik\=aya in which the text is named \pali{Dhammacakkappavattanasutta}. In Mah\=avagga of Sa\d myuttanik\=aya, there are totally 12 collections (\pali{sa\d myutta}). The \pali{Dhammacakka} is the first (of 10) sutta of the second (of 11) group in the last collection, \pali{Saccasa\d myutta} (the collection concerning truth).

\paragraph*{About the author} The very source of the text is attributed to the Buddha himself. As the textual formation tells us, however, it is far from the exact wording. No one talks in such a way. It is a retelling, at best. Therefore, the compilers of the canon from the early councils are supposed to be the text producer. The Buddha is supposed to be the sender, but it can be seen that he has nothing to do with the later textual formation. So, the Sangha at the time of compilations is more likely to be the sender.

\paragraph*{About the audience} Out of question, the target audience of this text is Buddhist monks and lay adherents. So, the text is supposed to be used within the religion, as it is used in ceremonies today. Although the text depicts that the first five disciples of the Buddha is the listeners of this sermon, those are not the audience of this text, because we do not know exactly what the Buddha said at the time. It is impossible to be this short. And as shown by the text itself, as will shall see, the text also includes the context of the sermon. So, it is better to be seen as a rework of the real sermon which is scantly known to us.

\paragraph*{About time and place} As the content suggests, the first sermon happened at a deer park where seers usually came (in the past), supposedly from the air (\pali{isipatane migad\=aya}), near Varanasi or Benares (\pali{b\=ar\=a\d nas\=i}). As held by Buddhists, the event occurred two months after the enlightenment. However, where and when the text came into the present form are uncertain. The tradition holds that the text has its form even in the Buddha's life time, because it is supposed to come directly from the master's mouth, at least some part of it.\footnote{The exclamation at the end, ``\pali{a\~n\~n\=asi vata, bho, ko\d n\d da\~n\~no}'' (Oh!, friends, Ko\d n\d da\~n\~na understood [it]), can be a testimony. See also the discussion of Sentence \ref{sen:annaasi} below.}

\paragraph*{About motives} Anything put in the first sermon of the founder is important. That is the very message the founder wants others to know after him. The motive of the Buddha could be that he just wanted to tell what he realized. It is too early to say that his motive was to establish the religion as we know it today. However, the motive of the compilers might be different. After the religion became institutionalized, preserving the first sermon is amount to preserving the religion itself. It has to contain the essence of the teaching. So, the motive of the Sangha could be to assert what the Buddha really knew and in what manner.

\paragraph*{About text function} It may be thought that the text has only informative function, to provide some facts. But its use in ceremonies nowadays show that ceremonial function can also be the case. And I want to add that we cannot rule out operative function of the text as well. Reading it closely, we can also see the persuasive aspect of the text.

\phantomsection
\addcontentsline{toc}{section}{Reading with a draft translation}
\section*{2.\ Reading with a draft translation}

My way of doing this part is quite fussy, for the benefit of new learners. We have the P\=ali text here with its rough rendition. I will break the text down to sentence level and work on them one by one, sometimes two or three if they are very short or closely connected. All chunks are numbered for the sake of referencing. I mark translations with a triangle ($\triangleright$). To remind the readers again, my translation is far from eloquent. I try to keep the sentence structure as much as possible. So, sometimes it sounds weird or awkward. And some words are intentionally untranslated, particularly what is deemed as a proper name.

For a very long sentence, I help the readers to take a breath by inserting break-markers, such as \fbox{1}, \fbox{2}, etc. Contents within borders with slightly smaller font are my commentary. This includes, explanation, analysis, notes, criticism, opinions, and other things should be said apart from the translation. Here what is counted as (critical) discourse analysis will be marked by a dagger (\dag).

According to what we have learned in Chapter \ref{chap:sentence}, I will demonstrate sentence decomposition (without partial translations) only in this example. If draft translations look confusing, seeing the diagrams can clarify my analysis. Long boxes in the diagrams are cut into pieces to fit our pages. The cut points are marked by this symbol: $\neg$. Sentence decomposition takes time and space to do, but for new learners it can be helpful. So, I encourage those who are very new to P\=ali text reading to do the diagramming first. This can also be a good activity in classroom environment. Then here we go.

\bigskip
\begin{center}
\textbf{\pali{Dhammacakkappavattanasutta\d m}}\par
$\triangleright$ A discourse concerning\\the forward moving of the wheel of Dhamma 
\end{center}

\begin{longtable}[c]{|p{0.9\linewidth}|}
\hline
\hspace{5mm}\small T.\,W.\,Rhys Davids translated this name as `The Foundation of the Kingdom of Righteousness.'\footnote{\citealp[p.~146]{rhys:suttas}} This is a good example of what we call \emph{domestication} (see Chapter \ref{chap:ethics}). It sounds familiar and sensible, but has a high degree of manipulation or suggestion.\\
\hspace{5mm}\small I put it more straightforward to keep it \emph{foreignized}. The meaning thus sounds hazier. It is like a metaphor---the wheel of Dhamma is being set in motion. More or less, it sounds like the vehicle called `Buddhism' was about to run.\\
\hspace{5mm}\small What is this Dhamma can be open to interpretation: the teaching, the righteousness, or the religion as a whole. According to translation techniques (discussed in Chapter \ref{chap:techniques}), we use Dhamma as a borrowed term from Pa\=li. The word itself has a vast area of meaning. The term is difficult to be pinned down by an English word, if the context is not clear. So, our best solution here is maintaining its original form.\\
\hline
\end{longtable}

\medskip
\setcounter{sennum}{1}
\arabic{sennum}. \pali{Eka\d m samaya\d m bhagav\=a b\=ar\=a\d nasiya\d m viharati isipatane migad\=aye.}\\
\small
\fbox{\pali{Eka\d m samaya\d m} [L]} \fbox{\pali{bhagav\=a} [S]} \fbox{\pali{b\=ar\=a\d nasiya\d m} [L]} \fbox{\pali{viharati} [V]} \\\fbox{\pali{isipatane migad\=aye} [L]}\\
\normalsize
$\triangleright$ In one occasion, the Blessed One lives in Isipatana\footnote{For the origin of the term, see Chapter \ref{chap:comcak}.} Deer Park near B\=ar\=a\d nas\=i.\\

\begin{longtable}[c]{|p{0.9\linewidth}|}
\hline
\hspace{5mm}\small Do you wonder why present tense (\pali{viharati}) is used here? It seems that this sentence was not a part of the original content, but later addition with a stock pattern. This sentence does not appear in the Vinaya. So, it is better do not take the question seriously.\\
\hline
\end{longtable}

\stepcounter{sennum}
\arabic{sennum}. \pali{Tatra kho bhagav\=a pa\~ncavaggiye bhikkh\=u \=amantesi}\\
\small
\fbox{\pali{Tatra} [L]} \fbox{\pali{kho} [P]} \fbox{\pali{bhagav\=a} [S]} \fbox{\pali{pa\~ncavaggiye bhikkh\=u} [R]} \\\fbox{\pali{\=amantesi} [V]}\\
\normalsize
$\triangleright$ In that place, the Blessed One called the group of five monks.\\

\begin{longtable}[c]{|p{0.9\linewidth}|}
\hline
\hspace{5mm}\small In the Vinaya (Mv\,1.13), it is ``\pali{Atha kho bhagav\=a pa\~ncavaggiye bhikkh\=u \=amantesi}.'' If you take a look at that section in the Vinaya, you will find that the use of \pali{atha} (then) is contextually appropriate, because it connects the sequence of events.\\
\hline
\end{longtable}

\stepcounter{sennum}
\arabic{sennum}. \pali{dveme, bhikkhave, ant\=a pabbajitena na sevitabb\=a. Katame dve?}\\
\small
\fbox{\pali{dve ime} [M]} \fbox{\pali{bhikkhave} [A]} \fbox{\pali{ant\=a} [S]} \fbox{\pali{pabbajitena} [R]} \\\fbox{\pali{na} [P]} \fbox{\pali{sevitabb\=a} [V]}. \fbox{\pali{Katame} [M]} \fbox{\pali{dve} [S]} [\fbox{\pali{honti?} [V]}]\\
\normalsize
$\triangleright$ ``These two extremes, monks, should not be practiced by a renouncer. What [are] the two?\\

\stepcounter{sennum}
\arabic{sennum}. \pali{Yo c\=aya\d m k\=amesu k\=amasukhallik\=anuyogo [so] h\=ino gammo pothujjaniko anariyo anatthasa\d mhito, yo c\=aya\d m attakilamath\=anuy\-ogo [so] dukkho anariyo anatthasa\d mhito.}\\
\small
\fbox{\pali{Yo} [M]} \fbox{\pali{ca} [P]} \fbox{\pali{aya\d m k\=amesu} [M]} \fbox{\pali{k\=amasukhallik\=anuyogo} [S]} \\\fbox{\pali{[so] h\=ino gammo pothujjaniko anariyo anatthasa\d mhito} [M]} \\{[\fbox{\pali{hoti} [V]}]}, \fbox{\pali{yo} [M]} \fbox{\pali{ca} [P]} \fbox{\pali{aya\d m} [M]} \fbox{\pali{attakilamath\=anuyogo} [S]} \\\fbox{\pali{[so] dukkho anariyo anatthasa\d mhito} [M]} [\fbox{\pali{hoti} [V]}]\\
\normalsize
$\triangleright$ [First,] which this [extreme called] the practice of indulging in sensual pleasure, [that is] inferior, belonging to villagers, belonging to common people, not noble, not equipped with benefit. [Second,] which this [extreme called] the practice of making oneself exhausted, [that is] suffering, not noble, not equipped with benefit.\\

\begin{longtable}[c]{|p{0.9\linewidth}|}
\hline
\hspace{5mm}\small This \pali{ya-ta} structure has only one marker, so I insert the missing term to make it more recognizable. Do not be misled by `this.' In the sentence, \pali{yo} clause correlates with \pali{so} clause. The two words work as relative pronoun, whereas \pali{aya\d m} (this) works as demonstrative pronoun pointing out that it is `this' extreme we have mentioned earlier.\\
\hspace{5mm}\small Traditional explanations of each word in this sentence can be found in the commentary. See Chapter \ref{chap:comcak}.\\
\hspace{5mm}\dag\ \small We have two contrasting pictures here: `common' vs.\ `noble' practice and `austere' vs.\ `noble' practice. By the account, the noble practice is preferable. The word `noble' (\pali{ariya}) here may cause some problem concerning racism or class domination. The translation is a straight one, thus foreignization. When it comes to our context, the noble practice sounds like ``That what gentlemen do.'' Some common people might think ``Why should we care, then?'' The problem is not about the idea the Buddha tried to present, but the word itself, which has certain value laden. So, it is not unbiased when used. Perhaps, modern translators should reconsider the use of `noble' here.\\
\hspace{5mm}\dag\ \small Another line of thought related to `noble,' is the connection between enlightenment and civilization. That is to say, enlightenment or any form of salvation makes sense only in civic societies. It is beyond savage people can think about. Not because of their mental capability, but rather the value of survival overpowers any philosophical digression. The pressing problem is how to survive until tomorrow, not prosperity in the next life. Why birth is painful is also a nonsensical question. This suggests that somehow the concept of salvation in a product of civilization.\\
\hline
\end{longtable}

\stepcounter{sennum}
\arabic{sennum}. \pali{Ete kho, bhikkhave, ubho ante anupagamma majjhim\=a pa\d tipad\=a tath\=agatena abhisambuddh\=a cakkhukara\d n\=i \~n\=a\d nakara\d n\=i upasam\=aya abhi\~n\~n\=aya sambodh\=aya nibb\=an\=aya sa\d mvattati.}\\
\small
\fbox{\fbox{\pali{Ete ubho ante} [R]} \fbox{\pali{anupagamma} [V]} [C1]} \fbox{\pali{kho} [P]} \\\fbox{\pali{bhikkhave} [A]} \fbox{\pali{majjhim\=a} [M]} \fbox{\pali{pa\d tipad\=a} [S]} \\\fbox{\fbox{\pali{tath\=agatena} [R]} \fbox{\pali{abhisambuddh\=a} [V]} [C2]} \\\fbox{\pali{cakkhukara\d n\=i \~n\=a\d nakara\d n\=i upasam\=aya abhi\~n\~n\=aya} [R]} \\\fbox{\pali{sambodh\=aya nibb\=an\=aya} [R]} \fbox{\pali{sa\d mvattati} [V]}\\
\normalsize
$\triangleright$ The practice of moderation, monks, not inclining to these both extremes, having been realized by the Buddha; making the wisdom eye [and] insight [happen]; leads to pacification, supreme knowledge, enlightenment, [and] nirvana.\\
\begin{longtable}[c]{|p{0.9\linewidth}|}
\hline
\hspace{5mm}\small As suggested by the commentaries (Sentence \ref{sen:cakkhu1} and \ref{sen:cakkhu2} in Chapter \ref{chap:comcak}), I translate \pali{cakkhukara\d n\=i} as `making the wisdom eye [happen],' as well as \pali{\~n\=a\d nakara\d n\=i} in the same manner. Often accompanied with dative case, \pali{sa\d mvattati} may be translated more precisely as `is useful for pacification, etc.'\\
\hline
\end{longtable}

\stepcounter{sennum}
\arabic{sennum}. \pali{Katam\=a ca s\=a, bhikkhave, majjhim\=a pa\d tipad\=a tath\=agatena abhisambuddh\=a cakkhukara\d n\=i \~n\=a\d nakara\d n\=i upasam\=aya abhi\~n\~n\=aya sambodh\=aya nibb\=an\=aya sa\d mvattati?}\\
$\triangleright$ What is that practice of moderation, monks, having been realized by the Buddha; making the wisdom eye [and] insight [happen]; leads to pacification, supreme knowledge, enlightenment, [and] nirvana?\\

\stepcounter{sennum}
\arabic{sennum}. \pali{Ayameva ariyo a\d t\d tha\.ngiko maggo, seyyathida\d m -- samm\=adi\d t\d thi samm\=asa\.nkappo samm\=av\=ac\=a samm\=akammanto samm\=a\=aj\=ivo samm\=av\=ay\=amo samm\=asati samm\=asam\=adhi.}\\
\small
\fbox{\pali{Aya\d m eva} [M]} \fbox{\pali{ariyo a\d t\d tha\.ngiko} [M]} \fbox{\pali{maggo} [S]} [\fbox{\pali{hoti} [V]}] \\\fbox{\pali{seyyathida\d m} [P]} \fbox{\pali{samm\=adi\d t\d thi samm\=asa\.nkappo samm\=av\=ac\=a}}$\neg$ \\\fbox{\pali{samm\=akammanto samm\=a\=aj\=ivo samm\=av\=ay\=amo samm\=asati}}$\neg$ \\\fbox{\pali{samm\=asam\=adhi} [M]}\\
\normalsize
$\triangleright$ [It is] this noble eight-fold path, i.e.\ right view, right intention, right speech, right action, right livelihood, right effort, right mindfulness, [and] right concentration.\\

\stepcounter{sennum}
\arabic{sennum}. \pali{Aya\d m kho s\=a, bhikkhave, majjhim\=a pa\d tipad\=a tath\=agatena abhisambuddh\=a cakkhukara\d n\=i \~n\=a\d nakara\d n\=i upasam\=aya abhi\~n\~n\=aya sambodh\=aya nibb\=an\=aya sa\d mvattati.}\\
$\triangleright$ This is that practice of moderation, monks, having realized by the Buddha; \ldots\\

\stepcounter{sennum}
\arabic{sennum}. \pali{Ida\d m kho pana, bhikkhave, dukkha\d m ariyasacca\d m -- j\=atipi dukkh\=a, jar\=api dukkh\=a, by\=adhipi dukkho, mara\d nampi dukkha\d m, appiyehi sampayogo dukkho, piyehi vippayogo dukkho, yampiccha\d m na labhati tampi dukkha\d m -- sa\d mkhittena pa\~ncup\=ad\=anakkhandh\=a dukkh\=a.}\\
\small
\fbox{\pali{Ida\d m} [M]} \fbox{\pali{kho pana} [P]} \fbox{\pali{bhikkhave} [A]} \fbox{\pali{dukkha\d m} [M]} \\\fbox{\pali{ariyasacca\d m} [S]} [\fbox{\pali{hoti} [V]}]; \fbox{\pali{j\=ati pi} [S]} \fbox{\pali{dukkh\=a} [M]} [\fbox{\pali{hoti} [V]}]; \\\fbox{\pali{jar\=a pi} [S]} \fbox{\pali{dukkh\=a} [M]} [\fbox{\pali{hoti} [V]}]; \fbox{\pali{by\=adhi pi} [S]} \fbox{\pali{dukkho} [M]} \\{}[\fbox{\pali{hoti} [V]}]; \fbox{\pali{mara\d na\d m pi} [S]} \fbox{\pali{dukkha\d m} [M]} [\fbox{\pali{hoti} [V]}]; \\\fbox{\pali{appiyehi} [M]} \fbox{\pali{sampayogo} [S]} \fbox{\pali{dukkho} [M]} [\fbox{\pali{hoti} [V]}]; \\\fbox{\pali{piyehi} [M]} \fbox{\pali{vippayogo} [S]} \fbox{\pali{dukkho} [M]} [\fbox{\pali{hoti} [V]}]; \\\fbox{\pali{ya\d m pi} [M]} \fbox{\pali{iccha\d m} [S]} \fbox{\pali{na labhati} [V]} \fbox{\pali{ta\d m pi dukkha\d m} [M]}; \\\fbox{\pali{sa\d mkhittena} [R]} \fbox{\pali{pa\~ncup\=ad\=anakkhandh\=a} [S]} \fbox{\pali{dukkh\=a} [M]} \\{[\fbox{\pali{honti} [V]}]}\\
\normalsize
$\triangleright$ This, monks, [is] the noble truth [of] suffering. Birth [is] suffering, also decay, illness, [and] death. Association with unpleasant [things] [is] suffering, also separation with pleasant [things]. That [one] does not get what one wishes [is] also suffering. In short, the attached five aggregates [are] suffering.\\
\begin{longtable}[c]{|p{0.9\linewidth}|}
\hline
\hspace{5mm}\small In the first part, we can see either \pali{dukkha\d m} or \pali{ariyasacca\d m} as the subject and the other as a modifier, because both agree with \pali{ida\d m} (nt.). If you choose \pali{dukkha\d m} as the subject, the translation should be thus, ``This [is] suffering as a noble truth.'' And this looks more suitable for the whole sentence. I choose \pali{ariyasacca\d m} as the subject to make the fourth truth in agreement (see below).\\
\hspace{5mm}\small In the correlative sentence ``\pali{yampiccha\d m na labhati tampi dukkha\d m},'' \pali{ya\d m-ta\d m} stands for a clause, not a particular word. For more detail, see Chapter \ref{chap:yata}.\\
\hspace{5mm}\dag\ \small To my view, the use of \pali{ariya} together with \pali{sacca} here is a discursive construction. The compound \pali{ariyasacca} was probably coined by the Buddha himself. I assert, again, the concept of the four noble truth was invented to represent what the Buddha had seen. And this is a discursive making that has a great impact lasting for more than two thousand years, comparable to the discourse of platonic idealism.\\
\hspace{5mm}\dag\ \small If we take the word `truth' (\pali{sacca}) seriously, we can see what I try to say. The discourse clearly establishes what should be held as true. Hence, it is `noble' (\pali{ariya}) truth. Suffering (\pali{dukkha}) is one of the truth in this respect. What `truth' really means is in this context far from clear. It by no means suggests that suffering really exists ontologically, because it depends on the experiencer's perspective. A well-trained person can feel less pain in such unpleasant events, for example. And from the nature's perspective, birth and death, etc., have no implication of suffering (in this sense) at all.\\
\hspace{5mm}\dag\ \small That brings us to the common Three Characteristics, i.e.\ \pali{anicca\d m} (impermanence), \pali{dukkha\d m} (unbearableness, or better, unsustainableness), and \pali{anatt\=a} (selflessness). This entails a lot to explain and discuss. Furthermore, suffering is not what we call neither conventional truth nor ultimate truth. See some more discussion in the concluding section.\\
\hline
\end{longtable}

\stepcounter{sennum}
\arabic{sennum}. \pali{Ida\d m kho pana, bhikkhave, [ta\d m] dukkhasamudaya\d m ariyas\-acca\d m -- y\=aya\d m ta\d nh\=a ponobbhavik\=a nandir\=agasahagat\=a tatratatr\=abhinandin\=i, seyyathida\d m -- k\=amata\d nh\=a, bhavata\d nh\=a, vibhavata\d nh\=a.}\\
\small
\fbox{\pali{Ida\d m} [M]} \fbox{\pali{kho pana} [P]} \fbox{\pali{bhikkhave} [A]} [\fbox{\pali{ta\d m} [M]}] \\\fbox{\pali{dukkhasamudaya\d m} [M]} \fbox{\pali{ariyasacca\d m} [S]} [\fbox{\pali{hoti} [V]}]; \\\fbox{\pali{y\=a aya\d m} [M]} \fbox{\pali{ta\d nh\=a} [S]} [\fbox{\pali{hoti} [V]}] \fbox{\pali{ponobbhavik\=a}}$\neg$ \\\fbox{\pali{nandir\=agasahagat\=a tatratatr\=abhinandin\=i} [M]} \fbox{\pali{seyyathida\d m} [P]} \\\fbox{\pali{k\=amata\d nh\=a, bhavata\d nh\=a, vibhavata\d nh\=a} [M]}\\
\normalsize
$\triangleright$ [That], monks, [is] this noble truth of the origin of suffering, which this craving [is] leading to rebirth, endowed with pleasure and lust, seeking for pleasure in that and that [thing], i.e.\ craving for sensual pleasure, craving for existence [and] craving for non-existence.\\

\newpage
\begin{longtable}[c]{|p{0.9\linewidth}|}
\hline
\hspace{5mm}\small In this sentence and the following one, I insert \pali{ta\d m} to make this \pali{ya-ta} structure recognizable. It pairs with \pali{y\=a} in this instance, and \pali{yo} in the next one. In English translation, you may shift the `which' clause to the beginning, resulting in a more familiar arrangement.\\
\hline
\end{longtable}

\stepcounter{sennum}
\arabic{sennum}. \pali{Ida\d m kho pana, bhikkhave, [ta\d m] dukkhanirodha\d m ariyasacca\d m -- yo tass\=ayeva ta\d nh\=aya asesavir\=aganirodho c\=ago pa\d tinissaggo mutti an\=alayo.}\\
\small
\fbox{\pali{Ida\d m} [M]} \fbox{\pali{kho pana} [P]} \fbox{\pali{bhikkhave} [A]} [\fbox{\pali{ta\d m} [M]}] \\\fbox{\pali{dukkhanirodha\d m} [M]} \fbox{\pali{ariyasacca\d m} [S]} [\fbox{\pali{hoti} [V]}]; \\\fbox{\pali{yo tass\=a eva ta\d nh\=aya} [M]} \fbox{\pali{asesavir\=aganirodho} [S]} [\fbox{\pali{hoti} [V]}] \\\fbox{\pali{c\=ago pa\d tinissaggo mutti an\=alayo} [M]}\\
\normalsize
$\triangleright$ [That], monks, [is] this noble truth of the cessation of suffering, which totally-craving-free cessation of that craving [is] abandonment, forsaking, liberation, [and] non-attachment.\\

\stepcounter{sennum}
\arabic{sennum}. \pali{Ida\d m kho pana, bhikkhave, dukkhanirodhag\=amin\=i pa\d tipad\=a ariyasacca\d m -- ayameva ariyo a\d t\d tha\.ngiko maggo, seyyathida\d m -- samm\=adi\d t\d thi \ldots pe\ldots\ samm\=asam\=adhi.}\\
\small
\fbox{\pali{Ida\d m} [M]} \fbox{\pali{kho pana} [P]} \fbox{\pali{bhikkhave} [A]} \fbox{\pali{dukkhanirodhag\=amin\=i}}$\neg$ \\\fbox{\pali{pa\d tipad\=a} [M]} \fbox{\pali{ariyasacca\d m} [S]} [\fbox{\pali{hoti} [V]}]; \fbox{\pali{aya\d m eva} [M]} \\\fbox{\pali{ariyo a\d t\d tha\.ngiko} [M]} \fbox{\pali{maggo} [S]} [\fbox{\pali{hoti} [V]}] \fbox{\pali{seyyathida\d m} [P]} \\\fbox{\pali{samm\=adi\d t\d thi \ldots pe\ldots\ samm\=asam\=adhi} [M]}\\
\normalsize
$\triangleright$ This, monks, [is] the noble truth of the practice leading to the cessation of suffering. This [is] the noble eight-fold path, i.e.\ right view, \ldots, right concentration.\\

\stepcounter{sennum}
\arabic{sennum}. \pali{`Ida\d m dukkha\d m ariyasaccan'ti me, bhikkhave, pubbe ananussutesu dhammesu cakkhu\d m udap\=adi, \~n\=a\d na\d m udap\=adi, pa\~n\~n\=a udap\-\=adi, vijj\=a udap\=adi, \=aloko udap\=adi.}\\
\small
\fbox{\fbox{\pali{Ida\d m} [M]} \fbox{\pali{dukkha\d m ariyasacca\d m} [S]} [\fbox{\pali{hoti} [V]}] [I]} \fbox{\pali{iti} [P]} \\\fbox{\pali{me} [R]} \fbox{\pali{bhikkhave} [A]} \fbox{\pali{pubbe ananussutesu dhammesu} [R]} \\\fbox{\pali{cakkhu\d m} [S]} \fbox{\pali{udap\=adi} [V]}; \fbox{\pali{\~n\=a\d na\d m} [S]} \fbox{\pali{udap\=adi} [V]}; \fbox{\pali{pa\~n\~n\=a} [S]} \\\fbox{\pali{udap\=adi} [V]}; \fbox{\pali{vijj\=a} [S]} \fbox{\pali{udap\=adi} [V]}; \fbox{\pali{\=aloko} [S]} \fbox{\pali{udap\=adi} [V]}\\
\normalsize
$\triangleright$ Monks, vision, insight, wisdom, knowledge, [and] light arose to me in the teaching never heard before thus, ``This [is] suffering as a noble truth.''\\
\begin{longtable}[c]{|p{0.9\linewidth}|}
\hline
\hspace{5mm}\dag\ \small It is quite puzzling if the Buddha had never realized that death and other things mentioned is suffering before his enlightenment. If this is not really meant by the statement, what does it mean then?\\
\hline
\end{longtable}

\stepcounter{sennum}
\arabic{sennum}. \pali{Ta\d m kho pan`ida\d m dukkha\d m ariyasacca\d m pari\~n\~neyyan'ti me, bhikkhave, pubbe \ldots pe\ldots\ udap\=adi.}\\
\small
\fbox{\pali{Ta\d m} [M]} \fbox{\pali{kho pana} [P]} \fbox{\fbox{\pali{ida\d m} [M]} \fbox{\pali{dukkha\d m ariyasacca\d m} [S]}}$\neg$ \\\fbox{\fbox{\pali{pari\~n\~neyya\d m} [M]} [\fbox{\pali{hoti} [V]}] [I]} \fbox{\pali{iti} [P]} \fbox{\pali{bhikkhave} [A]} \\\fbox{\pali{pubbe} [R]} \ldots\\
\normalsize
$\triangleright$ Monks, that vision, insight, wisdom, knowledge, [and] light arose to me in the teaching never heard before thus, ``This suffering as a noble truth should be well understood.''\\

\stepcounter{sennum}
\arabic{sennum}. \pali{Ta\d m kho panida\d m dukkha\d m ariyasacca\d m pari\~n\~n\=atan'ti me, bhikkhave, pubbe \ldots pe\ldots\ udap\=adi.}\\
$\triangleright$ Monks, that vision, insight, \ldots\ thus, ``This suffering as a noble truth had been well understood [by me].''\\

\stepcounter{sennum}
\arabic{sennum}. \pali{Ida\d m dukkhasamudaya\d m ariyasaccan'ti me, bhikkhave, pubbe ananussutesu dhammesu cakkhu\d m udap\=adi, \~n\=a\d na\d m udap\=adi, pa\~n\~n\=a udap\=adi, vijj\=a udap\=adi, \=aloko udap\=adi.}\\
$\triangleright$ Monks, vision, insight, wisdom, knowledge, [and] light arose to me in the teaching never heard before thus, ``This [is] the origin of suffering as a noble truth.''\\

\stepcounter{sennum}
\arabic{sennum}. \pali{Ta\d m kho panida\d m dukkhasamudaya\d m ariyasacca\d m pah\=atabban'ti me, bhikkhave, pubbe \ldots pe\ldots\ udap\=adi.}\\
$\triangleright$ Monks, vision, insight, \ldots\ thus, ``That origin of suffering as a noble truth should be abandoned.''\\

\stepcounter{sennum}
\arabic{sennum}. \pali{Ta\d m kho panida\d m dukkhasamudaya\d m ariyasacca\d m pah\=inan'ti me, bhikkhave, pubbe \ldots pe\ldots\ udap\=adi.}\\
$\triangleright$ Monks, vision, insight, \ldots\ thus, ``That origin of suffering as a noble truth had been abandoned [by me].''\\

\stepcounter{sennum}
\arabic{sennum}. \pali{Ida\d m dukkhanirodha\d m ariyasaccan'ti me, bhikkhave, pubbe ananussutesu dhammesu cakkhu\d m udap\=adi, \~n\=a\d na\d m udap\=adi, pa\~n\~n\=a udap\=adi, vijj\=a udap\=adi, \=aloko udap\=adi.}\\
$\triangleright$ Monks, vision, insight, wisdom, knowledge, [and] light arose to me in the teaching never heard before thus, ``This [is] the cessation of suffering as a noble truth.''\\

\stepcounter{sennum}
\arabic{sennum}. \pali{Ta\d m kho panida\d m dukkhanirodha\d m ariyasacca\d m sacchik\=atabban'ti me, bhikkhave, pubbe \ldots pe\ldots\ udap\=adi.}\\
$\triangleright$ Monks, vision, insight, \ldots\ thus, ``That cessation of suffering as a noble truth should be experienced.''\\

\stepcounter{sennum}
\arabic{sennum}. \pali{Ta\d m kho panida\d m dukkhanirodha\d m ariyasacca\d m sacchikata\-n'ti me, bhikkhave, pubbe \ldots pe\ldots\ udap\=adi.}\\
$\triangleright$ Monks, vision, insight, \ldots\ thus, ``That cessation of suffering as a noble truth had been experienced [by me].''\\

\stepcounter{sennum}
\arabic{sennum}. \pali{Ida\d m dukkhanirodhag\=amin\=i pa\d tipad\=a ariyasaccan'ti me, bhik\-khave, pubbe ananussutesu dhammesu cakkhu\d m udap\=adi, \~n\=a\d na\d m udap\=adi, pa\~n\~n\=a udap\=adi, vijj\=a udap\=adi, \=aloko udap\=adi.}\\
$\triangleright$ Monks, vision, insight, wisdom, knowledge, [and] light arose to me in the teaching never heard before thus, ``This [is] the path leading to the cessation of suffering as a noble truth.''\\

\stepcounter{sennum}
\arabic{sennum}. \pali{Ta\d m kho panida\d m dukkhanirodhag\=amin\=i pa\d tipad\=a ariyasacca\d m bh\=avetabban'ti me, bhikkhave, pubbe \ldots pe\ldots\ udap\=adi.}\\
$\triangleright$ Monks, vision, insight, \ldots\ thus, ``That path leading to the cessation of suffering as a noble truth should be cultivated.''\\

\stepcounter{sennum}
\arabic{sennum}. \pali{Ta\d m kho panida\d m dukkhanirodhag\=amin\=i pa\d tipad\=a ariyasacca\d m bh\=avita'nti me, bhikkhave, pubbe \ldots pe\ldots\ udap\=adi.}\\
$\triangleright$ Monks, vision, insight, \ldots\ thus, ``That path leading to the cessation of suffering as a noble truth had been cultivated [by me].''\\

\stepcounter{sennum}
\arabic{sennum}. \pali{\fbox{\upshape 1} Y\=avak\=iva\~nca me, bhikkhave, imesu cat\=usu ariyasaccesu eva\d m tipariva\d t\d ta\d m dv\=adas\=ak\=ara\d m yath\=abh\=uta\d m \~n\=a\d nadassana\d m na suvisuddha\d m ahosi, \fbox{\upshape 2} neva t\=av\=aha\d m, bhikkhave, sadevake loke sam\=arake sabrahmake sassama\d nabr\=ahma\d niy\=a paj\=aya sadevamanus\-s\=aya `anuttara\d m samm\=asambodhi\d m abhisambuddho'ti pacca\~n\~n\=asi\d m.}\\
\small
\fbox{1} \fbox{\pali{Y\=ava} [R]} \fbox{\pali{k\=iva\d m ca} [P]} \fbox{\pali{me} [R]} \fbox{\pali{bhikkhave} [A]} \\\fbox{\pali{imesu cat\=usu ariyasaccesu} [M]} \fbox{\pali{eva\d m tipariva\d t\d ta\d m} [M]} \\\fbox{\pali{dv\=adas\=ak\=ara\d m yath\=abh\=uta\d m} [M]} \fbox{\pali{\~n\=a\d nadassana\d m} [S]} \\\fbox{\pali{na suvisuddha\d m} [M]} \fbox{\pali{ahosi} [V]}; \\\fbox{2} \fbox{\pali{na eva} [P]} \fbox{\pali{t\=ava} [R]} \fbox{\pali{aha\d m} [S]} \fbox{\pali{bhikkhave} [A]} \\\fbox{\pali{sadevake loke sam\=arake sabrahmake sassama\d nabr\=ahma\d niy\=a}}$\neg$ \\\fbox{\pali{paj\=aya sadevamanuss\=aya} [L]} \\\fbox{\fbox{\pali{anuttara\d m samm\=asambodhi\d m} [R]} \fbox{\pali{abhisambuddho} [V]} [I]} \\\fbox{\pali{iti} [P]} \fbox{\pali{pacca\~n\~n\=asi\d m} [V]}\\
\normalsize
$\triangleright$ \fbox{1} To which extent, monks, the perfect knowledge [of] seeing things as they are, twelves manners [in] three rounds as such, in these four truths, did not happen to me very clearly, \fbox{2} to that extent, monks, I would not acknowledge that [I] attained the unsurpassed highest wisdom, the perfect enlightenment, in the world, with its gods, M\=aras, Brahm\=as, with its population [including] ascetics and brahmans, gods and human beings.\\
\begin{longtable}[c]{|p{0.9\linewidth}|}
\hline
\hspace{5mm}\small Here, \pali{y\=avak\=iva\d m} is equal to just \pali{y\=ava}. In the decomposition part, I separate \pali{y\=ava} from \pali{k\=iva\d m} to make it correlate with \pali{t\=ava} in the latter part. I mark \pali{y\=ava} and \pali{t\=ava} as R because they relate to the verb rather than a particular noun. And it will be more readable, if you change `to which extent' to `as long as' and drop `to that extent.' To new students, however, following my way of rendering can be helpful for understanding the sentence structure more clearly.\\
\hspace{5mm}\small In the \pali{iti} clause near the end, \pali{abhisambuddho} is a verb in \pali{ta} form with \pali{aha\d m} as subject, and \pali{samm\=asambodhi\d m} as its object.\\
\hspace{5mm}\small Another point worth noting here is that some locative terms are treated as subject modifier if they relate to that noun, e.g.\ \pali{imesu cat\=usu ariyasaccesu}. Yet some are treated as locative marker if they relate to the verb, e.g.\ \pali{sadevake loke}, etc.\\
\hspace{5mm}\dag\ \small I render \pali{yath\=abh\=uta\d m \~n\=a\d nadassana\d m}, as most translators do, as `the perfect knowledge [of] seeing things as they are.' To put it literally, \pali{yath\=abh\=uta} means just `being in which way' or `being in such a way.' It can be interpreted in two ways: (a) being in the way it is, and (b) being in the way mentioned. Most Buddhists understand it as (a). From my perspective, however, seeing things as they are is the seeing that conforms to what is stipulated, hence agreeable to the established discourse, thus (b).\\
\hspace{5mm}\dag\ \small In reality, we never, and will never, see things as they really are. Not because we lack the capability, but rather that is what our cognition does. And by very Buddhist position, everything has no substantial state to be seen. Some can argue that that condition is indeed what \pali{yath\=abh\=uta} means. So, seeing by insight is equal to understanding by rationality. Some might also think of an esoteric kind of seeing. That only makes the matter more obscure, hence easily manipulated. So, the phrase ``seeing things as they are'' is ambiguous, possibly misleading and manipulative.\\
\hline
\end{longtable}

\stepcounter{sennum}
\arabic{sennum}. \pali{\fbox{\upshape 1} Yato ca kho me, bhikkhave, imesu cat\=usu ariyasaccesu eva\d m tipariva\d t\d ta\d m dv\=adas\=ak\=ara\d m yath\=abh\=uta\d m \~n\=a\d nadassana\d m suvisuddha\d m ahosi, \fbox{\upshape 2} ath\=aha\d m, bhikkhave, sadevake loke sam\=arake sabrahmake sassama\d nabr\=ahma\d niy\=a paj\=aya sadevamanuss\=aya `anuttara\d m samm\=asambodhi\d m abhisambuddho'ti pacca\~n\~n\=asi\d m.}\\
$\triangleright$ \fbox{1} Because, monks, the perfect knowledge [of] seeing things as they are, twelves manners [in] three rounds as such, in these four truths, happened to me very clearly, \fbox{2} then I, monks, acknowledged that [I] attained the unsurpassed highest wisdom, the perfect enlightenment, in the world, with its gods, M\=aras, Brahm\=as, with its population [including] ascetics and brahmans, gods and human beings.\\

\begin{longtable}[c]{|p{0.9\linewidth}|}
\hline
\hspace{5mm}\small This is a strange case of \pali{ya-ta} structure. Here, \pali{yato} pairs with \pali{atha}. You may translate the pair as `on which/ that account' to keep the structure more visible.\\
\hline
\end{longtable}

\stepcounter{sennum}
\arabic{sennum}. \pali{\~N\=a\d na\~nca pana me dassana\d m udap\=adi -- `akupp\=a me vimutti, ayamantim\=a j\=ati, natthid\=ani punabbhavo'ti.}\\
\small
\fbox{\pali{\~N\=a\d na\d m ca} [S]} \fbox{\pali{pana} [P]} \fbox{\pali{me} [R]} \fbox{\pali{dassana\d m} [S]} \fbox{\pali{udap\=adi} [V]} \\\fbox{\fbox{\pali{akupp\=a} [M]} \fbox{\pali{me} [M]} \fbox{\pali{vimutti} [S]} [\fbox{\pali{hoti} [V]}];} \\\fbox{\fbox{\pali{aya\d m} [M]} \fbox{\pali{antim\=a} [M]} \fbox{\pali{j\=ati} [S]} [\fbox{\pali{hoti} [V]}];} \\\fbox{\fbox{\pali{natthi} [V]} \fbox{\pali{id\=ani} [L]} \fbox{\pali{punabbhavo} [S]} [I]} \fbox{\pali{iti} [P]}\\
\normalsize
$\triangleright$ Knowledge and insight arose to me thus, ``My liberation [is] unshakable; this [is my] final birth; now there is no more becoming.''\\

\begin{longtable}[c]{|p{0.9\linewidth}|}
\hline
\hspace{5mm}\small We can see the first \pali{me} in two ways: dative or genitive case. I choose the former and mark the term as R because it relates to \pali{udap\=adi}. If you choose genitive case, it will be marked as M because it modifies the subject, hence `my knowledge and insight.'\\
\hline
\end{longtable}

\stepcounter{sennum}
\arabic{sennum}. \pali{Idamavoca bhagav\=a. Attaman\=a pa\~ncavaggiy\=a bhikkh\=u bhagavato bh\=asita\d m abhinandun'ti.}\\
\small
\fbox{\pali{Ida\d m} [R]} \fbox{\pali{avoca} [V]} \fbox{\pali{bhagav\=a} [S]}; \fbox{\pali{Attaman\=a pa\~ncavaggiy\=a} [M]} \\\fbox{\pali{bhikkh\=u} [S]} \fbox{\pali{bhagavato bh\=asita\d m} [R]} \fbox{\pali{abhinandu\d m} [V]} \fbox{\pali{iti} [P]}\\
\normalsize
$\triangleright$ The Blessed One said this. The delighted five monks rejoiced [in] the speech of the Blessed One.\\
\begin{longtable}[c]{|p{0.9\linewidth}|}
\hline
\hspace{5mm}\small It seems better to ignore \pali{iti} here, because it just marks the ending this part. We will see such a use several times in due course. Sometimes, it is added without a good reason, I guess.\\
\hline
\end{longtable}

\stepcounter{sennum}
\arabic{sennum}. \pali{Imasmi\~nca pana veyy\=akara\d nasmi\d m bha\~n\~nam\=ane \=ayasmato ko\d n\d da\~n\~nassa viraja\d m v\=itamala\d m dhammacakkhu\d m udap\=adi -- `ya\d m ki\~nci samudayadhamma\d m, sabba\d m ta\d m nirodhadhamman'ti.}\\
\small
\fbox{\fbox{\pali{Imasmi\d m} [M]} \fbox{\pali{ca pana} [P]} \fbox{\pali{veyy\=akara\d nasmi\d m} [S]}} \\\fbox{\fbox{\pali{bha\~n\~nam\=ane} [V]} [C]} \fbox{\pali{\=ayasmato ko\d n\d da\~n\~nassa} [M]} \\\fbox{\pali{viraja\d m v\=itamala\d m} [M]} \fbox{\pali{dhammacakkhu\d m} [S]} \fbox{\pali{udap\=adi} [V]} \\\fbox{\fbox{\pali{ya\d m ki\~nci} [M]} \fbox{\pali{samudayadhamma\d m} [S]} [\fbox{\pali{hoti} [V]}];} \\\fbox{\fbox{\pali{sabba\d m ta\d m} [M]} \fbox{\pali{nirodhadhamma\d m} [S]} [\fbox{\pali{hoti} [V]}] [I]} \fbox{\pali{iti} [P]}\\
\normalsize
$\triangleright$ While this explanation was being said, the Venerable Ko\d n\d da\~n\~na's Dhamma eye arose immaculately thus, ``Whichever [is normally] arising nature, that all [is normally] ceasing nature.''\\
\begin{longtable}[c]{|p{0.9\linewidth}|}
\hline
\hspace{5mm}\small There is an absolute construction in locative here: \pali{imasmi\d m veyy\=akara\d nasmi\d m bha\~n\~nam\=ane}. The construction is in passive structure, marked by \pali{bha\~n\~nam\=ane} (\pali{m\=ana} form of \pali{bh\=asati} + ya).\footnote{The double \pali{\~n\~n} we see here is a result of \pali{ya} operation. For more information, see Chapter \externalref{37} of PNL.}\\
\hspace{5mm}\dag\ \small What does Dhamma eye really mean? I have no idea to put it specifically, so I retain the original word. If I have to explain it anyway, I will put it simply as ``the seeing that conforms to the teaching.'' There is no esoteric meaning to me here. Dealing with \pali{dhamma} in \pali{samudayadhamma\d m} and \pali{nirodhadhamma\d m} is more difficult. The word seems to mean a natural condition. So, the idea what the Venerable understood can be simply put as ``What naturally arises, it naturally ceases.'' Associating \pali{dhamma} with nature has a suggestive intention, because it implies objective or scientific truth, independent of our seeing. We can say that this is what \pali{yath\=abh\=ata} is intended to mean.\\
\hspace{5mm}\dag\ \small Whether things really arise and cease by their nature can be a matter of philosophical discussion. I will demonstrate one line of thought here.\\
\hspace{5mm}\dag\ \small The statement ``What naturally arises, it naturally ceases'' makes sense only when a thing has its own substantial state. For example, on what condition we can say that a rock arises? A small rock might be cracked from a bigger one. We may see that the small rock arises at that point. One day the rock loses its tiny piece. Does it cease or not? As you might realize when we say a thing arises and ceases, we must presuppose there is the `thing' to be the subject of arising and ceasing. If we do not do as such, there will be nothing to talk about. My point here is once a discourse is made it has to be understood in some way, and every point of view is always disputable. What we call natural way of things is also a discursive construction. Asking yourself what is not counted as `natural' may give you some inkling.\\
\hline
\end{longtable}

\stepcounter{sennum}
\arabic{sennum}. \pali{Pavattite ca pana bhagavat\=a dhammacakke bhumm\=a dev\=a saddamanus\-s\=avesu\d m -- `eta\d m bhagavat\=a b\=ar\=a\d nasiya\d m isipatane migad\=aye anuttara\d m dhammacakka\d m pavattita\d m appa\d tivattiya\d m sama\d nena v\=a br\=ahma\d nena v\=a devena v\=a m\=arena v\=a brahmun\=a v\=a kenaci v\=a lokasmin'ti.}\\
\small
\fbox{\fbox{\pali{Pavattite} [V]} \fbox{\pali{ca pana} [P]}}$\neg$ \\\fbox{\fbox{\pali{bhagavat\=a} [R]} \fbox{\pali{dhammacakke} [S]} [C]} \\\fbox{\pali{bhumm\=a dev\=a} [S]} \fbox{\pali{saddamanuss\=avesu\d m} [V]} \\\fbox{\fbox{\pali{eta\d m} [M]} \fbox{\pali{bhagavat\=a} [R]}}$\neg$ \\\fbox{\fbox{\pali{b\=ar\=a\d nasiya\d m isipatane migad\=aye} [L]} \fbox{\pali{anuttara\d m} [M]}}$\neg$ \\\fbox{\fbox{\pali{dhammacakka\d m} [S]} \fbox{\pali{pavattita\d m} [M]} \fbox{\pali{appa\d tivattiya\d m} [M]}}$\neg$ \\\fbox{[\fbox{\pali{hoti} [V]}] \fbox{\pali{sama\d nena v\=a br\=ahma\d nena v\=adevena v\=a}}}$\neg$ \\\fbox{\fbox{\pali{m\=arena v\=a brahmun\=a v\=a kenaci v\=a} [R]} \fbox{\pali{lokasmi\d m} [L]} [I]} \\\fbox{\pali{iti} [P]}\\
\normalsize
$\triangleright$ When the wheel of Dhamma was set forward by the Blessed One, gods on earth announced [this] thus, ``This unsurpassed wheel of Dhamma set forward by the Blessed One in Isipatana Deer Park, B\=ar\=a\d nas\=i, [is] unable to set backward by [any] ascetic or brahman or god or M\=ara or Brahm\=a or anyone in the world.''\\
\begin{longtable}[c]{|p{0.9\linewidth}|}
\hline
\hspace{5mm}\small By its form, \pali{saddamanuss\=aveti} (\pali{sadda\d m + anuss\=aveti}) is in causative structure. Hence, it means precisely `to cause the sound to be heard.'\\
\hline
\end{longtable}

\stepcounter{sennum}
\arabic{sennum}. \pali{Bhumm\=ana\d m dev\=ana\d m sadda\d m sutv\=a c\=atumah\=ar\=ajik\=a dev\=a saddamanuss\=avesu\d m -- `eta\d m bhagavat\=a \ldots pe\ldots\ lokasmin'ti.}\\
\small
\fbox{\fbox{\pali{Bhumm\=ana\d m dev\=ana\d m sadda\d m} [R]} \fbox{\pali{sutv\=a} [V]} [C]} \\\fbox{\pali{c\=atumah\=ar\=ajik\=a dev\=a} [S]} \fbox{\pali{saddamanuss\=avesu\d m} [V]} \ldots\\
\normalsize
$\triangleright$ Having heard the sound of gods on earth, gods on the realm of the four great kings announced [this] thus, \ldots\\

\stepcounter{sennum}
\arabic{sennum}. \pali{C\=atumah\=ar\=ajik\=ana\d m dev\=ana\d m sadda\d m sutv\=a t\=avati\d ms\=a dev\=a \ldots pe\ldots\ y\=am\=a dev\=a \ldots pe\ldots\ tusit\=a dev\=a \ldots pe\ldots\ nimm\=anarat\=i dev\=a \ldots pe\ldots\ paranimmitavasavatt\=i dev\=a \ldots pe\ldots\ brahmak\=ayik\=a dev\=a saddamanus\-s\=avesu\d m -- `eta\d m bhagavat\=a \ldots pe\ldots\ lokasmin'ti.}\\
$\triangleright$ Having heard the sound of gods on the realm of the four great kings, gods on T\=avati\d msa \ldots\ Y\=am\=a \ldots\ Tusita \ldots\ Nimm\=anarat\=i \ldots\ Paranimmitavasavatt\=i \ldots\ [and] gods in Brahm\=a's world announced [this] thus, \ldots\\
\begin{longtable}[c]{|p{0.9\linewidth}|}
\hline
\hspace{5mm}\small This repetitive part is not meant to be informative, like the repetition of the twelve manners above. Rather, it is used for recitation in ritualistic performances nowadays. In Thai chanting books, this part is extended to cover all sixteen heavens of Brahm\=as.\\
\hline
\end{longtable}

\stepcounter{sennum}
\arabic{sennum}. \pali{Itiha tena kha\d nena tena muhuttena y\=ava brahmalok\=a saddo abbhuggacchi.}\\
\small
\fbox{\pali{Itiha} [P]} \fbox{\pali{tena kha\d nena tena muhuttena} [R]} \\\fbox{\pali{y\=ava brahmalok\=a} [R]} \fbox{\pali{saddo} [S]} \fbox{\pali{abbhuggacchi} [V]}\\
\normalsize
$\triangleright$ In such a manner, the sound rose up to the Brahm\=a's worlds by that moment.\\

\begin{longtable}[c]{|p{0.9\linewidth}|}
\hline
\hspace{5mm}\small We can see \pali{ya-ta} structure here as \pali{y\=ava brahmalok\=a [t\=ava] saddo abbhuggacchi} (To which extent the Brahm\=a's worlds [are located], to that extent the sound rose up). Because \pali{kha\d nena} and \pali{muhuttena} has the same meaning, I translate them only once. And I render \pali{itiha} as `in such a manner.' So, the term is more or less equal to just \pali{iti}.\\
\hspace{5mm}\small Here is another example of \pali{itiha}: \pali{Itiha bhagavato pa\d tisa\~ncikkhato appossukkat\=aya citta\d m namati, no dhammadesan\=aya}\footnote{Mv\,1.7} (Having considered in that way, the mind of the Blessed One inclines for inactivity, not for teaching the Dhamma).\\
\hline
\end{longtable}

\stepcounter{sennum}
\arabic{sennum}. \pali{Aya\~nca dasasahassilokadh\=atu sa\.nkampi sampakampi sampavedhi, appam\=a\d no ca u\d l\=aro obh\=aso loke p\=aturahosi atikkamma dev\=ana\d m dev\=anubh\=avan'ti.}\\
\small
\fbox{\pali{Aya\d m} [M]} \fbox{\pali{ca} [P]} \fbox{\pali{dasasahassilokadh\=atu} [S]} \\\fbox{\pali{sa\.nkampi sampakampi sampavedhi} [V]}; \\\fbox{\pali{appam\=a\d no ca u\d l\=aro} [M]} \fbox{\pali{obh\=aso} [S]} \fbox{\pali{loke} [L]} \fbox{\pali{p\=aturahosi} [V]}\\ \fbox{\fbox{\pali{atikkamma} [V]} \fbox{\pali{dev\=ana\d m dev\=anubh\=ava\d m} [R]} [C]}\\
\normalsize
$\triangleright$ This 10,000-world-system vibrated. Infinite brilliant light appeared in the world, surpassing the divine power of gods.\\
\begin{longtable}[c]{|p{0.9\linewidth}|}
\hline
\hspace{5mm}\small We can see a stylistic use of a series of synonyms here, \pali{sa\.nkampi sampakampi sampavedhi}.\\
\hspace{5mm}\dag\ \small The vibration and illumination are better seen as a metaphor---the teaching would shake and illuminate the world in thousand years to come. This makes expressive aspect of the text more conspicuous. If the author also had an intention to mean them literally, it can be seen as insinuation. For believers, the event looks spectacular and significant miraculously. For people with sceptical mind, it can be interpreted figuratively. That is the reason why adding spectacular and dramatic elements in religious texts is so effective.\\
\hline
\end{longtable}

\refstepcounter{sennum}\label{sen:annaasi}
\arabic{sennum}. \pali{Atha kho bhagav\=a ima\d m ud\=ana\d m ud\=anesi -- `a\~n\~n\=asi vata, bho, ko\d n\d da\~n\~no, a\~n\~n\=asi vata, bho, ko\d n\d da\~n\~no'ti!}\\
\small
\fbox{\pali{Atha kho} [P]} \fbox{\pali{bhagav\=a} [S]} \fbox{\pali{ima\d m ud\=ana\d m} [R]} \fbox{\pali{ud\=anesi} [V]}\\ \fbox{\fbox{\pali{a\~n\~n\=asi} [V]} \fbox{\pali{vata} [P]} \fbox{\pali{bho} [A]} \fbox{\pali{ko\d n\d da\~n\~no} [S]} \ldots [I]} \fbox{\pali{iti} [P]}\\
\normalsize
$\triangleright$ Then the Blessed One uttered this exclamation, ``Oh!, friends, Ko\d n\d da\~n\~na understood [it]. Oh!, friends, Ko\d n\d da\~n\~na understood [it].''\\
\begin{longtable}[c]{|p{0.9\linewidth}|}
\hline
\hspace{5mm}\dag\ \small Why does only this part contain a direct speech quote? At surface level, it provides the source of the foremost disciple's name. If we look deeper than that, we will see that this quote asserts effectively, in a dramatic way, that the first sermon is fruitful. There must be someone who discerns the message, and Ko\d n\d da\~n\~na is the best candidate.\\
\hspace{5mm}\dag\ \small In the real situation of the first delivery, there must be a discussion or argumentation between the Buddha and the five monks, not just a one-way preaching. But that dialogue was not recorded, only the exclamation was portrayed.\\
\hspace{5mm}\dag\ \small This can create a picture that the Buddha just gave the monks certain information. When one of them understood the message, the Buddha knew by himself that the delivery was effective. So, he uttered out of getting delighted, or perhaps surprised (as \pali{vata} is used here). This picture rules out the atmosphere of philosophical discussion between the two parties, which I think it is a more realistic scene in that situation, because, as I discuss elsewhere, the content of the sermon, particularly about the four noble truths, is not undoubtedly clear.\\
\hspace{5mm}\dag\ \small In discourse analysis, this can be seen as framing or emphasizing a desirable picture as foreground while sidestepping another picture into background. This technique works so well and naturally here that very few can see and think in other way.\\
\hline
\end{longtable}

\stepcounter{sennum}
\arabic{sennum}. \pali{Iti hida\d m \=ayasmato ko\d n\d da\~n\~nassa `a\~n\~n\=asiko\d n\d da\~n\~no' tveva n\=ama\d m ahos\=i'ti.}\\
\small
\fbox{\pali{Iti hi} [P]} \fbox{\pali{\=ayasmato ko\d n\d da\~n\~nassa} [M]} \\\fbox{\pali{ida\d m `a\~n\~n\=asiko\d n\d da\~n\~no' tu eva} [M]} \fbox{\pali{n\=ama\d m} [S]} \fbox{\pali{ahosi} [V]} \\\fbox{\pali{iti} [P]}\\
\normalsize
$\triangleright$ By this manner, this `A\~n\~n\=asiko\d n\d da\~n\~na' became a name of the Venerable Ko\d n\d da\~n\~na.\\

\stepcounter{sennum}
\arabic{sennum}. \pali{Pa\d thama\d m.}\\
$\triangleright$ The first [was finished].\\

\phantomsection
\addcontentsline{toc}{section}{Conclusion and discussion}
\section*{3.\ Conclusion and discussion}

If the readers follow my reading closely, they should see that the sentence structure of the early texts is not so complicated as we see in commentaries. This confirms that the original form of the text is probably oral, unlike the commentaries which seem to be in written form at the beginning. Composing a complicated text without writing materials is unthinkable to me. So, what is counted as early text should be simple in structure. The only problem with early texts is they have an archaic form of words that may cause the readers a headache when they try to crack its coding. However, for this sutta, the text is relatively easy to read and substantially informative, if we shorten the repetitions.

After we read the sutta we should come up with what the subject matter is. A way to get that is trying to figure out what title should be given to this if we write it as an article. We can end up with different titles depending on our understanding of the text and what we pick up to underline. Here is my version of it: ``The four noble truths and the enlightenment of the Buddha.'' I think this covers what the first sermon is all about.

One discursive aspect of the text, as I have pointed out above, is the postulation of certain reality as the four noble truths. What kind of reality of these truths, if we think it seriously, is far from clear. Peter Harvey renders \pali{ariya-sacca} as ``true reality for the spiritually ennobled.''\footnote{\citealp[p.~51]{harvey:buddhism}} In his view, `truth' is different from `reality,' hence \pali{sacca} is not just `truth' but `true reality.' That sounds weird to me because it implies that `false reality' is thinkable. What does that really mean after all?

This shows that \pali{ariya-sacca} is indeed a problematic term. Those who have an enterprise to explain it have to give a treatment, one way or another. That is the nature of discursive practice. If the explanation looks smooth and eloquent, it can give us an illusion of the transparency of translation.

Concerning suffering as reality, let me illustrate my point in this way. Does suffering described in the sutta cause the Buddha to suffer? The answer has to be `No.' Because enlightened beings are not supposed to suffer from such incidents. But he definitely underwent painful states before he got enlightened. So, suffering exists conditionally. A more sensible explanation of Buddhist reality can be that everything is conditional. However, this is an over-reading here, because it is not said as such in the sutta.

In its simple form, the Buddha just said in the first delivery concerning this matter thus, ``Look!, birth, death, etc., are real. They should be known, and I knew them well.'' Interpreting it in this way brings up further philosophical problems. For example, in what sense exactly birth is real? We regard that birth and death are real as long as certain personhood is expected, otherwise they are meaningless.

One line of thought to treat this problem is to admit that the Buddha was making a discourse, a postulation of things being seen as truth, for teaching purpose only. That means the noble truths are just an instrument. This can be seen as a pragmatic reading of the four noble truths. They are not the ultimate reality independent from us. They are just useful truths discursively constructed. 

Look it in another way, we can see the assertion of the noble truths by the Buddha is equivalent to a \emph{discovery} of scientists at the present time. To summarize, the Buddha confirmed that there are sufferings in the world as we know them, and he discovered that desires are the very causes of those. He also found that the state free from sufferings really exists. To achieve that, he discovered the way and followed it to the end.

It makes more sense here if we see \pali{sacca} as \emph{discovery} (= discovering the nature) like the simple picture described above. However, equating \pali{ariya-sacca} to scientific discovery cannot escape the notion of discourse. As we have known so far, no findings in science are absolutely true. One finding may be closer to the nature than others. Or preferably put, one finding may be more useful than others in certain situations. Bringing Newtonian physics, Einsteinian relativity, and modern quantum physics into consideration can make the picture clearer.
