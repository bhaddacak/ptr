\chapter{Dependent origination}\label{chap:paticca}

It is will be a big miss if we have no oppotunity to read the principle of dependent origination in P\=ali. As we have seen when we read \pali{Anattalakkha\d nasutta}, the notion of nonself is often grasped by a wrong way. The obvious problem of this is that if we (the operating aggregates) are nonself, how come one does an action and gets its result? One remedy of this is to elaborate the account of causation. This account is called \pali{pa\d ticcasamupp\=ada}, or dependent origination.

I select one sutta from Sa\d myuttanik\=aya that covers the formula and its explanation. The sutta is easy to read, but not much to understand. When we go through the text, we will understand why explaining the causation clearly in philosophical terms is very difficult.

\phantomsection
\addcontentsline{toc}{section}{Pre-reading introduction}
\section*{1.\ Pre-reading introduction}

\paragraph*{About the text} The text I choose here is the second (of 10) sutta of the first (of 9) group in the first (of 10) collection of Nid\=anavagga, Sa\d myuttanik\=aya (S2\,2, SN\,12). It is called \pali{Vibha\.nga} (classification) by name.

\paragraph*{About the author} The content was initiated by the Buddha, at least the formula part. For this textual instance, the text sender was the Sangha in an early council. And the text producer was the compilers in that council.

\paragraph*{About the audience} In the content, the Buddha said to unspecified monks. And there is no much detail about the context. This may show that the teaching was delivered many times in various occasions. For the text itself, the target audience is Buddhist monks. Lay Buddhists may find this too technical.
 
\paragraph*{About time and place} The exact time of the event is unknown. The place of the event was in S\=avatth\=i. For the text, it was possibly produced by an early compilation in India.

\paragraph*{About motives} To explain the key terms in \pali{pa\d ticcasamupp\=ada} is a visible motive of this sutta. Another motive can be to assert certain interpretation of the terms.

\paragraph*{About text function} Informative function is marked in this text. In ceremonies, particularly those related to funeral, the formula part is also used in chanting. So, the text can do ceremonial function as well.

\phantomsection
\addcontentsline{toc}{section}{Reading with a draft translation}
\section*{2.\ Reading with a draft translation}

To comfort the readers, I change the format of presentation here by inserting translation and explanation paragraph by paragraph.

\bigskip
\begin{center}
\textbf{\pali{Vibha\.ngasutta\d m}}\par
$\triangleright$ A discourse concerning\\the classification [of dependent origination]
\end{center}

\setcounter{sennum}{0}
\pali{\fbox{\stepcounter{sennum}\arabic{sennum}} S\=avatthiya\d m viharati \ldots pe\ldots\ `pa\d ticcasamupp\=ada\d m vo, bhikkhave, desess\=ami vibhajiss\=ami. Ta\d m su\d n\=atha, s\=adhuka\d m manasi karotha; bh\=asiss\=am\=i'ti. \fbox{\stepcounter{sennum}\arabic{sennum}} `Eva\d m, bhante'ti kho te bhikkh\=u bhagavato paccassosu\d m. Bhagav\=a etadavoca --}

\addtocounter{sennum}{-2}
$\triangleright$ \fbox{\stepcounter{sennum}\arabic{sennum}} [The Buddha] lives in S\=avatth\=i. \ldots\ [Said the Buddha], ``monks, I will teach you the dependent origination, I will classify [it for you]. Listen to that. Think it over carefully. I will say.'' \fbox{\stepcounter{sennum}\arabic{sennum}} ``Yes, sir.'' Those monks answered the Blessed One. [Then] the Blessed One said [as follows]:

\newpage
\begin{longtable}[c]{|p{0.9\linewidth}|}
\hline
\hspace{5mm}\small The use of ellipsis (\ldots pe\ldots) here tells us something. If you examine suttas in this set, you will find that some start with \pali{eva\d m me suta\d m} and no ellipsis, but some start with \pali{s\=avatthiya\d m viharati \ldots}\footnote{In other Sa\d myutta, \pali{s\=avatthinid\=ana\d m} is also used for short introduction.} In very rare cases, such openings are not used, and the suttas start with the content right away. All these forms, therefore, look like templates for making the text in format. Also the incessant use of \pali{bhikkhave}, as we shall see below, can be a part of the template. These are the result of textual redaction of the past.\\
\hspace{5mm}\small My point here is this. The belief that the tradition in its early days tried to keep the original form of the teaching intact should be called into question. The tradition indeed manipulated the texts in order to make them congruent, in form and also possibly in content.\\
\hspace{5mm}\small The early Sangha might have an intention to make the texts in order, and to glorify them. But by this shortsighted vision, the texts had been contaminated accumulatively for more than a thousand years or so, until they were crystallized into the current form.\\
\hspace{5mm}\small Anyway, the texts in our hand today is the best we have. We only know the ancient world through this lens nonetheless. Depending solely on the texts is not enough to understand them. Critical reasoning skill is also indispensable.\\
\hline
\end{longtable}

\pali{\fbox{\stepcounter{sennum}\arabic{sennum}} Katamo ca, bhikkhave, pa\d ticcasamupp\=ado? \fbox{\stepcounter{sennum}\arabic{sennum}} Avijj\=apaccay\=a, bhikkhave, sa\.nkh\=ar\=a; sa\.nkh\=arapaccay\=a vi\~n\~n\=a\d na\d m; vi\~n\~n\=a\d napaccay\=a n\=amar\=upa\d m; n\=amar\=upapaccay\=a sa\d l\=ayatana\d m; sa\d l\=ayatanapaccay\=a phasso; phassapaccay\=a vedan\=a; vedan\=apaccay\=a ta\d nh\=a; ta\d n\-h\=apaccay\=a up\=ad\=ana\d m; up\=ad\=anapaccay\=a bhavo; bhavapaccay\=a j\=ati; j\=atipaccay\=a jar\=amara\d na\d m sokaparidevadukkhadomanassup\=ay\=as\=a \linebreak sambhavanti. \fbox{\stepcounter{sennum}\arabic{sennum}} Evametassa kevalassa dukkhakkhandhassa samudayo hoti.}

\addtocounter{sennum}{-3}
$\triangleright$ \fbox{\stepcounter{sennum}\arabic{sennum}} Monks, what is dependent origination? \fbox{\stepcounter{sennum}\arabic{sennum}} With ignorance as condition, monks, volitional activities [occur]; with volitional activities as condition, consciousness; with consciousness as condition, name-and-form; with name-and-form as condition, the six sense bases; with the six sense bases as condition, contact; with contact as condition, feeling; with feeling as condition, craving; with craving as condition, clinging; with clinging as condition, existence; with existence as condition, birth; with birth as condition, old-age-and-death, sorrow, lamentation, pain, dejection, and misery come to be.\footnote{For all technical terms here, I follow Ven.\,Bhikkhu Bodhi's translation (\citealp[pp.~294--5]{bodhi:reading}).} \fbox{\stepcounter{sennum}\arabic{sennum}} In that way the origin of this whole mass of suffering exists.\\

\pali{\fbox{\stepcounter{sennum}\arabic{sennum}} Katama\~nca, bhikkhave, jar\=amara\d na\d m? \fbox{\stepcounter{sennum}\arabic{sennum}} Y\=a tesa\d m tesa\d m satt\=ana\d m tamhi tamhi sattanik\=aye jar\=a [t\=a] j\=ira\d nat\=a [hoti] kha\d n\d dicca\d m p\=alicca\d m valittacat\=a \=ayuno sa\d mh\=ani indriy\=ana\d m parip\=ako; aya\d m vuccati jar\=a. \fbox{\stepcounter{sennum}\arabic{sennum}} Y\=a tesa\d m tesa\d m satt\=ana\d m tamh\=a tamh\=a sattanik\=ay\=a cuti [t\=a] cavanat\=a [hoti] bhedo antaradh\=ana\d m maccu mara\d na\d m k\=alakiriy\=a khandh\=ana\d m bhedo ka\d levarassa nikkhepo, ida\d m vuccati mara\d na\d m. \fbox{\stepcounter{sennum}\arabic{sennum}} Iti aya\~nca jar\=a, ida\~nca mara\d na\d m. Ida\d m vuccati, bhikkhave, jar\=amara\d na\d m.}

\addtocounter{sennum}{-4}
$\triangleright$ \fbox{\stepcounter{sennum}\arabic{sennum}} Monks, what is old-age-and-death? \fbox{\stepcounter{sennum}\arabic{sennum}} Which decay in each class of those beings, [that is] the state of decaying, [as well as] breaking [of teeth], greyness of hair, wrinkled skin, decreasing of vitality, [and] overripeness of faculties. This is called `old-age.' \fbox{\stepcounter{sennum}\arabic{sennum}} Which passing away from each class of those beings, [that is] the state of shifting, [as well as] decomposition, disappearance, death, disunion of the [five] aggregates, [and] abandonment of the body. This is called `death.' \fbox{\stepcounter{sennum}\arabic{sennum}} Thus, this is old-age, and this is death. Monks, this is called `old-age-and-death.'\\

\begin{longtable}[c]{|p{0.9\linewidth}|}
\hline
\hspace{5mm}\small The use of repetition here, i.e.\ \pali{tesa\d m tesa\d m, tamhi tamhi,} and \pali{tamh\=a tamh\=a}, denotes distributiveness (see Chapter \externalref{28} of PNL for more detail). So, they are equivalent to `each' in English.\\
\hspace{5mm}\small Some words are just synonymous to each other, such as \pali{maccu mara\d na\d m k\=alakiriy\=a} (death). Only one English word is used for them.\\
\hspace{5mm}\small The use of \pali{ya-ta} structure in this passage is not clear. My insertion of \pali{t\=a}s is a reasonable guess, otherwise we can drop \pali{y\=a} altogether without any serious effect.\\
\hline
\end{longtable}

\pali{\fbox{\stepcounter{sennum}\arabic{sennum}} Katam\=a ca, bhikkhave, j\=ati? \fbox{\stepcounter{sennum}\arabic{sennum}} Y\=a tesa\d m tesa\d m satt\=ana\d m tamhi tamhi sattanik\=aye j\=ati [t\=a] sa\~nj\=ati [hoti] okkanti nibbatti abhinibbatti khandh\=ana\d m p\=atubh\=avo \=ayatan\=ana\d m pa\d til\=abho. \fbox{\stepcounter{sennum}\arabic{sennum}} Aya\d m vuccati, bhikkhave, j\=ati.}

\addtocounter{sennum}{-3}
$\triangleright$ \fbox{\stepcounter{sennum}\arabic{sennum}} Monks, what is birth? \fbox{\stepcounter{sennum}\arabic{sennum}} Which birth in each class of those beings, [that is] the origin [of life], [as well as] coming into being, coming forth, becoming, appearance of the [five] aggregates, [and] obtaining of the six sense bases. \fbox{\stepcounter{sennum}\arabic{sennum}} Monks, this is called `birth.'\\

\begin{longtable}[c]{|p{0.9\linewidth}|}
\hline
\hspace{5mm}\small In the commentary of this sutta (Srp2\,2, SN-a\,12), there is an attempt to differentiate synonyms used here, sometimes confusingly. For example, \pali{j\=ati} is different from \pali{sa\~nj\=ati} in the way that the latter is a birth with complete sense bases, whereas the former is not (\pali{j\=ayana\d t\d thena j\=ati, s\=a aparipu\d n\d n\=ayatanavasena yutt\=a. Sa\~nj\=ayana\d t\d thena sa\~nj\=ati, s\=a paripu\d n\d n\=ayatanavasena yutt\=a}).\\
\hspace{5mm}\small Different by type of birth, \pali{okkanti} means a birth in egg and in womb (\pali{a\d n\d dajajal\=abujavasena yutt\=a}), and \pali{abhinibbatti} means a birth in moisture (microorganism?) and an instant birth [of gods, for example] (\pali{sa\d msedajaopap\=atikavasena yutt\=a}). I suppose that \pali{nibbatti} is just a general term of birth, for the term has no further explanation.\\
\hline
\end{longtable}

\pali{\fbox{\stepcounter{sennum}\arabic{sennum}} Katamo ca, bhikkhave, bhavo? \fbox{\stepcounter{sennum}\arabic{sennum}} Tayo me, bhikkhave, bhav\=a -- k\=amabhavo, r\=upabhavo, ar\=upabhavo. \fbox{\stepcounter{sennum}\arabic{sennum}} Aya\d m vuccati, bhikkhave, bhavo.}

\addtocounter{sennum}{-3}
$\triangleright$ \fbox{\stepcounter{sennum}\arabic{sennum}} Monks, what is existence? \fbox{\stepcounter{sennum}\arabic{sennum}} Monks, these [are] three existences: existence in the realm of pleasure, existence in the realm of form, [and] existence in the realm of formless beings. \fbox{\stepcounter{sennum}\arabic{sennum}} Monks, this is called `existence.'\\

\begin{longtable}[c]{|p{0.9\linewidth}|}
\hline
\hspace{5mm}\small The explanation of existence is disappointing, both here and in the commentary. It basically means realms of existence which are a part of Buddhist cosmology. In this wheel of causation, existence precedes birth. So, it is not `existence' in the sense used in existentialism---when someone is born, he or she exists in some way. Rather, it is a realm suitable for a person to be born into.\\
\hline
\end{longtable}

\pali{\fbox{\stepcounter{sennum}\arabic{sennum}} Katama\~nca, bhikkhave, up\=ad\=ana\d m? \fbox{\stepcounter{sennum}\arabic{sennum}} Catt\=arim\=ani, bhik\-khave, up\=ad\=an\=ani -- k\=amup\=ad\=ana\d m, di\d t\d thup\=ad\=ana\d m, s\=ilabbatup\=ad\=ana\d m, attav\=adup\=ad\=ana\d m. \fbox{\stepcounter{sennum}\arabic{sennum}} Ida\d m vuccati, bhikkhave, up\=ad\=ana\d m.}

\addtocounter{sennum}{-3}
$\triangleright$ \fbox{\stepcounter{sennum}\arabic{sennum}} Monks, what is clinging? \fbox{\stepcounter{sennum}\arabic{sennum}} Monks, [they are] four clingings: clinging to sensual pleasure, clinging to views, clinging to ritualistic observance, [and] clinging to the self. \fbox{\stepcounter{sennum}\arabic{sennum}} Monks, this is called `clinging.'\\

\begin{longtable}[c]{|p{0.9\linewidth}|}
\hline
\hspace{5mm}\small The first clinging is clear. The second one has its object as a view or belief. An example given by the commentary is the view that ``The self is eternal, so is the world. This is the only truth. Other view is empty'' (\pali{sassato att\=a ca loko ca, idameva sacca\d m moghama\~n\~na\d m}).\footnote{D3\,191 (DN\,29); M3\,27 (MN\,102)} Clinging to such a view is \pali{di\d t\d thup\=ad\=ana}.\\
\hspace{5mm}\small Clinging to ritualistic observance is exemplified in the commentary by the belief that practicing like a cow makes one purified (\pali{gos\=ilagovat\=ad\=ini hi eva\d m suddhi}). Clinging to the self is explained by the commentary as the grasping that only ones's words or position is true or only the self is true (\pali{attav\=adamattameva v\=a att\=ati up\=adiyanti}).\\
\hspace{5mm}\small After seeing this explanation, I have an inkling that the account did not come directly from the Buddha. It was an attempt of the compliers to match the terms with any explanation found elsewhere seemingly fitted. So, it looks out of context. It is better to mean just clinging, holding fast, to anything, not just these four.\\
\hline
\end{longtable}

\pali{\fbox{\stepcounter{sennum}\arabic{sennum}} Katam\=a ca, bhikkhave, ta\d nh\=a? \fbox{\stepcounter{sennum}\arabic{sennum}} Chayime, bhikkhave, ta\d nh\=ak\=ay\=a -- r\=upata\d nh\=a, saddata\d nh\=a, gandhata\d nh\=a, rasata\d nh\=a, pho\-\d t\d thabbata\d nh\=a, dhammata\d nh\=a. \fbox{\stepcounter{sennum}\arabic{sennum}} Aya\d m vuccati, bhikkhave, ta\d nh\=a.}

\addtocounter{sennum}{-3}
$\triangleright$ \fbox{\stepcounter{sennum}\arabic{sennum}} Monks, what is craving? \fbox{\stepcounter{sennum}\arabic{sennum}} Monks, [they are] these six groups of craving: craving for image, craving for sound, craving for smell, craving for taste, craving for touch, [and] craving for mental object. \fbox{\stepcounter{sennum}\arabic{sennum}} Monks, this is called `craving.'\\

\pali{\fbox{\stepcounter{sennum}\arabic{sennum}} Katam\=a ca, bhikkhave, vedan\=a? \fbox{\stepcounter{sennum}\arabic{sennum}} Chayime, bhikkhave, vedan\=ak\=ay\=a -- cakkhusamphassaj\=a vedan\=a, sotasamphassaj\=a vedan\=a, gh\=anasamphassaj\=a vedan\=a, jivh\=asamphassaj\=a vedan\=a, k\=ayasamphassaj\=a vedan\=a, manosamphassaj\=a vedan\=a. \fbox{\stepcounter{sennum}\arabic{sennum}} Aya\d m vuccati, bhikkhave, vedan\=a.}

\addtocounter{sennum}{-3}
$\triangleright$ \fbox{\stepcounter{sennum}\arabic{sennum}} Monks, what is feeling? \fbox{\stepcounter{sennum}\arabic{sennum}} Monks, [they are] these six groups of feeling: feeling arising from contact of eyes, ears, nose, tongue, body, [and] mind. \fbox{\stepcounter{sennum}\arabic{sennum}} Monks, this is called `feeling.'\\

\pali{\fbox{\stepcounter{sennum}\arabic{sennum}} Katamo ca, bhikkhave, phasso? \fbox{\stepcounter{sennum}\arabic{sennum}} Chayime, bhikkhave, phassak\=ay\=a -- cakkhusamphasso, sotasamphasso, gh\=anasamphasso, jivh\=asamphasso, k\=ayasamphasso, manosamphasso. \fbox{\stepcounter{sennum}\arabic{sennum}} Aya\d m vuccati, bhikkhave, phasso.}

\addtocounter{sennum}{-3}
$\triangleright$ \fbox{\stepcounter{sennum}\arabic{sennum}} Monks, what is contact? \fbox{\stepcounter{sennum}\arabic{sennum}} Monks, [they are] these six groups of contact: contact of eyes, ears, nose, tongue, body, [and] mind. \fbox{\stepcounter{sennum}\arabic{sennum}} Monks, this is called `contact.'\\

\pali{\fbox{\stepcounter{sennum}\arabic{sennum}} Katama\~nca, bhikkhave, sa\d l\=ayatana\d m? \fbox{\stepcounter{sennum}\arabic{sennum}} Cakkh\=ayatana\d m, sot\=ayatana\d m, gh\=an\=ayatana\d m, jivh\=ayatana\d m, k\=ay\=ayatana\d m, man\=ayatana\d m -- ida\d m vuccati, bhikkhave, sa\d l\=ayatana\d m.}

\addtocounter{sennum}{-2}
$\triangleright$ \fbox{\stepcounter{sennum}\arabic{sennum}} Monks, what is six sense bases? \fbox{\stepcounter{sennum}\arabic{sennum}} Eyes, ears, nose, tongue, body, [and] mind -- this, monks, is called six sense bases.

\pali{\fbox{\stepcounter{sennum}\arabic{sennum}} Katama\~nca, bhikkhave, n\=amar\=upa\d m? \fbox{\stepcounter{sennum}\arabic{sennum}} Vedan\=a, sa\~n\~n\=a, cetan\=a, phasso, manasik\=aro -- ida\d m vuccati n\=ama\d m. \fbox{\stepcounter{sennum}\arabic{sennum}} Catt\=aro ca mah\=abh\=ut\=a, catunna\~nca mah\=abh\=ut\=ana\d m up\=ad\=ayar\=upa\d m. Ida\d m vuccati r\=upa\d m. \fbox{\stepcounter{sennum}\arabic{sennum}} Iti ida\~nca n\=ama\d m, ida\~nca r\=upa\d m. \fbox{\stepcounter{sennum}\arabic{sennum}} Ida\d m vuccati, bhikkhave, n\=amar\=upa\d m.}

\addtocounter{sennum}{-5}
$\triangleright$ \fbox{\stepcounter{sennum}\arabic{sennum}} Monks, what is name-and-form? \fbox{\stepcounter{sennum}\arabic{sennum}} Feeling, perception, volition, contact, [and] ideation -- this is called `name.' \fbox{\stepcounter{sennum}\arabic{sennum}} The great four elements and forms depending on these great four elements -- this is called `form.' \fbox{\stepcounter{sennum}\arabic{sennum}} Thus, this is name, this is form. \fbox{\stepcounter{sennum}\arabic{sennum}} Monks, this is called `name-and-form.'\\

\begin{longtable}[c]{|p{0.9\linewidth}|}
\hline
\hspace{5mm}\small The same account is also found in \pali{Pa\d tisambhid\=amagga}.\footnote{Psm\,1:170} And \pali{Samm\=adi\d t\d thisutta} in Majjhimanik\=a has a retelling of this.\footnote{M1\,100 (MN\,9). By its arrangement, this sutta looks not so old as others. It is like a later compilation of existing materials.} So, it has a scent of later development. Interestingly, in Abhidhammattasa\.ngaha of Anuruddha (around 10th/11th century), seven common mental properties, namely \pali{phassa, vedan\=a, sa\~n\~n\=a, cetan\=a, ekaggat\=a, j\=ivitindr\=i,} and \pali{manasik\=ara}, are listed.\footnote{Abhidhammattasa\.ngaha 2.2. This book is a part of subcommentaries in the collection.} Five out of seven are in our list.\\
\hspace{5mm}\small In other places, \pali{n\=amar\=upa} means simply the five aggregates.\footnote{\pali{N\=amanti catt\=aro ar\=upino khandh\=a. R\=upanti catt\=aro ca mah\=abh\=ut\=a, catunna\~nca mah\=abh\=ut\=ana\d m up\=ad\=aya r\=upa\d m} (Nidd1\,185).} In the Abhidhamma, \pali{n\=ama} refers to only three mental aggregates, except consciousness.\footnote{Abhidhamma, Vibha\.nga 6.228} The account here looks more elaborate. But whether this is the intented meaning can be a matter of dispute.\\
\hline
\end{longtable}

\pali{\fbox{\stepcounter{sennum}\arabic{sennum}} Katama\~nca, bhikkhave, vi\~n\~n\=a\d na\d m? \fbox{\stepcounter{sennum}\arabic{sennum}} Chayime, bhikkhave, vi\~n\~n\=a\d nak\=ay\=a -- cakkhuvi\~n\~n\=a\d na\d m, sotavi\~n\~n\=a\d na\d m, gh\=anavi\~n\~n\=a\d na\d m, jivh\=avi\~n\~n\=a\d na\d m, k\=ayavi\~n\~n\=a\d na\d m, manovi\~n\~n\=a\d na\d m. \fbox{\stepcounter{sennum}\arabic{sennum}} Ida\d m vuccati, bhikkhave, vi\~n\~n\=a\d na\d m.}

\addtocounter{sennum}{-3}
$\triangleright$ \fbox{\stepcounter{sennum}\arabic{sennum}} Monks, what is consciousness? \fbox{\stepcounter{sennum}\arabic{sennum}} Monks, [they are] these six groups of consciousness: consciousness of eyes, ears, nose, tongue, body, [and] mind. \fbox{\stepcounter{sennum}\arabic{sennum}} Monks, this is called `consciousness.'\\

\pali{\fbox{\stepcounter{sennum}\arabic{sennum}} Katame ca, bhikkhave, sa\.nkh\=ar\=a? \fbox{\stepcounter{sennum}\arabic{sennum}} Tayome, bhikkhave, sa\.nkh\=ar\=a -- k\=ayasa\.nkh\=aro, vac\=isa\.nkh\=aro, cittasa\.nkh\=aro. \fbox{\stepcounter{sennum}\arabic{sennum}} Ime vuccanti, bhikkhave, sa\.nkh\=ar\=a.}

\addtocounter{sennum}{-3}
$\triangleright$ \fbox{\stepcounter{sennum}\arabic{sennum}} Monks, what are volitional activities? \fbox{\stepcounter{sennum}\arabic{sennum}} Monks, [they are] these three volitional activities: volitional activities of the body, of speech, [and] of thought.\footnote{In the Abhidhamma, this is explained as \pali{pu\~n\~n\=abhisa\.nkh\=aro, apu\~n\~n\=abhisa\.nkh\=aro, \=ane\~nj\=abhisa\.nkh\=aro, k\=ayasa\.nkh\=aro, vac\=isa\.nkh\=aro,} and \pali{cittasa\.nkh\=aro} (Vibha\.nga 6.226).} \fbox{\stepcounter{sennum}\arabic{sennum}} Monks, these are called `volitional activities.'\\

\pali{\fbox{\stepcounter{sennum}\arabic{sennum}} Katam\=a ca, bhikkhave, avijj\=a? \fbox{\stepcounter{sennum}\arabic{sennum}} Ya\d m kho, bhikkhave, dukkhe a\~n\~n\=a\d na\d m, dukkhasamudaye a\~n\~n\=a\d na\d m, dukkhanirodhe a\~n\~n\-\=a\d na\d m, dukkhanirodhag\=aminiy\=a pa\d tipad\=aya a\~n\~n\=a\d na\d m. Aya\d m vuccati, bhikkhave, avijj\=a.}

\addtocounter{sennum}{-3}
$\triangleright$ \fbox{\stepcounter{sennum}\arabic{sennum}} Monks, what is ignorance? \fbox{\stepcounter{sennum}\arabic{sennum}} Monks, which [is called] the ignorance in suffering, in the origin of suffering, in the cessation of suffering, [and] in the path leading to the cessation of suffering, [that] monks, this is called `ignorance.'\\

\pali{\fbox{\stepcounter{sennum}\arabic{sennum}} Iti kho, bhikkhave, avijj\=apaccay\=a sa\.nkh\=ar\=a; sa\.nkh\=arapaccay\=a vi\~n\~n\=a\d na\d m \ldots pe\ldots \ evametassa kevalassa dukkhakkhandhassa samudayo hoti.} 

\addtocounter{sennum}{-1}
$\triangleright$ \fbox{\stepcounter{sennum}\arabic{sennum}} Thus, monks, with ignorance as condition, volitional activities [occur]; \ldots\\

\pali{\fbox{\stepcounter{sennum}\arabic{sennum}} Avijj\=aya tveva asesavir\=aganirodh\=a sa\.nkh\=aranirodho; sa\.nkh\=a\-ranirodh\=a vi\~n\~n\=a\d nanirodho; vi\~n\~n\=a\d nanirodh\=a n\=amar\=upanirodho; n\=amar\=upanirodh\=a sa\d l\=ayatananirodho; sa\d l\=ayatananirodh\=a phassaniro\-dho; phassanirodh\=a vedan\=anirodho; vedan\=anirodh\=a ta\d nh\=anirodho; ta\d nh\=anirodh\=a up\=ad\=ananirodho; up\=ad\=ananirodh\=a bhavanirodho; bhavanirodh\=a j\=atinirodho; j\=atinirodh\=a jar\=amara\d na\d m sokaparidevaduk\-khadomanassup\=ay\=as\=a nirujjhanti. \fbox{\stepcounter{sennum}\arabic{sennum}} Evametassa kevalassa duk\-khakkhandhassa nirodho hot\=i'ti.}

\addtocounter{sennum}{-2}
$\triangleright$ \fbox{\stepcounter{sennum}\arabic{sennum}} Because of the cessation of ignorance [by getting rid of] desire without remainder, cessation of volitional activities [occurs]; because of the cessation of volitional activities, cessation of consciousness; because of the cessation of consciousness, cessation of name-and-form; because of the cessation of name-and-form, cessation of the six sense bases; because of the cessation of the six sense bases, cessation of contact; because of the cessation of contact, cessation of feeling; because of the cessation of feeling, cessation of craving; because of the cessation of craving, cessation of clinging; because of the cessation of clinging, cessation of existence; because of the cessation of existence, cessation of birth; because of the cessation of birth, old-age-and-death, sorrow, lamentation, pain, dejection, and misery cease. \fbox{\stepcounter{sennum}\arabic{sennum}} In that way the cessation of this whole mass of suffering occurs.\\

\pali{\fbox{\stepcounter{sennum}\arabic{sennum}} Dutiya\d m.}

\addtocounter{sennum}{-1}
$\triangleright$ \fbox{\stepcounter{sennum}\arabic{sennum}} The second [was finished].\\

\phantomsection
\addcontentsline{toc}{section}{Conclusion and discussion}
\section*{3.\ Conclusion and discussion}

After we read this sutta and looked closely to its form and relevancy, we have a reasonable doubt that the classifications within the sutta unlikely came from the Buddha by this very purpose. It looks more plausible that the accounts were compiled later from various sources to explain the formula. Perhaps, this may be one of the first generation commentaries which was added to the canon in the early time.

As we have seen, the text itself is quite easy to translate. But with many of technical terms involved, the very idea of dependent origination is difficult to understand clearly. Even Buddhaghosa said it is not easy to explain dependent origination.\footnote{\pali{na sukar\=a pa\d ticcasamupp\=adassatthava\d n\d nan\=a} (Vism\,17.581)} The difficulty is confirmed in the canon by attributing the subject as unfathomably profound.\footnote{In S2\,60 (SN\,12), when Ven.\,\=Ananda said that dependent origination appears to him as shallow (\pali{utt\=anakutt\=anako viya kh\=ayati}), the Buddha disagreed and insisted that it is indeed profound (\pali{gambh\=iro}).} However, the formula has to be explained one way or another, because of its importance. Hence, we have seen many ways of interpretation of dependent origination so far, including Tibetan graphical illustrations as we often see.

Why is dependent origination so important? First, it appears many times in the canon with various arrangements. By its standard form, the Buddha pondered upon the idea after he got enlightened.\footnote{Mv\,1.1} It can be seen as the detailed version of the four noble truths, especially the second and the third ones, suitable to competent listeners. Second, it can address the pressing question of how nonself can be reborn and many metaphysical problems. Third, it looks, in a way, scientific by illustrating a chain of causation. The formula itself also looks systematic in a charming way. This last point can uplift feeling of the adherents by making them think that Buddhism really has something substantial and `real' comparable to natural science.

Now I will not tell you what dependent origination is all about. One reason of this is I cannot pretend that I understand it all. Readers should find explanations by their own. Most books on Buddhist philosophy have an explanation of it. Visuddhimagga may be a good place to start with, and a decent translation of it is easy to find. I will just make some remarks after we read the P\=ali text of it.

\paragraph*{1) It is not clear at what level dependent origination explains.} Here are possible answers I can think of.
\begin{compactenum}[(1)]
\item Dependent origination explains the origin of the world in general, not of an individual. It looks like a Buddhist version of the Book of Genesis. The main idea of this is worldlings get born into the world because they lack the vision of reality. There is no direct support to this view in the canon, but it makes some sense and is worth pondering about. I call this \emph{world-level} explanation.
\item Dependent origination explains the connection between three lifetimes of an individual. We can cut the twelve factors into three portions, denoting conditions in the past, present, and future. There are two possible schemes: (1) ignorance and volitional activities (past), consciousness to existence (present), birth and death (future); (2) craving to existence (past), birth to volitional activities (present), consciousness to feeling (future). This is the explanation the tradition prefers, endorsed by the sutta we are reading now, as well as by Visuddhimagga. I call this \emph{person-level} explanation.
\item Dependent origination explains only when an experience occurs. It goes like this. We all are endowed with ignorance that conditions us when we respond to the world. When we have a sensation, the process is kicked off, and we likely end up with suffering. This explanation sounds scientific, and probably what the Buddha really meant to. But this view is mentioned less in the canon.\footnote{There are a handful of suttas that can be read in this way, for example, S2\,38 (SN\,12). In the Abhidhamma, we can see this line of account in Vibha\.nga 6.248.} I call this \emph{mind-level} explanation.
\end{compactenum}

Every level mentioned above has its own problem. The first view proposes a grand theory which is unlikely able to survive rigorous or scientific scrutiny. The second view is metaphysical. It posits what cannot be verified and tested. More seriously, it involves in some way the notion of personhood who gets born and dies in the process.\footnote{Many Buddhist scholars may argue relentlessly that dependent origination dismantles the notion of self, and there is no such thing in the process. However, when we talk about the continuity of something, there must be certain identity that holds the sameness. This is at heart a metaphysical problem which is really hard to defend.} The last view is promising in modern days, but its support is not predominant, as we have seen in the majority of suttas explaining the subject, as well as the attitude of the tradition itself.\footnote{We can understand this in socio-economical terms. Any explanation involving karma and rebirth is more lucrative and government-friendly than intellectual elucidations. What makes the Sangha and the state survive within the doctrinal boundary should be promoted.}

\paragraph*{2) Dependent origination is not causation in strict sense.} By using \pali{paccaya} in this context, it can mean a prior factor contributes to the happening of the successive one in any fashion, not just a direct or a lone cause.\footnote{In the last part of the Abhidhamma, called \pali{Pa\d t\d th\=ana}, systematic expositions of various casual relations (24 types with numerous combinations) are elaborated. This part occupies around half the content of the whole Abhidhamma. Unfortunately, \pali{Pa\d t\d th\=ana} is incomprehensible to most of us.} And as far as I am concerned, relations of each factor to one another are not direct cause. For example, ignorance (\pali{avijj\=a}) is by no means a direct or lone cause of volitional activities (\pali{sa\.nkh\=ar\=a}).\footnote{In some other translation, \pali{sa\.nkh\=ar\=a} means dispositions. This can give us a slightly different picture.} It can be a condition at best. That means volitional activities are also conditioned by other things else, which are unmentioned here, for example, an external or internal stimulant. Ignorance itself has no causal efficacy. An activity must be triggered by something else, not ignorance alone. To illustrate, no one is not urged to kill somebody because just he has ignorance. Only with certain instigation, those who have ignorance are prone to harm others.

Think it further, if ignorance is a condition of volitional activities, what kind of condition is it: necessary or sufficient? If we put this relation into a logical proposition, we get this: If ignorance is present, then volitional activities are present (if p then q). Logically speaking, this means ignorance is sufficient condition of volitional activities.\footnote{By the presence of sufficient condition, the result is assured. By the absence of necessary condition, the result is prevented.} Unpacking this further we get that, according to person-level explanation mentioned above, we all are born with ignorance, so everything we do is volitional activities. However, sometimes we can have a volition to do things without ignorance, say, a selfless donation, for instance.\footnote{In logical terms, if volitional activities are present, they do not necessarily come from ignorance (q does not imply p in if p then q).}

\paragraph*{3) Dependent origination resists logical analysis.} Any attempt to explain the idea with logical analysis dooms to failure. Or, put it another way, dependent origination is not logically sound. We cannot understand it with reasoning, so to speak. The main reason of this is the Buddha was not indeed interested in logical argument in a systematic way.\footnote{When we talk about Buddhist logic, mostly it means the work formulated by Dign\=aga (c.\,480--540) onwards. There is no such thing as early Buddhist logic. This does not mean the Buddha did not use reason. He did but in a simple, down-to-earth, practical way, which is inevitably illogical sometimes.} Some Buddhists might see this as a sign of superiority over modern science, but this in fact causes a serious problem. Before, we go to what kind of problem it entails. Let us see some examples of my logical analysis first.

\begin{compactenum}[(1)]
\item Ignorance is the state of lacking wisdom. So, we can put it as the lack of wisdom causes volitional activities (if not p then q). This also means for those who have no volitional activities, they also have wisdom (if not q then p). This is in line with the tradition's position which holds that enlightened ones have no volition, because they do not produce karma anymore.\footnote{In A6\,63, the Buddha said thus, ``\pali{Cetan\=aha\d m, bhikkhave, kamma\d m vad\=ami. Cetayitv\=a kamma\d m karoti -- k\=ayena v\=ac\=aya manas\=a.}'' This means what is counted as karma has to come from volition.} So, arhants do only mere action (\pali{kiriy\=a}), from Abhidhammic account. A kind of automaton may come to our mind.\par
\hspace{5mm}How do we understand this in modern context? The problem of this issue is in fact linguistic one. There is no clear definition of \pali{sa\.nkh\=ar\=a} because it can mean many things at the same time. In this context, even if it implies \pali{cetan\=a}, it cannot mean exactly `volition' as we use nowadays. That reduces an arhant to just a robot.
\item As described elsewhere, consciousness and name-and-form are in fact mutually causal.\footnote{D2\,96 (DN\,15)} This condition complicates the casual chain to the point that it cannot be understood, let alone explained in terms of logic. Another point concerning name-and-form is that the six sense bases, contact, and feeling are already included here.
\item When someone gets born, he or she has to fall ill and die, as the link between \pali{j\=ati} and \pali{jar\=amara\d na} illustrates. Strictly speaking, however, birth does not cause death and illness, or even is an immediate condition of them. It sounds likes I have suffering because of the Big Bang. It is true logically, but uninformatively and uselessly. By a matter of fact, a common cause of birth, illness, and death is that we are a biological organism subject to natural law.\footnote{Perhaps, I guess, this is what \pali{bhava} (existence) really means. When we are determined to be born into certain form, we are then subject to the law governing that existence. This is just my speculation, like many other teachers try to propose their own interpretation. It sounds understandable, but whether it is really the case can be disputable.} Still, the belief that we all suffer because we can't help getting born is entrenched in the mind of fundamentalist Buddhists.
\end{compactenum}

If you are a logician, you can find fallecies at any point in these causal relations. But we should stop here, and try to think about it seriously. Even if many logical flaws are located, the tradition maintains that dependent origination is very profound beyond capability of normal people to understand it. Only those who get enlightened can do. Forget it, if you use reasoning. It is alright if you ordinary folks cannot take it, but just accept it as true. I think this is the real situation concerning the subject in Buddhist cultures today.

Here comes the serious problem. If we deny the role of reason to understand this matter, we likely deny it as well in other obscure issues. This can render Buddhism to a religion based mainly on faith and authoritative accounts. By `reason,' I strictly mean \emph{the} reason produced by critically deliberate thinking, not just `my' reason or `your' reason to believe or deny such and such things. If we downplay reasoning in religious understanding, any attempt to explain the Dhamma reasonably turns pointless. In modern era, therefore, we should no longer adopt `the Middle Ages' attitude in religious matter, but we should turn to `the Enlightenment' attitude instead.\footnote{Only in academic Buddhism, `the Enlightenment' mode of thinking is adopted in large part. In living folk Buddhism, people mostly still use medieval mode of reasoning.} Anyone can disagree with my argument here, but he or she has to use certain reasoning nonetheless.

To understand the situation clearer, let me draw you a plausible scenario. Once the Buddha realized about the causation, he thought it is very difficult to understand. Yet, he tried to put it into words and expounded them in a minimalist fashion in a hope that a capable one could get it. After the idea was delivered, a few understood it, while many could not. Then confusion on the subject grew. When the account was retold, the confusion grew larger. Elaborations then were added. The confusion grew even larger. At some time, the true understanding of this teaching disappeared.\footnote{I do not subscribe to the view that all arhants understand the causation by its words clearly. Even if arhants may exist nowadays, they are not necessary to understand the subject by its notation.} When later generations see the text, they are bewildered and mistake profundity for incomprehensibility. By this view, any systematic account of the idea, even a good one, is alien to the original exposition. You can understand it in some way, however, but there is no guarantee that what you know is exactly what the Buddha intended.

That is a good place for textual manipulation. When something is obscure, it is easy to be exploited. So, dependent origination is often used as a panacea for Buddhist philosophical conundrums, such as, why \pali{anatt\=a} can be reborn, how rebirth exactly happens, how to go between eternalism and nihilism, or between strict determinism and indeterminism, and so on. Many associations between dependent origination and other matters have been produced, and often they work charmingly, but what they really mean can be questionable.

\paragraph*{4) Dependent origination is natural process ethicized.}

As described in the canon, the Buddha insisted that regardless of his arising, the process depicted in dependent origination runs by itself naturally and invariably.\footnote{S2\,20 (SN\,12)} The Buddha only discovered it and made it known. In Buddhists' mind, his discovery is on a par with scientific findings. It unveils a secret of nature.\footnote{I also do not subscribe to the view that the Buddha knew dependent origination instantly as if he accessed to the database of universal knowledge, a kind of cosmic encyclopedia. Instead he used reasoning to formulate it, and he tried several configurations until the best one came up. One obvious reason is that ignorance itself cannot be seen and known directly because it is a state of something absent, wisdom in this case. We can, a best, observe and ponder upon a person who lacks wisdom and see how his life goes. To come up with the complete circle of causation, a lot of inferences have to be done.}

Comparing dependent origination with a scientific account of natural phenomena can bring a lot of discussion nowadays. There are some areas in the causal chain overlapping with what we known as cognitive psychology today. And the studies of human cognition in recent decades are very advanced beyond antiquity could imagine. Buddhist scholars may have to bring this new knowledge into consideration before they can compare the explanations impartially. For example, if Buddhist teachers are not aware of how much we know nowadays about consciousness, attention, memory, perception and brain processes, etc., how they can discuss mental entities in an understandable way to modern audience.

I will not assess the accuracy of the picture of reality that dependent origination gives us here. Because, as we have known, the Buddha was not interested in metaphysical issues, as well as what we call science today. The only thing he was concerned is how to do away suffering. And he knew that it has something to do with knowledge---to know the world in the right way. Thus when he explained the process of causation, he grounded it on morality with a therapeutic goal in mind.

By this view, any confusion generated by dependent origination turns insignificant, because it is not meant to be understood as scientific, nor it should be. Hence, any attempt to elaborate or systematize dependent origination goes against the original intention. Regardless of what the process exactly means, we all are encouraged to do right actions, as described in the forth noble truth, nonetheless.

Put it in another scenario, when the Buddha entertained the idea, he found it fascinating. He was reluctant to teach it at first because it is too technical and unnecessary to know in detail. So, he put it succinctly for some could appreciate it. And he used it sometimes when he discussed with competent listeners. It was a good topic of conversations. After that for some time he did not bother to elaborate it, but instead directed the listeners to a more practical issues.

\bigskip
That can conclude my fair discussion on dependent origination. I know and I admit that I do not understand it fully, and I do not want to sell my interpretation. I also always suspect those who claim they understand the issue by its original intention and can explain it thoroughly without discrepancy in textual evidence. However, the subject is worth studying and discussing, in fact crucial to understand early Buddhist philosophy. By discursive practice, dependent origination can be understood in any way depending on what purpose the understanding is used for. The picture of early Buddhism is shaped by which facet of this understanding is presupposed.

