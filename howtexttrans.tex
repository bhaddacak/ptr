\chapter{How is text transmitted?}\label{chap:howtexttrans}

My approach to the problem of textual transmission is not historical. That is why I use `is' rather than `was.' What I try to tackle is the general case of transmission of understanding. So, it can be applied to religious texts as well. Let us draw a draft picture of the Buddhist transmission of text first.

\begin{figure}[!hbt]
\centering
\setlength{\unitlength}{1mm}
\begin{picture}(90,40)(0,0)
\linethickness{1mm}
\put(0,10){\line(1,0){90}}
\thinlines
\put(0,10){\line(0,1){20}}
\put(-1,32){\makebox(0,0)[l]{\footnotesize Enlightenment of the Buddha}}
\put(5,13){\makebox(0,0)[c]{\footnotesize 45 yrs}}
\put(5,8){\makebox(0,0)[t]{A}}
\put(10,10){\line(0,1){16}}
\put(9,28){\makebox(0,0)[l]{\footnotesize Death of the Buddha, the 1st council}}
\put(20,13){\makebox(0,0)[c]{\footnotesize 400 yrs}}
\put(20,8){\makebox(0,0)[t]{B}}
\put(30,10){\line(0,1){12}}
\put(29,24){\makebox(0,0)[l]{\footnotesize The canon was written down}}
\put(42,13){\makebox(0,0)[c]{\footnotesize 550 yrs}}
\put(42,8){\makebox(0,0)[t]{C}}
\put(55,10){\line(0,1){8}}
\put(54,20){\makebox(0,0)[l]{\footnotesize Buddhaghosa and}}
\put(57,17){\makebox(0,0)[l]{\footnotesize post-canonical literature}}
\end{picture}
\caption{A rough timeline of the development of the P\=ali scripture}
\label{fig:timeline}
\end{figure}

As shown in Figure \ref{fig:timeline}, I divide the development of the P\=ali scripture into three periods. The first period started from the Buddha's enlightenment to his death, the period of the Buddha's exposition of the Dhamma (A). This lasted 45 years. The second period started roughly from the death of the Buddha and the first council of compilation, three months after the death, to the time the canon was written down (B). This period lasted about 400 years. The third started from the writing down to the age of Buddhaghosa and the formation of P\=ali commentaries and other post-canonical literature (C). This lasted around 550 years. After that there was no such big events, so I ignore them.

The scenario of the development of the canon depicted by the tradition can be recounted in terms of key points as follows:

\begin{compactenum}
\item \=Ananda, the Buddha's attendant, memorized the main body of the doctrine, the Suttanta. Up\=al\=i memorized the disciplinary part, the Vinaya.
\item At the first council, \=Ananda and Up\=al\=i retold what they remembered. The contents were confirmed, formulated, classified, finalized, and recited by the council. This became the first version of P\=ali canon. The outcome was memorized by monks harmoniously and handed down.
\item The subsequent councils repeated the process in the same manner. Some new materials were added to the canon. Some spurious contents were removed.
\item In Sri Lanka around 100 BCE, the whole canon was written down. This is supposedly the end of oral transmission.
\item Every council was supported by the king or the government at the time. 
\end{compactenum}

The account sounds simple and idealistic. Most Buddhists accept the reliability of the process and believe that most of the teaching was well-preserved. However, when I look closely to the process and think about it carefully, I no longer take this for granted. I do not want to go against the tradition, but I want to be more reasonable. So, I reassess the transmission process under the light of the best knowledge we have today, some of which have been told in the preceding chapters. Here are details of my argumentation.

\paragraph*{1.\ \=Ananda is a set-up character.} This is a bold claim that can make many furious at me. I do not say Ven.\,\=Ananda did not exist historically, even if I am not sure of that but it has a possibility. Rather I doubt that the role of \=Ananda was set up for the sake of reliability of the canon. He is said to be the (final) attendant of the Buddha, even though both were of the same age (\pali{sahaj\=ati}). As the attendant, \=Ananda was able to hear all of the Buddha's sermons. If both were not at the same place at the time, he could ask the Buddha to retell the teaching later. This position buttressed him as the curator of the teaching.

Most of discourses in the Suttanta start with \pali{eva\d m me suta\d m} (thus it was heard by me). It suggests that \=Ananda was the narrator of stories told by someone else, the Buddha supposedly. There is no place that \pali{eva\d m me di\d t\d tha\d m} (thus it was seen by me) is used. That means, if we take the word seriously, \=Ananda did not see the events or was not in the same place of them. In fact, if we read suttas closely, the narrator was not \=Ananda but someone else.

In some suttas, even though \=Ananda was in the event, \pali{eva\d m me suta\d m} is still used, and \=Ananda is mentioned by name not pronoun of first person. For example, at the end of MN 18 it is read thus, ``Satisfied, Ven.\,\=Ananda was happy with what the Buddha said.''\footnote{\pali{Attamano \=ayasm\=a \=anando bhagavato bh\=asita\d m abhinand\=i'ti.} (M1\,205)} This shows by the text that the narrator was not \=Ananda himself.

We can explain this reasonably as in the formulation of the suttas there was a kind of normalization or unification process that made text looks unified in format, easy to remember. This suggests that the phrase \pali{eva\d m me suta\d m} does not really mean exactly as such. A sutta is a recount nonetheless from whoever, but when \=Ananda is said to be the narrator, it raises credibility of the story, like celebrities have louder voice in commercials.

One serious problem when every sutta was told by \=Ananda is that he was not an arhant at the time. By this character, \=Ananda could be moved emotionally. That makes stories more dramatic and memorable. The weak point is that how are we really sure he got the teaching right? As we have known, memory is not videorecorder and has constructive nature. It is more likely that he grasped the message by his own terms. When stories was retold, they came from \=Ananda's construction, not the exact words of the Buddha.\footnote{I should add that even though \=Ananda was an arhant at the time, he still constructed the message by his own understanding. This might be better than the Stream Enterer's version, but construction nonetheless.}

I think a more accurate picture what happened at the first council, if there was such an event at all\footnote{As noted by Charles Prebish concerning the first council, ``a council held in the grand style described in the scriptures is almost certainly a fiction.'' \citep[p.~188]{prebish:councils} I do not take this historical fact seriously because the explanation is made for general cases. So, it can be applied to all councils, real and fictional.}, is monks who can memorize events shared their experience at the meeting. By this means, one story in the text might come from various sources with different accounts. There must be a process of finalization by authority. That is the way such a council should be. \=Ananda, if he took part at all, might be one in the committee, perhaps the chair in some occasion. This picture look very modern, but I think it is the common way when we organize such an event.

By this view, some problem still persists. Let us consider this scenario. In one occasion, the Buddha gave a sermon to a group of monks. At the end, all of them attained the arhantship. The question is ``Did they remember the same thing from the same teaching?'' If you follow the tradition the answer is inevitably `Yes.' But that is not really the case, as far as we know about our cognition. Each monk understood the teaching in their own way, because all of them have different knowledge structure or pre-understanding. The messages they reconstructed are unlikely to be the same, maybe slightly different on perspectives. 

Then the next question comes, ``What or who determines the final version of a sermon?'' I think this question rarely comes to the mind of the traditional adherents, because in such a naive view the final version existed before the communal rehearsal (\pali{sa\.ng\=ayan\=a}) was performed. If you think it carefully, such a scenario is impossible in real world. The answer of this question will come to light later.

\paragraph*{2.\ Changes happened in every transmission.} After the formation of the canon in the first council, the outcome was preserved by monks orally, according to the timeline above, for 400 years. In traditional mind, the transmission was near perfect because monks did it dearly and faithfully. I do not doubt that enthusiasm and endeavor to preserve the teaching. But I think faithfulness and sincerity do not guarantee the accuracy of the preservation. Every time memorization is reproduced new construction occurs.

When we memorize something, we have to understand it first, otherwise we just commit nonsensical sounds to the memory. If we memorize rubbish, we cannot hold it for long because it changes easily. When we make sense of something learned, we also create our own understanding according to our existing knowledge structure. That means handing down text orally generation by generation does not guarantee its accuracy.

Another view on textual transmission is that the first priority was not the attempt to keep it word by word, but rather to use text as a teaching material. Paul Williams tells us that:

\begin{quote}
It is likely that Buddhist texts were intended as no more than mnemonic devices, scaffolding, the framework for textual exposition by a teacher in terms of \emph{his own experience and also the tradition}, the transmission from his teachers, traced back to the Buddha himself.\footnote{\citealp[p.~45]{williams:mahayana}, emphasis added}
\end{quote}

By this account, students learned a text under the guidance of a teacher who understood the text by his own experience that was in turn based on understanding of the former generations. This means transmission of text did not maintain its intact form. Instead, understanding of the text was transmitted, and the very content has changed generation after generation.

As the evidence shows, we can demythologize the accuracy of oral transmission of the P\=ali text. If the text was really well-preserved, the subsequent rehearsals were not needed. As a matter of fact, the text did really change and a new normalization process was needed. A recurring question comes, ``What or who determines the final version of a new compilation?'' Buddhists are tempted to think that the new one still agrees with the original teaching. That is a circular thinking. If we know what is the original, this problem will never arise. The task of this inevitably comes to the hand of authority. Decisions have to be done with great care.

If you understand the situation, you will see why early Buddhism was broken into several sects in just the first century after the death of the Buddha.\footnote{There were eighteen schools mentioned in Therav\=ada sources.} The very problem was about authority who decided what should be counted as the real teaching. And there is no single correct answer. What does constitute such authority? You might think, for example, those who remember a lot, or those who have a great number of followers, or those who have great expositive power, or those who claim to attain certain stages of liberation, or those who claim to have certain miraculous power. How are you sure which authority gives you the `right' answer?

In my view, the real authority comes from the king or the government at the time of compilations. With such power, you can appoint the committees you see fit, you can set agendas, you can sanction what you see proper, and you can reject what you see inappropriate. All these come in the guise of the supporter of the events. If you have power, won't you do anything with beliefs of the mass that have potential to undermine your power? Weapon of thought is far more effective than real weapons; think of capitalism or communism, for example. That means the king or the government is an unnoticed but powerful factor that determines the result of each compilation. The `right' teaching have to favor power, so to speak.

Why do I love to draw a gloomy picture, you might wonder? If you suspend your beliefs and think it straightly upon the issues, following my logic you can see that my picture is in fact a realistic one. When I draw power into consideration, it does not mean bad things happen. Utilization of power can bring both good and bad result. Certain religious beliefs can definitely be a part of good governance. However, it is also true that power tends to corrupt\footnote{This is a well-known remark of Lord Acton (1834--1902), a British historian. The full version is ``Power tends to corrupt and absolute power corrupts absolutely.''} if there is no counterbalance. We have to be aware of the underlying power of our social activities.

\paragraph*{3.\ Oral transmission is not better than transmission of written materials.} Whereas the tradition firmly asserts the accuracy of oral transmission, I hold the opposite view for a number of reasons. First, we can trace changes in written materials but not in the memory. As we have seen previously, changes inevitably occur by several factors, but oral transmission creates an illusion that they did not happen. The main reason is there is no other version to compare with. Once a new compilation finished, the old materials were discarded. Old stuff was no more remembered. In contrast, even though errors occur every time written materials are copied or recompiled, they leave traces that enable us to retrospect, if any remnant is left.\footnote{It may not be a good idea to keep the old versions of sacred texts. It might be better to destroy the old stuff, once the new edition comes into being. This can unify the teaching. I have heard that Islam did just that with its Quran. Maybe Buddhaghosa did likewise with the old Sinhalese commentaries. Such doing is good from power's point of view, but not so from scholarly stance.} 

Second, errors produced by writing are not so wild as by remembering.\footnote{Think about the game of Chinese whispers, you may get the idea.} As you might see, dealing with ambiguity in P\=ali is not easy. With only memory how can anyone differentiate the retroflexes (\pali{\d t \d th \d d \d dh \d n}) from the dentals (\pali{t th d dh n}), or \pali{r} from \pali{l} or \pali{\d l}? How can anyone memorize spaces between words correctly?\footnote{To see how this can be a problem, see Chapter \externalref{25} (on numerals) in PNL.} 

If you are really curious on this issue, you can conduct a research by making some participants remember a P\=ali text, then making them write it back and comparing to one another. It is better to separate two group of participants, one for those who understand P\=ali, and another for those who do not. The latter case tends to make more errors, I suppose. And let us think that ``Did rememberers understand what they put into memory throughout 400 years of oral transmission?''\footnote{I am not sure to what extent understanding of P\=ali help us to remember better, because I did not conduct this research myself. But from my experience, some P\=ali learners I know recite some excerpts (\pali{paritta}) wrongly here and there all the time. Understanding might help, but it also makes us overconfident in some points. Hence, we tend to bend the text to our understanding when thinking we get it right. This line of research is really interesting. I hope someone will do it. I have no academic interest to do that.}

\paragraph*{4.\ Sincerity does not guarantee correctness.} One line of argument I have often heard is that we should belief in the P\=ali canon because it was the product of arhants who have no bias whatsoever and they have done so for the benefit of all beings. That sounds very idealistic to me. If the compilers of the canon really had no bias, we would not find instances that show partiality towards other religions, other sects, and female gender, for example. We would find less aggrandizement of revered figures, like the Buddha and famous disciples.

To the point that the compilers had done it out of compassion towards all beings, it is true as long as all beings adopt the same kind of belief and attitude. That sounds like ``you all will be granted for salvation if you believe me and do what I say.'' Is this sound familiar? I do not argue that those monks had any ill-will. I believe they all had a good intention to alleviate human suffering. But that does not means they could escape a parochial view towards their own belief system. The content of the P\=ali canon clearly shows exclusivist position towards other religions---my teaching is true or better, yours are false or not complete. Therefore saying that the canon expresses impartiality is downright false.

If in-group interest always has high priority, do you really expect the `true' correctness from that? By `true' here I mean `objective' independent from any self-interest. Hence, correctness in this context always means our correctness not yours. Another obvious point is that sincerity does not prevent us to make a mistake. We all can do wrong sincerely.

I think this issue is really disturbing for Buddhists. Some even cannot put their thought on it. Buddhists, like all adherents of other religions in this matter, are used to think that their religion is the best of the world (in fact, the universe). I do not want to touch much on this sensitive point. I have a simple line of thought for you to follow, though.

If the ultimate concern of Buddhism is about the end of suffering, all teaching able to lead to that state can be counted as equivalent. Many Buddhists still argue that Buddhist salvation is far better off. If you are one of those who think as such, I have questions for you to think over. In what respect is Buddhist salvation better? Is Buddhist happiness better? Why aren't Buddhist countries the happiest of all? How about revered figures like Ramana Maharshi or Nisargadatta Maharaj, and many more outside Buddhism, who are regarded as enlightened beings in the modern age? Do they have things to do with Buddhist salvation? Do Japanese Roshis who have family like other worldly beings attain the salvation in Therav\=ada standard?

I cannot give you all the answers, you have to think them yourselves. Let me repeat a question thus, ``What do you think when you are happy in your own way and someone say you are not really happy and you have to adopt certain belief and practice to reach the `real' happiness?'' This question is not easy to deal with. It can lead to relativism that makes recreational drugs your valid happiness. But if you believe in objective happiness and try to tackle this problem honestly, you may come up with something `real' not just advertized happiness.

\section*{Concluding remarks}

In this chapter we deal with how text is transmitted. From the first period (A), the teachings were handed down disorderly. The Buddha gave sermons here and there. The memorization was done casually. One event might have different narrations. In the second period (B), once the first compilation had done, if any, the chaotic text had been systematized and unified. The memorization continued for 400 years. It is clear that the canon underwent changes along the way. The division into three baskets, as well as the present of Kath\=avatthu, is marked evidence.\footnote{Some may argue that adding a new item is not the same as changing the existing one. If new things can be added, there is no restriction that the existing things could not be changed or deleted. All things could happen including the insertion of newly composed suttas.} In the third period (C), textual materials were changed to written forms. That enables the proliferation of post-canonical literature.

The key issue needed to be stressed is text formation and transmission cannot be separated from constructive understanding. Therefore, when a text was transmitted, it was not just reproduced like photocopy. It was reconstructed each time when understanding was transferred. Holding that P\=ali texts we have in hand today are the same as the original is simply an illusion.
