\chapter{What is understanding?}\label{chap:whatunder}

We will start with the problem of `understanding.' This is not seen, by me, as a philosophical problem, but a scientific one. Understanding `understanding' is essential to our theoretical foundation. Also, understanding this problem clearly, in a way, can shake our old belief. So, please go through carefully and think along with me. I will not go technically deep.

To put it simply, `understanding' is a mental state that happens when we know something. A synonym of this is `comprehension.' It is more or less equal to `knowledge,' but I will not use this term because it has a specific use in epistemology, and I do not want to approach it philosophically.\footnote{Differentiating between `knowledge' and `understanding' is a headachy philosophical problem. For a treatment of this problem, see \citealp{mason:understanding}, particularly in chapter 3. I do not play that game because I adopt a simple view: to know is to understand.}

When I say ``I know that today is Monday,'' I have an understanding. In this case, I can mistake, say, it is really Tuesday today. So, I can have wrong understanding. If I take it rightly, I have right understanding. I may know that the sky is beautiful and I have a reason for that. You might disagree with this with another reason, but we cannot say who has the right understanding in this case. So, understanding can be objective or subjective. We will not pay attention to veracity or truthfulness of understanding here. That is a task of philosophers. We care only there is a mental state that happens to us as a comprehension.\footnote{You may see this as `sensation' or `perception' or `cognition' or `thought' or `idea' or whatever. I do not want to make it so technical that a deliberate definition is needed. Or you may see this as a kind of \emph{intentionality} in phenomenological sense. But this is still in the sphere of philosophy that needs further explanation.}

To be a little more precise, to understand something I mean to consciously see something, or hear, touch, think, and so on. Once an understanding happens, its content is stored in our memory, for a short or long period. Therefore when we talk about understanding, consciousness and memory are inevitably involved. By `consciousness' I simply mean the state of being awake and able to see, hear, think, etc. And `memory' means the mental faculty of retaining and recalling past experience. All these terms have no transcendental or supernatural implication.\footnote{I do not say that there is no transcendental entity, if any. I just delimit our focus only to things that can be verified, at least by our common experiences.} They are undeniable parts of our human nature.

Why does understanding matter then? It is a crucial term that we have to understand it clearly before anything else can be understood. To anticipate our discussion and see the big picture, when certain text is translated, it means one understanding is transferred from one context to another context. In naive view, this means understanding somehow can be reproduced without distortion. But as we shall learn later on, it is not the case that one textual comprehension can stay the same all the time.

By `text' here, I mean roughly a record of certain understanding, in oral or written form (we will talk more about `text' later). Text is dependent on language but understanding is partly not.\footnote{This is a vulnerable issue. Some may argue that understanding is indeed bounded by language (Sapir-Whorf hypothesis). I assert that some understanding does not require language, for example, when we eat tasty food we know directly that it is delicious.} This means there is a transformation from metal state of understanding to textual media bounded by certain language. Therefore, translation can be seen as a reversed process of textual formation. I depict the process in Figure \ref{fig:relation}.

\begin{figure}[!hbt]
\centering
\setlength{\unitlength}{1mm}
\begin{picture}(80,35)(0,0)
\put(0,12){\framebox(25,10)[c]{Understanding}}
\put(25,17){\vector(1,0){10}}
\put(35,12){\framebox(10,10)[c]{Text}}
\put(45,17){\vector(1,0){10}}
\put(55,12){\framebox(25,10)[c]{Understanding}}
%
\put(0,2){\line(0,1){6}}
\put(5,5){\line(-1,0){5}}
\put(6,0){\makebox(25,10)[l]{Textual formation}}
\put(35,5){\vector(1,0){5}}
\put(40,2){\line(0,1){6}}
\put(45,5){\line(-1,0){5}}
\put(50,0){\makebox(25,10)[l]{Translation}}
\put(75,5){\vector(1,0){5}}
\put(80,2){\line(0,1){6}}
%
\put(0,25){\makebox(35,10)[r]{Source language}}
\multiput(30,17)(0,2){6}{\line(0,1){1}}
\put(45,25){\makebox(35,10)[l]{Target language}}
\multiput(50,17)(0,2){6}{\line(0,1){1}}
\end{picture}
\caption{Relation between understanding and translation}
\label{fig:relation}
\end{figure}

To put it another way, we can see textual formation as writing process, the transformation from understanding (mental state) into language (its representation). And we can see translation as reading process, the transformation from language back to understanding. That is to say, translation is a specific case of reading, the interlingual reading from textual materials, in audible or visible form.\footnote{Precisely speaking, the final outcome of translation is achieved by writing the understanding into textual form of a target language. This is a matter of definition we used for translation. We will discuss more on translation in due course.}

When I use `reading' here, it is not just when we process a visual image to gain certain understanding. Generally, reading is the process of sense making regardless of textual form. You can also read by touching, or by hearing, and so on. As you might see, understanding has a close relation to meaning. Basically, when we say we understand some text, we understand its meaning. By this, it looks like that meaning belongs to text, and understanding belongs to our mind. Relation between understanding and meaning is crucial to our course, so we will come to this in detail in the following chapters.
