\chapter{Commentary reading}\label{chap:commentary}

In P\=ali literature, commentaries (\pali{atthakath\=a}) are the texts explaining the canon. Most P\=ali students may think that reading the P\=ali canon is enough to understand the early teachings of the Buddha. So, they generally ignore commentaries when they study P\=ali texts. One reason that makes they think as such, I think, is they are afraid of being suggested or misled by the tradition, so they try to make sense of the canon by their own judgement. Another reason can be that even the authenticity and reliability of the canon can be questionable, why should we rely on the commentaries?

Here are some reasons why we should take P\=ali commentaries into consideration. First, the main reason, texts in the P\=ali canon are not always clear. Consulting the former efforts to tackle the problems is better than doing a guesswork on your own. It is true that sometimes traditional commentaries are very suggestive, but we do not need to swallow all of them. Having some information, albeit useless sometimes, is better than having none.

Second, from a perspective of the language, the style of post-canonical texts is more sophisticated. Reading commentaries and the later literature help us get more understanding of P\=ali. If the learners can read commentaries, they will feel less difficult on reading the canon, grammatically speaking. They will know how to deal with problematic points properly.

Third, it is about the tradition. The P\=ali canon is a common asset of all Buddhist denominations. It can be read in several ways, so to speak. To understand Therav\=ada position, we have to read it from the commentaries and subsequent texts. It is not only the scholarly understanding that we gain from commentary reading. We also know how the tradition manipulates the religious beliefs by exegetic activities.

By the reasons mentioned above, reading commentaries of the P\=ali canon is worthwhile, if not indispensable for a rigorous study. Now we will look into the structure of P\=ali commentaries. Basically, we can see these four parts in a commentary:

\paragraph*{1.\ The explanation (exegesis) of the main text} This is what a commentary is all about, so it is the core part of the text. Normally, the commentator takes a difficult word or phrase\footnote{Technically, this is called \pali{m\=atik\=a}.} into consideration and makes certain interpretation. In our text collection, the terms taken from the canon are marked by boldface.
\paragraph*{2.\ The main text reproduced} If the main text has verses, the whole stanza is often reproduced. We normally find this part in the commentaries to the sections that have verses as their main body, such as J\=ataka and Dhammapada.
\paragraph*{3.\ Quotations} Sometimes, a part of other texts is quoted, maybe a stanza or a short excerpt. Fortunately, in our collection the source of the quotation is also given. So, we can trace back to its whole part without much difficulty. Despite often seen, they are optional.
\paragraph*{4.\ Stories} This is like the backstory\footnote{Backstory is used in fiction. In my view it is close to what I call `stories' here. But I do not suggest that all backstories are fictional, even if some obviously are. Thinking about historical fiction may give you what I mean. May I add that it is often said that Indian people have very poor sense of history. So, telling myth from history in their literature is very hard.} of the event related to the text being explained. This part is also optional. Some collections in the Suttanta may already have well-told stories, like D\=ighanik\=aya and Majjhimanik\=aya. The commentaries to these parts normally have no story. Some collections may need additional stories, like many parts in the Vinaya and A\.nguttaranik\=aya. So, their commentaries have some related stories. Commentaries to the collections that have only or mainly verses as their content, like J\=ataka and Dhammapada, may have full-blown stories as their substantial parts.

\bigskip
Now I will give an example from Dhammapada commentary (Dhp-a\,1.1). The excerpt below appears right after the long backstory of the first verse in Dhammapada is told. The text in this commentary has no reference point, except paragraph numbers corresponding to the main text. Normally, stories, which are sometimes very long, appear before the referred verses from the canon. Commentaries to other parts of the canon may follow a slightly different pattern.

\begin{quote}
\fbox{\small the end of story} $\downarrow$ \\[1mm]
\ldots\ \pali{Athass\=a dve akkh\=ini d\=ipasikh\=a viya vijjh\=ayi\d msu. So vejjo cakkhup\=alo ahosi.}\\[1.5mm]
\fbox{\small the bridge to commentary} $\downarrow$ \\[1mm]
\pali{Bhikkhave, tad\=a mama puttena katakamma\d m pacchato pacchato anubandhi. P\=apakamma\~nhi n\=ameta\d m dhura\d m vahato balibaddassa pada\d m cakka\d m viya anugacchat\=i'ti. ida\d m vatthu\d m kathetv\=a anusandhi\d m gha\d tetv\=a pati\d t\d th\=apitamattika\d m s\=asana\d m r\=ajamudd\=aya la\~nchanto viya dhammar\=aj\=a ima\d m g\=atham\=aha} --\\[1.5mm]
1. $\leftarrow$ \fbox{\small paragraph number} \hspace{6mm}$\downarrow$\ \fbox{\small text from the canon} \\[1mm]
\pali{``Manopubba\.ngam\=a dhamm\=a, manose\d t\d th\=a manomay\=a;\\
Manas\=a ce padu\d t\d thena, bh\=asati v\=a karoti v\=a;\\
Tato na\d m dukkhamanveti, cakka\d mva vahato padan''ti.}\\[1.5mm]
\fbox{\small the commentary started} $\downarrow$ \\[1mm]
\pali{Tattha \textbf{mano}ti k\=am\=avacarakusal\=adibheda\d m sabbampi catubh\=umikacitta\d m. Imasmi\d m pana pade tad\=a tassa vejjassa uppannacittavasena niyamiyam\=ana\d m vavatth\=a\-piyam\=ana\d m paricchijjiyam\=ana\d m domanassasahagata\d m pa\d tighasampayuttacittameva labbhati. \textbf{Pubba\.ngam\=a}ti tena pa\d thamag\=amin\=a hutv\=a samann\=agat\=a. \textbf{Dhamm\=a}ti gu\d nadesan\=apariyattinissattanijj\=ivavasena catt\=aro dhamm\=a n\=ama. Tesu} --\\[1.5mm]
\fbox{\small a quotation} $\downarrow$ \\[1mm]
\pali{``Na hi dhammo adhammo ca, ubho samavip\=akino;}\\
\pali{Adhammo niraya\d m neti, dhammo p\=apeti suggatin''ti. (therag\=a.\,304; j\=a.\,1.15.386)}\footnote{Thag\,304; Ja\,19:81}\\
\fbox{\small the commentary continued (quotations embedded)} $\downarrow$ \\[1mm]
\pali{Aya\d m gu\d nadhammo n\=ama. ``Dhamma\d m vo, bhikkhave, desess\=ami \=adikaly\=a\d nan''ti (ma.\,ni.\,3.420)\footnote{M3\,420 (MN\,148)} aya\d m desan\=adhammo n\=ama.} \ldots\\
\ldots\\[1.5mm]
\fbox{\small conclusion} $\downarrow$ \\[1mm]
\pali{G\=ath\=apariyos\=ane ti\d msasahass\=a bhikkh\=u saha pa\d tisambh\-id\=ahi arahatta\d m p\=apu\d ni\d msu. Sampattaparibh\=ayapi desan\=a s\=atthik\=a saphal\=a ahos\=i'ti.}\\[1.5mm]
\fbox{\small the end of this section} $\downarrow$ \\[1mm]
\pali{Cakkhup\=alattheravatthu pa\d thama\d m}\\
\end{quote}

My rough translation is shown below. I cannot say that I get all the meaning clear, but this sounds sensible to me.

\begin{quotation}
\ldots\ Then her two eyes became blind like a lamp's flame [goes out]. That physician was Cakkhup\=alo [in this time].

{[The Buddha said]} thus, ``Monks, from that time the action being done by my son followed [him] now and again. Such an evil action follows [the doer] like the wheel follows the footstep of an ox carrying a load.''

Having told this story, having made a conclusion, [then] the king of Dhamma said this verse, like the king stamps a clay edict with the royal seal.

\begin{quote}
1.\par
``All [mental] phenomena [is] forerun by the mind, led by the mind, made by the mind;\par
If by an evil mind, one says or does;\par
Suffering follows him/her because of that, like the wheel [follows] the footstep [of an ox] carrying a load.''
\end{quote}

In that [verse], even all kinds of consciousness in the four realms of existence\footnote{precisely, which dominated by sensual pleasures} [are] called `\pali{mano}.' In this word, when [the word] being determined, defined, [and] limited, [the physician] gets only the mind associated with anger, connected with displeasure, by the power of his mind arisen at that time. Having been the first go, what was endowed with that [mind] is called `\pali{pubba\.ngam\=a}.' Kinds of good conduct by the influence of virtue, preaching, studying, soullessness and lifelessness [are] called the four kinds of good conduct, [these are] called `\pali{dhamm\=a}.' In those [four], --

\begin{quote}
``Both good conduct and bad conduct do not yield the same result;\par
Bad conduct leads to the hell, good conduct leads to a good existence.''
\end{quote}

This is called `\pali{gu\d nadhamma}' (the good conduct by the influence of virtue). ``Monks, I will teach the good conduct, which is beautiful in the beginning, to you,'' thus this is called `\pali{desan\=adhamma}' (the good conduct by the influence of preaching). \ldots

\ldots

By the end of the verse recited, 3,000 monks attained the arhantship with the discriminating knowledge. The teaching was useful [and] fruitful even to the people in the assembly. [That is all.]

The story of Cakkhup\=ala, the first one.
\end{quotation}

\phantomsection
\addcontentsline{toc}{section}{Various approaches to explanations}
\section*{Various approaches to explanations}

In this section, I will summarize the variation of what an explanation looks like, apart from the above example which \pali{tattha} signals the explanation. There are a number of techniques we can see as follows:

\setcounter{parnum}{1}
\paragraph*{\arabic{parnum}. With `\pali{attho}' (sense):}\ \par
- \pali{Tattha \textbf{bhayan}ti bh\=ayitabba\d m, corehi pariyu\d t\d thitatt\=a sappa\d tibhayanti attho.}\footnote{Dhp-a\,9.123. \pali{Attha} has multiple meaning. In commentary parts, it means `sense' or `meaning' mostly. In story parts, it is more likely to mean `need' or `want' (see Chapter \ref{chap:minors}). In general use, however, it means `use' or `benefit.'} (In that [verse], [what is] fearful is called `\pali{bhaya\d m}.' The sense is that [the path] is dangerous because [it is] pervaded by thieves.)\par

\stepcounter{parnum}
\paragraph*{\arabic{parnum}. With `\pali{vuccati}' for definition:}\ \par
- \pali{\textbf{Amatapadan}ti amata\d m vuccati nibb\=ana\d m.}\footnote{Dhp-a\,2.21} ([For] `\pali{amatapada\d m}' [mentioned above], nibb\=ana is said to be the deathless state.)\par

\stepcounter{parnum}
\paragraph*{\arabic{parnum}. With `\pali{p\=a\d tho}' for an alternative reading:}\ \par
- \pali{\textbf{Asa\~n\~nat\=a}ti k\=ayasa\~n\~nat\=adirahit\=a. Acetas\=atipi p\=a\d tho, acittak\=ati attho.}\footnote{Dhp-a\,18.248} (Those who are deprived of body control, etc., are called `\pali{asa\~n\~nat\=a}.' In another reading, it is called `\pali{acetas\=a},' in the sense of `mindless.')\footnote{See also `\pali{p\=a\d tha}' in PTSD.}\par

\stepcounter{parnum}
\paragraph*{\arabic{parnum}. With `\pali{p\=a\d thaseso}' for a suggestion of omitted terms:}\ \par
- \pali{Tattha \textbf{diso disan}ti coro cora\d m. `Disv\=a'ti p\=a\d thaseso.}\footnote{Dhp-a\,3.42} (In that [verse], \pali{diso disa\d m} means ``A thief, [having seen] a thief.'' The remaining [term] is `\pali{disv\=a}' [which should be added or understood].)\footnote{This means the full clause of it should be ``\pali{diso disa\d m disv\=a}.'' But in the verse, `\pali{disv\=a}' is left out by meter constraint.}\par

\stepcounter{parnum}
\paragraph*{\arabic{parnum}. With `\pali{tassattho}' for introducing the Buddha's words:}\ \par
- \pali{Tassattho -- \textbf{putt\=a me atthi, dhana\d m me atthi, iti b\=alo}}\footnote{Dhp-a\,5.62} ([This is] his meaning, a fool [suffers] thus ``I have children, I have wealth.'')\par

\stepcounter{parnum}
\paragraph*{\arabic{parnum}. With `\pali{ayamattho}' (this meaning):}\ \par
- \pali{okamokatoti udakasa\.nkh\=at\=a \=alay\=ati ayamattho.}\footnote{Dhp-a\,3.34} (This meaning [is that] `\pali{okamokato}' means ``from what is counted as water, from the dwelling.'')\par

\stepcounter{parnum}
\paragraph*{\arabic{parnum}. With `\pali{adhippeto}' (intended):}\ \par
-\pali{Tattha \textbf{gabbhan}ti idha manussagabbhova adhippeto.}\footnote{Dhp-a\,9.126} (In that [verse], \pali{gabbha\d m} [was] intended [to mean] the human womb here.)\par

\stepcounter{parnum}
\paragraph*{\arabic{parnum}. With `\pali{nip\=ato}' for identifying a particle:}\ \par
-\pali{Tattha \textbf{y\=avadev\=a}ti avadhiparicchedanatthe nip\=ato.}\footnote{Dhp-a\,5.72} (In that [verse], \pali{y\=avadeva} is a particle in/for setting a boundary.)\par

\stepcounter{parnum}
\paragraph*{\arabic{parnum}. With `\pali{nip\=atamatta\d m}' (a mere particle):}\ \par
-\pali{Tattha \textbf{upan\=itavayo}ti \textbf{up\=a}ti nip\=atamatta\d m}\footnote{Dhp-a\,18.237} \ \ (In that [verse], \pali{upa} is a mere particle in \pali{upan\=itavayo}.)\par

\stepcounter{parnum}
\paragraph*{\arabic{parnum}. With `\pali{padacchedo}' for breaking down a term:}\ \par
-\pali{\textbf{Iccev\=ah\=a}ti iti eva\d m \=ah\=ati padacchedo}\footnote{Ja-a\,12:49} (\pali{Iccev\=aha} is broken down to \pali{iti + eva\d m + \=aha}.)\par
-\pali{\textbf{Paccet\=i}ti patieti.}\footnote{Dhp-a\,9.125. In the commentary of Dhammapada, \pali{padacchedo} is not presented, but it is understood by the context.} (\pali{Pacceti} is broken down to \pali{pati + eti}.)\par

\stepcounter{parnum}
\paragraph*{\arabic{parnum}. With `\pali{vacana\d m}' (a word):}\ \par
-\pali{Tattha \textbf{sahassamap\=i}ti paricchedavacana\d m}\footnote{Dhp-a\,8.100} \ \ (In that [verse], \pali{sahassamapi} is a word for setting a limit.)\par

\stepcounter{parnum}
\paragraph*{\arabic{parnum}. With `\pali{\=alapati}' (to address):}\ \par
-\pali{\textbf{Atul\=a}ti ta\d m up\=asaka\d m n\=amena \=alapati.}\footnote{Dhp-a\,17.227} ([The Buddha] calls that lay devotee by name as `\pali{Atula}.')\par

\stepcounter{parnum}
\paragraph*{\arabic{parnum}. With `\pali{dasseti}' (to show):}\ \par
-\pali{\textbf{Chuddho}ti apaviddho, apagatavi\~n\~n\=a\d nat\=aya tuccho hutv\=a sessat\=iti dasseti.}\footnote{Dhp-a\,3.41} (By \pali{chuddo}, [the Buddha] shows that the discarded [body], having been empty because of departing consciousness, will lie down.)\par

\stepcounter{parnum}
\paragraph*{\arabic{parnum}. With `\pali{d\=ipeti}' (to explain):}\ \par
-\pali{Tattha \textbf{appam\=ado}ti pada\d m mahanta\d m attha\d m d\=ipeti}\footnote{Dhp-a\,2.21} (In that [verse], the term `\pali{appam\=ado}' explains the broad meaning.)\par

\stepcounter{parnum}
\paragraph*{\arabic{parnum}. With `\pali{li\.ngavipall\=aso}' (gender deviation):}\ \par
-\pali{\textbf{Jitan}ti li\.ngavipall\=aso}\footnote{Dhp-a\,8.104} (\pali{Jita\d m} is gender deviation.)\footnote{It is supposed to be \pali{jito}, hence \pali{jito att\=a} (the self conquered).}\par

\stepcounter{parnum}
\paragraph*{\arabic{parnum}. With `\pali{adhivacana\d m}' (a designation):}\ \par
-\pali{Tattha \textbf{jant\=u}ti satt\=adhivacanameta\d m.}\footnote{Dhp-a\,8.107} (In that [verse], this \pali{jantu} is a designation of a being.)\par

\stepcounter{parnum}
\paragraph*{\arabic{parnum}. With `\pali{eta\d m}' (this/that):}\ \par
-\pali{\textbf{Sukha\d m set\=i}ti desan\=amattameveta\d m}\footnote{Dhp-a\,6.79. In Thai edition, it is `\pali{desan\=as\=isamattameta\d m}.'} (This `\pali{sukha\d m seti}' is a mere [topic of the] teaching.)\par

\stepcounter{parnum}
\paragraph*{\arabic{parnum}. With `\pali{n\=ama\d m}' (a name):}\ \par
-\pali{\textbf{Muddhan}ti pa\~n\~n\=ayeta\d m n\=ama\d m.}\footnote{Dhp-a\,5.72} (This `\pali{muddha\d m}' is a name of wisdom.)\par

\stepcounter{parnum}
\paragraph*{\arabic{parnum}. With `\pali{ettha}' (here):}\ \par
-\pali{Atha v\=a \textbf{pare c\=a}ti pubbe may\=a \ldots\ ayamettha attho.}\footnote{Dhp-a\,1.6} (To put it another way, this meaning, thus ``\pali{pubbe may\=a \ldots},'' is [the explanation] of `\pali{pare ca}' here.)\par

\stepcounter{parnum}
\paragraph*{\arabic{parnum}. With `\pali{ida\d m-vutta\d m}' (this \ldots\ is said):}\ \par
- \pali{\textbf{P\=atheyyan}ti ida\d m ki\~nc\=api he\d t\d th\=a vuttameva}\footnote{Dhp-a\,18.237} (Although this \pali{p\=atheyya\d m} is said in the latter part, \ldots)\par

\stepcounter{parnum}
\paragraph*{\arabic{parnum}. Introducing an equivalent:}\ \par
- \pali{\textbf{Koci lokasmin}ti evar\=upo puggalo dullabho}\footnote{Dhp-a\,10.143} (Someone in the world means such a person [who is] rare [to find].)\footnote{In this instance, \pali{ko(ci)} is equivalent to \pali{dullabho}. Both are adjective.}\par
- \pali{Tattha \textbf{p\=ajet\=i}ti cheko \textbf{gop\=alo} ked\=arantara\d m pavisantiyo \textbf{g\=avo da\d n\d dena} niv\=aretv\=a teneva pothento sulabhati\d nodaka\d m gocara\d m neti.}\footnote{Dhp-a\,10.135. In this excerpt, \pali{pothento} is possibly \pali{yodhento}. See \pali{pothetv\=a} in PTSD.} (In that [verse], concerning \pali{p\=ajeti}, a wise cowherd, having prevented the cattle entering the field area with that stick hitting, leads [the cattle] to a pasture full with grass and water.)\footnote{Here, \pali{p\=ajeti} is equivalent to \pali{neti}. Both are verb.}\par
- \pali{P\=apassa hi uccayo vu\d d\d dhi idhalokepi sampar\=ayepi dukkhameva \=avahati}\footnote{Dhp-a\,9.117} (Because accumulation, increase, of evil [actions] brings only suffering in this world and the next world.)\footnote{Here, \pali{vu\d d\d dhi} (increase) is introduced as an equivalent of \pali{uccayo} (accumulation).}\par
- \pali{Tassa akara\d nakkha\d nepi \textbf{tamhi} pu\~n\~ne \textbf{chanda\d m} ruci\d m uss\=aha\d m karotheva.}\footnote{Dhp-a\,9.118} ([People] should make a wish, an inclination, an effort, in that merit even in the moment of doing that.)\footnote{In addition to \pali{chanda\d m} in the verse, \pali{ruci\d m} and \pali{uss\=aha\d m} are presented as equivalents.}\par
- \pali{\textbf{Anuyu\~njat\=i}ti sevati bahul\=ikaroti.}\footnote{Dhp-a\,18.247} (\pali{Anuyu\~njati} means `to practice,' [or] `to do frequently.')\par
