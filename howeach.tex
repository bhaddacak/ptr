\chapter{How do we understand each other?}\label{chap:howeach}

Human beings in essence are social animals. This means communication is a fundamental instrument for our survival and prosperity. And communication basically is transference of understanding from one person to another. It can be non-verbal or verbal.

Non-verbal communication in human beings has a biological basis. We all have ability of mind-reading, a simple version of telepathy. In philosophy of mind and psychology, it is called \emph{theory of mind}. We can normally interact with other people because we have a kind of theory of mind module (TOMM) in our brain. The impairment of this module is a marked symptom of autism.\footnote{\citealp[pp.~447ff]{baarsgage:cognition}} This ability is not mysterious. Everyone has it. Attest yourself by reading a good novel or seeing a decent movie. If your TOMM works properly, you can feel joy, pain, and other feelings as if you are the characters.

Verbal communication is more complicated than that and it is the main part of our concern in this chapter. For complex understanding, it cannot be simply induced by TOMM like feelings. When we have certain understanding, and want to let another have it too, we have to convert the understanding into words and send them to the receiver. In a naive view, it looks like we put a thing into a box, and we give the box to someone. When that person gets the box and open it, the thing is obtained. So, understanding is transferred perfectly. In reality, however, it is more than just handing a thing to someone because our perception is constructive in nature and our cultural matrix plays a significant role on meaning making. There is an explanation from communication studies read as follows:

\begin{quote}
For communication to take place I have to create a message out of signs. This message stimulates you to create a meaning for yourself that relates in some way to the meaning that I generated in my message in the first place. The more we share the same \emph{codes}, the more we use the same sign systems, the closer our two `meanings' of the message will approximate to each other.\footnote{\citealp[p.~39]{fiske:communication}, emphasis added}
\end{quote}

Let me retell it in this way. When I have something to say, I have to put it in a system of signs (or language). What signs carry is message. When I tell this message to you, you cannot simply get the meaning. You have to know how I create the message. That `how' is called `code.'\footnote{Here is a definition given by John Fiske: ``A code is a system of meaning common to the members of a culture or subculture. It consists both of signs (i.e.\ physical signals that stand for something other than themselves) and of rules or conventions that determine how and in what contexts these signs are used and how they can be combined to form more complex messages.'' \citep[pp.~19--20]{fiske:communication}} Technically, when I say something, I encode certain meaning in a message using a sign system. If you do not know the code, you cannot access to the meaning. 

For example, I tell you something in contemporary American English language. The code here is contemporary American English language. You have to recognize that and decode what I tell you accordingly, otherwise you understand nothing. One difficulty is that, unlike encryption we used in computers nowadays, decoding linguistic message does not yield the original exactly. There is always something lost in communication. 

I have a question for you to tackle right now concerning code we have mentioned previously. When P\=ali texts were translated or commented, which code was used by translators or commentators in reading the texts? I have choices: (a) code belonging to the Buddha's time, (b) code contemporary to the translators or commentators. 

Ideally, Buddhists are tempted to answer (a), but this is difficult to determine. To know the code used we have to know its context.\footnote{[W]e cannot identify what might be relevant codes without knowing the situational context \citep[p.~236]{chandler:semiotics}.} But the context itself is a reconstruction from the texts. Moreover, texts do not contain complete code or context---like novels do not include a dictionary in them. Texts may have some explanation on terms used (like glossary), but this is far from being a code we are talking about. That is to say, it is inevitably to read texts through a contemporary code. Then your only choice is (b). This reminds us that retaining the original meaning of ancient texts from modern reading is indeed questionable.

In communication studies there are two approaches to communication.\footnote{See \citealp[pp.~2--4]{fiske:communication} for more detail.} The first defines communication as the \emph{transmission of messages}. So, the main focus of this school is on efficiency and accuracy. If the receiver do not get the sender's intention right, the communication is regarded as failed.

On the other hand, the second approach defines communication as the \emph{production and exchange of meanings}. As such, meaning is not seen as static entity that is given from the sender to the receiver. Rather meaning is created dynamically by interaction between participants and text, as well as the environment like cultures. Therefore, misunderstanding does not necessarily mean failure of communication. For our concern, translation of ancient texts is more suitable for the second approach, because determining the exact intention of the texts not belonging to our time is very difficult, if not impossible. We will discuss more on this issue in due course.
