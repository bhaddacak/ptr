\chapter{Translation ethics}\label{chap:ethics}

Talking about the ethics of translation seems out of place. A reason of this, as noted by Ben Van Wyke, is that ethics in translation has not been addressed directly because ``it has been understood that the `correct' behaviour of the translator is fidelity to the text and author, and that a `good translation' is one that is most identical to the original.''\footnote{\citealp[p.~548]{wyke:ethics}}

The assumption above is based on the belief that we can and should reproduce the `true' meaning of the text. As we have discussed so far in several previous chapters, this assumption is untenable or not so worthwhile. Hence, the approaches of translation ethics have moved in various directions. For example, `good' translations for functional (skopos) theory are those which fulfill the specific purpose of the translation project. And according to Lawrence Venuti's ethics of difference\footnote{\citealp{venuti:scandals}}, it is unethical if translators just accept and conform to the dominating paradigms of the production without calling attention to differences of cultures. So, preferable translation promotes ``cultural innovation as well as the understanding of cultural difference.''\footnote{\citealp[p.~11]{venuti:scandals}}

Not every choice we have for our actions has the same value of worthiness. Basically we call the criteria of choosing what to do `norms.' In general terms, this belongs to ethics. According to Gideon Toury, norms can be defined as follows:

\begin{quote}
Norms\footnote{Strictly speaking, to Toury, norms sit between rules and conventions. Violation of a rule entails a formal penalty. Violation of conventions is not so serious. And violation of norms may lead to negative evaluation. \citealp[See][p.~177]{munday:translation}.} have long been regarded as the translation of general values or ideas shared by a community -- as to what would count as right or wrong, adequate or inadequate -- into performance `instructions' appropriate for and applicable to concrete situations.\footnote{\citealp[p.~63]{toury:descriptive}}
\end{quote}

The `instructions' mentioned are what is prescribed, forbidden, tolerated, and permitted in a certain situation. Norms are closely linked to ideology as Lawrence Venuti puts it:

\begin{quote}
Norms may be in the first instance linguistic or literary, but they will also include a diverse range of domestic values, beliefs, and social representations which carry ideological force in serving the interests of specific groups. And they are always housed in the social institutions where translations are produced and enlisted in cultural and political agendas.\footnote{\citealp[p.~29]{venuti:scandals}}
\end{quote}

There are models of translation norms proposed by theorists, such as Gideon Toury's descriptive norms. I see Toury's model a little hard to understand. Since our purpose is not to go deeply in theoretical space, so I present here Andrew Chesterman's model for we can get some idea.

According to Chesterman, translation norms can be of two kinds: product or expectancy norms and professional norms.\footnote{\citealp[pp.~62--8]{chesterman:memes}} Brief explanations can be seen in Table \ref{tab:chestnorms}.\footnote{adapted from \citealp[pp.~187--8]{munday:translation}}

\begin{table}[!hbt]
\centering
\caption{Chesterman's translation norms}
\label{tab:chestnorms}
\bigskip
\begin{tabular}{>{\raggedright\arraybackslash\bfseries}p{0.25\linewidth}>{\raggedright\arraybackslash}p{0.6\linewidth}} \toprule
Product or expectancy norms & What the readers expect of the TT. They relate to translation tradition, prevailing genre, discourse conventions, ideology, etc. \\
\midrule
Professional norms & \textbf{- Accountability norm} is ethical; the translator accepts responsibility. \\
& \textbf{- Communication norm} is social; translator is expert.\\
& \textbf{- Relation norm} is linguistic; judged according to text type, ST author intentions and needs of TT readers. \\
\bottomrule
\end{tabular}
\end{table}

Product or expectancy norms are regulated by the expectation of the readers concerning what a translation should be like. The factors contributing to these norms are: predominant translation tradition, the discourse conventions of the similar TL genre, as well as economic and ideological considerations. These norms allow evaluative judgements. That is to say, the readers can say that a product of translation is `good' or `bad' according to the given criteria. For example, with readability, we can say a translation is `hard' or `easy' to read. These norms also constitute authorities who can judge the work by the norms, such as literary critics, teachers, and readers of publishing houses. 

Professional norms are subordinate to and determined by expectancy norms. They regulate the translation process. Three kinds of professional norms are mentioned:
\begin{compactenum}[(1)]
\item The accountability norm is an ethical norm. It is about professional standards that make the translator accept responsibility for the product. \item The communication norm is a social norm. As communication expert, the translator should optimize the communication between the parties.
\item The relation norm is a linguistic norm to establish and maintain an appropriate relation of relevant similarity between ST and TT, according to ``the text-type, the wishes of the commissioner, the intentions of the original writer, and the assumed needs of the prospective readers.''\footnote{\citealp[p.~67]{chesterman:memes}}
\end{compactenum}

Still, Chesterman's normative model looks very conceptual and difficult to put into practice. To make our life easier, I suggest that we can divide translation norms into two parts. The first part concerns the product of translation, and the second part concerns the translator itself.

An important guideline of how to deal with ethical choices comes from Lawrence Venuti. He introduces two levels of assessment. The first uses the scale of \emph{domestication} and \emph{foreignization}, and the second \emph{fluency} and \emph{resistancy}. I summarize these in Figure \ref{fig:domestication}.\footnote{adapted from \citealp[p.~228]{munday:translation}}

\begin{figure}[!hbt]
\centering
\setlength{\unitlength}{1mm}
\begin{picture}(80,55)(0,0)
\put(40,50){\makebox(0,0)[c]{\textbf{Ethical level}}}
\put(40,45){\vector(1,0){15}}
\put(40,45){\vector(-1,0){15}}
\put(23,45){\makebox(0,0)[r]{\textbf{domestication}}}
\put(-2,36){\parbox[b]{28mm}{\small\raggedright (conforming to TL culture values)}}
\put(57,45){\makebox(0,0)[l]{\textbf{foreignization}}}
\put(57,36){\parbox[b]{25mm}{\small\raggedright (making visible the foreign)}}
%%
\put(40,25){\makebox(0,0)[c]{\textbf{Discursive level}}}
\put(40,20){\vector(1,0){15}}
\put(40,20){\vector(-1,0){15}}
\put(23,20){\makebox(0,0)[r]{\textbf{fluency}}}
\put(-2,7){\parbox[b]{30mm}{\small\raggedright (transparent reading assimilated to TL norms)}}
\put(57,20){\makebox(0,0)[l]{\textbf{resistancy}}}
\put(57,7){\parbox[b]{25mm}{\small\raggedright (resistant reading challenging TL norms)}}
\end{picture}
\caption{Venuti's model of ethical choices}
\label{fig:domestication}
\end{figure}

Here is the explanation given by Venuti:

\begin{quote}
The terms `domestication' and `foreignization' indicate fundamentally \emph{ethical} attitudes towards a foreign text and culture, ethical effects produced by the choice of a text for translation and by the strategy devised to translate it, whereas terms like `fluency' and `resistancy' indicate fundamentally \emph{discursive} features of translation strategies in relation to the reader's cognitive processing.\footnote{\citealp[p.~19]{venuti:invisibility}}
\end{quote}

To illustrate, suppose, we are translating a P\=ali sutta that mentions women. On ethical level, if we use our current norm of gender equality and translate the text by portraying that equality, this is domestication. On the other hand, if we adopt the norm at that time and make visible that inequality, this is foreignization.\footnote{We can also see these as translation strategies. Domestication is `author-to-reader' strategy---the source text is adapted to the conventions of the target language. And foreignization is `reader-to-author' strategy---the foreign elements are maintained in order that the target audience is exposed to cultural difference. See \citealp[p.~549]{wyke:ethics}. In \citealp{venuti:scandals}, foreignization is related to \emph{minoritizing}.} The scale is not binary, so we can set the intensity which pole should be approached, in what degree.

On discursive level, if the product of translation can be read smoothly as if the translator was the author itself, it enables transparent reading that comes close to the fluency pole. But if the translation is full of foreign elements alien to the reader's context, it entails resistant reading that comes close to the resistancy pole. The later approach makes the translator visible as the mediator.

Making the translator visible is important to the ethics, because the translator cannot avoid the responsibility of meaning production and keep itself behind the curtain like a puppeteer.

Choices we make according to the scales reflect our ethical stance of the translation, as well as the interest and constraint of that. There is no best solution for all situations. We have to position ourselves properly corresponding to our work environment and target readers. For example, positions close to domestication and fluency pole look suitable for pedagogic purpose and young readers. Moving more towards foreignization and resistancy pole may be more suitable for academic purpose and learned readers.
