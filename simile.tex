\chapter{Simile}\label{chap:simile}

P\=ali texts use simile (\pali{upam\=a}) quite a lot. So, it should have a chapter of its own. Basically, we use two particles to mark a simile: \pali{viya} and \pali{iva} (often fused to the preceding word as \pali{-va}).\footnote{Another form of simile can be found in correlative sentences with \pali{yath\=a-tath\=a} pair (see Chapter \ref{chap:yata}).} That means detecting a simile is supposed to be easy. According to Thai P\=ali teachers, there are four kinds of simile, described as follows:

\subsection*{1.\ Simile using subject modifier (in nom.)}

This kind of simile has a phrase or clause in nominative case embedded for comparison. Here are some examples:\par
- \pali{So se\d t\d thino akkhimhi ka\d n\d tako \textbf{viya} kh\=ayi}\footnote{Dhp-a\,2.21} (He appeared like a thorn in the rich man's eyes.)\par
- \pali{\=ananda, may\=a kathitadhammo n\=ama sakkaccamasu\d nantassa agga\d nhantassa asajjh\=ayantassa adesentassa va\d n\d nasampanna\d m ag\-andhakapuppha\d m \textbf{viya} aphalo hoti}\footnote{Dhp-a\,4.51} (\=Ananda, such a teaching said by me is fruitless for [the person] not listening carefully, not learning, not rehearsing, not expounding; like a colorful [but] scentless flower.)\par
- \pali{Ra\~n\~no pu\d n\d nacando val\=ahakantara\d m pavi\d t\d tho \textbf{viya} upa\d t\d th\=asi}\footnote{Dhp-a\,5.60} ([The girl] appeared to the king like the full moon enterred in between clouds.)\par
- \pali{Tato na\d m dukkhamanveti, cakka\d m\textbf{va} vahato pada\d m}\footnote{Dhp\,1.1} (From that [action], suffering follows [him/her] like the wheel [follows] the foot-step [of an ox] carrying [a load].)\par
- \pali{mama dve putt\=a aggikkhandh\=a \textbf{viya} jalant\=a vicaranti}\footnote{Dhp-a\,10.141} (My two sons, shining like heaps of fire, wander.)\par
- \pali{Mah\=aduggato tikhi\d n\=aya sattiy\=a kucchiya\d m paha\d to \textbf{viya}, ``s\=ami, kasm\=a ma\d m n\=asesi, \ldots''}\footnote{Dhp-a\,6.80} (Mah\=aduggata, [feeling] like being struck with a sharp spear in the stomach, [said], ``Master, why do you ruin me? \ldots'')\par

\subsection*{2.\ Simile using modifier phrases (in other cases)}

This is close to the previous one, but it appears in other cases except nominative\footnote{To make our life simpler, we can put these two groups together. But for a technical reason, I separate these two. The first is called \pali{upam\=ali\.ngattha} by P\=ali teachers, whereas this kind of simile is called \pali{upam\=avisesana}.}, for example:\par
- \pali{So pabbatena \textbf{viya} mahantena sokena avattha\d to hutv\=a anappaka\d m domanassa\d m pa\d tisa\d mvedesi.}\footnote{Dhp-a\,2.21} (He underwent non-trivial grief, overpowered by mountain-like great sorrow.)\par
- \pali{surattadupa\d t\d ta\d m niv\=asetv\=a vijjulata\d m \textbf{viya} k\=ayabandhana\d m bandhitv\=a}\footnote{Dhp-a\,2.25} ([The Buddha], having put on a well-dyed, two-layered cloth, having tied a lightning-like waist-band.)\par
- \pali{Hirottappasampann\=a kuladh\=it\=a paccorasmi\d m sattipah\=ara\d m \textbf{viya} va\d ne kh\=arodakasecanaka\d m \textbf{viya} ca patv\=a ``ki\d m, s\=am\=i''ti \=aha.}\footnote{Dhp-a\,2.21. Interesting words here are \pali{paccora} (\pali{pa\d ti + ura}), and \pali{kh\=arodakasecanaka} (\pali{kh\=ara + udaka + secanaka}).} (The daughter [of a respectful family], full of shame and fear for evil doing, having felt like hitting by a spear in the chest, like sprinkling of potash on the wound, said thus, ``What, master?'')\par

\subsection*{3.\ Simile using subject complement}

This also works like modifier but its sentence or clause is finished by verb `to be,' which sometimes is omitted. As you might guess, if you have read through Chapter \ref{chap:pos}, this is called \pali{upam\=avikatikatt\=a}. Here are some examples:\par
- \pali{nicca\d m so\d lasavassuddesik\=a \textbf{viya} ahosi}\footnote{Dhp-a\,4.53} ([Vis\=akh\=a] was always like [one] of the age of 16.)\par
- \pali{Asa\.nkhyeyya\d m \=ayu\d m gahetv\=a nibbatt\=a \textbf{viya} ajar\=amar\=a \textbf{viya} ca nicca\d m pamatt\=a, manuss\=a [honti]}\footnote{Dhp-a\,4.48} (Human beings [are] constantly careless, [thinking] like being born with incalculable age, and like decayless and deathless [beings].)\par
- \pali{M\=aga\d n\d diy\=a apar\=apara\d m vicarant\=i \textbf{viya} hutv\=a v\=i\d n\=achiddato pupphagu\d la\d m apanesi}\footnote{Dhp-a\,2.21} (M\=aga\d n\d diy\=a, acting like walking back and forth, took away the bunch of flower from the lute's hole.)\footnote{Interestingly, apart from denoting a metaphor, \pali{viya} (and \pali{eva} supposedly) can denote a pretension or fake action.}\par
- \pali{ekav\=ara\d m t\=alapattave\d thako \textbf{viya} j\=ato}\footnote{Dhp-a\,8.110. This instance shows that sometimes verbs of root \pali{jana} can take a complement.} (In one occasion, [the sword] became like a wrapped palm leaf.)\par

\subsection*{4.\ Simile using object complement}

As you may get the idea from Chapter \ref{chap:pos}, this has something to do with verb `to do/make.' It is called \pali{upam\=avikatikamma}. And here are examples:\par
- \pali{tesa\d m tesa\d m vay\=anur\=upa\d m piyavacana\d m vatv\=a pa\d n\d n\=ak\=ara\d m pesent\=i sakalanagarav\=asino \~n\=atake \textbf{viya} ak\=asi}\footnote{Dhp-a\,4.53} ([Vis\=akh\=a], having said pleasant words suitable to ages for those [people], sending out a present, has made people of the whole city like relatives.)\par
- \pali{``Aggi paricaritabbo''ti ida\d m pana sassumpi sasurampi s\=amika\-mpi aggikkhandha\d m \textbf{viya} uragar\=aj\=ana\d m \textbf{viya} ca katv\=a passitu\d m va\d t\d tati.}\footnote{Dhp-a\,4.53} (This ``Fire should be worshiped'' [means] it is suitable to see mother-in-law, father-in-law, and the husband like doing [a worship to] a heap of fire and the serpent king.)\par

