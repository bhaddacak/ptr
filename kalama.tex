\chapter{K\=al\=amasutta}\label{chap:kalama}

In this chapter we will study closely a popular sutta, widely known as \pali{K\=al\=amasutta}. It is interesting, to me, for a number of reasons. First, this sutta shows an attitude towards rationality in a direct way, contrasting to many other appealing-to-faith or just-take-it suttas. This account should be taken seriously in the modern time, because it can be an antidote for unhealthy irrationality. Second, the sutta looks very old by its structure, having several repetitive parts. Third, it has moderate length and quite easy to read.

\phantomsection
\addcontentsline{toc}{section}{Pre-reading introduction}
\section*{1.\ Pre-reading introduction}

\paragraph*{About the text} The official name of this sutta is \pali{Kesamuttisutta}.\footnote{This name appears in Myanmar edition. In Thai edition, it is \pali{Kesaputtisutta} instead.} K\=al\=ama is the name of clan or tribe who live in that area. The text belongs to the three-fold collection of A\.ngutaranik\=aya, Suttantapi\d taka. It is the fifth (of 10) sutta in the seventh (of 18) group (Mah\=avagga). In the text collection with my referencing scheme, it is A3\,66.

\paragraph*{About the author} This can be the same as main suttas exemplified earlier. The text sender is the Sangha of an early council. And the text producer is the compilers in that council.

\paragraph*{About the audience} Basically we can say that the text was targeted to Buddhist monks and lay people.
 
\paragraph*{About time and place} The exact time of the event is unknown. The place of the event was named Kesamutta or Kesaputta, somewhere in Kosala. For the text itself, it was possibly produced in an early compilation, according to its form.

\paragraph*{About motives} To preserve the teaching is the best we can say.

\paragraph*{About text function} Informative function is conspicuous in this text. But the emphasis on morality also makes operative function visible. 

\phantomsection
\addcontentsline{toc}{section}{Reading with a draft translation}
\section*{2.\ Reading with a draft translation}

Like the previous example, I will insert my translations and notes paragraph by paragraph to comfort the readers.

\bigskip
\begin{center}
\textbf{\pali{Kesamuttisutta\d m}}\par
$\triangleright$ A discourse concerning\\those who live in Kesamutta
\end{center}

\setcounter{sennum}{0}
\pali{\fbox{\stepcounter{sennum}\arabic{sennum}} Eva\d m me suta\d m -- eka\d m samaya\d m bhagav\=a kosalesu c\=arika\d m caram\=ano mahat\=a bhikkhusa\.nghena saddhi\d m yena kesamutta\d m n\=ama k\=al\=am\=ana\d m nigamo tadavasari. \fbox{\stepcounter{sennum}\arabic{sennum}} Assosu\d m kho kesamuttiy\=a k\=al\=am\=a -- ``sama\d no khalu, bho, gotamo sakyaputto sakyakul\=a pabbajito kesamutta\d m anuppatto.}

\addtocounter{sennum}{-2}
$\triangleright$ \fbox{\stepcounter{sennum}\arabic{sennum}} It is heard by me thus, in one occasion the Blessed One, traveling in Kosala together with a big group of monks, [by] which town of the K\=al\=amas named Kesamutta [is located], he arrived there. \fbox{\stepcounter{sennum}\arabic{sennum}} The K\=al\=amas in Kesamutta have heard that ``Ascetic Gotama, a son of the S\=akya, having gone forth from the S\=akya clan, has reached Kesamutta.\\

\begin{longtable}[c]{|p{0.9\linewidth}|}
\hline
\hspace{5mm}\small This opening conforms to a familiar template. We find a \pali{ya-ta} structure here in form of \pali{yena-ta\d m}, thus \pali{tadavasari} = \pali{ta\d m + avasari}. In fact, \pali{yena} should be \pali{yo} to agree with \pali{nigamo}. So, this use looks idiomatic here. By contructing from a template, \pali{bho} looks out of context here, because by its form \pali{bho} is vocative. It should not be used here, for this is not a dialogue.\\
\hline
\end{longtable}

\pali{\fbox{\stepcounter{sennum}\arabic{sennum}} Ta\d m kho pana bhavanta\d m gotama\d m eva\d m kaly\=a\d no kittisaddo abbhuggato -- \fbox{\refstepcounter{sennum}\arabic{sennum}\label{sen:itipi}} `itipi so bhagav\=a araha\d m samm\=asambuddho vijj\=acara\d nasampanno sugato lokavid\=u anuttaro purisadammas\=arathi satth\=a devamanuss\=ana\d m buddho bhagav\=a. \fbox{\refstepcounter{sennum}\arabic{sennum}\label{sen:soimam}} So ima\d m loka\d m sadevaka\d m sam\=araka\d m sabrahmaka\d m sassama\d nabr\=ahma\d ni\d m paja\d m sadevamanussa\d m saya\d m abhi\~n\~n\=a sacchikatv\=a pavedeti. \fbox{\stepcounter{sennum}\arabic{sennum}} So dhamma\d m deseti \=adikaly\=a\d na\d m majjhekaly\=a\d na\d m pariyos\=anakaly\=a\d na\d m s\=attha\d m sabya\~njana\d m kevalaparipu\d n\d na\d m parisuddha\d m brahmacariya\d m pak\=aseti; \fbox{\stepcounter{sennum}\arabic{sennum}} s\=adhu kho pana tath\=ar\=up\=ana\d m arahata\d m dassana\d m hot\=i'\,''ti.}

\addtocounter{sennum}{-5}
$\triangleright$ \fbox{\stepcounter{sennum}\arabic{sennum}} That Venerable Gotama [has] a charming reputation diffused thus: \fbox{\stepcounter{sennum}\arabic{sennum}} The Buddha [is] an arahant, the perfectly Enlightened One, one who has perfect wisdom and virtue, one who has gone well, one who knows the world, the Unsurpassed One, the trainer of humans, the teacher of gods and humans, the Enlightened One, [and] the Blessed One. \fbox{\stepcounter{sennum}\arabic{sennum}} He, having realized this world including gods, M\=aras, Brahm\=as, ascetics and brahmans, mankind, gods and humans by his own special knowledge, [then] makes [it] known. \fbox{\stepcounter{sennum}\arabic{sennum}} He expounds the Dhamma beautifully in the beginning, in the middle, and in the end. [He] makes known completely the purified religious life including sense and letters. \fbox{\stepcounter{sennum}\arabic{sennum}} Seeing arhants as such is good.\\

\begin{longtable}[c]{|p{0.9\linewidth}|}
\hline
\hspace{5mm}\small Sooner or later P\=ali students have to come across this stock passage. In the text we have, it is cut short by ellipsis. I reproduce it fully because it is really worth reading. This can show us how an old P\=ali text looks like. The passage may be primarily used in chanting as most Buddhists do today. It is clearly a product of oral transmission.\\
\hspace{5mm}\small We can drop \pali{itipi} when translating because it just stands for a quotation mark. In Thai translation, the word is translated nonetheless confusingly as ``Even by this reason.''\footnote{We can find this Thai translation in the MCU edition of the P\=ali canon.} Even if \pali{pi} can be seen as an emphatic particle, there is no suitable English word here.\\
\hspace{5mm}\small In \fbox{\ref{sen:itipi}}, we have two instances of \pali{bhagav\=a}. They mean different things. The first \pali{bhagav\=a} refers to the Buddha, so I translate it as `the Buddha.' The second one means `a fortunate one,' which is normally used to call the Buddha as `the Blessed One.'\\
\hspace{5mm}\small That is to say, we use the first \pali{bhagav\=a} for referencing and use the second one for meaning.\footnote{See a discussion on sense and reference in Chapter \ref{chap:howtextmean}.} Knowing this distinction may cause you a headache, but it can give you a sharp eye when you read anything. Mistaking one for another can cause a reading or translation failure, ending up with a pointless argumentation.\\
\hspace{5mm}\small In \fbox{\ref{sen:soimam}}, the sentence started with \pali{So ima\d m} is the most interesting here. Be careful, the object of \pali{sacchikatv\=a} is not \pali{abhi\~n\~n\=a} but \pali{ima\d m loka\d m} and all following accusative words, except \pali{saya\d m} (adv.). The full form of \pali{abhi\~n\~n\=a} is actually \pali{abhi\~n\~n\=aya}, so by this meaning it is in instrumental case.\footnote{See \pali{abhij\=an\=ati} in PTSD.}\\
\hline
\end{longtable}

\pali{\fbox{\stepcounter{sennum}\arabic{sennum}} Atha kho kesamuttiy\=a k\=al\=am\=a yena bhagav\=a tenupasa\.nkami\-\d msu; upasa\.nkamitv\=a appekacce bhagavanta\d m abhiv\=adetv\=a ekamanta\d m nis\=idi\d msu, appekacce bhagavat\=a saddhi\d m sammodi\d msu, sammodan\=iya\d m katha\d m s\=ara\d n\=iya\d m v\=itis\=aretv\=a ekamanta\d m nis\=idi\d msu, appekacce yena bhagav\=a tena\~njali\d m pa\d n\=ametv\=a ekamanta\d m nis\=idi\-\d msu, appekacce n\=amagotta\d m s\=avetv\=a ekamanta\d m nis\=idi\d msu, appe\-kacce tu\d nh\=ibh\=ut\=a ekamanta\d m nis\=idi\d msu. \fbox{\stepcounter{sennum}\arabic{sennum}} Ekamanta\d m nisinn\=a kho te kesamuttiy\=a k\=al\=am\=a bhagavanta\d m etadavocu\d m --}

\addtocounter{sennum}{-2}
$\triangleright$ \fbox{\stepcounter{sennum}\arabic{sennum}} Then the K\=al\=amas of Kesamutta approached where the Blessed One [lived]. Having approached [the Buddha], some, having bowed down to the Blessed One, [then] sat in one side; some exchanged greetings with the Blessed One; having greeted [him] with pleasant, reminding speech, [then they] sat in one side; some, having saluted the Blessed One with lotus-shaped palms towards where [he sat], [then] sat in one side; some, having announced [their] name and clan, [then] sat in one side; some sat in one side in silence. \fbox{\stepcounter{sennum}\arabic{sennum}} Having sat in one side, those K\=al\=amas of Kesamutta said to the Blessed One thus:\\

\begin{longtable}[c]{|p{0.9\linewidth}|}
\hline
\hspace{5mm}\small Some confusing words for new learners here are \pali{appekacce} (\pali{api + ekacce})\footnote{See \pali{api} in PTSD.}, and \pali{v\=itis\=aretv\=a} (\pali{vi + ati + s\=aretv\=a})\footnote{See \pali{v\=iti} in PTSD}. \\
\hline
\end{longtable}

\pali{\fbox{\stepcounter{sennum}\arabic{sennum}} Santi, bhante, eke sama\d nabr\=ahma\d n\=a kesamutta\d m \=agacchanti. Te saka\d myeva v\=ada\d m d\=ipenti jotenti, parappav\=ada\d m pana khu\d msenti vambhenti paribhavanti omakkhi\d m karonti. \fbox{\stepcounter{sennum}\arabic{sennum}} Aparepi, bhante, eke sama\d nabr\=ahma\d n\=a kesamutta\d m \=agacchanti. Tepi saka\d m\-yeva v\=ada\d m d\=ipenti jotenti, parappav\=ada\d m pana khu\d msenti vambhenti paribhavanti omakkhi\d m karonti. \fbox{\refstepcounter{sennum}\arabic{sennum}\label{sen:tesadmno}} Tesa\d m no, bhante, amh\=aka\d m hoteva ka\.nkh\=a hoti vicikicch\=a -- `ko su n\=ama imesa\d m bhavata\d m sama\d nabr\=ahma\d n\=ana\d m sacca\d m \=aha, ko mus\=a'ti? \fbox{\stepcounter{sennum}\arabic{sennum}} Ala\~nhi vo, k\=al\=am\=a, ka\.nkhitu\d m ala\d m vicikicchitu\d m. \fbox{\stepcounter{sennum}\arabic{sennum}} Ka\.nkhan\=i\-yeva pana vo \d th\=ane vicikicch\=a uppann\=a.}

\addtocounter{sennum}{-5}
$\triangleright$ \fbox{\stepcounter{sennum}\arabic{sennum}} There are [those], sir. Some ascetics and brahmans come to Kesamutta. They explain [and] elucidate only their own view, but scold, despise, abuse, [and] make inferior other view. \fbox{\stepcounter{sennum}\arabic{sennum}} Yet, sir, some other ascetics and brahmans come to Kesamutta. They also explain [and] elucidate only their own view, but scold, despise, abuse, [and] make inferior other view. \fbox{\stepcounter{sennum}\arabic{sennum}} Sir, only doubt [and] uncertainty [in] their [views] happens to us thus, ``As for such these Venerable ascetics and brahmans, who said the truth, who lied?'' \fbox{\stepcounter{sennum}\arabic{sennum}} ``It is suitable, K\=al\=amas, to doubt, to be uncertain by you.'' \fbox{\stepcounter{sennum}\arabic{sennum}} ``The uncertainty in a doubtworthy case happened to you.''\\

\begin{longtable}[c]{|p{0.9\linewidth}|}
\hline
\hspace{5mm}\small To keep the structure, I split the first sentence into two English sentences, because there are two P\=ali verbs here.\\
\hspace{5mm}\small In \fbox{\ref{sen:tesadmno}}, \pali{no} looks redundant. An omitted noun related to \pali{tesa\d m} is \pali{v\=adesu} (loc.). By its arrangement, \pali{amh\=aka\d m} is dative, not genitive. If it is genitive, it would be ``\pali{hoti eva amh\=aka\d m ka\.nkh\=a}'' (Only our doubt happens) instead. In the quotation, it will be easier to rearrange the sentence to ``\pali{ko sacca\d m \=aha, ko mus\=a}'' and the rest. An awkward word here is \pali{n\=ama}. It is always difficult to put in English if it does not simply mean `name.' I mostly use `such' in this kind of use. It sounds close to me.\\
\hline
\end{longtable}

\pali{\fbox{\stepcounter{sennum}\arabic{sennum}} Etha tumhe, k\=al\=am\=a, m\=a anussavena, m\=a parampar\=aya, m\=a itikir\=aya, m\=a pi\d takasampad\=anena, m\=a takkahetu, m\=a nayahetu, m\=a \=ak\=araparivitakkena, m\=a di\d t\d thinijjh\=anakkhantiy\=a, m\=a bhabbar\=upat\=aya, m\=a sama\d no no gar\=u'ti.} 

\addtocounter{sennum}{-1}
$\triangleright$ \fbox{\stepcounter{sennum}\arabic{sennum}} Come, K\=al\=amas. Do not [take merely] by hearsay. Do not [take merely] by traditional practice. Do not [take merely] by hearing thus. Do not [take merely] by referring to text. Do not [take merely] by thinking. Do not [take merely] by inference. Do not [take merely] by appearance consideration. Do not [take merely] by agreement with our view. Do not [take merely] by appearance of likelihood. Do not [take merely] by [seeing this] ascetic as [our] teacher.\\

\begin{longtable}[c]{|p{0.9\linewidth}|}
\hline
\hspace{5mm}\small The full form of \pali{m\=a} sentences suggested by the commentary is, for example: \pali{M\=a anussaven\=a'ti anussavakath\=aya\textbf{pi} m\=a ga\d nhittha.} A particle worth noting in this explanation is \pali{pi}. It can mean either `even' or `also.' If we take the former, it sounds too strong, hence ``Do not take \emph{even} by hearsay.'' So, we should take the latter sense, yielding ``Do not take also by hearsay.'' This makes sense only when multiple items are in the list. A more suitable word in this context, I think, is \pali{eva} (only, merely). So, I render them by this notion.\\
\hspace{5mm}\small The meaning of \pali{anussavena} and \pali{itikir\=aya} is close, or maybe synonymous.\\
\hspace{5mm}\dag\ \small The use of \pali{pi\d taka} to mean text or canon is noteworthy here. Normally, in the canon we find the word is used for `basket.' Using \pali{pi\d taka} to denote a collection of text makes sense only when a compilation occurs. This suggests that perhaps the sutta may not be old as we expect. Or maybe it is old but the word was changed to make it more understandable in later time.\\
\hspace{5mm}\small \pali{Takkahetu} by no means refers to any system of logic, because there was no such thing in the Buddha's time. It also cannot mean `reasoning' in modern sense, because, as we shall see, the Buddha encouraged the use of reasoning in certain way, and he himself mainly taught by reasoning. So, \pali{takkahetu} should mean simply thinking, or better speculation.\\
\hspace{5mm}\small If a kind of reasoning is meant, it is \pali{nayahetu} or inference. It is like drawing a conclusion from information we already have. if done properly, it can give us knowledge. So, in this context it should mean `bad' inference, a kind of hasty jump to conclusion, I think.\\
\hspace{5mm}\small In the commentary, the explanation given for \pali{\=ak\=araparivitakkena} is ``\pali{sundaramida\d m k\=ara\d nan'ti eva\d m k\=ara\d naparivitakkenapi m\=a ga\d nhittha}.'' It sounds like we say ``This reason sounds good/nice, so it must be true.''\\
\hspace{5mm}\small For \pali{di\d t\d thinijjh\=anakkhantiy\=a}, the commentary says ``\pali{amh\=aka\d m [di\d t\d thi\d m] nijjh\=ayitv\=a khamitv\=a gahitadi\d t\d thiy\=a saddhi\d m sameti.''} It sounds like when we say ``That is what I have thought/meant.'' See also \pali{nijjh\=ana} in PTSD.\\
\hspace{5mm}\small For \pali{bhabbar\=upat\=aya}, it is explained as ``\pali{aya\d m bhikkhu bhabbar\=upo, imassa katha\d m gahetu\d m yutta\d m}.'' This sounds like appealing to authority thus, ``This monk is capable, so taking his speech is suitable.'' See also \pali{r\=upat\=a} in PTSD.\\
\hline
\end{longtable}

\pali{\fbox{\stepcounter{sennum}\arabic{sennum}} Yad\=a tumhe, k\=al\=am\=a, attan\=ava j\=aneyy\=atha -- \ `ime dhamm\=a akusal\=a, ime dhamm\=a s\=avajj\=a, ime dhamm\=a vi\~n\~nugarahit\=a, ime dhamm\=a samatt\=a sam\=adinn\=a ahit\=aya dukkh\=aya sa\d mvattant\=i'ti, atha tumhe, k\=al\=am\=a, pajaheyy\=atha.}

\addtocounter{sennum}{-1}
$\triangleright$ \fbox{\stepcounter{sennum}\arabic{sennum}} Inasmuch as, K\=al\=amas, you know only by yourselves thus, ``These teachings [are] unskillful. These teachings [are] blameworthy. These teachings [are] reproached by the wise. These teachings, [when] completely taken upon, lead to uselessness [and] suffering.'' Then, K\=al\=amas, you should abandon [them].\\

\begin{longtable}[c]{|p{0.9\linewidth}|}
\hline
\hspace{5mm}\dag\ \small After warning against various unreliable sources of knowledge, The Buddha went on proposing his own idea as described. Here, \pali{akusala} can also mean `wrong' or `foolish.' Then circular questions come: Who will judge whether a certain idea is wrong or foolish or blameworthy? What kind of the wise should be, if not our teachers? Can we trust ourselves on this matter after all? The answer may lie ahead, read on.\\
\hline
\end{longtable}

\pali{\fbox{\stepcounter{sennum}\arabic{sennum}} Ta\d m ki\d m ma\~n\~natha, k\=al\=am\=a, lobho purisassa ajjhatta\d m uppajjam\=ano uppajjati hit\=aya v\=a ahit\=aya v\=a'ti? \fbox{\stepcounter{sennum}\arabic{sennum}} Ahit\=aya, bhante.}

\addtocounter{sennum}{-2}
$\triangleright$ \fbox{\stepcounter{sennum}\arabic{sennum}} How do you think about that, K\=al\=amas? [When] arising inside a person, does greed arise for usefulness or uselessness? \fbox{\stepcounter{sennum}\arabic{sennum}} Uselessness, sir.\\

\pali{\fbox{\stepcounter{sennum}\arabic{sennum}} Luddho pan\=aya\d m, k\=al\=am\=a, purisapuggalo lobhena abhibh\=uto pariy\=adinnacitto p\=a\d nampi hanati, adinnampi \=adiyati, parad\=a\-rampi gacchati, mus\=api bha\d nati, parampi tathatt\=aya sam\=adapeti, ya\d m sa hoti d\=igharatta\d m ahit\=aya dukkh\=ay\=a'ti. \fbox{\stepcounter{sennum}\arabic{sennum}} Eva\d m, bhante.}

\addtocounter{sennum}{-2}
$\triangleright$ \fbox{\stepcounter{sennum}\arabic{sennum}} K\=al\=amas, this greedy person, overpowered [and] fully taken up by greed, kills a living being, takes the ungiven [things], goes to other's wife, tells a lie, and makes others do likewise. [When] that occurs, [it is] for uselessness [and] suffering  for a long time. \fbox{\stepcounter{sennum}\arabic{sennum}} It is so, sir.\\

\begin{longtable}[c]{|p{0.9\linewidth}|}
\hline
\hspace{5mm}\small As noted in the text, \pali{tathatt\=aya} is possibly \pali{tadatth\=aya} (for that benefit). And isolated \pali{sa} looks unusual here. It is possibly \pali{tassa}, but the redactor notes that only `\pali{ya\d m sa}' is found throughout the text.\\
\hline
\end{longtable}

\pali{\fbox{\stepcounter{sennum}\arabic{sennum}} Ta\d m ki\d m ma\~n\~natha, k\=al\=am\=a, doso purisassa ajjhatta\d m uppajjam\=ano uppajjati hit\=aya v\=a ahit\=aya v\=a'ti? \fbox{\stepcounter{sennum}\arabic{sennum}} Ahit\=aya, bhante.}

\addtocounter{sennum}{-2}
$\triangleright$ \fbox{\stepcounter{sennum}\arabic{sennum}} How do you think about that, K\=al\=amas? [When] arising inside a person, does anger arise for usefulness or uselessness? \fbox{\stepcounter{sennum}\arabic{sennum}} Uselessness, sir.\\

\pali{\fbox{\stepcounter{sennum}\arabic{sennum}} Du\d t\d tho pan\=aya\d m, k\=al\=am\=a, purisapuggalo dosena abhibh\=uto pariy\=adinnacitto p\=a\d nampi hanati, adinnampi \=adiyati, parad\=arampi gacchati, mus\=api bha\d nati, parampi tathatt\=aya sam\=adapeti, ya\d m sa hoti d\=igharatta\d m ahit\=aya dukkh\=ay\=a'ti. \fbox{\stepcounter{sennum}\arabic{sennum}} Eva\d m, bhante.}

\addtocounter{sennum}{-2}
$\triangleright$ \fbox{\stepcounter{sennum}\arabic{sennum}} K\=al\=amas, this angry person, overpowered [and] fully taken up by anger, kills a living being, takes the ungiven [things], goes to other's wife, tells a lie, and makes others do likewise. [When] that occurs, [it is] for uselessness [and] suffering  for a long time. \fbox{\stepcounter{sennum}\arabic{sennum}} It is so, sir.\\

\pali{\fbox{\stepcounter{sennum}\arabic{sennum}} Ta\d m ki\d m ma\~n\~natha, k\=al\=am\=a, moho purisassa ajjhatta\d m uppajjam\=ano uppajjati hit\=aya v\=a ahit\=aya v\=a'ti? \fbox{\stepcounter{sennum}\arabic{sennum}} Ahit\=aya, bhante.}

\addtocounter{sennum}{-2}
$\triangleright$ \fbox{\stepcounter{sennum}\arabic{sennum}} How do you think about that, K\=al\=amas? [When] arising inside a person, does delusion arise for usefulness or uselessness? \fbox{\stepcounter{sennum}\arabic{sennum}} Uselessness, sir.\\

\pali{\fbox{\stepcounter{sennum}\arabic{sennum}} M\=u\d lho pan\=aya\d m, k\=al\=am\=a, purisapuggalo mohena abhibh\=uto pariy\=adinnacitto p\=a\d nampi hanati, adinnampi \=adiyati, parad\=arampi gacchati, mus\=api bha\d nati, parampi tathatt\=aya sam\=adapeti, ya\d m sa hoti d\=igharatta\d m ahit\=aya dukkh\=ay\=a'ti. \fbox{\stepcounter{sennum}\arabic{sennum}} Eva\d m, bhante.}

\addtocounter{sennum}{-2}
$\triangleright$ \fbox{\stepcounter{sennum}\arabic{sennum}} K\=al\=amas, this deluded person, overpowered [and] fully taken up by delusion, kills a living being, takes the ungiven [things], goes to other's wife, tells a lie, and makes others do likewise. [When] that occurs, [it is] for uselessness [and] suffering  for a long time. \fbox{\stepcounter{sennum}\arabic{sennum}} It is so, sir.\\

\pali{\fbox{\stepcounter{sennum}\arabic{sennum}} Ta\d m ki\d m ma\~n\~natha, k\=al\=am\=a, ime dhamm\=a kusal\=a v\=a akusal\=a v\=a'ti? \fbox{\stepcounter{sennum}\arabic{sennum}} Akusal\=a, bhante.}

\addtocounter{sennum}{-2}
$\triangleright$ \fbox{\stepcounter{sennum}\arabic{sennum}} How do you think about that, K\=al\=amas? Are these kinds of nature skillful or unskillful? \fbox{\stepcounter{sennum}\arabic{sennum}} Unskillful, sir.\\

\pali{\fbox{\stepcounter{sennum}\arabic{sennum}} S\=avajj\=a v\=a anavajj\=a v\=a'ti? \fbox{\stepcounter{sennum}\arabic{sennum}} S\=avajj\=a, bhante.}

\addtocounter{sennum}{-2}
$\triangleright$ \fbox{\stepcounter{sennum}\arabic{sennum}} [Are they] blameworthy or praiseworthy? \fbox{\stepcounter{sennum}\arabic{sennum}} Blameworthy, sir.\\

\pali{\fbox{\stepcounter{sennum}\arabic{sennum}} Vi\~n\~nugarahit\=a v\=a vi\~n\~nuppasatth\=a v\=a'ti? \fbox{\stepcounter{sennum}\arabic{sennum}} Vi\~n\~nugarahit\=a, bhante.}

\addtocounter{sennum}{-2}
$\triangleright$ \fbox{\stepcounter{sennum}\arabic{sennum}} [Are they] reproached or praised by the wise? \fbox{\stepcounter{sennum}\arabic{sennum}} Reproached by the wise, sir.\\

\pali{\fbox{\stepcounter{sennum}\arabic{sennum}} Samatt\=a sam\=adinn\=a ahit\=aya dukkh\=aya sa\d mvattanti, no v\=a? Katha\d m v\=a ettha hot\=i'ti. \fbox{\stepcounter{sennum}\arabic{sennum}} Samatt\=a, bhante, sam\=adinn\=a ahit\=aya dukkh\=aya sa\d mvattant\=i'ti. Eva\d m no ettha hot\=i'ti.}

\addtocounter{sennum}{-2}
$\triangleright$ \fbox{\stepcounter{sennum}\arabic{sennum}} [When] completely taken upon, [do they] lead to uselessness [and] suffering or not? Or how [do you think] about this? \fbox{\stepcounter{sennum}\arabic{sennum}} [When] completely taken upon, sir, [they] lead to uselessness [and] suffering. That is our [view] in this [matter].\\

\pali{\fbox{\stepcounter{sennum}\arabic{sennum}} Iti kho, k\=al\=am\=a, ya\d m ta\d m avocumh\=a -- `etha tumhe, k\=al\=am\=a! M\=a anussavena, m\=a paparampar\=aya, m\=a itikir\=aya, m\=a pi\d takasampad\=anena, m\=a takkahetu, m\=a nayahetu, m\=a \=ak\=araparivitakkena, m\=a di\d t\d thinijjh\=anakkhantiy\=a, m\=a bhabbar\=upat\=aya, m\=a sama\d no no gar\=u'ti. Yad\=a tumhe k\=al\=am\=a attan\=ava j\=aneyy\=atha -- `ime dhamm\=a akusal\=a, ime dhamm\=a s\=avajj\=a, ime dhamm\=a vi\~n\~nugarahit\=a, ime dhamm\=a samatt\=a sam\=adinn\=a ahit\=aya dukkh\=aya sa\d mvattant\=i'ti, atha tumhe, k\=al\=am\=a, pajaheyy\=ath\=a'ti, \fbox{\stepcounter{sennum}\arabic{sennum}} iti ya\d m ta\d m vutta\d m, idameta\d m pa\d ticca vutta\d m.}

\addtocounter{sennum}{-2}
$\triangleright$ \fbox{\stepcounter{sennum}\arabic{sennum}} Therefore, K\=al\=amas, which [teaching I] have said, that is [as follows]: ``Come, K\=al\=amas. Do not [take merely] by hearsay. \ldots you should abandon [them]. \fbox{\stepcounter{sennum}\arabic{sennum}} Thus, that is what was said. This was said on account of that [teaching].\\

\pali{\fbox{\stepcounter{sennum}\arabic{sennum}} Etha tumhe, k\=al\=am\=a, m\=a anussavena, m\=a parampar\=aya, m\=a itikir\=aya, m\=a pi\d takasampad\=anena, m\=a takkahetu, m\=a nayahetu, m\=a \=ak\=araparivitakkena, m\=a di\d t\d thinijjh\=anakkhantiy\=a, m\=a bhabbar\=upat\=aya, m\=a sama\d no no gar\=u'ti. \fbox{\stepcounter{sennum}\arabic{sennum}} Yad\=a tumhe, k\=al\=am\=a, attan\=ava j\=aneyy\=atha -- `ime dhamm\=a kusal\=a, ime dhamm\=a anavajj\=a, ime dhamm\=a vi\~n\~nuppasatth\=a, ime dhamm\=a samatt\=a sam\=adinn\=a hit\=aya sukh\=aya sa\d mvattant\=i'ti, atha tumhe, k\=al\=am\=a, upasampajja vihareyy\=atha.}

\addtocounter{sennum}{-2}
$\triangleright$ \fbox{\stepcounter{sennum}\arabic{sennum}} Come, K\=al\=amas. Do not [take merely] by hearsay. \ldots\ \fbox{\stepcounter{sennum}\arabic{sennum}} Inasmuch as, K\=al\=amas, you know only by yourselves thus, ``These teachings [are] skillful. These teachings [are] not blameworthy. These teachings [are] praised by the wise. These teachings, [when] completely taken upon, lead to usefulness [and] happiness.'' Then, K\=al\=amas, taking upon [those], you should live [by them].\\

\pali{\fbox{\stepcounter{sennum}\arabic{sennum}} Ta\d m ki\d m ma\~n\~natha, k\=al\=am\=a, alobho purisassa ajjhatta\d m uppajjam\=ano uppajjati hit\=aya v\=a ahit\=aya v\=a'ti? \fbox{\stepcounter{sennum}\arabic{sennum}} Hit\=aya, bhante.}

\addtocounter{sennum}{-2}
$\triangleright$ \fbox{\stepcounter{sennum}\arabic{sennum}} How do you think about that, K\=al\=amas? [When] arising inside a person, does the absence of greed arise for usefulness or uselessness? \fbox{\stepcounter{sennum}\arabic{sennum}} Usefulness, sir.\\

\begin{longtable}[c]{|p{0.9\linewidth}|}
\hline
\hspace{5mm}\small In English, it is difficult to find an opposite term of greed that fits \pali{alobha}. Some suggest `disinterestedness' or `generosity.' But these still sound not right to me.\\
\hspace{5mm}\small At first, I thought negating the term straightly as `non-greed' could work, but it can mean all things except greed. So, `the absence of greed' is the closest meaning of \pali{alobha}. Other two terms are treated in the same way.\\
\hline
\end{longtable}

\pali{\fbox{\stepcounter{sennum}\arabic{sennum}} Aluddho pan\=aya\d m, k\=al\=am\=a, purisapuggalo lobhena ana\-bhibh\=uto apariy\=adinnacitto neva p\=a\d na\d m hanati, na adinna\d m \=adiyati, na parad\=ara\d m gacchati, na mus\=a bha\d nati, na parampi tathatt\=aya sam\=adapeti, ya\d m sa hoti d\=igharatta\d m hit\=aya sukh\=ay\=a'ti. \fbox{\stepcounter{sennum}\arabic{sennum}} Eva\d m, bhante.}

\addtocounter{sennum}{-2}
$\triangleright$ \fbox{\stepcounter{sennum}\arabic{sennum}} K\=al\=amas, this greedless person, not overpowered [and] fully taken up by greed, does not kill a living being, does not take the ungiven [things], does not go to other's wife, does not tell a lie, and does not make others do likewise. [When] that occurs, [it is] for usefulness [and] happiness for a long time. \fbox{\stepcounter{sennum}\arabic{sennum}} It is so, sir.\\

\pali{\fbox{\stepcounter{sennum}\arabic{sennum}} Ta\d m ki\d m ma\~n\~natha, k\=al\=am\=a, adoso purisassa ajjhatta\d m uppajjam\=ano uppajjati \ldots\ amoho purisassa ajjhatta\d m uppajjam\=ano uppajjati \ldots\ hit\=aya sukh\=ay\=a'ti. Eva\d m bhante.}

\addtocounter{sennum}{-1}
$\triangleright$ \fbox{\stepcounter{sennum}\arabic{sennum}} How do you think about that, K\=al\=amas? [When] arising inside a person, does the absence of anger arise for usefulness or uselessness? \ldots does the absence of delusion arise for usefulness or uselessness? \ldots\ It is so, sir.\\

\pali{\fbox{\stepcounter{sennum}\arabic{sennum}} Ta\d m ki\d m ma\~n\~natha, k\=al\=am\=a, ime dhamm\=a kusal\=a v\=a akusal\=a v\=a'ti? Kusal\=a, bhante.}

\addtocounter{sennum}{-1}
$\triangleright$ \fbox{\stepcounter{sennum}\arabic{sennum}} How do you think about that, K\=al\=amas? Are these kinds of nature skillful or unskillful? Skillful, sir.\\

\pali{\fbox{\stepcounter{sennum}\arabic{sennum}} S\=avajj\=a v\=a anavajj\=a v\=a'ti? Anavajj\=a, bhante.}

\addtocounter{sennum}{-1}
$\triangleright$ \fbox{\stepcounter{sennum}\arabic{sennum}} [Are they] blameworthy or praiseworthy? Praiseworthy, sir.\\

\pali{\fbox{\stepcounter{sennum}\arabic{sennum}} Vi\~n\~nugarahit\=a v\=a vi\~n\~nuppasatth\=a v\=a'ti? Vi\~n\~nuppasatth\=a, bhante.}

\addtocounter{sennum}{-1}
$\triangleright$ \fbox{\stepcounter{sennum}\arabic{sennum}} [Are they] reproached or praised by the wise? Praised by the wise, sir.\\

\pali{\fbox{\stepcounter{sennum}\arabic{sennum}} Samatt\=a sam\=adinn\=a hit\=aya sukh\=aya sa\d mvattanti no v\=a? Katha\d m v\=a ettha hot\=i'ti? Samatt\=a, bhante, sam\=adinn\=a hit\=aya sukh\=aya sa\d mvattanti. Eva\d m no ettha hot\=i'ti.}

\addtocounter{sennum}{-1}
$\triangleright$ \fbox{\stepcounter{sennum}\arabic{sennum}} [When] completely taken upon, [do they] lead to usefulness [and] happiness or not? Or how [do you think] about this? [When] completely taken upon, sir, [they] lead to usefulness [and] happiness. That is our [view] in this [matter].\\

\pali{\fbox{\stepcounter{sennum}\arabic{sennum}} Iti kho, k\=al\=am\=a, ya\d m ta\d m avocumh\=a -- `etha tumhe, k\=al\=am\=a! M\=a anussavena, m\=a parampar\=aya, m\=a itikir\=aya, m\=a pi\d takasampad\=anena, m\=a takkahetu, m\=a nayahetu, m\=a \=ak\=araparivitakkena, m\=a di\d t\d thinijjh\=anakkhantiy\=a, m\=a bhabbar\=upat\=aya, m\=a sama\d no no gar\=u'ti. Yad\=a tumhe, k\=al\=am\=a, attan\=ava j\=aneyy\=atha -- `ime dhamm\=a kusal\=a, ime dhamm\=a anavajj\=a, ime dhamm\=a vi\~n\~nuppasatth\=a, ime dhamm\=a samatt\=a sam\=adinn\=a hit\=aya sukh\=aya sa\d mvattant\=i'ti, atha tumhe, k\=al\=am\=a, upasampajja vihareyy\=ath\=a'ti, iti ya\d m ta\d m vutta\d m idameta\d m pa\d ticca vutta\d m.}

\addtocounter{sennum}{-1}
$\triangleright$ \fbox{\stepcounter{sennum}\arabic{sennum}} Therefore, K\=al\=amas, which [teaching I] have said, that is [as follows]: ``Come, K\=al\=amas. Do not [take merely] by hearsay. \ldots taking upon [those], you should live [by them]. Thus, that is what was said. This was said on account of that [teaching].\\

\pali{\fbox{\stepcounter{sennum}\arabic{sennum}} Sa [= Yo] kho so, k\=al\=am\=a, ariyas\=avako eva\d m vigat\=abhijjho vigataby\=ap\=ado asamm\=u\d lho sampaj\=ano patissato [so] mett\=asahagatena cetas\=a eka\d m disa\d m pharitv\=a viharati, tath\=a dutiya\d m, tath\=a tatiya\d m, tath\=a catuttha\d m, iti uddhamadho tiriya\d m sabbadhi sabbattat\=aya sabb\=avanta\d m loka\d m mett\=asahagatena cetas\=a vipulena mahaggatena appam\=a\d nena averena aby\=apajjhena pharitv\=a viharati.}

\addtocounter{sennum}{-1}
$\triangleright$ \fbox{\stepcounter{sennum}\arabic{sennum}} That noble disciple, who was free from avarice, free form ill-will, not deluded, thoughtful, [and] mindful, lives with the mind equipped with loving-kindness spreading in one direction, so does in the second [direction], in the third, in the fourth. Thus, [he] lives with the mind equipped with loving-kindness spreading to the whole world, everywhere upward, downward [and] crosswise, extensively, greatly, boundlessly, cordially, humanely.\\

\begin{longtable}[c]{|p{0.9\linewidth}|}
\hline
\hspace{5mm}\small Again, the isolated \pali{sa} at the beginning looks unusual. It should be \pali{yo}, corresponding to \pali{ariyas\=avako}. I also insert \pali{so} to make the \pali{ya-ta} structure easier to recognize.\\
\hspace{5mm}\small I split this long instance into two English sentences. To understand how I work on the latter part, you have to jump back and forth. For the nouns with instrumental case (\pali{vipulena} to \pali{aby\=apajjhena}), I treat them as adverbs, hence `extensively' and so on.\\
\hspace{5mm}\small A techinical term here is \pali{mahaggata} (gone great). It is a state of mind in meditation, as we see in \pali{mahaggata\d m citta\d m}.\footnote{in D2\,381 (DN\,22), for example}\\
\hspace{5mm}\small In Visuddhimagga, it is explained shortly as ``\pali{bh\=umivasena pana eta\d m mahaggata\d m}'' (By the contribution of the ground, [it is] this \pali{mahaggata}).\footnote{Vism\,9.254} I think it means a great mind like the ground or the earth that can uphold everything on it.\\
\hspace{5mm}\small From \pali{mett\=asahagatena} onwords, it is a template for the so-called \emph{loving-kindness meditation}. We can find this portion, and the other three factors, in many places in the canon where this kind of meditation is mentioned.\footnote{in D1\,556 (DN\,13), D2\,262 (DN\,17), M1\,77 (MN\,7), for example}\\
\hline
\end{longtable}

\pali{\fbox{\stepcounter{sennum}\arabic{sennum}} Karu\d n\=asahagatena cetas\=a \ldots\ mudit\=asahagatena cetas\=a \ldots\ upekkh\=asahagatena cetas\=a eka\d m disa\d m pharitv\=a viharati, tath\=a dutiya\d m, tath\=a tatiya\d m, tath\=a catuttha\d m, iti uddhamadho tiriya\d m sabbadhi sabbattat\=aya sabb\=avanta\d m loka\d m upekkh\=asahagatena cetas\=a vipulena mahaggatena appam\=a\d nena averena aby\=apajjhena pharitv\=a viharati.}

\addtocounter{sennum}{-1}
$\triangleright$ \fbox{\stepcounter{sennum}\arabic{sennum}} [That noble disciple] lives with the mind equipped with compassion, \ldots\ with sympathetic joy, \ldots\ with equanimity \ldots\\

\pali{\fbox{\stepcounter{sennum}\arabic{sennum}} Sa [= Yo] kho so, k\=al\=am\=a, ariyas\=avako eva\d m averacitto eva\d m aby\=apajjhacitto eva\d m asa\d mkili\d t\d thacitto eva\d m visuddhacitto. Tassa di\d t\d theva dhamme catt\=aro ass\=as\=a adhigat\=a honti.}

\addtocounter{sennum}{-1}
$\triangleright$ \fbox{\stepcounter{sennum}\arabic{sennum}} K\=al\=amas, which that noble disciple, having such a friendly mind, having such a merciful mind, having such an impeccable mind, having such a pure mind, his obtained four consolations in this world exist.\\

\begin{longtable}[c]{|p{0.9\linewidth}|}
\hline
\hspace{5mm}\small We find \pali{ya-ta} structure again here, if we replace the odd \pali{sa} with \pali{yo}, pairing with \pali{tassa}. So, we have to combine two P\=ali sentences into one.\\
\hline
\end{longtable}

\pali{\fbox{\stepcounter{sennum}\arabic{sennum}} Sace kho pana atthi paro loko, atthi sukatadukka\d t\=ana\d m kamm\=ana\d m phala\d m vip\=ako, ath\=aha\d m k\=ayassa bhed\=a para\d m mara\d n\=a sugati\d m sagga\d m loka\d m upapajjiss\=am\=i'ti, ayamassa pa\d thamo ass\=aso adhigato hoti.}

\addtocounter{sennum}{-1}
$\triangleright$ \fbox{\stepcounter{sennum}\arabic{sennum}} (1) If the other world exists, [and] the fruit of good and bad actions exists, then I will get reborn in a good heaven after the death of the body. This is the first obtained consolation.\\

\pali{\fbox{\stepcounter{sennum}\arabic{sennum}} Sace kho pana natthi paro loko, natthi sukatadukka\d t\=ana\d m kamm\=ana\d m phala\d m vip\=ako, ath\=aha\d m di\d t\d theva dhamme avera\d m aby\=apajjha\d m an\=igha\d m sukhi\d m att\=ana\d m parihar\=am\=i'ti, ayamassa dutiyo ass\=aso adhigato hoti.}

\addtocounter{sennum}{-1}
$\triangleright$ \fbox{\stepcounter{sennum}\arabic{sennum}} (2) If the other world does not exist, [and] the fruit of good and bad actions does not exist, then I keep up being generous, being merciful, being free from trouble, [and] my own happiness in this world. This is the second obtanied consolation.\\

\pali{\fbox{\stepcounter{sennum}\arabic{sennum}} Sace kho pana karoto kar\=iyati p\=apa\d m, na kho pan\=aha\d m kassaci p\=apa\d m cetemi. Akaronta\d m kho pana ma\d m p\=apakamma\d m kuto dukkha\d m phusissat\=i'ti, ayamassa tatiyo ass\=aso adhigato hoti.}

\addtocounter{sennum}{-1}
$\triangleright$ \fbox{\stepcounter{sennum}\arabic{sennum}} (3) If when [a person] is doing [an evil action], the evil action is done, [and] I do not intend the evil action for anybody, how will suffering reach me who do no [such an action]? This is the third obtained consolation.\\

\begin{longtable}[c]{|p{0.9\linewidth}|}
\hline
\hspace{5mm}\small This instance is tricky. A difficult word is \pali{karoto}, a genitive form of present participle \pali{karonta} (doing).\footnote{See Sadd-Pad Ch.\,7 by searching `\pali{Karontasaddassa}.' See also declensions of irregular nouns in Appendix \externalref{B.4} of PNL.} This means it is a form of absolute construction. Making it `\pali{[puggalassa p\=apa\d m] karoto}' may help. And \pali{kar\=iyati} is in passive form. To make our if-clause grammatical, we have to combine P\=ali sentences into one.\\
\hline
\end{longtable}

\pali{\fbox{\stepcounter{sennum}\arabic{sennum}} Sace kho pana karoto na kar\=iyati p\=apa\d m, ath\=aha\d m \linebreak ubhayeneva visuddha\d m att\=ana\d m samanupass\=am\=i'ti, ayamassa catuttho ass\=aso adhigato hoti.}

\addtocounter{sennum}{-1}
$\triangleright$ \fbox{\stepcounter{sennum}\arabic{sennum}} (4) If when [a person] is doing [an evil action], the evil action is not done, then I see myself flawless by both sides. This is the fourth obtained consolation.\\

\begin{longtable}[c]{|p{0.9\linewidth}|}
\hline
\hspace{5mm}\small In the commentary, the explanation for `by both sides' goes as follows: \pali{yañca p\=apa\d m na karomi, ya\~nca karotopi na kar\=iyati.} This means roughly, in the first case, I do no evil action, and in the second case, if I do but it has no result.\\
\hline
\end{longtable}

\pali{\fbox{\stepcounter{sennum}\arabic{sennum}} Sa kho so, k\=al\=am\=a, ariyas\=avako eva\d m averacitto eva\d m aby\=apajjhacitto eva\d m asa\d mkili\d t\d thacitto eva\d m visuddhacitto. Tassa di\d t\d theva dhamme ime catt\=aro ass\=as\=a adhigat\=a hont\=i'ti.}

\addtocounter{sennum}{-1}
$\triangleright$ \fbox{\stepcounter{sennum}\arabic{sennum}} Thus, K\=al\=amas, which that noble disciple, having such a friendly mind, having such a merciful mind, having such an impeccable mind, having such a pure mind, his obtained four consolations in this world exist.\\

\pali{\fbox{\stepcounter{sennum}\arabic{sennum}} Evameta\d m, bhagav\=a, evameta\d m, sugata! \fbox{\stepcounter{sennum}\arabic{sennum}} Sa kho so, bhante, ariyas\=avako eva\d m averacitto eva\d m aby\=apajjhacitto eva\d m asa\d mkili\d t\d thacitto eva\d m visuddhacitto. Tassa di\d t\d theva dhamme catt\=aro ass\=as\=a adhigat\=a honti. Sace kho pana atthi paro loko, atthi sukatadukka\d t\=ana\d m kamm\=ana\d m phala\d m vip\=ako, ath\=aha\d m k\=ayassa bhed\=a para\d m mara\d n\=a sugati\d m sagga\d m loka\d m upapajjiss\=am\=i'ti, ayamassa pa\d thamo ass\=aso adhigato hoti. Sace kho pana natthi paro loko, natthi sukatadukka\d t\=ana\d m kamm\=ana\d m phala\d m vip\=ako, ath\=aha\d m di\d t\d theva dhamme avera\d m aby\=apajjha\d m an\=igha\d m sukhi\d m att\=ana\d m parihar\=am\=i'ti, ayamassa dutiyo ass\=aso adhigato hoti. Sace kho pana karoto kar\=iyati p\=apa\d m, na kho pan\=aha\d m -- kassaci p\=apa\d m cetemi, akaronta\d m kho pana ma\d m p\=apakamma\d m kuto dukkha\d m phusissat\=i'ti, ayamassa tatiyo ass\=aso adhigato hoti. Sace kho pana karoto na kar\=iyati p\=apa\d m, ath\=aha\d m ubhayeneva visuddha\d m att\=ana\d m samanupass\=am\=i'ti, ayamassa catuttho ass\=aso adhigato hoti.}

\addtocounter{sennum}{-2}
$\triangleright$ \fbox{\stepcounter{sennum}\arabic{sennum}} That is so, the Blessed One. That is so, the Thus Gone. \fbox{\stepcounter{sennum}\arabic{sennum}} Sir, which that noble disciple, having such a friendly mind, \ldots\ This is the fourth obtained consolation.\\

\pali{\fbox{\stepcounter{sennum}\arabic{sennum}} Sa kho so, bhante, ariyas\=avako eva\d m averacitto eva\d m aby\=apajjhacitto eva\d m asa\d mkili\d t\d thacitto eva\d m visuddhacitto. Tassa di\d t\d theva dhamme ime catt\=aro ass\=as\=a adhigat\=a honti.}

\addtocounter{sennum}{-1}
$\triangleright$ \fbox{\stepcounter{sennum}\arabic{sennum}} Sir, which that noble disciple, \ldots\\

\pali{\fbox{\stepcounter{sennum}\arabic{sennum}} abhikkanta\d m, bhante, abhikkanta\d m, bhante, seyyath\=api, bhante, nikkujjita\d m v\=a ukkujjeyya, pa\d ticchanna\d m v\=a vivareyya, m\=u\d lhassa v\=a magga\d m \=acikkheyya, andhak\=are v\=a telapajjota\d m dh\=ar\-eyya `cakkhumanto r\=up\=ani dakkhant\=i'ti evameva\d m bhagavat\=a anekapariy\=ayena dhammo pak\=asito.}

\addtocounter{sennum}{-1}
$\triangleright$ \fbox{\stepcounter{sennum}\arabic{sennum}} Very enjoyable, sir. Very enjoyable, sir. The Dhamma illustrated by the Blessed One in various ways as such, sir, [is] like turning up what is faced downwards, or exposing what is covered, or telling the direction to a lost one, or holding an oil lamp in the dark [by thinking that] ``Those having eyes will see images.''\\

\begin{longtable}[c]{|p{0.9\linewidth}|}
\hline
\hspace{5mm}\small In the text we have, the full part of this is omitted by ellipsis. This is a frequently used template. It often occurs after an exposition. I reproduce the whole of it here, because it is worth a translation.\footnote{The full text of this part can be found in, for example, Mv\,1.27, D1\,404 (DN\,8).}\\
\hspace{5mm}\small To put it more precisely, for example, ``\pali{pa\d ticchanna\d m v\=a vivareyya}'' can be translated as ``or [like] one should expose what is covered.''\\
\hline
\end{longtable}

\pali{\fbox{\stepcounter{sennum}\arabic{sennum}} Ete maya\d m, bhante, bhagavanta\d m sara\d na\d m gacch\=ama dhamma\~nca bhikkhusa\.ngha\~nca. Up\=asake no, bhante, bhagav\=a dh\=aretu ajjatagge p\=a\d nupete sara\d na\d m gate'ti.}

\addtocounter{sennum}{-1}
$\triangleright$ \fbox{\stepcounter{sennum}\arabic{sennum}} We, sir, go to the Blessed One, also the Dhamma and the Sangha, as the refuge. May the Blessed One, sir, hold that [we are] lay followers who took the refuge from now on to the end of life.\\

\begin{longtable}[c]{|p{0.9\linewidth}|}
\hline
\hspace{5mm}\small It is unusual to say `these we' in English, so `\pali{ete maya\d m}' is just `we.' Maybe, `all of us here' sounds closer.\\
\hline
\end{longtable}

\pali{\fbox{\stepcounter{sennum}\arabic{sennum}} Pa\~ncama\d m.}

\addtocounter{sennum}{-1}
$\triangleright$ \fbox{\stepcounter{sennum}\arabic{sennum}} The fifth [was finished].\\

\phantomsection
\addcontentsline{toc}{section}{Conclusion and discussion}
\section*{3.\ Conclusion and discussion}

Now we have an idea how an early sutta is constructed. The text itself looks old, at least the main content, because we can see some glitches left unedited by the redactors. It is also put into format by using standard templates. Its own content may have not so many words in essence, but repetition makes the sutta long. By construction as such, the sutta can be memorized and recited together. That is perhaps the best way to deal with texts in oral transmission, or the best way that those in the time were able to think of.

By modern view, it is reasonable to call into question the originality of the idea expressed in the text. As we have seen, the text looks like a patchwork of normalized materials. It does not represent exactly what the conversation in the event really is. The text is a composition of ideas, not a direct report.

When I mention patchwork, a picture of text Frankenstein comes to my mind. It is always disputable whether this way of textual formation can be counted as authentic. If we see that Buddhism is not just the Buddha, but the whole body of intellectual enterprises throughout the history, what we learn from the existing texts is always relevant to the understanding of Buddhism, one way or another. By this view, an attempt to reconstruct the original Buddhism turns futile, and perhaps impossible.

After reading the sutta, the readers should answer this question: What is the subject matter of this discourse? Unlike other suttas that their subject matters look obvious, in this sutta it is a little difficult to figure out. We may grasp the main idea differently depending on what we look for. I will tell my answer at the end, after the discussion.

Let us look at the content of this sutta. The well-known part of this is the ten sources of knowledge that should be treated carefully, reproduced as the list below.\footnote{The same account is also found in S\=a\d lhasutta (A3\,67), next to this one, and Bhaddiyasutta (A4\,193).}

\begin{quote}
\begin{compactenum}[(1)]
\item hearsay
\item traditional practice
\item thus-hearing
\item referring to text
\item just thinking
\item (poor) inference
\item mere appearance
\item agreement with our view
\item appearance of likelihood
\item just regarding the speaker as our teacher
\end{compactenum}
\end{quote}

It does not mean that all these sources cannot give us some truth. It can to some extent. Rather the Buddha stresses that we should not fully rely on those shaky sources. As we have read the sutta, we can see that these ten sources in fact play little role to the main content of the discourse. That is why this sutta is not grouped in the ten-fold collection, but in this three-fold collection.

As suggested by the three-fold group, the main focus of this sutta formulated by the compilers is about three defilements (greed, anger, delusion) and their opposites. The account of these is a clich\'e, so I will not bring them to discussion. What I want to underline is the way reasoning is used in this teaching.

The criteria to decide what should be followed proposed by the Buddha, reworded by me, are whether the claim is likely to bring a good or bad result, whether it is blameworthy or praiseworthy, how the wise say about it, and when the claim is put into practice, whether it succeeds or not. If the results of these tests turn positive, one should embrace that claim and live by it.

This can assert the Buddha's practical attitude towards truth claims. He does not judge a particular idea right or wrong. Instead, he encourages people to exercise their reasoning. They have to think over the claim critically by themselves, not just accept it at face value. Authority is also important in this regard, because we often lack information, especially technical knowledge about the issue. So, consulting an expert can yield a good result. Moreover, the Buddha suggests that the claim have to put into test, by practicing it in real situation. Only if it survives the test, the claim is worth the acceptance.

By this understanding, we can say that the Buddha encourages followers and people to use reasoning to assess any truth claim, including claims in Buddhism itself. Examples the Buddha illustrates are greed, anger, and delusion. The result is obvious. But the very point the Buddha wants to make is not about the criteria as such. It is about how to lead people to the realization that these three things make us suffer. It is the application part of the method he has provided.

In the last part, the Buddha also uses reasoning to deal with undetermined issues, ending up with a line of thought partly similar to Pascal's wager. It really does not matter whether the afterlife exists or not. If we do good things, the result is always good. By this `good,' it is not relative to anything, culturally or individually. In Buddhist view, an action is good because it does not come from greed, anger, and delusion (this includes unhealthy obsession, fear, belief, and so on). 

Because all of us is the same kind of being, what are counted as greed, anger, and delusion are commonly discernible. So, the Buddha's method here is not relativism in the sense that the individuals have to think and test by themselves. Rather he uses his method to expose our objective human nature, independent of personal or cultural preferences.

That can answer why this sutta is so popular in modern time. And this is my favorite one. If you are not Buddhist, you should study it. If you are a Buddhist, you even have to read it carefully. And reading it in P\=ali is highly recommended.

Here comes the subject matter of this sutta. My method is to figure out what the title of this discourse should be if it is published as an article. Here is my minimalist answer: ``The Buddha's rationality,'' or more generalized, ``Rationality in Buddhism.''
